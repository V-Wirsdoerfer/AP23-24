\documentclass[
  bibliography=totoc,     % Literatur im Inhaltsverzeichnis
  captions=tableheading,  % Tabellenüberschriften
  titlepage=firstiscover, % Titelseite ist Deckblatt
]{scrartcl}

% Paket float verbessern
\usepackage{scrhack}

% Warnung, falls nochmal kompiliert werden muss
\usepackage[aux]{rerunfilecheck}

% unverzichtbare Mathe-Befehle
\usepackage{amsmath}
% viele Mathe-Symbole
\usepackage{amssymb}
% Erweiterungen für amsmath
\usepackage{mathtools}

% Fonteinstellungen
\usepackage{fontspec}
% Latin Modern Fonts werden automatisch geladen
% Alternativ zum Beispiel:
%\setromanfont{Libertinus Serif}
%\setsansfont{Libertinus Sans}
%\setmonofont{Libertinus Mono}

% Wenn man andere Schriftarten gesetzt hat,
% sollte man das Seiten-Layout neu berechnen lassen
\recalctypearea{}

% deutsche Spracheinstellungen
\usepackage[ngerman]{babel}


\usepackage[
  math-style=ISO,    % ┐
  bold-style=ISO,    % │
  sans-style=italic, % │ ISO-Standard folgen
  nabla=upright,     % │
  partial=upright,   % │
  mathrm=sym,        % ┘
  warnings-off={           % ┐
    mathtools-colon,       % │ unnötige Warnungen ausschalten
    mathtools-overbracket, % │
  },                       % ┘
]{unicode-math}

% traditionelle Fonts für Mathematik
\setmathfont{Latin Modern Math}
% Alternativ zum Beispiel:
%\setmathfont{Libertinus Math}

\setmathfont{XITS Math}[range={scr, bfscr}]
\setmathfont{XITS Math}[range={cal, bfcal}, StylisticSet=1]

% Zahlen und Einheiten
\usepackage[
  locale=DE,                   % deutsche Einstellungen
  separate-uncertainty=true,   % immer Unsicherheit mit \pm
  per-mode=symbol-or-fraction, % / in inline math, fraction in display math
]{siunitx}

% chemische Formeln
\usepackage[
  version=4,
  math-greek=default, % ┐ mit unicode-math zusammenarbeiten
  text-greek=default, % ┘
]{mhchem}

% richtige Anführungszeichen
\usepackage[autostyle]{csquotes}

% schöne Brüche im Text
\usepackage{xfrac}

% Standardplatzierung für Floats einstellen
\usepackage{float}
\floatplacement{figure}{htbp}
\floatplacement{table}{htbp}

% Floats innerhalb einer Section halten
\usepackage[
  section, % Floats innerhalb der Section halten
  below,   % unterhalb der Section aber auf der selben Seite ist ok
]{placeins}

% Seite drehen für breite Tabellen: landscape Umgebung
\usepackage{pdflscape}

% Captions schöner machen.
\usepackage[
  labelfont=bf,        % Tabelle x: Abbildung y: ist jetzt fett
  font=small,          % Schrift etwas kleiner als Dokument
  width=0.9\textwidth, % maximale Breite einer Caption schmaler
]{caption}
% subfigure, subtable, subref
\usepackage{subcaption}

% Grafiken können eingebunden werden
\usepackage{graphicx}

% schöne Tabellen
\usepackage{tabularray}
\UseTblrLibrary{booktabs, siunitx}

% Verbesserungen am Schriftbild
\usepackage{microtype}

% Literaturverzeichnis
\usepackage[
  backend=biber,
]{biblatex}
% Quellendatenbank
\addbibresource{lit.bib}
\addbibresource{programme.bib}

% Hyperlinks im Dokument
\usepackage[
  german,
  unicode,        % Unicode in PDF-Attributen erlauben
  pdfusetitle,    % Titel, Autoren und Datum als PDF-Attribute
  pdfcreator={},  % ┐ PDF-Attribute säubern
  pdfproducer={}, % ┘
]{hyperref}
% erweiterte Bookmarks im PDF
\usepackage{bookmark}

% Trennung von Wörtern mit Strichen
\usepackage[shortcuts]{extdash}

\author{%
  Vincent Wirsdörfer\\%
  \href{mailto:vincent.wirsdoerfer@udo.edu}{authorA@udo.edu}%
  \and%
  Joris Daus\\%
  \href{mailto:joris.daus@udo.edu}{authorB@udo.edu}%
}
\publishers{TU Dortmund – Fakultät Physik}


\begin{document}
\section{Versuchsaufbau}
Das Geiger-Müller-Zählrohr ist mit einer Strahlenschutzwand umgeben. Es ist an ein Oszilloskop und eine Spannungsquelle angeschlossen.
Das Oszilloskop dient dazu die Totzeit des Zählrohres zu messen, indem es das Ausgangssignal misst. Die Spannungsquelle dient dazu eine 
konstante und kontrollierte Betriebsspannung aufzubauen. Diese kann auf \qty{1}{\volt} genau eingestellt werden und kann von 
\qty{0}{\volt} bis \qty{1000}{\volt} geregelt werden. Das Ausgangssignal geht neben dem Oszilloskop noch in einen Vorverstärker, um 
anschließend von einem Zähler detektiert zu werden. Der Vorverstärker hebt das Signal auf eine für den Zähler messbare Intensität. Der 
Zähler kann lediglich bis 67200 zählen. Messungen, die über 67200 Zählungen gehen, müssen genau beobachtet werden. Dabei muss gezählt 
werden wie oft das Maximum erreicht wird, um dann die Anzahl der Maxima mal 67200 plus den angezeigten Wert zusammenzuaddieren. Die 
radioaktiven Proben werden in einem Stativ in einem definierten Abstand vor dem Zählrohr eingespannt.  


\section{Versuchsdurchführung}
Die Messzeit wird auf dem Zähler eingestellt, sodass die Messung nach dem manuellen Starten automatisch gestoppt wird. Es wird immer 
der selbe Abstand der Proben zum Geiger-Müller-Zählrohr verwendet. Die Spannungsquelle hat drei Drehkknäufe mit der Beschriftung 0 bis 
10. Jeder dieser Drehregler ist für eine Dezimalstelle zuständig. Glatte Hunderterstellen werden durch die 10er Stellung 10 eingestellt, 
da dies sensibler ist als durch den Hunterregler. So wird als Beispiel 400 die 3 10 0 erreicht.
\subsection{Kennlinie des Geiger-Müller-Zählrohres}
\noindent Zur Bestimmung der Kennlinie muss die Anzahl der Pulse über einen gewissen Spannungsbreich gemessen werden. Dazu wird eine Probe 
vor dem Zählrohr eingespannt. Es werden über \qty{60}{\second} die Anzahl der Pulse gezählt. Die Messreihe beginnt mit einer Spannung 
von \qty{300}{\volt}. Es muss nicht geringer angefangen werden, da die erste Messung keine Zählung ergibt. Nach jeder Messung wird 
die Spannung um \qty{20}{\volt} erhöht. Dies wird solange durchgeführt, bis ein sehr starke Anstieg der Zählrate zu verzeichnen ist. 
Zu diesem Zeitpunkt ist die Dauerentladung erreicht. Weiteres Messen würde die Messapparatur beschädigen. 

\subsection{Strom des Geiger-Müller-Zählrohres}
Um die Funktion des Stroms zu messen werden definierte Spannungen eingestellt. Dies sind die selben Spannungen wie schon bei der 
Vermessung der Kennlinie. Der Strom wird durch die Ladungen erzeugt und kann auf dem Geiger-Müller-Zählrohr abgelesen werden. Dort 
ist ein analoger Strommesser, welcher Ströme in \unit{\micro\meter} Größe angibt. Der Strom ist jedoch nicht stabil und schwingt 
zwischenzeitig.

\subsection{Bestimmung der Totzeit}
Um die Totzeit des Geiger-Müller-Zählrohres zu bestimmen, wird sowohl die Spannungskurve am Oszilloskop abgelesen, als auch 
die zwei-Quellen-Methode angewandt. \\
\noindent Die Totzeit mithilfe des Oszilloskopes zu bestimmen bedarf keine großen Durchführung. Die Totzeit kann direkt abgelesen werden. 
Dazu werden der Tiefpunkt einer Kurve und der Punkt, an dem die Kurve abgeflacht ist, bestimmt. Die Differenz beider Punkte ist die Totzeit. \\
\noindent Für die Zwei-Quellen-Methode bedarf es, wie der Name es schon sagt, zwei unterschiedlich starker radioaktiver Proben. Es werden 
zunächst die Zerfallsraten beider Proben einzeln ermittelt. Im Anschluss werden beide Proben gleichzeitig mit dem Geiger-Müller-Zählrohr 
vermessen. Alle drei Messungen dauern jeweils \qty{120}{\second}.
Der Zähler läuft merhmals über, weshalb mitgezählt werden muss, wie oft dies passiert.

%\section{Messwerte}

\end{document}

