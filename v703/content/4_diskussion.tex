%\input{../../header.tex}

%\begin{document}

\section{Diskussion}
\label{sec:Diskussion}

\noindent Prinizipiell sind die gemessenen Ergebnisse des Versuchs zun Geiger-Müller-Zähler als positiv zu bewerten. Im Wesentlichen 
gilt es sowohl die Steigung des GM-Plateaus, als auch die Totzeit des Detektors auf zwei verschiedene Art und Weisen zu bestimemen. 
Im Optimalfall sind die jeweiligen Werte der Methoden selbstverständlich identisch. Aufgrund systematischer Fehlereinflüsse
sind Differenzen jedoch unvermeidbar. Diese sollen im Folgenden näher betrachtet werden.\\

\noindent Wie dem vorherigen Kapitel \ref{sec:Auswertung} zu entnehmen ist, wird die Steigung einerseits durch eine lineare 
Regression und andererseits durch die Formel \eqref{eqn:mPlateau} berechnet. Mittels dieser Methoden werden die folgenden 
Werte bestimmt:

\begin{align*}
    a_\text{Reg.} &= \qty{0.032\pm0.011}{\per \volt \per \second} \cdot 100\% / \qty{100}{\volt}\\
    a_\text{Form.} &= \qty{0.038\pm0.001}{\per \volt \per \second} \cdot 100\% / \qty{120}{\volt}
\end{align*}

\noindent Die Mittelwerte beider Messergebnisse unterscheiden sich erst in der dritten Nachkommastelle, was auf eine erfolgreiche 
und gelungene Untersuchung des Plateaus schließen lässt. Zusätzlich soll die Totzeit, also das Zeitintervall der Inaktivitätet des 
Geiger-Müller-Zälers untersucht werden. Hierzu wird die auf dem Oszilloskop erkennbare Graphik und die Formel \eqref{eqn:Totzeit}
verwendet.

\begin{align*}
    \tau_\text{Osz.} &= \qty{260\pm5}{\micro \second}\\
    \tau_\text{Form.} &= \qty{235\pm7}{\micro \second}
\end{align*}

\noindent Auch diese Werte weichen um nur etwa \qty{10}{\per\cent} voneinander ab, was ebenfalls eine korrekte Durchführung und keine 
signifikanten Messfehler impliziert.\\

\noindent Aufgrund der Tatsache, dass Großteile des Experiments bereits vorbereitet sind und sich der studentische Arbeitsverlauf 
essenziell auf die Bedienung der vorhandenen Gerät beschränkt, ist eine direkte Auflistung systematischer Fehlerquellen 
erschwert. Nichtsdestotrotz können auch beim Ablesen von analogen und digitalen Geräten Fehler passieren. So zum Beispiel 
zeigt die Abbildung \ref{fig:Oszilloskop}, wie komplex die Erkennung der tatsächlichen Totzeit mit dem Oszilloskop wegen der 
Verzerrtheit der Graphik ist. Dies könnte somit einen direkten Einfluss auf die Abweichung der bestimmten Totzeiten haben.
Ferner ist der Parallaxenfehler beim Ablesen der Stromstärken nicht zu vernachlässigen, welcher durch die Schwingung der Messnadel
verstärkt wird. Ein weiterer Bereich des Fehlerspektrums wird durch die Trägheit der Instrumente wie zum Beispiel des Vorverstärkers 
oder des Zählers abgedeckt. Somit können tatsächliche, durch ionisierende Strahlung verursachte, Impulse nicht gemessen werden, 
was den Verlauf der Kennlinie und daher auch die Steigung der Regressionsgerade maßgeblich beeinflusst. Die Akkumukation dieses 
Fehlerspektrums sind dementsprechend potentielle Gründe für die abweichenden Ergebnisse der jeweiligen Methoden.   
 
\section{Anhang}

\begin{figure}[H]
    \centering 
    \includegraphics[width=\textwidth]{content/v703_Laborbuch1.jpg}
\end{figure}

\begin{figure}[H]
    \centering 
    \includegraphics[width=\textwidth]{content/v703_Laborbuch2.jpg}
\end{figure}

%\end{document}
