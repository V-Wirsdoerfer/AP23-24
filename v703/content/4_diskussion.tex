\input{../../header.tex}

\begin{document}

\section{Diskussion}

\noindent Prinizipiell sind die gemessenen Ergebnisse des Versuchs zun Geiger-Müller-Zähler als positiv zu bewerten. Im Wesentlichen 
gilt es sowohl die Steigung des GM-Plateaus, als auch die Totzeit des Detektors auf zwei verschiedene Art und Weisen zu bestimemen. 
Im Optimalfall sind die jeweiligen Werte der Methoden selbstverständlich identisch. Aufgrund systematischer Fehlereinflüsse
sind Differenzen jedoch unvermeidbar. Diese sollen im Folgenden näher betrachtet werden.\\

\noindent Wie dem vorherigen Kapitel \ref{sec:Auswertung} zu entnehmen ist, wird die Steigung einerseits durch eine lineare 
Regression und andererseits durch die Formel \eqref{eqn:mPlateau} berechnet. Mittels dieser Methoden werden die folgenden 
Werte bestimmt:

\begin{align*}
    a_\text{Reg.} &= \qty{0.032\pm0.011}{\per \volt \per \second} \cdot 100\% / \qty{100}{\volt}\\
    a_\text{Form.} &= \qty{0.038\pm0.001}{\per \volt \per \second} \cdot 100\% / \qty{120}{\volt}
\end{align*}

\noindent Die Mittelwerte beider Messergebnisse unterscheiden sich erst in der dritten Nachkommastelle, was auf eine erfolgreiche 
und gelungene Untersuchung des Plateaus schließen lässt. Zusätzlich soll die Totzeit, also das Zeitintervall der Inaktivitätet des 
Geiger-Müller-Zälers untersucht werden. Hierzu wird die auf dem Oszilloskop erkennbare Graphik und die Formel \eqref{eqn:Totzeit}
verwendet.

\begin{align*}
    \tau_\text{Osz.} &= \qty{260\pm5}{\micro \second}\\
    \tau_\text{Form.} &= \qty{235\pm7}{\micro \second}
\end{align*}

\noindent 

\label{sec:Diskussion}


\end{document}
