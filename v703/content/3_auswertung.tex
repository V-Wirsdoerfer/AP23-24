\input{../../header.tex}

\begin{document}
\section{Auswertung}
\label{sec:Auswertung}

In diesem Versuch wird das Geiger-Müller-Zählrohr vermessen. So werden typische Charakteristika wie die Kennlinie und 
die Totzeit bestimmt.

\subsection{Kennlinie des Geiger-Müller-Zählrohrs}
Um die Kennlinie zu vermessen, werden die Betriebsspannung und die Zählrate vermessen, wie in der Durchführung 
beschrieben. So werden die folgenden Werte aufgenommen:

\begin{table}[H]
    \caption{Messwerte der Spannung und der Zählrate auf 120s.}
    \label{tab:Kennlinie}
    \begin{minipage}[t]{0.5\textwidth}
        \vspace{0pt}
        \centering
    \begin{tblr}{
    colspec = {S[table-format=3.0] S[table-format=5.0]},
    row{1} = {guard, mode = math} 
    }
    \toprule
    \text{Spannung} \mathbin{/} \unit{\volt} & \text{Zählrate N} \\
    \midrule
        300 &   0       \\
        320 &   0       \\
        340 &   15380   \\
        360 &   15886   \\
        380 &   16143   \\
        400 &   16272   \\
        420 &   16317   \\
        440 &   16477   \\
        460 &   16430   \\
        480 &   16539   \\
        500 &   16429   \\
    \end{tblr}
\end{minipage}\hfill
\begin{minipage}[t]{0.5\textwidth}
    \vspace{0pt}
    \centering
    \begin{tblr}{
        colspec = {S[table-format=3.0] S[table-format=5.0]},
        row{1} = {guard, mode = math} 
        }
        \toprule
        \text{Spannung} \mathbin{/} \unit{\volt} & \text{Zählrate N} \\
        \midrule
            520 &   16620   \\
            540 &   16846   \\
            560 &   16814   \\
            580 &   16662   \\
            600 &   16736   \\
            620 &   16947   \\
            640 &   16680   \\
            660 &   16762   \\
            680 &   17101   \\
            700 &   17400   \\
            720 &   17790   \\
            740 &   18738   \\
        \end{tblr}
    \end{minipage}\hfill
\end{table}

\noindent Diese Daten können nun gegeneinander aufgetragen werden. Durch die dann sichtbare Kennlinie 
des Zählrohrs kann das Arbeitsplateau bestimmt werden. 
Aus technischen Gründen misst das Zählrohr erst Teilchen ab einer Betriebsspannung von \qty{340}{\volt} 
Aus diesem Grund werden die ersten beiden Werte nicht mit in den Plot genommen.
Die Werte werden jeweils mit dem für Poissonverteilungen üblichen Fehler $\sqrt{n}$ versehen, wobei $n$ 
der jeweilige Messwert ist.
Anschließend wählt man nach Augenmaß die Werte aus, die auf einer Geraden, welche dem Arbeitsplateau 
entspricht, liegen. Dies sind die Werte von \qty{480}{\volt} bis \qty{660}{\volt}. Durch diese Auswahl 
der Messdaten wird nun eine Ausgleichsgerade gelegt. Diese Ausgleichsgerade besitzt eine Steigung $a$ von

\begin{align}
    a = \qty{2.26\pm0.35}{\per \volt}
\end{align}

\noindent Nun werden sowohl die Messdaten als auch die Ausgleichsgerade visualisiert.
So ergibt sich die folgende Kennlinie mit dem Arbeitsplateau des Geiger-Müller-Zählrohrs.

\begin{figure}[H]
    \centering
    \includegraphics[width=0.9\textwidth]{../build/Kennlinie.pdf}
    \caption{Kennlinie des Geiger-Müller-Zählrohrs.}
\end{figure}

\noindent Die Steigung des Arbeitsplateaus kann auch 


\subsection{Bestimmung der Totzeit}


\end{document}
