\input{../../header.tex}

\begin{document}
%\section{Versuchsaufbau}

\section{Versuchsdurchführung}

Der allgemeine Versuchsaufbau wird in der unten stehenden Abbildung visualisiert.

\begin{figure}
    \centering
    \includegraphics[height=6cm]{Aufbau.png}
    \caption{Versuchsaufbau des Franck-Hertz-Experiments\cite{Versuchsanleitung_v601}.}
    \label{fig:Aufbau}
\end{figure}

\noindent Zwangsweise wurden die grundlegenden Elemente des Franck-Hertz-Versuchs in der Theorie \ref{sec:Theorie} nahgelegt. Hauptbestandteil ist dabei 
das heizbare Gehäuse, welches die Quecksilber-Kammer sowie alle notwendigen Kathoden und Elektroden enthält. Der Großteil der quantenphysikalischen Entdeckung 
des Experiments wird somit bereits abgedeckt. Über ein elektrisches Thermometer kann die Temperatur des Gehäuses. Ein weiterer Bestandteil des Aufbaus ist 
das Picoamperemeter, welches den Auffängerstrom $I_\text{A}$ misst und an den XY-Schreiber weitergibt. Die detektierten Elektronen können somit auf Papier 
gebracht werden.\\

\noindent Zu Beginn des Versuchs werden alle Ein- und Ausgänge der relevanten Geräte verkabelt und von der Betreuerin überprüfen lassen. Im ersten Teil des Versuchs wird 
die integrale Energieverteilung der Elektronen untersucht. Hierzu wird bei einer konstanten Beschleunigungsspannung von $U_\text{B} = \qty{9}{\volt}$ der 
Auffängerstrom $I_\text{A}$ als Funktion der Bremsspannung $U_\text{A}$ gemessen. Das Intervall der Bremsspannung beträgt dabei [$\qty{0}{\volt};\qty{10}{\volt}$]. 
Hiebei wird je eine Messung bei Zimmertemperatur ($T \approx \qty{298.15}{\kelvin}$) und eine Messung bei einer erhöhten Temperatur von $T = \qty{414.55}{\kelvin}$
durchgeführt.\\

\noindent Im Anschluss wird die Franck-Hertz-Kurve für verschiedenen Temperaturen aufgenommen. Dementsprechend wird die Versuchsapparatur erneut verkabelt, sodass 
nun der Auffängerstrom als Funktion der Beschleunigunsspannung in einem Intervall [$\qty{0}{\volt};\qty{60}{\volt}$] gemessen werden kann. Die jetzt kosntante 
Bremsspannung hat den den Wert $U_\text{A} = \qty{1}{\volt}$. Die Franck-Hertz-Kurven werden bei den folgenden Temperaturen aufgezeichnet:

\begin{align*}
    T1 &= \qty{438.15}{\kelvin} \\ 
    T2 &= \qty{447.15}{\kelvin} \\  
    T3 &= \qty{459.15}{\kelvin} \\  
\end{align*}


%\section{Messwerte}

\end{document}

