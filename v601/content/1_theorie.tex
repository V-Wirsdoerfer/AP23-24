\input{../../header.tex}

\begin{document}

\section{Zielsetzung}
\label{sec:Zielsetung}

Der Franck-Hertz-Versuch ist ein grundlegendes Experiment im Hinblick auf die Diskretisierung von Energieniveaus und somit auf den quantenmechanischen 
Charakter eines Atoms. Das Experiment kann der Kategorie der \textbf{Elektronenstoßexperimente} zugeordnet werden, weshalb ein Großteil des Informationen 
aus dem Energieverlust gewonnen werden kann, welcher durch Stöße zweier Teilchen erfolgt. Ziel des Versuches ist es daher eine präzisere Vorstellung 
der Energiequantenlung innerhalb der Atomhülle zu erhalten und somit direkte Rückschlüsse auf die Energieniveaus zu ziehen.

\section{Theorie}
\label{sec:Theorie}

Wie bereits erwähnt repräsentiert der Franck-Hertz-Versuch ein Elektronenstoßexperiment. In diesem speziellen Fall lässt sich das grundlegende Prinzip 
demnach folgenderweise erklären. Die approximativ monoenergetischen Elektronen treffen auf die Atome eines Hg-Dampfes und wechselwirken dabei in Form von 
elastischen und elastischen Stößen. Hierbei besteht eine Äquivalanz zwischen jener Energie, die von den Hg-Atomen aufgenommen wird und der Energiedifferenz 
der Elektronen vor und nach dem Stoßprozess. Auf atomarer Ebene betrachtet werden die Hg-Atome von ihrem Grundzustand $E_0$ in den ersten angeregten
Zustand versetzt $E_1$. Mathematisch gilt somit der Zusammenhang 

\begin{equation*}
    E_\text{kin,vor} - E_\text{kin,nach} = E_1 - E_0
\end{equation*}, 

\noindent wobei sich die kinetische Energie des Elektrons über die Ruhemasse $m_0$ und die Geschwindigkeiten $v_\text{vor}$ vor dem Stoß bzw. $v_\text{nach}$
nach dem Stoß berechnen lässt. Um die Ermitlung der Elektronengeschwindigkeiten nachvollziehen zu können, ist es unabdingbar den allgemeinen Aufbau und die 
Funktionsweise des Franck-Hertz-Versuchs vorwegzunehmen.\\

\noindent Das Kernelement des Versuchs ist ein evakuiertes Gefäß mit Quecksilber-Dampf, welcher sich durch Sprühen winziger Tropfen langfristig einstellt.
Die Dichte das Dampfes kann dabei über die Umgebungsemperatur $T$ kontrolliert werden. Zusätzlich wird dem Gefäß ein Metalldraht aus Wolfram hinzugefügt, welcher 
durch den Anschluss einer Spannungsquelle erhitzt werden kann. Infolge des \emph{glühelektrischen Effekts} werden Elektronen emittiert und sind lokal 
um den Draht verteilt. Mittels einer positiv geladenen Elektrode können die Elektronen nun in z-Richtung beschleunigt werden. Visualisiert wird diese 
Konstruktion mithilfe der folgenden Abbildung.

\begin{figure}[H]
    \centering
    \includegraphics[height=6cm]{FH_Skizze.png}
    \caption{Skizzierter Versuchsaufbau des Franck-Hertz-Versuchs\cite{Versuchsanleitung_v601}.}
    \label{fig:FH_Skizze}
\end{figure}

\noindent Die dazu notwenige Beschleunigungsspannung wird im Folgendne als $U_\text{B}$ gekennzeichent und erfüllt die folgende Energiebilanz

\begin{equation*}
    \frac{1}{2}m_0v²_\text{vor} = e_0U_\text{B},
\end{equation*}

\noindent wobei $e_0$ die Elementarladung darstellt. Nachdem die Elektronen die Beschleunigungselektrode passieren, werden sie abgebremst. Diese ist, wie 
in Abb. \ref{fig:FH_Skizze} zu erkennen, auf die Existenz einer negativ geladenen \emph{Auffängerelektrode} zurückzuführen. Die auf der Oberfläche der
Auffängerelektrode ankommenden Elektronen bewirken einen Auffängerstrom $I_\text{A}$, welcher über ein Messinstrument ermittelt wird. Demnach werden nur 
jene Elektronen detektiert, dessen kinetische Energie größer oder gleich der von der Gegenspannung $U_\text{A}$ erzeugten elektrischen Energie ist:

\begin{equation*}
    \frac{1}{2}m_0v²_\text{z} \geq e_0U_\text{A}
\end{equation*}

\noindent Der physikalisch interaktive Bereich des Versuchs ist der Beschleunigungsraum. Hier können Elektonen und Hg-Atome miteinander wechselwirken.
Einserseits können die beiden Partner einen elastischen Stoß ausführen. Bei genauerer Betrachtung der Stoßpartner wird jedoch deutlich, warum diese Art von 
Stößen in der Bedeutung des Experiments nur eine untergeordnete Rolle spielt. Das Massenverhältnis zwischen Elektron und Hg-Atom beträgt in etwa 
$\sfrac{m_0}{M} \approx \sfrac{1}{1836}$, weshalb die Energieabgabe des Elektrons vernachlässigbar ist. Bei größer werdender Beschleunigungsspannung besitzen 
die Elektronen ausreichend kinetische Energie, um die Hg-Atome vom Grundzustand $E_0$ in den Zustand erster Anregung $E_1$ zu versetzen. Das Elektron übergibt 
somit exakt den dafür notwenigen Energiebetrag an die Hg-Atome und verbleibt ggf. mit einer Energie von $E - \left(E_1 - E_0\right)$. Nach einer gewissen
Zeit kehrt das Hg-Atom erneut in den Grundzustand zurück und emittiert dabei ein Photon der Energie 

\begin{equation*}
    E = hf = E_1 - E_2. 
\end{equation*}

\noindent Dabei steht $h$ für das Plancksche Wirkungsquantum und $f$ für die Frequenz des Lichts.\\

\subsection{Modellvorstellung der Franck-Hertz-Kurve}
\label{sec:Modellvorstellung}

\noindent Um nun einen akkuraten Überblick der Interaktionen zwischen Elektronen und Quecksilber-Atomen zu erhalten, wird der Auffängerstrom $I_\text{A}$
in Abhängigkeit der Beschleunigungsspannung $U_\text{B}$ gemessen. Idealisiert sollte dabei die folgenden Kurve zu beobachten sein:

\begin{figure}
    \centering
    \includegraphics[height=5.5cm]{FH_Modellkurve.png}
    \caption{Modellierter Kurververlauf des Auffängerstroms als Funktion der Beschleunigungsspannung\cite{Versuchsanleitung_v601}.}
    \label{fig:FH_Modellkurve}
\end{figure}

\noindent Der erste Bereich starken Anstiegs kann auf eine leichte Erhöhung der Beschleunigungsspannung zurückgeführt werden. Die kinetische Energie der 
Elektronen reicht maximal zu elastischen Stößen mit den Hg-Atomen, was den energetischen Verlauf der Elektronen nicht wesentlich beeinflusst. Dementsprechend 
können Großteile der Raumladungswolke an die Oberfläche der Auffängerkathode gelangen, was den Auffängerstrom rasant steigen lässt. Der darauffolgende,
instantane Abfall von $I_\text{A}$ begründet sich durch unelastische Stöße unmittelabar vor der Beschleunigungskathode. Nach Abgabe der Energie $E_1 - E_0$
haben die Elektronen keine Zeit neue Energie aufzunehmen und erreichen deshalb die Auffängerkathode nicht. In dieser Modellvorstellung haben alle Elektronen 
von Beginn an die gleiche Energie, was zu einem kollektiven Abfall auf $I_\text{A} = \qty{0}{\ampere}$ führt. Bei sukzessiver Erhöhung verschiebt sich effektiv 
die Stoßzone weiter vor die Beschleunigungselektrode, sodass die Elektronen nach dem Stoß erneut genügend Energie aufnehmen können um einen Auffängerstrom 
zu erzeugen. Dieser Trend setzt sich so lang fort bis die Elektronen nach dem ersten Stoß so viel Energie aufnehmen können, um ein zweites mal zu wechselwirken.
Die zweite Stoßgrenze liegt nun unmittelbar vor der Beschleunigungselektrode, weshalb ein erneuter, unstetiger Abfall in der Kurve zu betrachten ist. Dieser 
Prozess setzt sich periodisch fort, was mithilfe der obigen Skizze verdeutlich werden soll. Der Abstand $U_1$ korrespondiert nach dieser Erklärung mit 
der ersten Anregungsenergie:

\begin{equation*}
    U_1 = \frac{1}{e_0}\left(E_1 - E_0\right)
\end{equation*}

\subsection{Realitätseinflüsse}
\label{sec:Realitaetseinfluesse}

Die im vorherigen Teilkapitel \ref{sec:Modellvorstellung} dargestellte Modellvorstellung der Franck-Hertz-Kurve entspricht nur bis zu einem gewissen Grad dem 
tatsächlichen Verlauf. Grund dafür sind Umwelteinflüsse bzw. reale Effekte, welche im Folgenen diskutiert werden.\\

\subsubsection*{Kontaktpotential}

Um die Emission von Elektronen an der Glühkathode zu ereichtern, wird ein Material mit geringer Austrittsarbeit $\Phi_\text{G}$ verwendet, sodass bereits bei 
Temperaturen der Austritt von Elektronen stattfinden. Demgegenüber steht eine Beschleunigungskathode mit einer zumeist unterschiedlichen Austrittsarbeit $\Phi_\text{B}$.
Diese realen Potentialverhältnisse sorgen dafür, dass das tatsächliche Beschleunigungspotential nicht dem Wert der von außen angelegten Spannung entspricht. 
Diese Potentialgefälle wird in der folgenden Abbildung verdeutlicht.

\begin{figure}
    \centering
    \includegraphics[height=6cm]{Kontaktpotential.png}
    \caption{Einfluss des Kontaktpotentials im Franck-Hertz-Versuch\cite{Versuchsanleitung_v601}.}
    \label{fig:Kontaktpotential}
\end{figure}

\noindent Dieser Darstellung zur Folge besitz das reale Beschleunigungspotential $U_\text{B,eff}$ den Wert

\begin{equation}
    U_\text{B,eff} = U_\text{B} - \frac{1}{e_0}\left(\Phi_\text{B} - \Phi_\text{G}\right),
\label{eqn:UBEFF}
\end{equation}

\noindent wobei der Ausdruck 

\begin{equation*}
    K \coloneqq \frac{1}{e_0}\left(\Phi_\text{B} - \Phi_\text{G}\right)
\end{equation*}

\noindent als \textbf{Kontaktpotential} bezeichnet wird. Nach Gleichung \eqref{eqn:UBEFF} lässt sich nachvollziehen, dass eine Verschiebung der Franck-Hertz-Kurve 
um den Wert $K$ die Konsequenz des Kontaktpotentials ist. 

\subsection*{Energie-Spektrum der Elektonen}

Eine weitere getroffende Annahme ist der monoenergetische Charakter der emittierten Elektronen. Aufgrund der Tatsache, dass die Leiterelektronen bereits im Metall 
einer Energieverteilung unterliegen (Fermi-Dirac-Verteilung), besitzen sie im Moment der Emission verschiedenen Anfangsgeschwindigkeiten, was sich nach Durchlaufen 
des Beschleunigungspotentials in einer Energieverteilung wiederspiegelt. Diese Verteilung beginnt bei $U\text{B.eff}$ und flacht bei Betrachtung höherer Energien 
rapide ab. Der somit nicht exakte Einsatz von unelastischen Stößen bedingt daher, dass der Anstieg der Franck-Hertz-Kurve bei Annäherung an den Maximalwert leicht 
abflacht und nach einem Maximum stetig auf einen Stromminmum zuläuft. Dies widerspricht der in Abb. \ref{fig:FH_Modellkurve}cgezeigten idealen Franck-Hertz-Kurve. \\

\noindent Nicht nur die unelastischen Stöße beeinflussen die Gestalt der Kurve, sondern auch die elastischen Stößen tragen einen gewissen Beitrag dazu. Auch wenn die 
Energieabnahme der Elektronen aufgrund des fundamentalen Masseunterschiedes weiterhin vernachlässigt werden kann, wird im Folgenden die Richtungsändrung der Elektonen
betrachtet. Besonders auffällig werden die Zusammenstöße zwischen Elektronen und Hg-Atomen zwischen Beschleunigungs- und Auffängerelektrode. Die Richtungsändrung der 
Elektronen kann die Geschwindigkeit in z-Richtung $v_\text{z}$ dratisch beeinträchtigen, weshalb die Franck-Hertz-Kurve abgeflacht und verbreitert wird. 

\subsection*{Dampfdruck}

Die Grundlage für unelastische Zusammenstöße zwischen Stoßpartnern ist eine hinreichend kleine mittlere Weglänge $\bar{w}$ vergleichen mit dem Abstand $a$ zwischen 
Glükathode und Beschleunigungselektrode. Um dies zu gewährleisten, kann der in der Röhre herrschende Sättigungsdampfdruck $p_\text{sät}$ speziell eingestellt werden. 
Die Beziehung 

\begin{equation*}
    \bar{w} = \frac{0.0029}{p_\text{sät}}
\end{equation*}

\noindent zeigt das reziproke Verhältnis zwischen diese beiden Größen. Dem vorherigen Teilkapitel ist zu entnehmen, dass der Quecksilber-Dampdfruck maßgeblich von der 
Temperatur T beeinflusst werden kann, was sich in der Gleichung

\begin{equation*}
    p_\text{sät}(T) = 5.5\cdot10⁷\exp\left(-6876/T\right)
\end{equation*}

\noindent niederschlägt. Effektiv kann somit über die Temperatur des Gefäßes die mittlere freie Weglänge und kohärent dazu auch die Anzahl an Stößen gesteuert werden, 
sodass die Franck-Hertz-Kurve beobachtet werden kann.

%\section{Vorbereitung}

%\section{Fehlerrechnung}
\end{document}