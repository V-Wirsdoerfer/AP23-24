%\input{../../header.tex}

%\begin{document}
%\section{Versuchsaufbau}
\section{Versuchsdurchführung}

%wrapfigure
\begin{minipage}[t]{1\textwidth}
    \begin{wrapfigure}{H}{0.65\textwidth}
        \vspace{-15pt}
        \begin{center}
            \includegraphics[width=0.65\textwidth]{content/Schaltung.png}
            \caption{Schaltung einer Hochvakuumdiode und Aufbau des Experiments \cite{Versuchsanleitung_v504}.}
        \end{center}
    \end{wrapfigure}
    Der Versuchsaufbau wird nach dem Grundsatz der nebenstehenden Schaltung aufgebaut. An der Glühkathode 
    befindet sich so eine Heizspannung. Die Anode ist mit der Kathode über eine Saugspannung verbunden. 
    Außerdem ist ein Strommessgerät in Reihe zur Anode geschaltet, welches den Saugstrom misst. So kann 
    die Kennlinie der Hochvakuumdiode aufgenommen werden. Die Glühkathode besteht aus einem Wolframdraht. \\
\end{minipage}
\noindent Da der Anlaufstrom sehr klein ist, wird für die Aufzeichnung des Anlaufstroms ein empfindlicheres 
Strommessgerät verwendet als für den restlichen Teil der Kennlinie.
Zur Aufzeichnung des Anlaufstroms wird außerdem ein Gegenfeld aufgebaut, sodass negative Spannungen 
realisiert werden und immer weniger Elektronen die Anode erreichen. 


\section{Messwerte}
\label{sec:Messwerte}

Die für die Anlaufspannung entstehenden Messwerte sind in der folgenden Tabelle aufgeführt. 
Die Heizspannung beträgt hierbei \qty{5 \pm 0.5}{\volt} bei einem Heizstrom von \qty{2.4\pm 0.1}{\ampere}.
\begin{table}[H]
    \centering 
    \caption{Messung des Anlaufstromgebiets mithilfe der Gegenfeldmethode.}
    \begin{tblr}{
        colspec = {S[table-format=1.2] S[table-format=1.2]},
        row{1} = {guard, mode=math},
        }
        \toprule 
            \text{Saugspannung} \mathbin{/} \unit{\volt} & \text{Saugstrom} \mathbin{/} \unit{\nano\ampere} \\
        \midrule
        0.00    &   4.00    \\
        0.04    &   2.60    \\
        0.10    &   2.00    \\
        0.20    &   1.25    \\
        0.24    &   0.97    \\
        0.26    &   0.85    \\
        0.30    &   0.68    \\
        0.34    &   0.50    \\
        0.40    &   0.32    \\
        0.50    &   0.15    \\
        \bottomrule
    \end{tblr}
    \label{tab:Anlaufstromgebiet}
\end{table}

\noindent Die Vermessung der Kennlinie wird für fünf verschiedene Heizströme durchgeführt. 
Es gibt eine größere Wertedichte für den Bereich von \qty{0}{\volt} bis \qty{60}{\volt}, da die Skala 
des Strommessgeräts in diesem Abschnitt feiner ist und somit genauer gemessen werden kann.\\
\noindent Für die Heizspannung von \qty{4\pm 0.5}{\volt} und einem Heizstrom von \qty{2\pm 0.1}{\ampere} 
ergibt die Vermessung der Kennlinie die folgenden Werte.

\begin{table}[H]
    \centering 
    \caption{Messung der Kennlinie bei einer Heizspannung von \qty{4 \pm 0.5}{\volt} und \qty{2.0\pm0.1}{\ampere}.}
    \begin{tblr}{
        colspec = {S[table-format=3.0] S[table-format=1.3]},
        row{1} = {guard, mode=math},
        }
        \toprule 
            \text{Saugspannung} \mathbin{/} \unit{\volt} & \text{Saugstrom} \mathbin{/} \unit{\milli\ampere} \\
        \midrule
        0       &   0.000   \\
        5       &   0.005   \\
        10      &   0.013   \\
        15      &   0.022   \\
        20      &   0.029   \\
        25      &   0.037   \\
        30      &   0.043   \\
        35      &   0.047   \\
        40      &   0.050   \\
        45      &   0.052   \\
        50      &   0.054   \\
        55      &   0.056   \\
        60      &   0.057   \\
        70      &   0.058   \\
        80      &   0.057   \\
        90      &   0.058   \\
        100     &   0.060   \\
        \bottomrule
    \end{tblr}
    \label{tab:Kennlinie2.0}
\end{table}

\noindent Die Messung wird für Heizströme von \qty{2.1\pm0.1}{\ampere} mit \qty{4\pm0.5}{\volt}, 
\qty{2.2\pm0.1}{\ampere} mit \qty{4.5\pm0.5}{\volt}, \qty{2.3\pm0.1}{\ampere} mit \qty{5\pm0.5}{\volt} und 
\qty{2.4\pm0.1}{\ampere} mit \qty{5\pm0.5}{\volt} wiederholt. Im Folgenden werden die zugehörigen Messwerte aufgeführt. 

\begin{table}[H]
    \caption{Messung der Kennlinie.}
    \label{tab:Kennlinie}
    \begin{minipage}[t]{0.5\textwidth}
        \vspace{0pt}
        \centering
        \subcaption{Heizstrom von \qty{2.1\pm0.1}{\ampere} bei \qty{4\pm0.5}{\volt}}
    \begin{tblr}{
    colspec = {S[table-format=3.0] S[table-format=1.3]},
    row{1} = {guard, mode = math} 
    }
    %subcaption 2.1A
    \toprule
    \text{Saugspannung} \mathbin{/} \unit{\volt} & \text{Saugstrom} \mathbin{/} \unit{\ampere} \\
    \midrule
    0       &   0.000   \\
    5       &   0.007   \\
    10      &   0.019   \\
    15      &   0.031   \\
    20      &   0.043   \\
    25      &   0.056   \\
    30      &   0.070   \\
    35      &   0.082   \\
    40      &   0.092   \\
    45      &   0.100   \\
    50      &   0.107   \\
    55      &   0.111   \\
    60      &   0.115   \\
    70      &   0.118   \\
    80      &   0.117   \\
    90      &   0.122   \\
    100     &   0.125   \\
    120     &   0.132   \\
    150     &   0.131   \\
    \end{tblr}
\end{minipage}\hfill
\begin{minipage}[t]{0.5\textwidth}
    \vspace{0pt}
    \centering
    \subcaption{Heizstrom von \qty{2.2\pm0.1}{\ampere} bei \qty{4.5\pm0.5}{\volt}}
    \begin{tblr}{
        colspec = {S[table-format=3.0] S[table-format=1.3]},
        row{1} = {guard, mode = math} 
        }
        %subcaption 2.2
        \toprule
        \text{Saugspannung} \mathbin{/} \unit{\volt} & \text{Saugstrom} \mathbin{/} \unit{\ampere} \\
        \midrule
        0       &   0.000   \\
        10      &   0.026   \\
        20      &   0.060   \\
        30      &   0.101   \\
        40      &   0.143   \\
        50      &   0.184   \\
        60      &   0.219   \\
        70      &   0.244   \\
        80      &   0.257   \\
        90      &   0.274   \\
        100     &   0.283   \\
        120     &   0.294   \\
        140     &   0.300   \\
        180     &   0.308   \\
        240     &   0.316   \\
        250     &   0.318   \\
        \end{tblr}
    \end{minipage}\hfill
\end{table}
%

%
\begin{table}[H]
    \caption{Messung der Kennlinie.}
    \label{tab:Kennlinie}
    \begin{minipage}[t]{0.5\textwidth}
        \vspace{0pt}
        \centering
        \subcaption{Heizstrom von \qty{2.3\pm0.1}{\ampere} bei \qty{5\pm0.5}{\volt}}
    \begin{tblr}{
    colspec = {S[table-format=3.0] S[table-format=1.3]},
    row{1} = {guard, mode = math} 
    }
    %subcaption 2.3A
    \toprule
    \text{Saugspannung} \mathbin{/} \unit{\volt} & \text{Saugstrom} \mathbin{/} \unit{\ampere} \\
    \midrule
    0       &   0.000\\
    10      &   0.030\\
    20      &   0.073\\
    30      &   0.126\\
    40      &   0.181\\
    50      &   0.238\\
    60      &   0.300\\
    70      &   0.354\\
    80      &   0.409\\
    90      &   0.464\\
    100     &   0.508\\
    120     &   0.572\\
    140     &   0.606\\
    160     &   0.627\\
    180     &   0.639\\
    200     &   0.648\\
    220     &   0.655\\
    240     &   0.661\\
    250     &   0.664\\
    \end{tblr}
\end{minipage}\hfill
\begin{minipage}[t]{0.5\textwidth}
    \vspace{0pt}
    \centering
    \subcaption{Heizstrom von \qty{2.4\pm0.1}{\ampere} bei \qty{5\pm0.5}{\volt}}
    \begin{tblr}{
        colspec = {S[table-format=3.0] S[table-format=1.3]},
        row{1} = {guard, mode = math} 
        }
        %subcaption 2.4
        \toprule
        \text{Saugspannung} \mathbin{/} \unit{\volt} & \text{Saugstrom} \mathbin{/} \unit{\ampere} \\
        \midrule
        0       &   0.000\\
        3       &   0.006\\
        6       &   0.014\\
        9       &   0.023\\
        12      &   0.033\\
        15      &   0.044\\
        18      &   0.058\\
        21      &   0.073\\
        24      &   0.090\\
        27      &   0.100\\
        30      &   0.123\\
        33      &   0.144\\
        36      &   0.163\\
        39      &   0.181\\
        42      &   0.202\\
        45      &   0.223\\
        48      &   0.246\\
        51      &   0.270\\
        54      &   0.295\\
        57      &   0.322\\
        60      &   0.349\\
        70      &   0.435\\
        80      &   0.521\\
        90      &   0.607\\
        100     &   0.681\\
        120     &   0.835\\
        140     &   0.959\\
        160     &   1.068\\
        180     &   1.146\\
        200     &   1.201\\
        220     &   1.238\\
        240     &   1.263\\
        250     &   1.272\\
        \end{tblr}
    \end{minipage}\hfill
\end{table}



%\end{document}

