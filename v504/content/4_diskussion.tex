\input{../../header.tex}

\begin{document}
\section{Diskussion}
\label{sec:Diskussion}


%Austrittsarbeit Wolfram 4.54-4.6
%Heizstrom von 2.4 A konvergiert nicht, daher nur höchster Wert
 
\cite{Austrittsarbeit}

\subsection{Kennlinie}


\subsection{Anlaufstrom}

Der Anlaufstrom hat die intrinsische Eigenschaft sehr klein zu sein. Er befindet sich im \unit{\nano \ampere} Bereich. 
Dementsprechend ist das Strommessgerät auch sehr empfindlich. In der Tat ist es so empfindlich, dass es kleinste Störungen 
der Umgebung das Messergebnis beeinflussen. Während der Aufnahme der Messreihen konnte beobachtet werden, dass selbst das 
bloße Halten der Hand neben der Hochvakuumdiode zu einer Veränderung des Stromausschlags führt. Das kann auf eine statische 
Ladung der Hand zurückzuführen sein. Da die Elektronen aufgrund eines elektrischen Feldes zur Anode gezogen werden, kann das 
E-Feld der Hand somit die Messung beeinflussen. Andere Störfaktoren werden nicht ausgeschlossen. Aus diesem Grund werden 
möglichst wenig Messdaten mit geringem Strom gemessen, und mehr mit etwas höherem Strom. \\
\noindent Dessen ungeachtet entspricht der Qualitative Verlauf der Anlaufspannung den theoretischen Erwartungen. Er fügt sich 
einem exponentiellen Verlauf, welcher für größer werdende Gegenspannungen zunimmt. \footnote{Da das Anlaufstromgebiet für 
negative Spannungen definiert wird, entspricht eine positive Gegenspannung einer negativen Spannung der Kennlinie} 


\subsection{Austrittsarbeit}

Der Literaturwert der Austrittsarbeit von Wolfram liegt zwischen \qty{4.54}{\electronvolt} und \qty{4.60}{\electronvolt} 
\cite{Austrittsarbeit}.




\end{document}
