\input{../../header.tex}

\begin{document}

\section{Auswertung}
\label{sec:Auswertung}

\subsection{Sättigungsstrom}
\label{sec:Saettigungsstrom}

Im ersten Teil der Auswertung werden die Sättigungsströme der jeweiligen Kennlinien ermittelt. Hierzu werden die aus den Tabellen entnommenen Messwerte
graphisch dargestellt und hinsichtlich ihres asymptotischen Verhaltens untersucht.

\begin{figure}[H]
    \centering
    \includegraphics[height=6cm]{../build/Kennlinie_2.0.pdf}
    \caption{Kennlinie bei einem Heizstrom von $I_A = \qty{2.0}{\milli\ampere}$}
    \label{fig:Kennlinie1}
\end{figure}

\noindent Der daraus abzulesende Sättigungsstrom beträgt bei einem Heizstrom von \qty{2}{\ampere} in etwa \qty{0.6}{\milli\ampere}.

\begin{figure}[H]
    \centering
    \includegraphics[height=6cm]{../build/Kennlinie_2.1.pdf}
    \caption{Kennlinie bei einem Heizstrom von $I_A = \qty{2.1}{\milli\ampere}$}
    \label{fig:Kennlinie2}
\end{figure}

\noindent Bei einem Heizstrom von \qty{2.1}{\ampere} beträgt der Sättigungsstrom ca. \qty{0.132}{\milli\ampere}.

\begin{figure}[H]
    \centering
    \includegraphics[height=6cm]{../build/Kennlinie_2.2.pdf}
    \caption{Kennlinie bei einem Heizstrom von $I_A = \qty{2.2}{\milli\ampere}$}
    \label{fig:Kennlinie3}
\end{figure}

\noindent Die Kennlinie im Falle eines Heizstroms von \qty{2.2}{\ampere} nähert sich einem Wert von \qty{0.664}{\milli\ampere}.

\begin{figure}[H]
    \centering
    \includegraphics[height=6cm]{../build/Kennlinie_2.3.pdf}
    \caption{Kennlinie bei einem Heizstrom von $I_A = \qty{2.3}{\milli\ampere}$}
    \label{fig:Kennlinie4}
\end{figure}

\noindent Der Sättigungsstrom beträgt schätzungsweise \qty{0.665}{\milli\ampere} bei einem Heizstrom von \qty{2.3}{\ampere}.
\begin{figure}[H]
    \centering
    \includegraphics[height=6cm]{../build/Kennlinie_2.4.pdf}
    \caption{Kennlinie bei einem Heizstrom von $I_A = \qty{2.4}{\milli\ampere}$}
    \label{fig:Kennlinie5}
\end{figure}

\noindent Die letzte Kennlinie weist einen Sättigungsstrom von \qty{1.275}{\milli\ampere} auf.

\end{document}
