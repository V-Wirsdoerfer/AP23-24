\input{../../header.tex}

\begin{document}

\section{Auswertung}
\label{sec:Auswertung}

\subsection{Sättigungsströme der Kennlinien}
\label{sec:Saettigungsstrom}

Im ersten Teilkapitel der Auswertung werden die Sättigungsströme der aufgenommenen Kennlinie ermittelt. Diese können einerseits über die Messwerte in der Tabelle 
oder mittels der asymptotischen Charakteristik der Kennlinienschar abgelesen werden. Zweitere wird in der folgenden Abbildung dargestellt.

\begin{figure}
    \centering
    \includegraphics[height=6cm]{../build/Kennlinie_2.0.pdf}
    \caption{Kennlinien im Heizstrombereich \qty{2.0}{\ampere} bis \qty{2.3}{\ampere}.}
    \label{fig:Kennlinien1}
\end{figure}

Aus dieser Graphik ergeben sich folgenden Sättigungsströme:

\begin{align*}
    I_{\text{S},\qty{2.0}{\ampere}} = \qty{0.060}{\milli\ampere}\\
    I_{\text{S},\qty{2.1}{\ampere}} = \qty{0.132}{\milli\ampere}\\   
    I_{\text{S},\qty{2.2}{\ampere}} = \qty{0.320}{\milli\ampere}\\   
    I_{\text{S},\qty{2.3}{\ampere}} = \qty{0.665}{\milli\ampere}
\end{align*}

\noindent 
\end{document}
