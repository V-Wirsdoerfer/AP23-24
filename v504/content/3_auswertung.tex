%\input{../../header.tex}

%\begin{document}

\section{Auswertung}
\label{sec:Auswertung}

\subsection{Sättigungsströme der Kennlinien}
\label{sec:Saettigungsstrom}

Im ersten Teilkapitel der Auswertung werden die Sättigungsströme der aufgenommenen Kennlinie ermittelt. Diese können einerseits über die Messwerte in den
dazugehörigen Tabellen des Messwertkapitels \ref{sec:Messwerte} oder durch die asymptotische Charakteristik der Kennlinienschar abgelesen werden. Zweitere 
wird in der folgenden Abbildung dargestellt.

\begin{figure}[H]
    \centering
    \includegraphics[height=6cm]{build/Kennlinie_2.0.pdf}
    \caption{Kennlinien im Heizstrombereich \qty{2.0}{\ampere} bis \qty{2.3}{\ampere}.}
    \label{fig:Kennlinien1}
\end{figure}

Aus dieser Graphik ergeben sich die zu den Heizströmen passende Sättigungsströme:

\begin{align*}
    I_{\text{S};\qty{2.0}{\ampere}} = \qty{0.060}{\milli\ampere}\\
    I_{\text{S};\qty{2.1}{\ampere}} = \qty{0.132}{\milli\ampere}\\   
    I_{\text{S};\qty{2.2}{\ampere}} = \qty{0.320}{\milli\ampere}\\   
    I_{\text{S};\qty{2.3}{\ampere}} = \qty{0.665}{\milli\ampere}
\end{align*}

\noindent Die Kennlinie bei einem Heizstrom von \qty{2.4}{\ampere} erfährt im Verlauf der Auswertung eine besondere Wichtigkeit, weshalb die dazugehörige Graphik 
separat abgebildet wird. 

\begin{figure}[H]
    \centering 
    \includegraphics[height=6cm]{build/Kennlinie_2.4.pdf}
    \caption{Kennlinie bei einem Heizstrom von \qty{2.4}{\ampere}.}
    \label{fig:Kennlinien2}
\end{figure}

\noindent Der dazugehörige Sättigungsstrom lässt sich mit $I_{\text{S};\qty{2.4}{\ampere}} = \qty{1.272}{\milli\ampere}$ abschätzen.

\subsection{Untersuchung des Raumladungsgebietes}
\label{sec:Raumladungsgebiet}

Ziel dieses Abschnitts ist die Überprüfung des Gültigkeitsbereichs des Langmuir-Schottkyschen Raumladungsgesetzes. Notwendigerweise muss dazu der Raumladungsbereich 
der Kennlinie herauskristallisiert. Anhand der obigen Abbildung \ref{fig:Kennlinien2} lässt sich nachvollziehen, wann dieses Gebiet beginnt und endet. Über die 
Maskenfunktion des \texttt{numpy.arrays} kann die Menge der Datenpunkte entsprechend eingegrenzt werden. In diesem Fall beginnt das Raumladungsgebiet bei $U_1 = \qty{3}{\volt}$
und endet bei $U_2 = \qty{120}{\volt}$. In diesem Intervall werden sowohl die Spannungen auf der horizontalen Achse als auch die Stromstärken auf der vertikalen Achse 
logarithmiert mit der Intention, eine numerisch stabile lineare Regression durchführen zu können. Die approximierte Gleichung hat daher die Form

\begin{equation*}
    \log(I) = \log(\tilde{b}\cdot{}U^m) = m\cdot\log(U) + b.
\end{equation*}

\noindent Die Ausgleichsgerade wird durch die logarithmierten Datenpunkte \emph{gefittet}: 

\begin{figure}[H]
    \centering 
    \includegraphics[height=6cm]{build/logRaum.pdf}
    \caption{Logarithmierte Kennlinie bei einem Heizstrom von \qty{2.4}{\ampere}.}
    \label{fig:Kennlinien2}
\end{figure}

\noindent Über eine lineare Ausgleichsrechnung der Funktion \texttt{numpy.polyfit} können die gesuchten Parameter $m$ und $b$ ausgegeben werden.
Diese Parameter lauten wie folgt:

\begin{align*}
    m &= 1.395\pm0.013\\
    b &= -13.72\pm0.05
\end{align*}

\subsection{Anlaufstromgebiet}
\label{sec:AnlaufstromgebietSec}


Ein ähnliches Verfahren der linearen Regression wird zur Untersuchung des Anlaufstromgebiets konsultiert. Zunächst werden dazu die tatsächlichen Wertepaare von 
Spannung und Stromstärke in einem Koordinatensystem dargestellt. Eine Möglichkeit der Auswertung wäre die Implementierung eines \texttt{curve-fits} der 
Bibliothek \texttt{scipy.optimize} in der Gestalt einer Exponentialfunktion. Aufgrund der Tatsache, dass die Rückgabeparameter des \texttt{curve-fits} stark 
abhängig von den notwendigen Startparametern sind, wird eine erneute Logarithmierung präfertiert. Zum direkten Vergleich beider Möglichkeiten wird trotz dessen 
der \emph{exponentielle fit} im Koordinatensystem des ursprünglichen Datensatzes visualisiert.

\begin{figure}[H]
    \centering 
    \includegraphics[height=6cm]{build/Anlaufstrom.pdf}
    \caption{Anlaufstromgebiet der Elektronenemission.}
    \label{fig:Kennlinien2}
\end{figure}

\noindent Auf Grundlage dieser Werte können auch die logarithmierten Datenpunkte dargestellt werden. Erneut wird über einen \texttt{numpy.polyfit},
eine Ausgleichsgerade durch die Wertepaare gelegt. Die folgende Abbildung zeigt den aus Abb. \ref{fig:Kennlinien2} logarithmierten Datensatz unter 
Hinzufügen der Regressionsgerade.

\begin{figure}[H]
    \centering 
    \includegraphics[height=6cm]{build/logAnlauf.pdf}
    \caption{Logarithmierte Datenpunkte des Anlaufstromgebiets.}
    \label{fig:Anlaufstromgebiet}
\end{figure}

\noindent Unter Berücksichtigung der allgemeinen Funktionsform 

\begin{equation*}
    f(x) = mx + b 
\end{equation*}

\noindent sind die Rückgabeparameter des \texttt{Polyfits}

\begin{align*}
    m &= \left(3.75\pm{}0.13\right)\cdot{}10^{-9}\\
    b &= -6.0\pm{}0.4.
\end{align*}

\noindent Ferner kann über den Zusammenhang 

\begin{equation*}
    m = -\frac{e_0}{k_\text{B}\cdot{}T_\text{K}}
\end{equation*}

\noindent die Kathodentemperatur $T_\text{K}$ direkt aus der Steigung der Ausgleichsgeraden ermittlet werden. Durch triviales Umstellen 
dieser Gleichung ergibt sich ein Wert von 

\begin{equation*}
    T_\text{K} = \qty{2020\pm90}{\kelvin}.
\end{equation*}

\subsection{Austrittsarbeit von Wolfram}
\label{sec:Austrittsarbeit}

Im letzten Teilkapitel der Auswertung soll die Austrittsarbeit von Wolfram bestimmt werden. Grundlage hierfür ist der Zusammenhang zwischen Kathodentemperatur
und Austrittsarbeit. Die Berechnung der jeweiligen Kathodentemperaturen in Abhängigkeit der Heizströme $I_\text{H}$ sowie Heizspannungen $U_\text{H}$ erfolgt 
mittels der Gleichung 

\begin{equation*}
    I_\text{H}U_\text{H} = f\eta\sigma{}T⁴ + N_\text{Wl},
\end{equation*}

\noindent welche Resultat des Stefan-Bolzmanm-Gesetzes ist. Nach Umstellen der Gleichung 

\begin{equation*}
    T = \left(\frac{I_\text{H}U_\text{H} - N_\text{Wl}}{f\eta\sigma}\right)
\end{equation*}

\noindent können die Kathodentemperaturen durch einsetzen der Tupel $\left(I_\text{H}U_\text{H}\right)$ sowie sämtlicher Apparaturkonstanten 

\begin{align*}
    f &= \qty{0.35}{\centi\meter\squared}\\
    \eta &= 0.28\\
    \sigma &= \qty{5.7e-12}{\watt\per\centi\meter\squared\per\kelvin\tothe{4}}\\
    N_\text{Wl} &= \qty{0.95}{\watt}
\end{align*}

\noindent ermittelt werden. In der untenstehenden Tabelle sind die zu den Wertepaaren $\left(I_\text{H}U_\text{H}\right)$ gehörigen Temperaturen 
aufgelistet.

\begin{table}[H]
    \centering
    \caption{Tablle der Kathodentemperaturen.}
    \label{tab:Temperaturen}
    \sisetup{table-format=1.1}
    \begin{tblr}{
        colspec = {S S S[separate-uncertainty=true, table-format=4.0(1)]},
        row{1} = {guard, mode=math},
      }
      \toprule
      I_\text{H} \mathbin{/} \unit{\nano\ampere} & U_\text{H} \mathbin{/} \unit{\volt} & T_\text{K} \mathbin{/} \unit{\kelvin}\\
      \midrule
      2.0  &  4.0  &  1930 \pm 70 \\
      2.1  &  4.0  &  1950 \pm 70 \\
      2.2  &  4.5  &  2050 \pm 70 \\
      2.3  &  5.0  &  2130 \pm 60 \\
      2.4  &  5.0  &  2160 \pm 60 \\
      \bottomrule
    \end{tblr}
\end{table}

\noindent Über Äquivalenzumformungen der \emph{Richardson-Gleichung} lässt sich ein direkter Ausdruck für die Austrittsarbeit $E_\text{A}$
finden, welcher ausschließlich von gemessenen bzw. zuvor bestimmten Größen abhängt:

\begin{equation*}
    E_\text{A} = e0\cdot\phi = -T_\text{K}k_\text{B}\ln\left(\frac{j_\text{S}h³}{4\pi\varepsilon_0m_0k²_\text{B}T²_\text{K}}\right)
\end{equation*}

\noindent Hierbei repräsentiert $j_\text{S}$ die Sättigungsstromdichte, bzw $j_\text{S} = \sfrac{I_\text{S}}{f}$. Zusätzlich muss das Ergebnis der obigen Gleichung
durch die Elementarladung geteilt werden, um eine Energie in der Einheit $\unit{\electronvolt}$ zu erhalten.
Werden die zuvor ermittelten 
Größen sowie Naturkonstanten in diese Gleichung eingesetzt, lassen sich die jeweiligen Austrittsarbeiten tabellarisch darstellen. 

\begin{table}[H]
    \centering
    \caption{Tablle der Austrittsarbeiten.}
    \label{tab:Temperaturen}
    \sisetup{table-format=1.1}
    \begin{tblr}{
        colspec = {S S[table-format=1.2(1)]},
        row{1} = {guard, mode=math},
      }
      \toprule
      I_\text{H} \mathbin{/} \unit{\nano\ampere} & E_\text{A} \mathbin{/} \unit{\electronvolt}\\
      \midrule
      2.0  &  4.76 \pm 0.18 \\
      2.1  &  4.68 \pm 0.18 \\
      2.2  &  4.78 \pm 0.18 \\
      2.3  &  4.84 \pm 0.15 \\
      2.4  &  4.80 \pm 0.14 \\
      \bottomrule
    \end{tblr}
\end{table}

\noindent Der Mittelwert der in der Tabelle aufgeführten Austrittsarbeiten liefert schließlich den empirisch erfassten Wert 
der Austrittsarbeit von Wolfram:

\begin{equation*}
    \bar{E}_\text{A} = \qty{4.77\pm0.07}{\electronvolt}
\end{equation*}

%\end{document}
