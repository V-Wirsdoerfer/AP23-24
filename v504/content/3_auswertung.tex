\documentclass[
  bibliography=totoc,     % Literatur im Inhaltsverzeichnis
  captions=tableheading,  % Tabellenüberschriften
  titlepage=firstiscover, % Titelseite ist Deckblatt
]{scrartcl}

% Paket float verbessern
\usepackage{scrhack}

% Warnung, falls nochmal kompiliert werden muss
\usepackage[aux]{rerunfilecheck}

% unverzichtbare Mathe-Befehle
\usepackage{amsmath}
% viele Mathe-Symbole
\usepackage{amssymb}
% Erweiterungen für amsmath
\usepackage{mathtools}

% Fonteinstellungen
\usepackage{fontspec}
% Latin Modern Fonts werden automatisch geladen
% Alternativ zum Beispiel:
%\setromanfont{Libertinus Serif}
%\setsansfont{Libertinus Sans}
%\setmonofont{Libertinus Mono}

% Wenn man andere Schriftarten gesetzt hat,
% sollte man das Seiten-Layout neu berechnen lassen
\recalctypearea{}

% deutsche Spracheinstellungen
\usepackage[ngerman]{babel}


\usepackage[
  math-style=ISO,    % ┐
  bold-style=ISO,    % │
  sans-style=italic, % │ ISO-Standard folgen
  nabla=upright,     % │
  partial=upright,   % │
  mathrm=sym,        % ┘
  warnings-off={           % ┐
    mathtools-colon,       % │ unnötige Warnungen ausschalten
    mathtools-overbracket, % │
  },                       % ┘
]{unicode-math}

% traditionelle Fonts für Mathematik
\setmathfont{Latin Modern Math}
% Alternativ zum Beispiel:
%\setmathfont{Libertinus Math}

\setmathfont{XITS Math}[range={scr, bfscr}]
\setmathfont{XITS Math}[range={cal, bfcal}, StylisticSet=1]

% Zahlen und Einheiten
\usepackage[
  locale=DE,                   % deutsche Einstellungen
  separate-uncertainty=true,   % immer Unsicherheit mit \pm
  per-mode=symbol-or-fraction, % / in inline math, fraction in display math
]{siunitx}

% chemische Formeln
\usepackage[
  version=4,
  math-greek=default, % ┐ mit unicode-math zusammenarbeiten
  text-greek=default, % ┘
]{mhchem}

% richtige Anführungszeichen
\usepackage[autostyle]{csquotes}

% schöne Brüche im Text
\usepackage{xfrac}

% Standardplatzierung für Floats einstellen
\usepackage{float}
\floatplacement{figure}{htbp}
\floatplacement{table}{htbp}

% Floats innerhalb einer Section halten
\usepackage[
  section, % Floats innerhalb der Section halten
  below,   % unterhalb der Section aber auf der selben Seite ist ok
]{placeins}

% Seite drehen für breite Tabellen: landscape Umgebung
\usepackage{pdflscape}

% Captions schöner machen.
\usepackage[
  labelfont=bf,        % Tabelle x: Abbildung y: ist jetzt fett
  font=small,          % Schrift etwas kleiner als Dokument
  width=0.9\textwidth, % maximale Breite einer Caption schmaler
]{caption}
% subfigure, subtable, subref
\usepackage{subcaption}

% Grafiken können eingebunden werden
\usepackage{graphicx}

% schöne Tabellen
\usepackage{tabularray}
\UseTblrLibrary{booktabs, siunitx}

% Verbesserungen am Schriftbild
\usepackage{microtype}

% Literaturverzeichnis
\usepackage[
  backend=biber,
]{biblatex}
% Quellendatenbank
\addbibresource{lit.bib}
\addbibresource{programme.bib}

% Hyperlinks im Dokument
\usepackage[
  german,
  unicode,        % Unicode in PDF-Attributen erlauben
  pdfusetitle,    % Titel, Autoren und Datum als PDF-Attribute
  pdfcreator={},  % ┐ PDF-Attribute säubern
  pdfproducer={}, % ┘
]{hyperref}
% erweiterte Bookmarks im PDF
\usepackage{bookmark}

% Trennung von Wörtern mit Strichen
\usepackage[shortcuts]{extdash}

\author{%
  Vincent Wirsdörfer\\%
  \href{mailto:vincent.wirsdoerfer@udo.edu}{authorA@udo.edu}%
  \and%
  Joris Daus\\%
  \href{mailto:joris.daus@udo.edu}{authorB@udo.edu}%
}
\publishers{TU Dortmund – Fakultät Physik}


\begin{document}

\section{Auswertung}
\label{sec:Auswertung}

\subsection{Sättigungsströme der Kennlinien}
\label{sec:Saettigungsstrom}

Im ersten Teilkapitel der Auswertung werden die Sättigungsströme der aufgenommenen Kennlinie ermittelt. Diese können einerseits über die Messwerte in der Tabelle 
oder mittels der asymptotischen Charakteristik der Kennlinienschar abgelesen werden. Zweitere wird in der folgenden Abbildung dargestellt.

\begin{figure}[H]
    \centering
    \includegraphics[height=6cm]{../build/Kennlinie_2.0.pdf}
    \caption{Kennlinien im Heizstrombereich \qty{2.0}{\ampere} bis \qty{2.3}{\ampere}.}
    \label{fig:Kennlinien1}
\end{figure}

Aus dieser Graphik ergeben sich folgenden Sättigungsströme:

\begin{align*}
    I_{\text{S};\qty{2.0}{\ampere}} = \qty{0.060}{\milli\ampere}\\
    I_{\text{S};\qty{2.1}{\ampere}} = \qty{0.132}{\milli\ampere}\\   
    I_{\text{S};\qty{2.2}{\ampere}} = \qty{0.320}{\milli\ampere}\\   
    I_{\text{S};\qty{2.3}{\ampere}} = \qty{0.665}{\milli\ampere}
\end{align*}

\noindent Die Kennlinie bei einem Heizstrom von \qty{2.4}{\ampere} erfährt im Verlauf der Auswertung eine besondere Wichtigkeit, weshalb die dazugehörige Graphik 
separat abgebildet wird. 

\begin{figure}[H]
    \centering 
    \includegraphics[height=6cm]{../build/Kennlinie_2.4.pdf}
    \caption{Kennlinie bei einem Heizstrom von \qty{2.4}{\ampere}.}
    \label{fig:Kennlinien2}
\end{figure}

\noindent Der dazugehörige Sättigungsstrom lässt sich mit $I_{\text{S};\qty{2.4}{\ampere}} = \qty{1.272}{\ampere}$ abschätzen.

\subsection{Untersuchung des Raumladungsgebietes}
\label{sec:Raumladungsgebiet}

Ziel dieses Abschnitts ist die Überprüfung des Gültigkeitsbereichs des Langmuir-Schottkyschen Raumladungsgesetzes. Hierfür wird der in Abbildung \ref{fig:Kennlinien2}
skizzierte Graph logarithmiert. Die approximierte Gleichung hat die Form

\begin{equation*}
    \log(I) = \log(\tilde{b}\cdot{}U^m) = m\cdot\log(U) + b.
\end{equation*}

\noindent Die Ausgleichsgerade wird durch die logarithmierten Datenpunkte gefittet: 

\begin{figure}[H]
    \centering 
    \includegraphics[height=6cm]{../build/logRaum.pdf}
    \caption{Logarithmierte Kennlinie bei einem Heizstrom von \qty{2.4}{\ampere}.}
    \label{fig:Kennlinien2}
\end{figure}

\noindent Über eine lineare Ausgleichsrechnung der Funktion \texttt{numpy.polyfit} können die gesuchten Parameter $m$ und $b$ ausgegeben werden.
Diese Parameter lauten wie folgt:

\begin{align*}
    m &= 1.395\pm0.013\\
    b &= -13.72\pm0.05
\end{align*}

\subsection{Anlaufstromgebiet}
\label{sec:AnlaufstromgebietSec}


Ein ähnliches Verfahren wird zur Untersuchung des Anlaufstromgebiets konsultiert. Die aus Tabelle \ref{tab:Anlaufstromgebiet} entnommenen Werte 
liefern die folgende Graphik:

\begin{figure}[H]
    \centering 
    \includegraphics[height=6cm]{../build/Anlaufstrom.pdf}
    \caption{Anlaufstromgebiet der Elektronenemission.}
    \label{fig:Kennlinien2}
\end{figure}

\noindent Auf Grundlage dieser Werte können auch die logarithmierten Datenpunkte visualisiert werden. Erneut wird über einen \texttt{numpy.polyfit},
eine Ausgleichsgerade durch die Datenpunkte gelegt. Die folgende Abbildung zeigt den aus Abb. \ref{fig:Kennlinien2} logarithmierten Datensatz unter 
Hinzufügen der Regressionsgerade.

\begin{figure}[H]
    \centering 
    \includegraphics[height=6cm]{../build/logAnlauf.pdf}
    \caption{Logarithmierte Datenpunkte des Anlaufstromgebiets.}
    \label{fig:Anlaufstromgebiet}
\end{figure}

\noindent Unter Berücksichtigung der allgemeinen Funktionsform 

\begin{equation*}
    f(x) = mx + b 
\end{equation*}

\noindent sind die Rückgabeparameter des \texttt{Polyfits}

\begin{align*}
    m &= \left(3.75\pm{}0.13\right)\cdot{}10^{-9}\\
    b &= -6.0\pm{}0.4.
\end{align*}

\noindent Über den Zusammenhang 

\begin{equation*}
    m = -\frac{e_0}{k_\text{B}\cdot{}T_\text{K}}
\end{equation*}

\noindent kann aus der Steigung der Ausgleichsgerade direkt auf die Kathodentemperatur $T_\text{K}$ geschlossen werden. Durch triviales Umstellen 
dieser Gleichung ergibt sich ein Wert von 

\begin{equation*}
    T_\text{K} = \qty{2020\pm90}{\kelvin}.
\end{equation*}

\subsection{Austrittsarbeit von Wolfram}
\label{sec:Austrittsarbeit}

Im letzten Teilkapitel der Auswertung soll die Austrittsarbeit von Wolfram bestimmt werden. Grundlage hierfür ist der Zusammenhang zwischen Kathodentemperatur
und Austrittsarbeit. Die Berechnung der jeweiligen Kathodentemperaturen in Abhängigkeit der Heizströme $I_\text{H}$ sowie Heizspannungen $U_\text{H}$ erfolgt 
mittels der Gleichung 

\begin{equation*}
    I_\text{H}U_\text{H} = f\eta\sigma{}T⁴ + N_\text{Wl},
\end{equation*}

\noindent welche Resultat des Stefan-Bolzmanm-Gesetzes ist. Nach Umstellen der Gleichung 

\begin{equation*}
    T = \left(\frac{I_\text{H}U_\text{H} - N_\text{Wl}}{f\eta\sigma}\right)
\end{equation*}

\noindent können die Kathodentemperaturen durch einsetzen der Tupel $\left(I_\text{H}U_\text{H}\right)$ sowie sämtlicher Apparaturkonstanten 

\begin{align*}
    f &= \qty{0.35}{\centi\meter\squared}\\
    \eta &= 0.28\\
    \sigma &= \qty{5.7e-12}{\watt\per\centi\meter\squared\per\kelvin\tothe{4}}\\
    N_\text{Wl} &= \qty{0.95}{\watt}
\end{align*}

\noindent ermittelt werden. In der untenstehenden Tabelle sind die zu den Wertepaaren $\left(I_\text{H}U_\text{H}\right)$ gehörigen Temperaturen 
aufgelistet.

\begin{table}[H]
    \centering
    \caption{Tablle der Kathodentemperaturen.}
    \label{tab:Temperaturen}
    \sisetup{table-format=1.1}
    \begin{tblr}{
        colspec = {S S S[separate-uncertainty=true, table-format=4.0(1)]},
        row{1} = {guard, mode=math},
      }
      \toprule
      I_\text{H} \mathbin{/} \unit{\nano\ampere} & U_\text{H} \mathbin{/} \unit{\volt} & T_\text{K} \mathbin{/} \unit{\kelvin}\\
      \midrule
      2.0  &  4.0  &  1880 \pm 70 \\
      2.1  &  4.0  &  1910 \pm 70 \\
      2.2  &  4.5  &  2000 \pm 70 \\
      2.3  &  5.0  &  2080 \pm 60 \\
      2.4  &  5.0  &  2110 \pm 60 \\
      \bottomrule
    \end{tblr}
\end{table}

\noindent Über Äquivalenzumformungen der \emph{Richardson-Gleichung} lässt sich ein direkter Ausdruck für die Austrittsarbeit $E_\text{A}$
finden, welcher ausschließlich von gemessenen bzw. zuvor bestimmten Größen abhängt:

\begin{equation*}
    E_\text{A} = e_0\phi = -T_\text{K}k_\text{B}\ln\left(\frac{j_\text{S}h³}{4\pi\varepsilon_0m_0k²_\text{B}T²_\text{K}}\right)
\end{equation*}

\noindent Hierbei repräsentiert $j_\text{S}$ die Sättigungsstromdichte, bzw $j_\text{S} = \sfrac{I_\text{S}}{f}$. Werden die zuvor ermittelten 
Größen sowie Naturkonstanten in diese Gleichung eingesetzt, lassen sich die jeweiligen Austrittsarbeiten tabellarisch darstellen.



\end{document}
