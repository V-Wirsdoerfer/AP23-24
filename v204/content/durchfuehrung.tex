\input{../../header.tex}

\begin{document}
\section{Versuchsdurchführung}
\label{sec:Versuchsdurchführung}

Zu Beginn des Versuches müssen divese Voraussetzungen geprüft werden, bevor die Wärmeleitfähigkeiten der Metalle experimentell unterscuht werden können.
Dazu gehört beispielsweise die vollständige Verkabelung aller relevanten Geräte, sowie die korrekte Sensorik des Datenloggers. Außerdem muss der Abstand 
zweier benachbarter Thermoelemente gemessen werde, welche \textbf{nicht} durch das Peltierelement getrennt sind. Wie in \ref{fig:Versuchsaufbau} abgelesen werden kann,
ist das zum Beispiel der Abstand zwischen $T_1$ und $T_2$. 

\begin{figure}
    \centering
    \includegraphics[width=\textwidth]{Versuchsaufbau_v204.png}
    \caption{Versuchsaufbau}
    \label{fig:Versuchsaufbau}
\end{figure}

\subsection{Statische Methode}
\label{sec:Statische Methode}

Das Grundkonzept der statischen Methode besteht darin, den zeitlichen Verlauf der Temperaturänderung der jeweiligen Stäbe zu beobachten und so Rückschlüsse auf die
Wärmeleitfähigkeit der Metalle zu ziehen. Dies gelingt, indem an jeweils zwei Stellen der Stäbe die Temperatur gemessen und schließlich als Funktion der Zeit aufgetragen wird. \\
Hierbei muss zunächst die Abtastrate des Datenloggers auf $\increment t_\text{GLX} = 10\unit{\second}$ ein. Ferner wird die Spannung des Power Supplys bei maximaler Stromstärke auf 
$U_\text{P} = 5\unit{\volt}$ eingestellt. Sofern die Probestäbe der Grundplatte hinreichend abgekühlt sind \footnote{Die kann am Datenlogger unter Menüpunkt \texttt{Digital} eingesehen werden}
werden die Wärmeisolierungen an den Stäben angebracht. Im Anschluss kann der Schalter auf \enquote{HEAT} gestellt werden bis das Thermoelement $T_7$ auf eine Temperatur von 45 \unit{\celsius}
erhitzt ist. Ist diese Schwelle erreicht, wird der Schalter erneut auf \enquote{COOL} umgestellt, die Wärmeisolierungen entfernt und die Probestäbe werden runtergekühlt. \\
Mittels des Menüpunktes \texttt{Tabellen} können nun die Temperaturwerte aller Thermoelemente zu den jeweiligen Zeitpunkten eingesehen werden. Folglich können somit die Werte der
sich fern vom Peltierelement befindenden Thermoelemente $T_1$, $T_4$, $T_5$ und $T_8$ nach ca. 700 \unit{\second} notiert werden. Der komplette Datensatz wird extrahiert und auf einem USB-Stick
gesichert. Anhand dieser Daten können nun Temperaturverläufe der fernen Thermoelemente sowie der Temperaturdifferenzen $T_{7,8} = T_7 - T_8$ und $T_{2,1} = T_2 - T_1$ graphisch dargestellt werden.

\subsection{Dynamische Mathode}
\label{sec:Dynamische Methode}

\end{document}

