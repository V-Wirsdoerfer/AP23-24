\documentclass[
  bibliography=totoc,     % Literatur im Inhaltsverzeichnis
  captions=tableheading,  % Tabellenüberschriften
  titlepage=firstiscover, % Titelseite ist Deckblatt
]{scrartcl}

% Paket float verbessern
\usepackage{scrhack}

% Warnung, falls nochmal kompiliert werden muss
\usepackage[aux]{rerunfilecheck}

% unverzichtbare Mathe-Befehle
\usepackage{amsmath}
% viele Mathe-Symbole
\usepackage{amssymb}
% Erweiterungen für amsmath
\usepackage{mathtools}

% Fonteinstellungen
\usepackage{fontspec}
% Latin Modern Fonts werden automatisch geladen
% Alternativ zum Beispiel:
%\setromanfont{Libertinus Serif}
%\setsansfont{Libertinus Sans}
%\setmonofont{Libertinus Mono}

% Wenn man andere Schriftarten gesetzt hat,
% sollte man das Seiten-Layout neu berechnen lassen
\recalctypearea{}

% deutsche Spracheinstellungen
\usepackage[ngerman]{babel}


\usepackage[
  math-style=ISO,    % ┐
  bold-style=ISO,    % │
  sans-style=italic, % │ ISO-Standard folgen
  nabla=upright,     % │
  partial=upright,   % │
  mathrm=sym,        % ┘
  warnings-off={           % ┐
    mathtools-colon,       % │ unnötige Warnungen ausschalten
    mathtools-overbracket, % │
  },                       % ┘
]{unicode-math}

% traditionelle Fonts für Mathematik
\setmathfont{Latin Modern Math}
% Alternativ zum Beispiel:
%\setmathfont{Libertinus Math}

\setmathfont{XITS Math}[range={scr, bfscr}]
\setmathfont{XITS Math}[range={cal, bfcal}, StylisticSet=1]

% Zahlen und Einheiten
\usepackage[
  locale=DE,                   % deutsche Einstellungen
  separate-uncertainty=true,   % immer Unsicherheit mit \pm
  per-mode=symbol-or-fraction, % / in inline math, fraction in display math
]{siunitx}

% chemische Formeln
\usepackage[
  version=4,
  math-greek=default, % ┐ mit unicode-math zusammenarbeiten
  text-greek=default, % ┘
]{mhchem}

% richtige Anführungszeichen
\usepackage[autostyle]{csquotes}

% schöne Brüche im Text
\usepackage{xfrac}

% Standardplatzierung für Floats einstellen
\usepackage{float}
\floatplacement{figure}{htbp}
\floatplacement{table}{htbp}

% Floats innerhalb einer Section halten
\usepackage[
  section, % Floats innerhalb der Section halten
  below,   % unterhalb der Section aber auf der selben Seite ist ok
]{placeins}

% Seite drehen für breite Tabellen: landscape Umgebung
\usepackage{pdflscape}

% Captions schöner machen.
\usepackage[
  labelfont=bf,        % Tabelle x: Abbildung y: ist jetzt fett
  font=small,          % Schrift etwas kleiner als Dokument
  width=0.9\textwidth, % maximale Breite einer Caption schmaler
]{caption}
% subfigure, subtable, subref
\usepackage{subcaption}

% Grafiken können eingebunden werden
\usepackage{graphicx}

% schöne Tabellen
\usepackage{tabularray}
\UseTblrLibrary{booktabs, siunitx}

% Verbesserungen am Schriftbild
\usepackage{microtype}

% Literaturverzeichnis
\usepackage[
  backend=biber,
]{biblatex}
% Quellendatenbank
\addbibresource{lit.bib}
\addbibresource{programme.bib}

% Hyperlinks im Dokument
\usepackage[
  german,
  unicode,        % Unicode in PDF-Attributen erlauben
  pdfusetitle,    % Titel, Autoren und Datum als PDF-Attribute
  pdfcreator={},  % ┐ PDF-Attribute säubern
  pdfproducer={}, % ┘
]{hyperref}
% erweiterte Bookmarks im PDF
\usepackage{bookmark}

% Trennung von Wörtern mit Strichen
\usepackage[shortcuts]{extdash}

\author{%
  Vincent Wirsdörfer\\%
  \href{mailto:vincent.wirsdoerfer@udo.edu}{authorA@udo.edu}%
  \and%
  Joris Daus\\%
  \href{mailto:joris.daus@udo.edu}{authorB@udo.edu}%
}
\publishers{TU Dortmund – Fakultät Physik}


\begin{document}
\section{Zielsetzung}
\label{sec:Zielsetzung}

In dem folgend protokollierten Versuch soll die Wärmeleitung von
Aluminium, Messing und Edelstahl untersucht werden. Das konkrete Ziel besteht
darin, den zeitlichen Verlauf der Temperaturänderung der jeweiligen Stäbe zu analysieren
und die Wellenänge sowie Frequenz der \enquote{Temperaturwelle} zu bestimmen. Zusätzlich soll
die Wärmeleitfähigkeit der zu Beginn aufgeführten Metalle ermittlet werden.

\section{Theorie}
\label{sec:Theorie}

In einem System kann genau dann ein Wärmetransport beobachtet werden, falls ein Temperaturgefälle vorhanden ist. Dieser Wärmetransport kann sich durch Wärmestrahlung, Konvektion und Wärmeleitung
realisieren. In dem folgenden Versuch wird sich jedoch auf die Wärmeleitung beschränkt. \\
Das soeben angesprochene System wird durch einen Stab der Länge $L$ und Querschnittsfläche $A$ dargestellt.
Die Massendichte wird im Folgenden als $\rho$ und die spezifische Wärme als $c$ bezeichnet. \\
Existiert nun eine Temperaturdifferenz an den Enden des Stabes, so fließt die Wärmemenge

\begin{equation}
\label{eqn:Waermemenge}
    \symup {d} Q = -\kappa{}A\frac{\partial{}T}{\partial{}x} \symup d t 
\end{equation}

in der Zeit $\symup d t$ entlang des Temperaturgefälles. Hierbei steht der Faktor $\kappa$ für die Wärmeleitfähigkeit
des Materials. Der Wärmestrom fließt dabei immer in Richtung der niedrigeren Temperatur was mathematisch durch 
das negative Vorzeichen in \eqref{eqn:Waermemenge} repräsentiert wird. Unter Hinzunahme der \emph{Kontinuitätsgleichung}

\begin{equation}
    \partial{}_t \rho + \nabla{}\cdot{}\vec{j} = 0
\end{equation}

und den Ausdrücken

\begin{equation}
    \symup d Q = mc \cdot \partial{}T
\end{equation}

für die Wärmemenge $dQ$, sowie

\begin{equation}
    j_\omega = -\kappa \frac{\partial{}T}{\partial{}x}
\end{equation}

für die Wärmestromdichte $j_\omega$, ergibt sich die eindimensionale \emph{Wärmeleitungsgleichung}

\begin{equation}
    \frac{\partial T}{\partial t} = \frac{\kappa}{\rho c} \frac{\partial² T}{\partial x²}.
\end{equation}

Der materialabhängige Vorfaktor $\frac{\kappa}{\rho c}$ gibt die Geschwindgkeit des Temperaturausgleichs 
an und wird daher als \emph{Temperaturleitfähigkeit} tituliert. \\
Findet nun ein periodisches Wechselspiel zwischen Erwärmung und Abkühlung des Stabes statt,
so breitet sich eine \emph{Temperaturwelle} der Form

\begin{equation}
\label{eqn:Temperaturwelle}
    T(x,t) = T_{\mathrm{max}} e^{-\sqrt{\frac{\omega \rho c}{2\kappa}}}\cos\left(\omega t - \sqrt{\frac{\omega \rho c}{2\kappa}}x\right)
\end{equation}

durch den Stab aus, wobei $T(x,t)$ die Periodendauer in Abhängigkeit von Ort und Zeit und $T_\text{max}$ die Amplitude bezeichnet.
Bei genauer Betrachtung von \eqref{eqn:Temperaturwelle} ergibt sich somit eine \emph{Phasengeschwindigkeit} $v$ mit 

\begin{equation}
\label{eqn:Phasengeschwindigkeit}
    v = \frac{\omega}{k} = \sfrac{\omega}{\sqrt{\frac{\omega \rho c}{2\kappa}}} = \sqrt{2\kappa \omega}{\rho c}.
\end{equation}

Aus dem Amplitudenverhältnis $\sfrac{A_\text{nah}}{A_\text{fern}}$ ergibt sich die Dämpfung der Welle. Um einen mathematischen Ausdruck 
für die Wärmeleitfähigkeit $\kappa$ zu erhalten, dienen die Ausdrücke 

\begin{align*}
    \omega &= \frac{2\pi}{T*} & \phi &= \frac{2\pi \increment t}{T*}
\end{align*}

mit der Periodendauer $T*$ und der Phase $\phi$. Damit resultiert aus \eqref{eqn:Phasengeschwindigkeit} die Gleichung

\begin{equation}
    \kappa = \frac{\rho c \left(\increment x\right)²}{2\increment t ln\left(\sfrac{A_\text{nah}}{A_\text{fern}}\right)}.
\end{equation}

Hierbei drückt $\increment x$ den Abstand zwischen den beiden Messtellen und $\increment t$ die Phasenverschiebung zwischen den Messtellen aus.


\section{Vorbereitung}
\label{sec:Vorbereitung}

Die Dichte $\rho$, spezifische Wärme $c$ und Wärmeleitfähigkeit $\kappa$ für Aluminium, Messing, Edelstahl und Wasser werden im Folgenden 
tabellarisch dargestellt:

\begin{table}
    \centering
    \caption{Stoffeigenschaften}
    \label{tab:Stoffeigenschaften}
    \begin{tblr}{
        colspec = {c c c c},
        row{1} = {guard, mode=math},
    }
    \toprule
    \text{Stoff} & \rho \mathbin{/} \unit[per-mode=fraction]{\kilo\gram\per\cubic\meter} & c \mathbin{/} \unit[per-mode=fraction]{\kilo\joule\per\kilo\gram\per\kelvin} & \kappa \mathbin{/} \unit[per-mode=fraction]{\watt\per\meter\per\kelvin} \\
    \midrule
    Aluminium & 2700 & 0.888 & 237 \\
    Messing & 8500 & 0.377 & 120 \\
    Edelstahl & 7480 - 8000 & 0.5 & 15 \\
    Wasser & 998 & 4.183 & 0.6 \\
    \bottomrule
    \end{tblr}
\end{table}

\cite{sample}

\end{document}

%https://www.yaclass.at/p/physik/9-schulstufe/kinematik-20670/dichte-20598/re-1c28828a-3952-4508-8213-00d97ae06a65
%https://www.sigmaaldrich.com/DE/de/substance/densitystandard998kgm318027732185
%https://www.gasparini.com/de/rechner-fuer-das-blechgewicht/
%https://www.chemie.de/lexikon/Spezifische_W%C3%A4rmekapazit%C3%A4t.html
%https://www.3d-activation.de/wp-content/uploads/2018/05/1.4404-316L-D.pdf
%https://www.baunetzwissen.de/glossar/w/waermeleitfaehigkeit-664148
%https://www.baulinks.de/webplugin/2012/0716.php4
%https://www.chemie.de/lexikon/Eigenschaften_des_Wassers.html
%https://www.chemie.de/lexikon/W%C3%A4rmeleitf%C3%A4higkeit.html