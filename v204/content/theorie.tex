\input{../../header.tex}

\begin{document}
\section{Zielsetzung}
\label{sec:Zielsetzung}

In dem folgend protokollierten Versuch soll die Wärmeleitung von
Aluminium, Messing und Edelstahl untersucht werden. Das konkrete Ziel besteht
darin, den zeitlichen Verlauf der Temperaturänderung der jeweiligen Stäbe zu analysieren
und die Wellenänge sowie Frequenz der \enquote{Temperaturwelle} zu bestimmen. Zusätzlich soll
die Wärmeleitfähigkeit der zu Beginn aufgeführten Metalle ermittlet werden.

\section{Theorie}
\label{sec:Theorie}

In einem System kann genau dann ein Wärmetransport beobachtet werden, falls ein Temperaturgefälle vorhanden ist. Dieser Wärmetransport kann sich durch Wärmestrahlung, Konvektion und Wärmeleitung
realisieren. In dem folgenden Versuch wird sich jedoch auf die Wärmeleitung beschränkt. \\
Das soeben angesprochene System wird durch einen Stab der Länge $L$ und Querschnittsfläche $A$ dargestellt.
Die Massendichte wird im Folgenden als $\rho$ und die spezifische Wärme als $c$ bezeichnet. \\
Existiert nun eine Temperaturdifferenz an den Enden des Stabes, so fließt die Wärmemenge

\begin{equation}
\label{eqn:Waermemenge}
    dQ = -\kappa{}A\frac{\partial{}T}{\partial{}x}dt 
\end{equation}

in der Zeit $dt$ entlang des Temperaturgefälles. Hierbei steht der Faktor $\kappa$ für die Wärmeleitfähigkeit
des Materials. Der Wärmestrom fließt dabei immer in Richtung der niedrigeren Temperatur was mathematisch durch 
das negative Vorzeichen in \eqref{eqn:Waermemenge} repräsentiert wird. Unter Hinzunahme der \emph{Kontinuitätsgleichung}

\begin{equation}
    \partial{}_t \rho + \nabla{}\cdot{}\vec{j} = 0
\end{equation}

und den Ausdrücken

\begin{equation}
    dQ = mc \cdot \partial{}T
\end{equation}

für die Wärmemenge $dQ$, sowie

\begin{equation}
    j_\omega = -\kappa \frac{\partial{}T}{\partial{}x}
\end{equation}

für die Wärmestromdichte $j_\omega$, ergibt sich die eindimensionale \emph{Wärmeleitungsgleichung}

\begin{equation}
    \frac{\partial T}{\partial t} = \frac{\kappa}{\rho c} \frac{\partial² T}{\partial x²}.
\end{equation}


\section{Vorbereitung}
\cite{sample}

\end{document}