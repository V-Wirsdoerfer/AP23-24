%\input{../../header.tex}

\begin{document}
\section{Diskussion}
\label{sec:Diskussion}

Ein Abgleich der experimentell ermittelten Werte der Wärmeleitfähigkeiten mit der Literatur zeigt, dass die Messergebnisse nah 
an den tatsächlichen Werten von $\kappa$ liegen. So beträgt die prozentuale Abweichung der gemittelten Wärmeleitfähigkeiten von 
Aluminium und Messing weniger als 4\,$\unit{\percent}$ von den Literaturwerten. Die größte Abweichung liegt bei Messung mit etwa
9\,$\unit{\percent}$ vor. Nichtsdestotrotz besitzen allen Werte einen vergleichsweise großen Fehler, wofür folgende Ursachen
verantwortlich sein könnten:\\\\

Der Temperaturen wurden digital über den Datenlogger ausgelesen, weshalb dort menschliches Versagen ausgeschlossen werden kann.
Bei der dynamischen Methode wurde alle $40\,\unit{\second}$ bzw. alle $100\,\unit{\second}$ der Schiebeschalter manuell
von \enquote{COOL} auf \enquote{HEAT} gestellt. Die Messungen wurden jeweils nach ganzen Perioden beendet. Daher sollten die 
Messdaten ganzzahlige Vielfache von $80\,\unit{\second}$ bzw. $200\,\unit{\second}$ sein. Jedoch ist die Messung der 
$80\,\unit{\second}$ Perioden insgesamt $900\,\unit{\second}$ und die Messung der $200\,\unit{\second}$ Perioden insgesamt 
$1104\,\unit{\second}$ lang.
Daher wurde die Periodendauer nicht genau eingehalten. Die Perioden können somit auch unterschiedlich lang sein.
Dies könnte ein Grund dafür sein, dass die Phasendifferenz eine Unischerheit von bis zu $23{\%}$ besitzt.
%Auswirkung ???????????????????????????????
Außerdem ist es zu erwähnen, dass die Wärmeisolierungen nicht perfekt sind und es am seitlichen Rand der Stäbe keine
Isolierung gab. Daher ist das System für Raumtemperaturschwankungen anfällig. Diese können sich auch durch Luftzirkulationen 
verursacht von umgebenden Menschen äußern.
\end{document}
