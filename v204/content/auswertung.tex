\input{../../header.tex}

\begin{document}
\section{Auswertung}
\label{sec:Auswertung}

\subsection{Statische Methode}
\label{sec:Statische Methode}

Zu Beginn werden die Temperaturverläufe der fernen Thermoelemente graphisch aufgetragen. Hierbei bezieht sich die erste Abbildung
auf den Temperaturverlauf des breiten und schmalen Messingstabes:

\begin{figure}
  \centering
  \includegraphics[width=\textwidth]{../build/statisch_T1_T4.pdf}
  \caption{Temperaturverläufe der Messingstäbe.}
  \label{fig:statisch1}
\end{figure}

Anhand der Graphik \ref{fig:statisch1} lässt sich erkennen, dass der breite und der schmale Messingstab in den ersten 50\,$\unit{\second}$
einen ähnlich starken, exponentiellen Temperaturanstieg verzeichen können. In der Folge lässt sich bis zu einer Zeit von etwa 
200\,$\unit{\second}$ ein linearer Anstieg beider Temperaturverläufe konstatieren, wobei die Temperatur des schmalen Stabes die 
des breiten Stabes kontinuierlich überwiegt. Nach einer Zeit von 180\,$\unit{\second}$ ändert sich dieser Trend jedoch und die
Temperatur des breiten Messingstabes übersteigt die Temperatur des schmalen Stabes. Mit voranschreitender Zeit flachen jedoch beide
Verläufe ab und der Temperaturwert des breiten Stabes ist fortwährend ca. $2\,\unit{\celsius}$ über der Temperatur des schmalen Stabes.
\newpage

\begin{figure}
  \centering
  \includegraphics[width=\textwidth]{../build/statisch_T5_T8.pdf}
  \caption{Temperaturverläufe von Aluminium und Edelstahl.}
  \label{fig:statisch2}
\end{figure}

Der Vergleich des Temperaturverlaufes von Aluminium und Edelstahl in \ref{fig:statisch2} zeigt, dass 
sich der Aluminiumstab signifikant schneller erwärmt, als der Edelstahlstab.
Die Temperatur des Aluminiumstabes steigt zu Beginn mit hoher Geschwindigkeit an. Nach etwa 180\,\unit{\second} 
flacht der Temperaturanstieg bei etwa 43\,\unit{\celsius} ab. Ab etwa 200\,\unit{\second} steigt die Temperatur des Aluminiumstabes 
nur noch leicht bis auf Insgesamt 49\,\unit{\celsius} in einem linear ähnlichem Verlauf an. Im Vegleich dazu steigt die Temperatur 
des Edelstahlstabes in gleichen Zeitintervallen nur geringfügig an. So herrscht bereits nach 100/,$\unit{\second}$ eine
Temperaturdifferenz von ca. $12/,\unit{\celsius}$. Analog zum Aluminium gliedert sich auch beim Edelstahl nach etwa $200/,\unit{\second}$
ein linearer Temperaturverlauf ein. Jedoch erreicht dieser Verlauf bei Beendigung der Messung einen vergleichsweise niedrigen
Wert von $37\,\unit{\celsius}$. \\\\
In einem zusätzlichen Vegleich aller vier Thermoelemente lässt sich herauskristallisieren, dass der Temperaturverlauf der Messingstäbe
zwischen den Verläufen von Aluminium und Edelstahl einzuordnen ist. Diese grafische Analyse wird erneut durch die Werte der Thermoelemente
nach $700\,\unit{\second}$ unterstrichen\footnote{Zur Umrechnung von Celsius auf Kelvin wurden zu den Celsius-Werten $273.15\,\unit{\degree}$ addiert}:

\begin{gather*}
  T_1 = 318.81\,\unit{\kelvin}\\ 
  T_4 = 317.06\,\unit{\kelvin}\\
  T_5 = 321.95\,\unit{\kelvin}\\
  T_8 = 309.68\,\unit{\kelvin}\\
\end{gather*}

In dieser Auflistung überragt der Wert des Thermoelements $T_5$, welches die Temperatur des Aluminiumstabes misst. Somit decken sich die grafischen
Beobachtungen mit diesen Daten. Aufgrund von nahezu identischen Anfangsbedingungen aller Stäbe lässt sich sagen, dass Aluminium die beste 
Wärmeleitung besitzt.

Nun wird gemäß Gleichung \eqref{eqn:Waermemenge} der Wärmestrom $\sfrac{\increment Q}{\increment t}$ zu fünf verschiedenen Messzeiten berechnet
werden. Dazu werden den Rohdaten fünf verschiedene Temperaturdifferenzen $\partial T = T_\text{nah} - T_\text{fern}$ der nahen und fernen Thermoelemente der jeweiligen Probestäbe entnommen.
Die Literaturwerte der Wärmeleitfähigkeiten stammen aus der Tabelle \ref{tab:Stoffeigenschaften} und die Werte der Querschnittsfläche $A$ aus der
Versuchsanleitung\cite{Versuchsanleitung_v204}. In der Formel bezeichnet $\partial x$ den zu Beginn gemessen Abstand der Thermoelemente. 
Dieser liegt bei $\partial x = \left(0.03 + 0.01\right)\,\unit{\meter}$ \\
Demnach gilt für die Messingtäbe:

\begin{table}
  \centering
  \caption{Wärmestrom von Messing.}
  \label{tab:Waermestrom_Messing}
  \sisetup{table-format=1.1, per-mode=reciprocal}
  \begin{tblr}{
      colspec = {S S S S S},
      row{1} = {guard, mode=math},
    }
    \toprule
    \text{Messzeit}\, t \mathbin{/} \unit{\second} & \left(T_2 - T_1\right) \mathbin{/} \unit{\kelvin}& \frac{\increment Q_{21}}{\increment t} \mathbin{/} \unit{\watt} & \left(T_3 - T_4\right) \mathbin{/} \unit{\kelvin}& \frac{\increment Q_{34}}{\increment t} \mathbin{/} \unit{\watt} \\
    \midrule
    100 & 5.13 & -0.089 & 5.6 & -0.056 \\
    200 & 3.71 & -0.064 & 4 & -0.040 \\ 
    400 & 2.67 & -0.046 & 3.1 & -0.031 \\ 
    600 & 2.43 & -0.042 & 2.94 & -0.030 \\ 
    750 & 2.39 & -0.041 & 2.9 & -0.029 \\  
    \bottomrule
  \end{tblr}
\end{table}

Die identische Herangehenweise wird für die Bestimmung des Wärmestroms von Aluminium und Edelstahl gewählt.
Somit gilt für den Aluminium- und Edelstahlstab:
\newpage 
\begin{table}
  \centering
  \caption{Wärmestrom von Aluminium und Edelstahl.}
  \label{tab:Waermestrom_Alu_Edelstahl}
  \sisetup{table-format=1.1, per-mode=reciprocal}
  \begin{tblr}{
      colspec = {S S S S S},
      row{1} = {guard, mode=math},
    }
    \toprule
    \text{Messzeit}\, t \mathbin{/} \unit{\second} & \left(T_6 - T_5\right) \mathbin{/} \unit{\kelvin}& \frac{\increment Q_{65}}{\increment t} \mathbin{/} \unit{\watt} & \left(T_7 - T_8\right) \mathbin{/} \unit{\kelvin}& \frac{\increment Q_{78}}{\increment t} \mathbin{/} \unit{\watt} \\
    \midrule
    100 & 3.09 & -0,105 & 9.37 & -0.020 \\
    200 & 1.85 & -0.063 & 10.00 & -0.021 \\ 
    400 & 1.33 & -0.045 & 8.86 & -0.019 \\ 
    600 & 1.26 & -0.043 & 8.28 & -0.018 \\ 
    750 & 1.23 & -0.042 & 8.06 & -0.017 \\  
    \bottomrule
  \end{tblr}
\end{table}

Im letzten Schritt der statischen Methode werden die Temperaturdifferenzen $\increment T_{21} = T_2 - T_1$ und
$\increment T_{78} = T_7 - T_8$
als Funktionen der Zeit grafisch aufgetragen.

\begin{figure}
  \centering
  \includegraphics[width=\textwidth]{../build/Temperaturdifferenz.pdf}
  \caption{Temperaturdifferenzen als Funktionen der Zeit.}
  \label{fig:statisch3}
\end{figure}

In der oberen Grafik wird der zeitliche Verlauf der Temperaturdifferenzen der Thermoelemente $T_2$ und $T_1$ abgebildet.
Zu Beginn der Messung steigt die Temperaturdifferenz exorbitant an, was sowohl an dem Anstieg selbst, aber auch an den 
großen Abständen der Messpunkte zu erkennen ist. Nach einer Zeit von ca. $70\,\unit{\second}$ bei einer Temperaturdifferenz von ungefähr 5.6\,\unit{\celsius} ändert sich dieser Trend jedoch
schlagartig und die Temperaturdifferenz wird weniger. Hierbei ist der zu beobachtende Abstieg jedoch nicht so gravierdend wie
der vorangegangene Anstieg. Nach einer Zeit von etwa 400\,$\unit{\second}$ verändert sich die Temperaturdifferenz nur noch 
geringfügig und pendelt sich mit einer absteigendne Tendenz bei ca.$2.4\,\unit{\celsius}$ ein.\\
Ähnlich zum Messing, steigt auch die Temperaturdifferenz der Thermoelemente des Edelstahlstabes anfänglich stark an. Hier erreicht die 
Temperaturdifferenz jedoch ein Maximum von ungefähr 10\,$\unit{\celsius}$. Analog zum Messing sinkt jedoch auch hier der Wert der 
Temperaturdifferenz. Vergleichsweise ist dieser Abstieg jedoch nicht so stark ausgeprägt wie beim Messing, weswegen sich die Temperaturdifferenz
zum Ende der Messung bei ca. $8.04\,\unit{\celsius}$ befindet.

\subsection{Dynamische Methode}

Wie bereits in vorherigen Teilen des Protokolls angekündigt, wird bei der dynamischen Methode die Wärmeleitfähigkeit $\kappa$ der einzelen Metalle
ermittelt. Dies gelingt durch periodisches Erhitzen und Abkülen der Probestäbe und durch die Anwendung von Gleichung \eqref{eqn:waermeleitfaehigkeit}.
Hierbei treten jedoch Größen auf, welche nicht durch die Literatur abgelesen, sondern bestimmt werden müssen. Sowohl die Phasendiffernez $\increment t$
aber auch der natürliche Logarithmus des Amplitudenverhältnisses werden durch einen spezifischen Python-Code berechnet. Die dazugehörigen Mittelwerte und
Fehler folgen den Gleichungen \eqref{eqn:Mittelwert} und \eqref{eqn:Mittelwertfehler}. Dementsprechend wird für die Berechnung Wärmeleitfähigkeit $\kappa$
die Gauß'sche Fehlerfortpflanzung \eqref{eqn:Fehlerfortpflanzung} angewendet. 

\section{Diskussion}
Test
\section{Literatur}

\section{Anhang}
Siehe \autoref{fig:plot} und \autoref{tab:tabelle}!

\end{document}
