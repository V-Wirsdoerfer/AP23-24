\documentclass[
  bibliography=totoc,     % Literatur im Inhaltsverzeichnis
  captions=tableheading,  % Tabellenüberschriften
  titlepage=firstiscover, % Titelseite ist Deckblatt
]{scrartcl}

% Paket float verbessern
\usepackage{scrhack}

% Warnung, falls nochmal kompiliert werden muss
\usepackage[aux]{rerunfilecheck}

% unverzichtbare Mathe-Befehle
\usepackage{amsmath}
% viele Mathe-Symbole
\usepackage{amssymb}
% Erweiterungen für amsmath
\usepackage{mathtools}

% Fonteinstellungen
\usepackage{fontspec}
% Latin Modern Fonts werden automatisch geladen
% Alternativ zum Beispiel:
%\setromanfont{Libertinus Serif}
%\setsansfont{Libertinus Sans}
%\setmonofont{Libertinus Mono}

% Wenn man andere Schriftarten gesetzt hat,
% sollte man das Seiten-Layout neu berechnen lassen
\recalctypearea{}

% deutsche Spracheinstellungen
\usepackage[ngerman]{babel}


\usepackage[
  math-style=ISO,    % ┐
  bold-style=ISO,    % │
  sans-style=italic, % │ ISO-Standard folgen
  nabla=upright,     % │
  partial=upright,   % │
  mathrm=sym,        % ┘
  warnings-off={           % ┐
    mathtools-colon,       % │ unnötige Warnungen ausschalten
    mathtools-overbracket, % │
  },                       % ┘
]{unicode-math}

% traditionelle Fonts für Mathematik
\setmathfont{Latin Modern Math}
% Alternativ zum Beispiel:
%\setmathfont{Libertinus Math}

\setmathfont{XITS Math}[range={scr, bfscr}]
\setmathfont{XITS Math}[range={cal, bfcal}, StylisticSet=1]

% Zahlen und Einheiten
\usepackage[
  locale=DE,                   % deutsche Einstellungen
  separate-uncertainty=true,   % immer Unsicherheit mit \pm
  per-mode=symbol-or-fraction, % / in inline math, fraction in display math
]{siunitx}

% chemische Formeln
\usepackage[
  version=4,
  math-greek=default, % ┐ mit unicode-math zusammenarbeiten
  text-greek=default, % ┘
]{mhchem}

% richtige Anführungszeichen
\usepackage[autostyle]{csquotes}

% schöne Brüche im Text
\usepackage{xfrac}

% Standardplatzierung für Floats einstellen
\usepackage{float}
\floatplacement{figure}{htbp}
\floatplacement{table}{htbp}

% Floats innerhalb einer Section halten
\usepackage[
  section, % Floats innerhalb der Section halten
  below,   % unterhalb der Section aber auf der selben Seite ist ok
]{placeins}

% Seite drehen für breite Tabellen: landscape Umgebung
\usepackage{pdflscape}

% Captions schöner machen.
\usepackage[
  labelfont=bf,        % Tabelle x: Abbildung y: ist jetzt fett
  font=small,          % Schrift etwas kleiner als Dokument
  width=0.9\textwidth, % maximale Breite einer Caption schmaler
]{caption}
% subfigure, subtable, subref
\usepackage{subcaption}

% Grafiken können eingebunden werden
\usepackage{graphicx}

% schöne Tabellen
\usepackage{tabularray}
\UseTblrLibrary{booktabs, siunitx}

% Verbesserungen am Schriftbild
\usepackage{microtype}

% Literaturverzeichnis
\usepackage[
  backend=biber,
]{biblatex}
% Quellendatenbank
\addbibresource{lit.bib}
\addbibresource{programme.bib}

% Hyperlinks im Dokument
\usepackage[
  german,
  unicode,        % Unicode in PDF-Attributen erlauben
  pdfusetitle,    % Titel, Autoren und Datum als PDF-Attribute
  pdfcreator={},  % ┐ PDF-Attribute säubern
  pdfproducer={}, % ┘
]{hyperref}
% erweiterte Bookmarks im PDF
\usepackage{bookmark}

% Trennung von Wörtern mit Strichen
\usepackage[shortcuts]{extdash}

\author{%
  Vincent Wirsdörfer\\%
  \href{mailto:vincent.wirsdoerfer@udo.edu}{authorA@udo.edu}%
  \and%
  Joris Daus\\%
  \href{mailto:joris.daus@udo.edu}{authorB@udo.edu}%
}
\publishers{TU Dortmund – Fakultät Physik}


\begin{document}
\section{Auswertung}
\label{sec:Auswertung}

\subsection{Statische Methode}
\label{sec:Statische Methode}

Zu Beginn werden die Temperaturverläufe der fernen Thermoelemente graphisch aufgetragen. Hierbei bezieht sich die erste Abbildung
auf den Temperaturverlauf des breiten und schmalen Messingstabes:

\begin{figure}[H]
  \centering
  \includegraphics[width=\textwidth]{../build/statisch_T1_T4.pdf}
  \caption{Temperaturverläufe der Messingstäbe.}
  \label{fig:statisch1}
\end{figure}

Anhand der Graphik \ref{fig:statisch1} lässt sich erkennen, dass der breite und der schmale Messingstab in den ersten 50\,$\unit{\second}$
einen ähnlich starken, exponentiellen Temperaturanstieg verzeichen können. In der Folge lässt sich bis zu einer Zeit von etwa 
200\,$\unit{\second}$ ein linearer Anstieg beider Temperaturverläufe konstatieren, wobei die Temperatur des schmalen Stabes die 
des breiten Stabes kontinuierlich überwiegt. Nach einer Zeit von 180\,$\unit{\second}$ ändert sich dieser Trend jedoch und die
Temperatur des breiten Messingstabes übersteigt die Temperatur des schmalen Stabes. Mit voranschreitender Zeit flachen jedoch beide
Verläufe ab und der Temperaturwert des breiten Stabes ist fortwährend ca. $2\,\unit{\celsius}$ über der Temperatur des schmalen Stabes.
\newpage

\begin{figure}[H]
  \centering
  \includegraphics[width=\textwidth]{../build/statisch_T5_T8.pdf}
  \caption{Temperaturverläufe von Aluminium und Edelstahl.}
  \label{fig:statisch2}
\end{figure}

Der Vergleich des Temperaturverlaufes von Aluminium und Edelstahl in \ref{fig:statisch2} zeigt, dass 
sich der Aluminiumstab signifikant schneller erwärmt, als der Edelstahlstab.
Die Temperatur des Aluminiumstabes steigt zu Beginn mit hoher Geschwindigkeit an. Nach etwa 180\,\unit{\second} 
flacht der Temperaturanstieg bei etwa 43\,\unit{\celsius} ab. Ab etwa 200\,\unit{\second} steigt die Temperatur des Aluminiumstabes 
nur noch leicht bis auf Insgesamt 49\,\unit{\celsius} in einem linear ähnlichem Verlauf an. Im Vegleich dazu steigt die Temperatur 
des Edelstahlstabes in gleichen Zeitintervallen nur geringfügig an. So herrscht bereits nach 100/,$\unit{\second}$ eine
Temperaturdifferenz von ca. $12/,\unit{\celsius}$. Analog zum Aluminium gliedert sich auch beim Edelstahl nach etwa $200/,\unit{\second}$
ein linearer Temperaturverlauf ein. Jedoch erreicht dieser Verlauf bei Beendigung der Messung einen vergleichsweise niedrigen
Wert von $37\,\unit{\celsius}$. \\\\
In einem zusätzlichen Vegleich aller vier Thermoelemente lässt sich herauskristallisieren, dass der Temperaturverlauf der Messingstäbe
zwischen den Verläufen von Aluminium und Edelstahl einzuordnen ist. Diese grafische Analyse wird erneut durch die Werte der Thermoelemente
nach $700\,\unit{\second}$ unterstrichen\footnote{Zur Umrechnung von Celsius auf Kelvin wurden zu den Celsius-Werten $273.15\,\unit{\degree}$ addiert}:

\begin{gather*}
  T_1 = 318.81\,\unit{\kelvin}\\ 
  T_4 = 317.06\,\unit{\kelvin}\\
  T_5 = 321.95\,\unit{\kelvin}\\
  T_8 = 309.68\,\unit{\kelvin}\\
\end{gather*}

In dieser Auflistung überragt der Wert des Thermoelements $T_5$, welches die Temperatur des Aluminiumstabes misst. Somit decken sich die grafischen
Beobachtungen mit diesen Daten. Aufgrund von nahezu identischen Anfangsbedingungen aller Stäbe lässt sich sagen, dass Aluminium die beste 
Wärmeleitung besitzt.

Nun wird gemäß Gleichung \eqref{eqn:Waermemenge} der Wärmestrom $\sfrac{\increment Q}{\increment t}$ zu fünf verschiedenen Messzeiten berechnet
werden. Dazu werden den Rohdaten fünf verschiedene Temperaturdifferenzen $\partial T = T_\text{nah} - T_\text{fern}$ der nahen und fernen Thermoelemente der jeweiligen Probestäbe entnommen.
Die Literaturwerte der Wärmeleitfähigkeiten stammen aus der Tabelle \ref{tab:Stoffeigenschaften} und die Werte der Querschnittsfläche $A$ aus der
Versuchsanleitung\cite{Versuchsanleitung_v204}. In der Formel bezeichnet $\partial x$ den zu Beginn gemessen Abstand der Thermoelemente. 
Dieser liegt bei $\partial x = \left(0.03 + 0.005\right)\,\unit{\meter}$ \\
Demnach gilt für die Messingtäbe:

\begin{table}
  \centering
  \caption{Wärmestrom von Messing.}
  \label{tab:Waermestrom_Messing}
  \sisetup{ per-mode=reciprocal}
  \begin{tblr}{
      colspec = {S[table-format=3.0] S[table-format=1.2] 
      S[separate-uncertainty=true,table-format=1.3(1)] 
      S[table-format=1.2] S[separate-uncertainty=true,table-format=1.1(1)]},
      row{1} = {guard, mode=math},
    }
    \toprule
    \text{Messzeit}\, t \mathbin{/} \unit{\second} & 
    \left(T_2 - T_1\right) \mathbin{/} \unit{\kelvin}& 
    \frac{\increment Q_{21}}{\increment t} \mathbin{/} \unit{\watt} & 
    \left(T_3 - T_4\right) \mathbin{/} \unit{\kelvin}& 
    \frac{\increment Q_{34}}{\increment t} \mathbin{/} \unit{\watt} \\
    \midrule
    100 & 5.13 & -0.98\pm0.16   & 5.6   & -6.3\pm1.0  \\
    200 & 3.71 & -0.71\pm0.12   & 4     & -4.5\pm0.7  \\ 
    400 & 2.67 & -0.51\pm0.09   & 3.1   & -3.5\pm0.6  \\ 
    600 & 2.43 & -0.042\pm0.08  & 2.94  & -3.3\pm0.5  \\ 
    750 & 2.39 & -0.041\pm0.08  & 2.9   & -3.2\pm0.5  \\  
    \bottomrule
  \end{tblr}
\end{table}

Die identische Herangehenweise wird für die Bestimmung des Wärmestroms von Aluminium und Edelstahl gewählt.
Somit gilt für den Aluminium- und Edelstahlstab:
\newpage 
\begin{table}
  \centering
  \caption{Wärmestrom von Aluminium und Edelstahl.}
  \label{tab:Waermestrom_Alu_Edelstahl}
  \sisetup{table-format=1.1, per-mode=reciprocal}
  \begin{tblr}{
      colspec = {S[table-format=3.0] S[table-format=1.2] 
      S[separate-uncertainty=true,table-format=1.2(1)] 
      S[table-format=2.2] S[separate-uncertainty=true,table-format=1.3(1)]},
      row{1} = {guard, mode=math},
    }
    \toprule
    \text{Messzeit}\, t \mathbin{/} \unit{\second} & 
    \left(T_6 - T_5\right) \mathbin{/} \unit{\kelvin}& 
    \frac{\increment Q_{65}}{\increment t} \mathbin{/} \unit{\watt} 
    & \left(T_7 - T_8\right) \mathbin{/} \unit{\kelvin}& 
    \frac{\increment Q_{78}}{\increment t} \mathbin{/} \unit{\watt} \\
    \midrule
    100 & 3.09 & -1.17\pm0.2  & 9.37  & -0.22 \pm0.04 \\
    200 & 1.85 & -0.7 \pm0.12 & 10.00 & -0.24 \pm0.04 \\ 
    400 & 1.33 & -0.5 \pm0.08 & 8.86  & -0.213\pm0.035 \\ 
    600 & 1.26 & -0.48\pm0.08 & 8.28  & -0.198\pm0.033 \\ 
    750 & 1.23 & -0.47\pm0.08 & 8.06  & -0.193\pm0.032 \\  
    \bottomrule
  \end{tblr}
\end{table}
\hfill \break
Im letzten Schritt der statischen Methode werden die Temperaturdifferenzen\\ 
$\increment T_{21}=T_2-T_1$ und $\increment T_{78} = T_7 - T_8$ als Funktionen der Zeit grafisch aufgetragen.

\begin{figure}[H]
  \centering
  \includegraphics[width=\textwidth]{../build/Temperaturdifferenz.pdf}
  \caption{Temperaturdifferenzen als Funktionen der Zeit.}
  \label{fig:statisch3}
\end{figure}

In der oberen Grafik wird der zeitliche Verlauf der Temperaturdifferenzen der Thermoelemente $T_2$ und $T_1$ abgebildet.
Zu Beginn der Messung steigt die Temperaturdifferenz exorbitant an, was sowohl an dem Anstieg selbst, aber auch an den 
großen Abständen der Messpunkte zu erkennen ist. Nach einer Zeit von ca. $70\,\unit{\second}$ bei einer Temperaturdifferenz von ungefähr 5.6\,\unit{\celsius} ändert sich dieser Trend jedoch
schlagartig und die Temperaturdifferenz wird weniger. Hierbei ist der zu beobachtende Abstieg jedoch nicht so gravierend, wie
der vorangegangene Anstieg. Nach einer Zeit von etwa 400\,$\unit{\second}$ verändert sich die Temperaturdifferenz nur noch 
geringfügig und pendelt sich mit einer absteigenden Tendenz bei ca. $2.4\,\unit{\celsius}$ ein.\\
Ähnlich zum Messing, steigt auch die Temperaturdifferenz der Thermoelemente des Edelstahlstabes anfänglich stark an. Hier erreicht die 
Temperaturdifferenz jedoch ein Maximum von ungefähr 10\,$\unit{\celsius}$. Analog zum Messing sinkt jedoch auch hier der Wert der 
Temperaturdifferenz. Vergleichsweise ist dieser Abstieg jedoch nicht so stark ausgeprägt wie beim Messing, weswegen sich die Temperaturdifferenz
zum Ende der Messung bei ca. $8.04\,\unit{\celsius}$ befindet.

\subsection{Dynamische Methode}

Wie bereits in vorherigen Teilen des Protokolls angekündigt, wird bei der dynamischen Methode die Wärmeleitfähigkeit $\kappa$ der einzelnen Metalle
ermittelt. Dies gelingt durch periodisches Erhitzen und Abkülen der Probestäbe und durch die Anwendung von Gleichung \eqref{eqn:waermeleitfaehigkeit}.
Hierbei treten jedoch Größen auf, welche nicht durch die Literatur abgelesen, sondern bestimmt werden müssen. 
Sowohl die Phasendifferenz $\increment t$ aber auch die Amplituden $A_{\text{nah}}$ und $A_{\text{fern}}$ werden durch einen spezifischen Python-Code 
berechnet. Im folgenden wird die Berechnung erläutert. \\
Zur Bestimmung der Phasendifferenz müssen die Temperaturverläufe von zwei Messpunkten des selben Thermoelementes angeschaut werden. Es werden nun 
jeweils die Minima und Maxima beider Verläufe ermittelt. Die Zeitdifferenz von dem ersten Maximum des nahen Messpunktes bis zum ersten Maximum des fernen 
Messpunktes ist ein Wert für die Phasendifferenz. Diese Zeitdifferenzen werden analog für alle Minima und Maxima bestimmt. Alle Zeitdifferenzen werden 
gemittelt, um die Phasendifferenz zu bestimmen.
Bei der Ermittelung der Amplitude wird sich auf einen Temperaturverlauf beschränkt. Dort werden jeweils Minima und Maxima der Verlaufes ermittelt.
Mithilfe von Gleichung \eqref{noch_nicht_vorhanden} werden nun die Amplituden der nahen und fernen Thermoelemente jeweils berechnet. Es wird nun wieder 
über alle Amplituden des jeweils nahen oder fernen Thermoelementes gemittelt, um einen Wert für $A_{\text{nah}}$ und $A_{\text{fern}}$ zu bestimmen.
Die jeweils bestimmten Mittelwerte und Fehler folgen den Gleichungen \eqref{eqn:Mittelwert} und \eqref{eqn:Mittelwertfehler}.Dementsprechend wird 
für die Berechnung der Wärmeleitfähigkeit $\kappa$ die Gauß'sche Fehlerfortpflanzung \eqref{eqn:Fehlerfortpflanzung} angewendet.\\
Vor der Kalkulation werden jedoch die Temperaturverläufe für den breiten Messingstab \ref{fig:dynamisch1} und den 
Aluminiumstab \ref{fig:dynamisch2} bei einer Periodendauer von 80\,$\unit{\second}$ grafisch dargetellt.

\begin{figure}[H]
  \centering
  \includegraphics[width=\textwidth-100pt]{../build/dynamisch_T1_T2.pdf}
  \caption{Temperaturverlauf von $T_1$ und $T_2$ mit Periodendauer 80\,\unit{\second}}
  \label{fig:dynamisch1}
\end{figure}

\begin{figure}[H]
  \centering
  \includegraphics[width=\textwidth-100pt]{../build/dynamisch_T5_T6.pdf}
  \caption{Temperaturverlauf von $T_5$ und $T_6$ mit Periodendauer 80\,\unit{\second}}
  \label{fig:dynamisch2}
\end{figure}

Mittels den daraus gewonnenen Daten und den im obigen Text angegebenen Formeln, kann nun die Wärmeleitfähigkeit ermittelt werden:

Die Werte von Messing
\begin{gather*}
  \rho_\text{Messing} = 8520\,\unit[per-mode=fraction]{\kilo\gram\per\cubic\meter} \\
  c_\text{Messing} = 0.385\,\unit{\kilo\joule\per\kilo\gram\per\kelvin} \\
  \increment x = \left(0.03\pm0.005\right)\,\unit{\meter} \\
  \increment t = \left(11.81\pm0.925\right)\,\unit{\second} \\
  A_\text{nah} = \left(4.61\pm0.021\right)\,\unit{\celsius}\\
  A_\text{fern} = \left(1.437\pm0.03\right)\,\unit{\celsius}\\
\end{gather*}

ergeben nach Gleichung \eqref{eqn:waermeleitfaehigkeit} und \eqref{eqn:Fehlerfortpflanzung}

\begin{equation}
  \kappa_\text{Messing} = \left(110\pm40\right)\,\unit{\watt\per\meter\per\kelvin}
\end{equation}

als Wärmeleitfähigkeit für Messing.
\hfill \break

Die Werte von Aluminium
\begin{gather*}
  \rho_\text{Aluminium} = 2800\,\unit[per-mode=fraction]{\kilo\gram\per\cubic\meter} \\
  c_\text{Aluminium} = 0.830\,\unit{\kilo\joule\per\kilo\gram\per\kelvin} \\
  \increment x = \left(0.03\pm0.005\right)\,\unit{\meter} \\
  \increment t = \left(6.476\pm0.363\right)\,\unit{\second} \\
  A_\text{nah} = \left(6.461\pm0.034\right)\,\unit{\celsius}\\
  A_\text{fern} = \left(3.209\pm0.039\right)\,\unit{\celsius}\\
\end{gather*}

ergeben nach Gleichung \eqref{eqn:waermeleitfaehigkeit} und \eqref{eqn:Fehlerfortpflanzung}

\begin{equation}
  \kappa_\text{Aluminium} = \left(230\pm80\right)\,\unit{\watt\per\meter\per\kelvin}
\end{equation}

als Wärmeleitfähigkeit für Aluminium.
\hfill \break

Die Werte von Edelstahl
\begin{gather*}
  \rho_\text{Edelstahl} = 8000\,\unit[per-mode=fraction]{\kilo\gram\per\cubic\meter} \\
  c_\text{Edelstahl} = 0.4\,\unit{\kilo\joule\per\kilo\gram\per\kelvin} \\
  \increment x = \left(0.03\pm0.005\right)\,\unit{\meter} \\
  \increment t = \left(152.222\pm35.045\right)\,\unit{\second} \\
  A_\text{nah} = \left(7.131\pm0.03\right)\,\unit{\celsius}\\
  A_\text{fern} = \left(3.873\pm0.186\right)\,\unit{\celsius}\\
\end{gather*}

ergeben nach Gleichung \eqref{eqn:waermeleitfaehigkeit} und \eqref{eqn:Fehlerfortpflanzung}

\begin{equation}
  \kappa_\text{Edelstahl} = \left(15\pm6\right)\,\unit{\watt\per\meter\per\kelvin}
\end{equation}

als Wärmeleitfähigkeit für Edelstahl.
\hfill \break

\section{Diskussion}
 
Ein Abgleich der experimentell ermittelten Werte der Wärmeleitfähigkeiten mit der Literatur zeigt, dass die Messergebnisse nah 
an den tatsächlichen Werten von $\kappa$ liegen. So beträgt die prozentuale Abweichung der gemittelten Wärmeleitfähigkeiten von 
Aluminium und Messing weniger als 4\,$\unit{\percent}$ von den Literaturwerten. Die größte Abweichung liegt bei Messung mit etwa
9\,$\unit{\percent}$ vor. Nichtsdestotrotz besitzen allen Werte einen vergleichsweise großen Fehler, wofür folgende Ursachen
verantwortlich sein könnten:\\\\

Der Temperaturen wurden digital über den Datenlogger ausgelesen, weshalb dort menschliches Versagen ausgeschlossen werden kann.
Bei der dynamischen Methode wurde alle $40\,\unit{\second}$ bzw. alle $100\,\unit{\second}$ der Schiebeschalter manuell
von \enquote{COOL} auf \enquote{HEAT} gestellt. Die Messungen wurden jeweils nach ganzen Perioden beendet. Daher sollten die 
Messdaten ganzzahlige Vielfache von $80\,\unit{\second}$ bzw. $200\,\unit{\second}$ sein. Jedoch ist die Messung der 
$80\,\unit{\second}$ Perioden insgesamt $900\,\unit{\second}$ und die Messung der $200\,\unit{\second}$ Perioden insgesamt 
$1104\,\unit{\second}$ lang.
Daher wurde die Periodendauer nicht genau eingehalten. Die Perioden können somit auch unterschiedlich lang sein.
Dies könnte ein Grund dafür sein, dass die Phasendifferenz eine Unischerheit von bis zu $23{\%}$ besitzt.
%Auswirkung ???????????????????????????????
Außerdem ist es zu erwähnen, dass die Wärmeisolierungen nicht perfekt sind und es am seitlichen Rand der Stäbe keine
Isolierung gab. Daher ist das System für Raumtemperaturschwankungen anfällig. Diese können sich auch durch Luftzirkulationen 
verursacht von umgebenden Menschen äußern.


\section{Anhang}
Siehe \autoref{fig:plot} und \autoref{tab:tabelle}!

\end{document}
