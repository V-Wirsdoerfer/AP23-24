\documentclass[
  bibliography=totoc,     % Literatur im Inhaltsverzeichnis
  captions=tableheading,  % Tabellenüberschriften
  titlepage=firstiscover, % Titelseite ist Deckblatt
]{scrartcl}

% Paket float verbessern
\usepackage{scrhack}

% Warnung, falls nochmal kompiliert werden muss
\usepackage[aux]{rerunfilecheck}

% unverzichtbare Mathe-Befehle
\usepackage{amsmath}
% viele Mathe-Symbole
\usepackage{amssymb}
% Erweiterungen für amsmath
\usepackage{mathtools}

% Fonteinstellungen
\usepackage{fontspec}
% Latin Modern Fonts werden automatisch geladen
% Alternativ zum Beispiel:
%\setromanfont{Libertinus Serif}
%\setsansfont{Libertinus Sans}
%\setmonofont{Libertinus Mono}

% Wenn man andere Schriftarten gesetzt hat,
% sollte man das Seiten-Layout neu berechnen lassen
\recalctypearea{}

% deutsche Spracheinstellungen
\usepackage[ngerman]{babel}


\usepackage[
  math-style=ISO,    % ┐
  bold-style=ISO,    % │
  sans-style=italic, % │ ISO-Standard folgen
  nabla=upright,     % │
  partial=upright,   % │
  mathrm=sym,        % ┘
  warnings-off={           % ┐
    mathtools-colon,       % │ unnötige Warnungen ausschalten
    mathtools-overbracket, % │
  },                       % ┘
]{unicode-math}

% traditionelle Fonts für Mathematik
\setmathfont{Latin Modern Math}
% Alternativ zum Beispiel:
%\setmathfont{Libertinus Math}

\setmathfont{XITS Math}[range={scr, bfscr}]
\setmathfont{XITS Math}[range={cal, bfcal}, StylisticSet=1]

% Zahlen und Einheiten
\usepackage[
  locale=DE,                   % deutsche Einstellungen
  separate-uncertainty=true,   % immer Unsicherheit mit \pm
  per-mode=symbol-or-fraction, % / in inline math, fraction in display math
]{siunitx}

% chemische Formeln
\usepackage[
  version=4,
  math-greek=default, % ┐ mit unicode-math zusammenarbeiten
  text-greek=default, % ┘
]{mhchem}

% richtige Anführungszeichen
\usepackage[autostyle]{csquotes}

% schöne Brüche im Text
\usepackage{xfrac}

% Standardplatzierung für Floats einstellen
\usepackage{float}
\floatplacement{figure}{htbp}
\floatplacement{table}{htbp}

% Floats innerhalb einer Section halten
\usepackage[
  section, % Floats innerhalb der Section halten
  below,   % unterhalb der Section aber auf der selben Seite ist ok
]{placeins}

% Seite drehen für breite Tabellen: landscape Umgebung
\usepackage{pdflscape}

% Captions schöner machen.
\usepackage[
  labelfont=bf,        % Tabelle x: Abbildung y: ist jetzt fett
  font=small,          % Schrift etwas kleiner als Dokument
  width=0.9\textwidth, % maximale Breite einer Caption schmaler
]{caption}
% subfigure, subtable, subref
\usepackage{subcaption}

% Grafiken können eingebunden werden
\usepackage{graphicx}

% schöne Tabellen
\usepackage{tabularray}
\UseTblrLibrary{booktabs, siunitx}

% Verbesserungen am Schriftbild
\usepackage{microtype}

% Literaturverzeichnis
\usepackage[
  backend=biber,
]{biblatex}
% Quellendatenbank
\addbibresource{lit.bib}
\addbibresource{programme.bib}

% Hyperlinks im Dokument
\usepackage[
  german,
  unicode,        % Unicode in PDF-Attributen erlauben
  pdfusetitle,    % Titel, Autoren und Datum als PDF-Attribute
  pdfcreator={},  % ┐ PDF-Attribute säubern
  pdfproducer={}, % ┘
]{hyperref}
% erweiterte Bookmarks im PDF
\usepackage{bookmark}

% Trennung von Wörtern mit Strichen
\usepackage[shortcuts]{extdash}

\author{%
  Vincent Wirsdörfer\\%
  \href{mailto:vincent.wirsdoerfer@udo.edu}{authorA@udo.edu}%
  \and%
  Joris Daus\\%
  \href{mailto:joris.daus@udo.edu}{authorB@udo.edu}%
}
\publishers{TU Dortmund – Fakultät Physik}


\begin{document}
\section{Auswertung}
\label{sec:Auswertung}

\subsection{Statische Methode}
\label{sec:Statische Methode}

Zu Beginn werden die Temperaturverläufe der fernen Thermoelemente graphisch aufgetragen. Hierbei bezieht sich die erste Abbildung
auf den Temperaturverlauf des breiten und schmalen Messingstabes:

\begin{figure}
  \centering
  \includegraphics[width=\textwidth]{../build/statisch_T1_T4.pdf}
  \caption{Temperaturverläufe der Messingstäbe.}
  \label{fig:statisch1}
\end{figure}

Anhand der Graphik \ref{fig:statisch1} lässt sich erkennen, dass der breite und der schmale Messingstab in den ersten 50\,$\unit{\second}$
einen ähnlich starken, exponentiellen Temperaturanstieg verzeichen können. In der Folge lässt sich bis zu einer Zeit von etwa 
200\,$\unit{\second}$ ein linearer Anstieg beider Temperaturverläufe konstatieren, wobei die Temperatur des schmalen Stabes die 
des breiten Stabes kontinuierlich überwiegt. Nach einer Zeit von 180\,$\unit{\second}$ ändert sich dieser Trend jedoch und die
Temperatur des breiten Messingstabes übersteigt die Temperatur des schmalen Stabes. Mit voranschreitender Zeit flachen jedoch beide
Verläufe ab und der Temperaturwert des breiten Stabes ist fortwährend ca. $2\,\unit{\celsius}$ über der Temperatur des schmalen Stabes.
\newpage

\begin{figure}
  \centering
  \includegraphics[width=\textwidth]{../build/statisch_T5_T8.pdf}
  \caption{Temperaturverläufe der Messingstäbe.}
  \label{fig:statisch2}
\end{figure}

hfgfgf
%\begin{table}
%  \centering
%  \caption{Eine Beispieltabelle mit Messdaten.}
%  \label{tab:tabelle}
%  \sisetup{table-format=1.1, per-mode=reciprocal}
%  \begin{tblr}{
%      colspec = {S[table-format=3.0] S[table-format=2.1] S},
%      row{1} = {guard, mode=math},
%      vline{4} = {2}{-}{text=\clap{$\pm$}},
%    }
%    \toprule
%    U \mathbin{/} \unit{\volt} & I \mathbin{/} \unit{\micro\ampere} & \SetCell[c=2]{c} N \mathbin{/} \unit{\per\second} & \\
%    \midrule
%    360 & 0.1 & 98.3 & 0.9 \\
%    400 & 0.2 & 99.8 & 1.0 \\
%    420 & 0.2 & 99.1 & 0.9 \\
%    \bottomrule
%  \end{tblr}
%\end{table}

\section{Diskussion}
Test
\section{Literatur}

\section{Anhang}
Siehe \autoref{fig:plot} und \autoref{tab:tabelle}!

\end{document}
