\documentclass[
  bibliography=totoc,     % Literatur im Inhaltsverzeichnis
  captions=tableheading,  % Tabellenüberschriften
  titlepage=firstiscover, % Titelseite ist Deckblatt
]{scrartcl}

% Paket float verbessern
\usepackage{scrhack}

% Warnung, falls nochmal kompiliert werden muss
\usepackage[aux]{rerunfilecheck}

% unverzichtbare Mathe-Befehle
\usepackage{amsmath}
% viele Mathe-Symbole
\usepackage{amssymb}
% Erweiterungen für amsmath
\usepackage{mathtools}

% Fonteinstellungen
\usepackage{fontspec}
% Latin Modern Fonts werden automatisch geladen
% Alternativ zum Beispiel:
%\setromanfont{Libertinus Serif}
%\setsansfont{Libertinus Sans}
%\setmonofont{Libertinus Mono}

% Wenn man andere Schriftarten gesetzt hat,
% sollte man das Seiten-Layout neu berechnen lassen
\recalctypearea{}

% deutsche Spracheinstellungen
\usepackage[ngerman]{babel}


\usepackage[
  math-style=ISO,    % ┐
  bold-style=ISO,    % │
  sans-style=italic, % │ ISO-Standard folgen
  nabla=upright,     % │
  partial=upright,   % │
  mathrm=sym,        % ┘
  warnings-off={           % ┐
    mathtools-colon,       % │ unnötige Warnungen ausschalten
    mathtools-overbracket, % │
  },                       % ┘
]{unicode-math}

% traditionelle Fonts für Mathematik
\setmathfont{Latin Modern Math}
% Alternativ zum Beispiel:
%\setmathfont{Libertinus Math}

\setmathfont{XITS Math}[range={scr, bfscr}]
\setmathfont{XITS Math}[range={cal, bfcal}, StylisticSet=1]

% Zahlen und Einheiten
\usepackage[
  locale=DE,                   % deutsche Einstellungen
  separate-uncertainty=true,   % immer Unsicherheit mit \pm
  per-mode=symbol-or-fraction, % / in inline math, fraction in display math
]{siunitx}

% chemische Formeln
\usepackage[
  version=4,
  math-greek=default, % ┐ mit unicode-math zusammenarbeiten
  text-greek=default, % ┘
]{mhchem}

% richtige Anführungszeichen
\usepackage[autostyle]{csquotes}

% schöne Brüche im Text
\usepackage{xfrac}

% Standardplatzierung für Floats einstellen
\usepackage{float}
\floatplacement{figure}{htbp}
\floatplacement{table}{htbp}

% Floats innerhalb einer Section halten
\usepackage[
  section, % Floats innerhalb der Section halten
  below,   % unterhalb der Section aber auf der selben Seite ist ok
]{placeins}

% Seite drehen für breite Tabellen: landscape Umgebung
\usepackage{pdflscape}

% Captions schöner machen.
\usepackage[
  labelfont=bf,        % Tabelle x: Abbildung y: ist jetzt fett
  font=small,          % Schrift etwas kleiner als Dokument
  width=0.9\textwidth, % maximale Breite einer Caption schmaler
]{caption}
% subfigure, subtable, subref
\usepackage{subcaption}

% Grafiken können eingebunden werden
\usepackage{graphicx}

% schöne Tabellen
\usepackage{tabularray}
\UseTblrLibrary{booktabs, siunitx}

% Verbesserungen am Schriftbild
\usepackage{microtype}

% Literaturverzeichnis
\usepackage[
  backend=biber,
]{biblatex}
% Quellendatenbank
\addbibresource{lit.bib}
\addbibresource{programme.bib}

% Hyperlinks im Dokument
\usepackage[
  german,
  unicode,        % Unicode in PDF-Attributen erlauben
  pdfusetitle,    % Titel, Autoren und Datum als PDF-Attribute
  pdfcreator={},  % ┐ PDF-Attribute säubern
  pdfproducer={}, % ┘
]{hyperref}
% erweiterte Bookmarks im PDF
\usepackage{bookmark}

% Trennung von Wörtern mit Strichen
\usepackage[shortcuts]{extdash}

\author{%
  Vincent Wirsdörfer\\%
  \href{mailto:vincent.wirsdoerfer@udo.edu}{authorA@udo.edu}%
  \and%
  Joris Daus\\%
  \href{mailto:joris.daus@udo.edu}{authorB@udo.edu}%
}
\publishers{TU Dortmund – Fakultät Physik}


\begin{document}
\section{Versuchsaufbau}
Das Höppler-Viskosimeter besteht aus zwei Zylindern, die mit destilliertem Wasser gefüllt sind. 
Der äußere Zylinder umschließt den inneren Zylinder. Im inneren Zylinder wird eine Kugel 
eingeschlossen. Ein beheizbares Wasserreservoir ist mit dem äußeren Zylinder verbunden. 
Der Zylinderverbund ist in einer gekippten Halterung verankert, welche über eine Libelle 
in Wage gebracht werden kann. Für eine schnellere Durchführung ist es möglich, die Zylinder um 
$\ang{180;;}$ zu drehen. Im Folgenden wird "unten" als die Seite mit der Verbindung zum Wasserreservoir 
bezeichnet. Oben ist dementsprechend die Seite ohne Wasseranschluss. 


\section{Versuchsdurchführung}
Drei Messreihen werden bei diesem Versuch aufgenommen. Eine Messreihe mit einer kleinen Kugel 
und die anderen beiden mit einer etwas größeren Kugel. Zunächst wird der innere Zylinder mit  
destilliertem Wasser gefüllt und die kleine Kugel in das Wasser gelegt. Der Zylinder wird nun 
ohne Lufteinschluss verschlossen. Es wird die Fallzeit der Kugel über $\qty{10}{\centi \meter}$
gemessen. Nach dem Fall wird die Apparatur gedreht und der Fall erneut gemessen. So werden zehn 
Messungen aufgenommen.\\
\noindent Analog werden ebenfalls zehn Messungen für die große Kugel protokolliert. \\
\noindent Temperaturabhängige Viskositäten können durch Erhöhung der Wassertemperatur beobachtet werden. 
Dazu wird das Wasser zwischen dem inneren und äußeren Zylinder erhitzt. Dies gibt dadurch Wärme 
an den inneren Zylinder ab, wodurch die Viskosität des Wassers, durch das die Kugel fällt, beeinflusst wird.
Das Wasserreservoir ist mit einem Heizstab, Thermometer und Pumpe ausgestattet. So zirkuliert das Wasser 
zwischen den beiden Zylindern und dem Reservoir.\\
Über ein Temperaturspektrum von $\qty{20}{\celsius}$ bis $\qty{52}{\celsius}$ werden elf Temperaturen vermessen. 
Das Wasserreservoir wird so lange beheizt, bis die gewünschte Temperatur erreicht ist. Anschließend wird eine 
Minute gewartet, sodass das Wasser im inneren Zylinder Zeit hat, sich zu erwärmen. Daraufhin werden pro Temperatur 
vier Fallzeiten, wie bereits beschrieben, vermessen.

------------------Beschleunigungsphase------------------------------------

\section{Messwerte}
Beide Kugeln sehen auf den ersten Blick ähnlich groß aus. Sie werden, auch aufgrund weiterer Rechnungen zunächst 
vermessen. So ergeben sich für die Durchmesser:

\begin{align*}
    D_\text{groß} = \qty{15.760 +- 0.005}{\milli \meter} \\
    D_\text{klein} = \qty{15.590 +- 0.005}{\milli \meter}
\end{align*}

\noindent
Die Massen der Kugeln waren mit

\begin{align*}
    m_\text{groß} = \qty{4.9521}{\gram} \\
    m_\text{klein} = \qty{4.4531}{\gram}
\end{align*}

\noindent
gegeben.
Im Raum herrscht eine Temperatur von $\qty{20}{\celsius}$. 

Bei den Messungen wird zwischen dem Fall von oben nach unten und anders herum unterschieden. Dies dient dazu, 
mögliche systematische Fehler zu identifizieren. Der Fall von Oben nach unten wird daher mit "$O \rightarrow U$" 
abgekürzt.
Die gemessene Zeit, die die kleine Kugel für eine Fallstrecke von $\qty{10}{\centi \meter}$ benötigt, wird in der 
folgenden Tabelle aufgelistet.
\begin{table}
    \centering
    \sisetup{ per-mode=reciprocal, table-format=2.2}
    \begin{tblr}{
        colspec = {S S},
        row{1} = {guard, mode=math},
        }
        \toprule
        O \rightarrow U \mathbin{/} \unit{\second} & 
        U \rightarrow O \mathbin{/} \unit{\second} \\
        \midrule
        12.23     &   12.73   \\
        12.35     &   12.42   \\
        12.23     &   12.62   \\
        12.30     &   12.55   \\
        12.89     &   12.15   \\
        12.33     &   12.54   \\
        12.70     &   12.55   \\
        12.56     &   12.84   \\
        12.57     &   12.70   \\
        12.46     &   12.61   \\ 
        \bottomrule
    \end{tblr}
    \caption{Fallzeit der kleinen Kugel bei Raumtemperatur.}
    \label{tab:klein}
  \end{table}

\noindent
Um Zeiteffizient die Fallzeit der großen Kugel zu messen, wird die Fallstrecke auf $\qty{5}{\centi \meter}$ halbiert. 
So können zwei Messungen direkt hintereinander ohne neue Beschleunigungsphase erfolgen. Der Zylinderverbund muss daher 
nur alles zwei Messungen gedreht werden. Folgende Fallzeiten wurden gemessen: %Formulierung scheiße vom letzten Satz

\begin{table}
    \centering
    \sisetup{ per-mode=reciprocal, table-format=2.2}
    \begin{tblr}{
        colspec = {S S S S},
        row{1} = {guard, mode=math},
        }
        \toprule
        \SetCell[c=2]{c} O \rightarrow U \mathbin{/} \unit{\second} & &
        \SetCell[c=2]{c} U \rightarrow O \mathbin{/} \unit{\second} \\
        \midrule
        47.59   &   46.72   &   45.97   &   47.39   \\
        46.50   &   46.50   &   45.31   &   47.50   \\
        46.61   &   45.98   &   45.77   &   47.15   \\
        46.91   &   45.83   &   46.53   &   47.47   \\
        46.70   &   46.71   &   46.01   &   46.71   \\
        \bottomrule
    \end{tblr}
    \caption{Fallzeit der großen Kugel bei Raumtemperatur.}
    \label{tab:groß_20}
  \end{table}




\end{document}

