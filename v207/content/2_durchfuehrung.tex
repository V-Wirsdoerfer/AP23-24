\input{../../header.tex}

\begin{document}
\section{Versuchsaufbau}
Das Höppler-Viskosimeter besteht aus zwei Zylinder, die mit destilliertem Wasser gefüllt sind. 
Der äußere Zylinder umschließt den inneren Zylinder. Im inneren Zylinder wird eine Kugel 
eingeschlossen. Der äußere Zylinder ist mit einem beheizbaren Wasserreservoir verbunden. 
Der Zylinderverbund ist in einer gekippten Halterung verankert, welche kann über eine Libelle 
in Wage gebracht werden. Für eine schnellere Durchführung ist es möglich, die Zylinder um 
$\ang{180;;}$ zu drehen.


\section{Versuchsdurchführung}
Drei Messreihen werden bei diesem Versuch aufgenommen. Eine Messreihe mit einer kleinen Kugel 
und die anderen beiden mit einer etwas größeren Kugel. Zunächst wird der innere Zylinder mit  
destilliertem Wasser gefüllt und die kleine Kugel in das Wasser gelegt. Der Zylinder wird nun 
ohne Lufteinschluss verschlossen. Es wird die Fallzeit der Kugel über $\qty{10}{\centi \meter}$
gemessen. Nach dem Fall wird die Apparatur gedreht und der Fall erneut gemessen. So werden zehn 
Messungen aufgenommen.\\
\noindent Analog werden ebenfalls zehn Messungen für die große Kugel protokolliert. \\
\noindent Temperaturabhängige Viskositäten können durch Erhöhung der Wassertemperatur beobachtet werden. 
Dazu wird das Wasser zwischen dem inneren und äußeren Zylinder erhitzt. Dies gibt dadurch Wärme 
an den inneren Zylinder ab, wodurch die Viskosität des Wassers, durch das die Kugel fällt, beeinflusst wird.
Das Wasserreservoir ist mit einem Heizstab, Thermometer und Pumpe ausgestattet. So zirkuliert das Wasser 
zwischen den beiden Zylindern. 




\section{Messwerte}

\end{document}

