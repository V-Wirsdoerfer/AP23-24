%\input{../../header.tex}

%\begin{document}
\section{Versuchsaufbau}
Das Höppler-Viskosimeter besteht aus zwei Zylindern, die mit destilliertem Wasser gefüllt sind. 
Der äußere Zylinder umschließt den inneren Zylinder. Im inneren Zylinder wird eine Kugel 
eingeschlossen. Ein beheizbares Wasserreservoir ist mit dem äußeren Zylinder verbunden. 
Der Zylinderverbund ist in einer gekippten Halterung verankert, welche über eine Libelle 
in Wage gebracht werden kann. Für eine schnellere Durchführung ist es möglich, die Zylinder um 
$\ang{180;;}$ zu drehen. Im Folgenden wird "unten" als die Seite mit der Verbindung zum Wasserreservoir 
bezeichnet. Oben ist dementsprechend die Seite ohne Wasseranschluss. 


\section{Versuchsdurchführung}
Drei Messreihen werden bei diesem Versuch aufgenommen. Eine Messreihe mit einer kleinen Kugel 
und die anderen beiden mit einer etwas größeren Kugel. Zunächst wird der innere Zylinder mit  
destilliertem Wasser gefüllt und die kleine Kugel in das Wasser gelegt. Der Zylinder wird nun 
ohne Lufteinschluss verschlossen. Es wird die Fallzeit der Kugel über $\qty{10}{\centi \meter}$
gemessen. Nach dem Fall wird die Apparatur gedreht und der Fall erneut gemessen. So werden zehn 
Messungen aufgenommen.\\
\noindent Analog werden ebenfalls zehn Messungen für die große Kugel protokolliert. \\
\noindent Temperaturabhängige Viskositäten können durch Erhöhung der Wassertemperatur beobachtet werden. 
Dazu wird das Wasser zwischen dem inneren und äußeren Zylinder erhitzt. Dies gibt dadurch Wärme 
an den inneren Zylinder ab, wodurch die Viskosität des Wassers, durch das die Kugel fällt, beeinflusst wird.
Das Wasserreservoir ist mit einem Heizstab, Thermometer und Pumpe ausgestattet. So zirkuliert das Wasser 
zwischen den beiden Zylindern und dem Reservoir.\\
Über ein Temperaturspektrum von $\qty{20}{\celsius}$ bis $\qty{52}{\celsius}$ werden elf Temperaturen vermessen. 
Das Wasserreservoir wird so lange beheizt, bis die gewünschte Temperatur erreicht ist. Anschließend wird eine 
Minute gewartet, sodass das Wasser im inneren Zylinder Zeit hat, sich zu erwärmen. Daraufhin werden pro Temperatur 
vier Fallzeiten, wie bereits beschrieben, vermessen.
Damit Fallzeiten möglichst repetitiv sind, wird eine konstante Geschwindigkeit benötigt. Die Beschleunigungsphase wird 
nach Möglichkeit so nicht mitgemessen, welches durch ein möglichst spätes Starten der Messung versucht wird zu erreichen.

\noindent

  \section{Fehlerrechnung}
  \label{sec:Fehlerrechnung}
  
  Alle im Protokoll vermerkten Mittelwerte lassen sich über die folgende Formel berechnen:
  
  \begin{equation}
  \label{eqn:Mittelwert}
      \bar{x} = \frac{1}{N}\sum_{i=1}^N x_i
  \end{equation}
  
  \noindent
  Zudem lässt sich der dazugehörige Fehler des Mittelwerts wie folgt berechnen:
  
  \begin{equation}
  \label{eqn:Mittelwertfehler}
      \increment \bar{x} = \sqrt{\frac{1}{N\left(N-1\right)}\sum_{i=1}^N \left(x_i - \bar{x}\right)²}
  \end{equation}
  
  \noindent
  Entsteht ein neuer Fehler durch bereits fehlerbehaftete Größen, so wird die Gauß'sche Fehlerfortpflanzung angewendet:
  
  \begin{equation}
  \label{eqn:Fehlerfortpflanzung}
      \increment f = \sqrt{\sum_{i=1}^N \left(\frac{\partial f}{\partial x_i}\right)²\cdot\left(\increment x_i\right)²}
  \end{equation}
  
  \section{Vorbereitung}
  Eine Strömung wird als linear bezeichnet, wenn keine Wirbel entstehen. Die Geschwindigkeit nimmt von der Mitte bis zum Rand linear ab.
  Destilliertes Wasser besitzt ein dynamische Viskosität von $\eta = 1.005\, \unit{\milli \pascal \second}$ bei $20\, \unit{\celsius}$ 
  \cite{Physikalisches_Praktikum} Seite 290. Die Dichte bei $20\, \unit{\celsius}$ beträgt $998.21\, \unit{\kilo \gram \per \cubic \meter}$
  \cite{Physikalisches_Praktikum}.\\
  \noindent
  Die Reynoldszahl ist abhängig von der Viskosität des Mediums, der Strömungsgeschwindigkeit $V$, der Dichte $\rho$ des Fluids und der 
  charakteristischen Länge $L$ eines Körpers. So ergibt sich die Formel 
  
  \begin{equation}
      Re = \frac{\rho V L}{\eta}
      \label{eqn:Reynolds}
  \end{equation}
  für die Reynoldszahl. Eine Strömung wird über einen Grenzbereich von $Re \approx 2300$ \cite{Reynolds} nicht mehr als laminar bezeichnet.

%\end{document}

