%\input{../../header.tex}

%\begin{document}
\section{Zielsetzung}
Im folgend protokollierten Versuch, soll zunächst überprüft werden, ob die Strömung im Viskosimeter laminar ist. Dies geschieht 
über die Reynoldsche Zahl.
Anschließend soll die temperaturabhängige Viskosität von destilliertem Wasser experimentell ermittelt werden. Diese soll als 
Funktion der Temperatur bestimmt werden.

\section{Theorie}
\label{sec:Theorie}
Ein Körper kann über Reibung mit der Umwelt wechselwirken. Es gibt verschiedene Arten von Reibungen. Im Folgenden wird sich auf die 
Stokessche Reibung beschränkt. Diese beschreibt die Reibung in Fluiden für geringe Geschwindigkeiten. Die Stokessche Reibungskraft 
$F_R$ setzt kugelförmige Körper voraus.
Die Reibung ist eine Kraft, welche von der Geschwindigkeit $v$ und dem Radius $r$ des Körpers, sowie von der Viskosität $\eta$ des 
Fluids um den Körper abhängig ist. So ergibt sich

\begin{equation*}
    F_R = 6 \pi \eta v r.
\end{equation*}

\noindent
Die Viskosität $\eta$ ist allerdings keine Konstante. Vielmehr ist sie von der Temperatur abhängig. Die \emph{Andradesche Gleichung} 
gibt diese Temperaturabhängigkeit durch 

\begin{equation}
\label{eqn:Andra}
    \eta(T) = A \exp\left({\frac{B}{T}}\right)
\end{equation}

\noindent
wider.
$A$ und $B$ sind Konstanten, die in der Auswertung ermittelt werden.\\

\noindent
Das Experiment, welches benutzt wird, um die beschriebenen Zusammenhänge zu untersuchen, ist das \emph{Kugelfall-Viskosimeter nach Höppler}.
In diesem Experiment wird eine Kugel in einem mit Wasser gefüllten Glasrohr fallen gelassen. Das Glasrohr ist dabei nur wenig breiter als 
die Kugel, damit nur laminare Strömungen und keine Verwirbelungen entstehen.\\
Auf die Kugel wirken neben der Gewichtskraft $F_G = m g$ und die eben beschriebene Reibung auch die Auftriebskraft 

\begin{equation*}
    F_A = g \rho V.
\end{equation*}

\noindent
Dabei beschreibt $\rho$ die Dichte des Körpers und $V$ das Volumen des Körpers.\\
\noindent
Wenn die Kugel fallen gelassen wird, beschleunigt sie, bis die Summe aus Reibungskraft und Auftriebskraft so groß wie die Gewichtskraft 
ist. Es stellt sich also das Kräftegleichgewicht $F_R + F_A = F_G$ ein.\\
\noindent
Um die Viskosität bestimmen zu können, misst man die Fallzeit.

\begin{equation}
    \eta = K(\rho_\text{Kugel} - \rho_\text{Fluid}) \cdot t
    \label{eqn:eta_Zeit}
\end{equation}

\noindent
Gleichung \eqref{eqn:eta_Zeit} ist Konstruktionsabhängig. Das bedeutet, dass $K$ eine Apparaturkonstante ist. Diese enthält die Fallhöhe 
und Kugelmaße.\\

\noindent
Um Verwirbelungen des Fluids zu verhindern, wird die Versuchsapparatur leicht gekippt. Dies hat zu Folge, dass die Kugel immer auf einer 
Seite des Rohres ist und das Fluid immer an den anderen Seiten vorbeiströmt. So wird versucht laminare Strömungen zu erhalten.

%\end{document}