%\documentclass[
  bibliography=totoc,     % Literatur im Inhaltsverzeichnis
  captions=tableheading,  % Tabellenüberschriften
  titlepage=firstiscover, % Titelseite ist Deckblatt
]{scrartcl}

% Paket float verbessern
\usepackage{scrhack}

% Warnung, falls nochmal kompiliert werden muss
\usepackage[aux]{rerunfilecheck}

% unverzichtbare Mathe-Befehle
\usepackage{amsmath}
% viele Mathe-Symbole
\usepackage{amssymb}
% Erweiterungen für amsmath
\usepackage{mathtools}

% Fonteinstellungen
\usepackage{fontspec}
% Latin Modern Fonts werden automatisch geladen
% Alternativ zum Beispiel:
%\setromanfont{Libertinus Serif}
%\setsansfont{Libertinus Sans}
%\setmonofont{Libertinus Mono}

% Wenn man andere Schriftarten gesetzt hat,
% sollte man das Seiten-Layout neu berechnen lassen
\recalctypearea{}

% deutsche Spracheinstellungen
\usepackage[ngerman]{babel}


\usepackage[
  math-style=ISO,    % ┐
  bold-style=ISO,    % │
  sans-style=italic, % │ ISO-Standard folgen
  nabla=upright,     % │
  partial=upright,   % │
  mathrm=sym,        % ┘
  warnings-off={           % ┐
    mathtools-colon,       % │ unnötige Warnungen ausschalten
    mathtools-overbracket, % │
  },                       % ┘
]{unicode-math}

% traditionelle Fonts für Mathematik
\setmathfont{Latin Modern Math}
% Alternativ zum Beispiel:
%\setmathfont{Libertinus Math}

\setmathfont{XITS Math}[range={scr, bfscr}]
\setmathfont{XITS Math}[range={cal, bfcal}, StylisticSet=1]

% Zahlen und Einheiten
\usepackage[
  locale=DE,                   % deutsche Einstellungen
  separate-uncertainty=true,   % immer Unsicherheit mit \pm
  per-mode=symbol-or-fraction, % / in inline math, fraction in display math
]{siunitx}

% chemische Formeln
\usepackage[
  version=4,
  math-greek=default, % ┐ mit unicode-math zusammenarbeiten
  text-greek=default, % ┘
]{mhchem}

% richtige Anführungszeichen
\usepackage[autostyle]{csquotes}

% schöne Brüche im Text
\usepackage{xfrac}

% Standardplatzierung für Floats einstellen
\usepackage{float}
\floatplacement{figure}{htbp}
\floatplacement{table}{htbp}

% Floats innerhalb einer Section halten
\usepackage[
  section, % Floats innerhalb der Section halten
  below,   % unterhalb der Section aber auf der selben Seite ist ok
]{placeins}

% Seite drehen für breite Tabellen: landscape Umgebung
\usepackage{pdflscape}

% Captions schöner machen.
\usepackage[
  labelfont=bf,        % Tabelle x: Abbildung y: ist jetzt fett
  font=small,          % Schrift etwas kleiner als Dokument
  width=0.9\textwidth, % maximale Breite einer Caption schmaler
]{caption}
% subfigure, subtable, subref
\usepackage{subcaption}

% Grafiken können eingebunden werden
\usepackage{graphicx}

% schöne Tabellen
\usepackage{tabularray}
\UseTblrLibrary{booktabs, siunitx}

% Verbesserungen am Schriftbild
\usepackage{microtype}

% Literaturverzeichnis
\usepackage[
  backend=biber,
]{biblatex}
% Quellendatenbank
\addbibresource{lit.bib}
\addbibresource{programme.bib}

% Hyperlinks im Dokument
\usepackage[
  german,
  unicode,        % Unicode in PDF-Attributen erlauben
  pdfusetitle,    % Titel, Autoren und Datum als PDF-Attribute
  pdfcreator={},  % ┐ PDF-Attribute säubern
  pdfproducer={}, % ┘
]{hyperref}
% erweiterte Bookmarks im PDF
\usepackage{bookmark}

% Trennung von Wörtern mit Strichen
\usepackage[shortcuts]{extdash}

\author{%
  Vincent Wirsdörfer\\%
  \href{mailto:vincent.wirsdoerfer@udo.edu}{authorA@udo.edu}%
  \and%
  Joris Daus\\%
  \href{mailto:joris.daus@udo.edu}{authorB@udo.edu}%
}
\publishers{TU Dortmund – Fakultät Physik}


%\begin{document}
\section{Zielsetzung}
Im folgend protokollierten Versuch, soll zunächst überprüft werden, ob die Strömung im Viskosimeter laminar ist. Dies geschieht 
über die Reynoldsche Zahl.
Anschließend soll die temperaturabhängige Viskosität von destilliertem Wasser experimentell ermittelt werden. Diese soll als 
Funktion der Temperatur bestimmt werden.

\section{Theorie}
\label{sec:Theorie}
Ein Körper kann über Reibung mit der Umwelt wechselwirken. Es gibt verschiedene Arten von Reibungen. Im Folgenden wird sich auf die 
Stokessche Reibung beschränkt. Diese beschreibt die Reibung in Fluiden für geringe Geschwindigkeiten. Die Stokessche Reibungskraft 
$F_R$ setzt kugelförmige Körper voraus.
Die Reibung ist eine Kraft, welche von der Geschwindigkeit $v$ und dem Radius $r$ des Körpers, sowie von der Viskosität $\eta$ des 
Fluids um den Körper abhängig ist. So ergibt sich

\begin{equation*}
    F_R = 6 \pi \eta v r.
\end{equation*}

\noindent
Die Viskosität $\eta$ ist allerdings keine Konstante. Vielmehr ist sie von der Temperatur abhängig. Die \emph{Andradesche Gleichung} 
gibt diese Temperaturabhängigkeit durch 

\begin{equation}
\label{eqn:Andra}
    \eta(T) = A \exp\left({\frac{B}{T}}\right)
\end{equation}

\noindent
wider.
$A$ und $B$ sind Konstanten, die in der Auswertung ermittelt werden.\\

\noindent
Das Experiment, welches benutzt wird, um die beschriebenen Zusammenhänge zu untersuchen, ist das \emph{Kugelfall-Viskosimeter nach Höppler}.
In diesem Experiment wird eine Kugel in einem mit Wasser gefüllten Glasrohr fallen gelassen. Das Glasrohr ist dabei nur wenig breiter als 
die Kugel, damit nur laminare Strömungen und keine Verwirbelungen entstehen.\\
Auf die Kugel wirken neben der Gewichtskraft $F_G = m g$ und die eben beschriebene Reibung auch die Auftriebskraft 

\begin{equation*}
    F_A = g \rho V.
\end{equation*}

\noindent
Dabei beschreibt $\rho$ die Dichte des Körpers und $V$ das Volumen des Körpers.\\
\noindent
Wenn die Kugel fallen gelassen wird, beschleunigt sie, bis die Summe aus Reibungskraft und Auftriebskraft so groß wie die Gewichtskraft 
ist. Es stellt sich also das Kräftegleichgewicht $F_R + F_A = F_G$ ein.\\
\noindent
Um die Viskosität bestimmen zu können, misst man die Fallzeit.

\begin{equation}
    \eta = K(\rho_\text{Kugel} - \rho_\text{Fluid}) \cdot t
    \label{eqn:eta_Zeit}
\end{equation}

\noindent
Gleichung \eqref{eqn:eta_Zeit} ist Konstruktionsabhängig. Das bedeutet, dass $K$ eine Apparaturkonstante ist. Diese enthält die Fallhöhe 
und Kugelmaße.\\

\noindent
Um Verwirbelungen des Fluids zu verhindern, wird die Versuchsapparatur leicht gekippt. Dies hat zu Folge, dass die Kugel immer auf einer 
Seite des Rohres ist und das Fluid immer an den anderen Seiten vorbeiströmt. So wird versucht laminare Strömungen zu erhalten.

%\end{document}