\documentclass[
  bibliography=totoc,     % Literatur im Inhaltsverzeichnis
  captions=tableheading,  % Tabellenüberschriften
  titlepage=firstiscover, % Titelseite ist Deckblatt
]{scrartcl}

% Paket float verbessern
\usepackage{scrhack}

% Warnung, falls nochmal kompiliert werden muss
\usepackage[aux]{rerunfilecheck}

% unverzichtbare Mathe-Befehle
\usepackage{amsmath}
% viele Mathe-Symbole
\usepackage{amssymb}
% Erweiterungen für amsmath
\usepackage{mathtools}

% Fonteinstellungen
\usepackage{fontspec}
% Latin Modern Fonts werden automatisch geladen
% Alternativ zum Beispiel:
%\setromanfont{Libertinus Serif}
%\setsansfont{Libertinus Sans}
%\setmonofont{Libertinus Mono}

% Wenn man andere Schriftarten gesetzt hat,
% sollte man das Seiten-Layout neu berechnen lassen
\recalctypearea{}

% deutsche Spracheinstellungen
\usepackage[ngerman]{babel}


\usepackage[
  math-style=ISO,    % ┐
  bold-style=ISO,    % │
  sans-style=italic, % │ ISO-Standard folgen
  nabla=upright,     % │
  partial=upright,   % │
  mathrm=sym,        % ┘
  warnings-off={           % ┐
    mathtools-colon,       % │ unnötige Warnungen ausschalten
    mathtools-overbracket, % │
  },                       % ┘
]{unicode-math}

% traditionelle Fonts für Mathematik
\setmathfont{Latin Modern Math}
% Alternativ zum Beispiel:
%\setmathfont{Libertinus Math}

\setmathfont{XITS Math}[range={scr, bfscr}]
\setmathfont{XITS Math}[range={cal, bfcal}, StylisticSet=1]

% Zahlen und Einheiten
\usepackage[
  locale=DE,                   % deutsche Einstellungen
  separate-uncertainty=true,   % immer Unsicherheit mit \pm
  per-mode=symbol-or-fraction, % / in inline math, fraction in display math
]{siunitx}

% chemische Formeln
\usepackage[
  version=4,
  math-greek=default, % ┐ mit unicode-math zusammenarbeiten
  text-greek=default, % ┘
]{mhchem}

% richtige Anführungszeichen
\usepackage[autostyle]{csquotes}

% schöne Brüche im Text
\usepackage{xfrac}

% Standardplatzierung für Floats einstellen
\usepackage{float}
\floatplacement{figure}{htbp}
\floatplacement{table}{htbp}

% Floats innerhalb einer Section halten
\usepackage[
  section, % Floats innerhalb der Section halten
  below,   % unterhalb der Section aber auf der selben Seite ist ok
]{placeins}

% Seite drehen für breite Tabellen: landscape Umgebung
\usepackage{pdflscape}

% Captions schöner machen.
\usepackage[
  labelfont=bf,        % Tabelle x: Abbildung y: ist jetzt fett
  font=small,          % Schrift etwas kleiner als Dokument
  width=0.9\textwidth, % maximale Breite einer Caption schmaler
]{caption}
% subfigure, subtable, subref
\usepackage{subcaption}

% Grafiken können eingebunden werden
\usepackage{graphicx}

% schöne Tabellen
\usepackage{tabularray}
\UseTblrLibrary{booktabs, siunitx}

% Verbesserungen am Schriftbild
\usepackage{microtype}

% Literaturverzeichnis
\usepackage[
  backend=biber,
]{biblatex}
% Quellendatenbank
\addbibresource{lit.bib}
\addbibresource{programme.bib}

% Hyperlinks im Dokument
\usepackage[
  german,
  unicode,        % Unicode in PDF-Attributen erlauben
  pdfusetitle,    % Titel, Autoren und Datum als PDF-Attribute
  pdfcreator={},  % ┐ PDF-Attribute säubern
  pdfproducer={}, % ┘
]{hyperref}
% erweiterte Bookmarks im PDF
\usepackage{bookmark}

% Trennung von Wörtern mit Strichen
\usepackage[shortcuts]{extdash}

\author{%
  Vincent Wirsdörfer\\%
  \href{mailto:vincent.wirsdoerfer@udo.edu}{authorA@udo.edu}%
  \and%
  Joris Daus\\%
  \href{mailto:joris.daus@udo.edu}{authorB@udo.edu}%
}
\publishers{TU Dortmund – Fakultät Physik}


\begin{document}
\section{Auswertung}
\label{sec:Auswertung}

Bevor die Fallzeiten der Kugeln im Viskosimeter ausgewertet werden, müssen zunächst die Dichten der Kugeln 
bestimmt werden. Diese berechnen sich aus dem Verhältnis der Masse $m$ und dem Volumen $V$ der Kugeln:

\begin{equation*}
    \rho = \frac{m}{V} = \frac{m}{\left(\sfrac{4}{3}\right)\pi\left(\sfrac{d}{2}\right)³}
\end{equation*}

\noindent Die dafür notwendigen Kenngrößen der Kugeln werden dem vorherigen Kapital \ref{sec:Messwerte} entnommen. Es 
ergeben sich somit die Werte

\begin{gather*}
    \rho_\text{Gr} = 2.0802\,\unit{\gram\per\centi\cubic\meter} \\
    \rho_\text{Kl} = 2.4165\,\unit{\gram\per\centi\cubic\meter}
\end{gather*}

\noindent als Dichten der jeweiligen Kugeln.

\subsection{Statische Methode}

Mit der Intention die dynamische Viskosität $\eta$ zu berechnen, werden zunächst die Fallzeiten der kleinen Kugel 
und die dazugehörigen Viskositäten grafisch dargestellt.

\begin{figure}[H]
    \centering
    \includegraphics[height=7cm]{../build/klein_20C.pdf}
    \caption{Fallzeiten der kleinen Kugel.}
\end{figure}

\noindent Aufgrund der Tatsache, dass die Apparaturkonstante der kleinen Kugel mit einem Wert von
$K_\text{L} = 0.07640\,\unit{\milli\pascal\centi\cubic\meter\per\gram}$ bereits als gegeben vorausgestzt werden darf, 
befindet sich ein Gleichung \eqref{eqn:eta_Zeit} noch eine Unbekannte, die Viskosität. Unter Konsultieren der
gemittelten Fallzeiten, sowie der Dichtewerte der kleinen Kugel und des Fluids kann durch \emph{python} die Viskosität 
berechnet werden:

\begin{equation}
\label{eqn:Viskositaet}
    \eta = (1.356 \pm 0.005)\,\unit{\gram\per\centi\meter\per\second}
\end{equation}

\noindent Im Anschluss kann somit ebenfalls die Apparaturkonstante der großen Kugel berechnet werden. Dafür wird zunächst die Menge 
aller Wertepaare \{$t,\eta(t)$\} aus Tabelle \ref{tab:gross_20} visualisiert.

\begin{figure}[H]
    \centering
    \includegraphics[height=7cm]{../build/gross_20C.pdf}
    \caption{Fallzeiten der großen Kugel.}
\end{figure}

\noindent Die soeben ermittelte Viskosität wird nun dazu verwendet, die Apparaturkonstante $K_\text{Gr}$ der großen Kugel zu 
bestimmen. Durch triviales Umstellen der Gleichung \eqref{eqn:eta_Zeit}, sowie die Mittlung der aufgenommenen Fallzeiten wird 
der Wert 

\begin{equation*}
    K_\text{Gr} = \left(002691 \pm 0.00008\right)\,\unit{\milli\pascal\centi\cubic\meter\per\gram}
\end{equation*}

\noindent als Apparaturkonstante der großen Kugel berechnet.

\subsection{Dynamische Methode}

Im Folgenden wird die Temperaturabhängigkeit der dynamischen Viskosität genauer betrachtet, indem die Fallzeiten 
der großen Kugel für eld verschiedene Temperaturen gemessen werden. Dabei wird der konkrete mathematiche Zusammenhang 
durch die graphische Darstellung von $\ln(\eta)$ als Funktion von $\sfrac{1}{T}$ verdeutlicht. Darüberhinaus wird die 
bereits erwähnte Andradesche Gleichung \eqref{eqn:Andra} mittels der Funktion \texttt{polyfit} an die geplotteten Messwerte 
angelegt.

\begin{figure}[H]
    \centering
    \includegraphics[height=7cm]{../build/dynamisch.pdf}
    \caption{Fallzeiten der großen Kugel bei verschiedenen Temperaturen.}
\end{figure}

\noindent Ausgegeben werden somit die Parameter $A$ und $B$ der Andradeschen Gleichung. Diesbezüglich werden durch 
python folgende Werte des \texttt{polyfits} berechnet:

\begin{align*}
    A &= 0.486 \pm 0.020 \\
    B &= 24.3 \pm 1.4
\end{align*}

\subsection{Reynoldszahl}

Im letzten Schritt der Auswertung wird geprüft, ob es sich bei dem betrachteten Versuch, um eine laminare Strömung handelt.
Wie bereits in der Theorie \ref{sec:Theorie} erwähnt, wird für diese Beurteilung die Reynoldszahl berechnet. Hierbei existiert
eine kritische Zahl $R_\text{crit}$, welche bei der vorliegenden Versuchskonstruktion einen Wert von $R_\text{crit} \approx 2300$
besitzt \cite{Physikalisches_Praktikum}. Ein Abgleich der mittels Gleichung \eqref{eqn:Reynolds} berechneten, tatsächlichen Reynoldszahl mit dem kritischen
Wert ermöglicht eine Urteilsbildung über die Art der Strömung. Sofern der berechnete Wert deutlich unter der kritischen Zahl liegt,
kann die Strömung als laminar bezeichnet werden. In diesem Fall werden die Reynoldszahlen der statischen und dynamischen 
Methode mittles der zugrundeliegenden Daten ermittelt:

\begin{table}
    \centering
    \sisetup{ per-mode=reciprocal, table-format=2.2}
    \begin{tblr}{
        colspec = {S S S S S S S},
        row{1} = {guard, mode=math},
        }
        \toprule
        \phantom & \rho\mathbin{/}\,\unit{\gram\per\centi\cubic\meter} &
         s\mathbin{/}\,\unit{\centi\meter} & 
         t\mathbin{/}\,\unit{\second} & 
         L\mathbin{/}\,\unit{\centi\meter} & 
         \eta\mathbin{/}\,\unit{\gram\per\centi\meter\per\second} & 
         Re \\
        \midrule
        1   & 0.99821 & 10 & 12.52 & 1.559 & 1.356 & 0.917 \\
        1   & 0.99821 & 5 & 46.59 & 1.576 & 1.356 & 0.1245 \\
        1   & 0.00821 & 5 & 25.35 & 1.576 & 0.738 & 0.421 \\
        \bottomrule 
    \end{tblr}
    \caption{Daten zur Bestimmung der Reynoldszahlen.}
\end{table}     

\end{document}
