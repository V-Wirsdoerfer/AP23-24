\input{../../header.tex}

\begin{document}
\section{Zielsetzung}
\label{sec:Theorie}
In diesem Experiment soll das Hooksche Gesetz experimentell nachgewiesen werden.

\section{Theorie}
Das Hooksche Gesetz besagt, dass die Kraft $F$, die es braucht um eine Feder auszulenken, proportional zur ausgelenkten Strecke $\increment x$ ist
und über die Federkonstante $D$ verknüpft ist.
So ergibt sich die Formel:
\begin{equation}
    F = D \cdot \increment x
    \label{eq:Hook}
\end{equation}

%\section{Vorbereitung}
%\cite{sample}

\end{document}