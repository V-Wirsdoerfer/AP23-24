\input{../../header.tex}

\begin{document}
\section{Versuchsaufbau}
Der Versuch findet digital auf der Seite der Universität Duisburg-Essen 
\cite{Hook_Interaktiv} statt. \
Um die Federkonstante zu bestimmen, ist ein Kraftmesser über eine Feder mit einer Auslenkvorrichtung verbunden.
Der digitale Kraftmesser ist zwischen einem Stativ und der Feder verbunden. Die Feder hängt senkrecht unter dem Kraftmesser.
Um die Feder auslenken zu können, ist diese über ein Seil, welches über eine Umlenkrolle geführt wird, mit einer 
Spannvorrichtung gekoppelt. Die Spanvorrichtung besteht aus einem $67 \unit{\centi \meter}$ langen Lineal und einem Pfeil, 
der anzeigt, wie weit die Feder ausgelenkt wird. Das lineal ist wagerecht mit dem Stativ verbunden. Der Pfeil lässt sich online 
mit dem Mauszeiger auf $0.5 \unit{\centi \meter}$ genau verschieben. %komma oder Punkt?
Die an der Feder wirkende Kraft wird auf einem mit dem Kräftemesser verbundenen Computer angezeigt. Der Kräftemesser 
gibt sein Signal in elektronisches Gerät, welches nicht näher bekannt ist. Von diesem Gerät geht das Signal dann in den 
Computer. Das elektronische Gerät und der Computer stehen neben dem Stativ mit der Feder. Die Kraftskala auf dem Laptop geht 
von $0 \unit{\Newton}$ bis $5 \unit{\Newton}$ und kann bis auf $0.01 \unit{\Newton}$ genau abgelesen werden
%evtl was zu Rahmenbedingungen also blaue Unterlage und Hintergrund?


\section{Versuchsdurchführung}
\label{sec:Versuhsdurchfuehrung}
Auf der beschriebenen Internetseite wird mit der Maus der Pfeil am lineal verschoben. Durch dieses verschieben wird die Feder 
gespannt und ausgelenkt. Die Umlenkrolle ändert ihre Stellung. Die Kraft, die auf die Feder wirkt, wird auf dem Laptop 
abgelesen.

\section{Messwerte}



\end{document}

