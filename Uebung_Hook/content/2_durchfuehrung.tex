%\documentclass[
  bibliography=totoc,     % Literatur im Inhaltsverzeichnis
  captions=tableheading,  % Tabellenüberschriften
  titlepage=firstiscover, % Titelseite ist Deckblatt
]{scrartcl}

% Paket float verbessern
\usepackage{scrhack}

% Warnung, falls nochmal kompiliert werden muss
\usepackage[aux]{rerunfilecheck}

% unverzichtbare Mathe-Befehle
\usepackage{amsmath}
% viele Mathe-Symbole
\usepackage{amssymb}
% Erweiterungen für amsmath
\usepackage{mathtools}

% Fonteinstellungen
\usepackage{fontspec}
% Latin Modern Fonts werden automatisch geladen
% Alternativ zum Beispiel:
%\setromanfont{Libertinus Serif}
%\setsansfont{Libertinus Sans}
%\setmonofont{Libertinus Mono}

% Wenn man andere Schriftarten gesetzt hat,
% sollte man das Seiten-Layout neu berechnen lassen
\recalctypearea{}

% deutsche Spracheinstellungen
\usepackage[ngerman]{babel}


\usepackage[
  math-style=ISO,    % ┐
  bold-style=ISO,    % │
  sans-style=italic, % │ ISO-Standard folgen
  nabla=upright,     % │
  partial=upright,   % │
  mathrm=sym,        % ┘
  warnings-off={           % ┐
    mathtools-colon,       % │ unnötige Warnungen ausschalten
    mathtools-overbracket, % │
  },                       % ┘
]{unicode-math}

% traditionelle Fonts für Mathematik
\setmathfont{Latin Modern Math}
% Alternativ zum Beispiel:
%\setmathfont{Libertinus Math}

\setmathfont{XITS Math}[range={scr, bfscr}]
\setmathfont{XITS Math}[range={cal, bfcal}, StylisticSet=1]

% Zahlen und Einheiten
\usepackage[
  locale=DE,                   % deutsche Einstellungen
  separate-uncertainty=true,   % immer Unsicherheit mit \pm
  per-mode=symbol-or-fraction, % / in inline math, fraction in display math
]{siunitx}

% chemische Formeln
\usepackage[
  version=4,
  math-greek=default, % ┐ mit unicode-math zusammenarbeiten
  text-greek=default, % ┘
]{mhchem}

% richtige Anführungszeichen
\usepackage[autostyle]{csquotes}

% schöne Brüche im Text
\usepackage{xfrac}

% Standardplatzierung für Floats einstellen
\usepackage{float}
\floatplacement{figure}{htbp}
\floatplacement{table}{htbp}

% Floats innerhalb einer Section halten
\usepackage[
  section, % Floats innerhalb der Section halten
  below,   % unterhalb der Section aber auf der selben Seite ist ok
]{placeins}

% Seite drehen für breite Tabellen: landscape Umgebung
\usepackage{pdflscape}

% Captions schöner machen.
\usepackage[
  labelfont=bf,        % Tabelle x: Abbildung y: ist jetzt fett
  font=small,          % Schrift etwas kleiner als Dokument
  width=0.9\textwidth, % maximale Breite einer Caption schmaler
]{caption}
% subfigure, subtable, subref
\usepackage{subcaption}

% Grafiken können eingebunden werden
\usepackage{graphicx}

% schöne Tabellen
\usepackage{tabularray}
\UseTblrLibrary{booktabs, siunitx}

% Verbesserungen am Schriftbild
\usepackage{microtype}

% Literaturverzeichnis
\usepackage[
  backend=biber,
]{biblatex}
% Quellendatenbank
\addbibresource{lit.bib}
\addbibresource{programme.bib}

% Hyperlinks im Dokument
\usepackage[
  german,
  unicode,        % Unicode in PDF-Attributen erlauben
  pdfusetitle,    % Titel, Autoren und Datum als PDF-Attribute
  pdfcreator={},  % ┐ PDF-Attribute säubern
  pdfproducer={}, % ┘
]{hyperref}
% erweiterte Bookmarks im PDF
\usepackage{bookmark}

% Trennung von Wörtern mit Strichen
\usepackage[shortcuts]{extdash}

\author{%
  Vincent Wirsdörfer\\%
  \href{mailto:vincent.wirsdoerfer@udo.edu}{authorA@udo.edu}%
  \and%
  Joris Daus\\%
  \href{mailto:joris.daus@udo.edu}{authorB@udo.edu}%
}
\publishers{TU Dortmund – Fakultät Physik}


%\begin{document}
\section{Versuchsaufbau}
Der Versuch findet digital auf der Seite der Universität Duisburg-Essen 
\cite{Hook_Interaktiv} statt. \
Um die Federkonstante zu bestimmen, ist ein Kraftmesser über eine Feder mit einer Auslenkvorrichtung verbunden.
Der digitale Kraftmesser ist zwischen einem Stativ und der Feder verbunden. Die Feder hängt senkrecht unter dem Kraftmesser.
Um die Feder auslenken zu können, ist diese über ein Seil, welches über eine Umlenkrolle geführt wird, mit einer 
Spannvorrichtung gekoppelt. Die Spanvorrichtung besteht aus einem $67 \unit{\centi \meter}$ langen Lineal und einem Pfeil, 
der anzeigt, wie weit die Feder ausgelenkt wird. Das lineal ist wagerecht mit dem Stativ verbunden. Der Pfeil lässt sich online 
mit dem Mauszeiger auf $0.5 \unit{\centi \meter}$ genau verschieben. %komma oder Punkt?
Die an der Feder wirkende Kraft wird auf einem mit dem Kräftemesser verbundenen Computer angezeigt. Der Kräftemesser 
gibt sein Signal in elektronisches Gerät, welches nicht näher bekannt ist. Von diesem Gerät geht das Signal dann in den 
Computer. Das elektronische Gerät und der Computer stehen neben dem Stativ mit der Feder. Die Kraftskala auf dem Laptop geht 
von $0 \unit{\newton}$ bis $5 \unit{\newton}$ und kann bis auf $0.01 \unit{\newton}$ genau abgelesen werden
%evtl was zu Rahmenbedingungen also blaue Unterlage und Hintergrund?


\section{Versuchsdurchführung}
\label{sec:Versuhsdurchfuehrung}
Auf der beschriebenen Internetseite wird mit der Maus der Pfeil am lineal verschoben. Durch dieses verschieben wird die Feder 
gespannt und ausgelenkt. Die Umlenkrolle ändert ihre Stellung. Die Kraft, die auf die Feder wirkt, wird auf dem Laptop 
abgelesen.

\section{Messwerte}
\label{sec:Messwerte}
Die Kraft $F$ und die Auslenkung $\increment x$ wurden gemessen. Die Federkonstante $D$ wurde aus den jeweiligen Messdaten 
berechnet.
\begin{table}
    \centering
    \caption{Messdaten.}
    \label{tab:tabelle}
    \sisetup{table-format=1.2, per-mode=reciprocal}
    \begin{tblr}{
        colspec = {S S S},
        row{1} = {guard, mode=math},
        %vline{4} = {2}{-}{text=\clap{$\pm$}},
      }
      \toprule
      \increment x \mathbin{/} \unit{\centi \meter} & F \mathbin{/} \unit{\newton} &  D \mathbin{/} \unit{\newton \per \meter} \\
      \midrule
      0.01 & 0.03 & 3    \\
      0.03 & 0.09 & 3    \\
      0.06 & 0.18 & 3    \\
      0.1  & 0.29 & 2.9  \\
      0.12 & 0.36 & 3    \\
      0.2  & 0.59 & 2.95 \\
      0.26 & 0.77 & 2.96 \\
      0.37 & 1.1  & 2.97 \\
      0.49 & 1.46 & 2.98 \\
      0.58 & 1.73 & 2.98 \\
      \bottomrule
    \end{tblr}
  \end{table}
  

%\end{document}

