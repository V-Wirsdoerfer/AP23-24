%\input{../../header.tex}

%\begin{document}

\section{Fehlerrechnung}
\label{sec:Fehlerrechnung}

Alle im Protokoll vermerkten Mittelwerte lassen sich über die folgende Formel berechnen:

\begin{equation}
\label{eqn:Mittelwert}
    \bar{x} = \frac{1}{N}\sum_{i=1}^N x_i
\end{equation}

\noindent Zudem lässt sich der dazugehörige Fehler des Mittelwerts wie folgt berechnen:

\begin{equation}
\label{eqn:Mittelwertfehler}
    \increment \bar{x} = \sqrt{\frac{1}{N\left(N-1\right)}\sum_{i=1}^N \left(x_i - \bar{x}\right)²}
\end{equation}

\noindent Entsteht ein neuer Fehler durch bereits fehlerbehaftete Größen, so wird die Gauß'sche Fehlerfortpflanzung angewendet:

\begin{equation}
\label{eqn:Fehlerfortpflanzung}
    \increment f = \sqrt{\sum_{i=1}^N \left(\frac{\partial f}{\partial x_i}\right)²\cdot\left(\increment x_i\right)²}
\end{equation}


\section{Auswertung}
\label{sec:Auswertung}

Zu Beginn der Auswertung, soll die Zeitkonstante $\tau$ eines RC-Gliedes durch die Analyse eines Auf- beziehungsweise Entladevorgangs
des Kondensators bestimmt werden. Dazu werden bei Beobachtung des Entladevorgangs zugehörige Wertepaare der Zeit $t$ und der Kondensatorspannug $U_C(t)$
entnommen und grafisch dargestellt:\footnote{Die dazugehörigen Werte befinden sich in Tabelle \ref{tab:t-U}}

\begin{figure}[H]
    \centering
    \includegraphics[height=7.5cm]{Entladungskurve.pdf}
    \caption{Entladungsvorgang des Plattenkondensators}
    \label{fig:Entladungskurve}
\end{figure}

\noindent Mit Hilfe von Gleichung \eqref{eqn:Entladung} und der Abbildung \ref{fig:Entladungskurve} lässt sich die lineare Ausgleichsrechnung 

\begin{equation}
    \ln\left(\frac{U_C}{U_0}\right) = \underbrace{-\frac{1}{RC}}_m \cdot t + b
\end{equation}

\noindent nachvollziehen. Der Wert für die Zeitkonstante $\tau$ ergibt sich somit aus dem negativen Kehrwert der Steigung $m$.\\
Der gemessene Parameter der Ausgleichsgerade lauten demnach:

\begin{equation}
    RC = \left(1.52 \pm 0.06\right)\,\cdot 10^{-3}\,\unit{\second}
\end{equation}

\noindent Die nächste Methode zur Bestimmung der Zeitkonstante unterliegt dem Zusammenhang zwischen der Frequenz $f$ und dem 
Amplitudenverhältnis $\sfrac{U_C}{U_0}$. Dementsprechend wird die Menge der Wertpaare ${\sfrac{U_C}{U_0},f}$ sowie eine 
nicht-lineare Ausgleichsgerade grafisch aufgetragen:\footnote{Die Werte für dieses Diagramm entstammen Tabelle \ref{tab:f-A-a}.}

\begin{figure}
    \centering
    \includegraphics[height=7.5cm]{Amplitudenspannung.pdf}
    \caption{Zeitlicher Verlauf des Amplitudenverhältnis.}
    \label{fig:Amplitudenverlauf}
\end{figure}

\noindent Durch korrektes der Umstellen der Gleichung \eqref{eqn:Phase_Amplitude} kann die Zeitkonstante bestimmt werden:

\begin{equation}
    RC = \left(2.19 \pm 0.18\right)\,\cdot 10^{-3}\,\unit{\second}
\end{equation}
\newpage
\noindent Neben der Amplitude $U_C$ der Kondensatorspannug ist auch die Phasenverschiebung $\varphi$ zwischen Generator- und
Kondensatorspannug abhängig von der Ausgangsfrequenz $f$. Dieser Zusammenhang wird in der nächsten Methode ausgenutzt, wobei 
die Menge der Wertepaare ${\varphi , f}$ grafisch aufgetragen werden: 

\begin{figure}[H]
    \centering
    \includegraphics[height=7.5cm]{phi(f).pdf}
    \caption{Zusammenhang zwischen der Frequenz $f$ und der Phasenverschiebung $\varphi$.}
    \label{fig:Phasenverschiebung}
\end{figure}

\noindent Der konkrete mathematische Zusammenhang lässt sich an 
Gleichung \eqref{eqn:Phase_Amplitude} ablesen. Durch äquivalentes Umformen ebenfalls dieser Gleichung kann somit die Zeitkonstante
berechnet werden. Eine Teilmenge\footnote{Warum hier nur eine Teilmenge der Werte verwendet wurde, wird in der Diskussion behandelt} der dafür verwendeten Werte befindet sich in Tabelle \ref{tab:f-A-a}
Daraus resultiert die folgende Zeitkonstante:

\begin{equation}
    RC = \left(2.5 \pm 1.7 \right)\,\cdot 10^{-3}\,\unit{\second}
\end{equation}


\noindent Die Menge der Wertepaare ${\sfrac{A(\omega)}{U_0},\varphi}$ aus Tabelle \ref{tab:f-A-a} können jedoch nicht nur in ein kartesisches Koordinatensystem eingetragen, sondern
auch als Polarplot dargestellt werden. Dies sähe dann folgendermaßen aus:

\begin{figure}[H]
    \centering
    \includegraphics[height=7.5cm]{polar2.pdf}
    \caption{Polarplot der Phasenverschiebung $\varphi$ und Relativamplitude $\sfrac{A(\omega)}{U_0}$.}
    \label{fig:polar}
\end{figure}

\noindent Zuletzt wird gezeigt, dass der RC-Kreis auch als Integrator verwendet. Wie in der Theorie postuliert muss dafür die Kreisfrequenz $\omega$
wesentlich größer als der Kehrwert der Zeitkonstante $\sfrac{1}{RC}$ sein. Dies kann mittels des abfotografierten Oszilloskops geprüft werden:

\begin{figure}[H]
    \begin{subfigure}{0.5\textwidth}
        \centering
        \includegraphics[width=6.5cm]{./content/Rechtecksspannung.jpg}
        \caption{Rechtecksspannung}
        \label{fig:Rechteck}
    \end{subfigure}
    \begin{subfigure}{0.5\textwidth}
        \centering
        \includegraphics[width=6.5cm]{./content/Dreiecksspannung.jpg}
        \caption{Dreiecksspannung}
        \label{fig:Rechteck}
    \end{subfigure}
    \begin{subfigure}{0.5\textwidth}
        \centering
        \includegraphics[width=6.5cm]{./content/Sinusspannung.jpg}
        \caption{Sinusspannung}
        \label{fig:Rechteck}
    \end{subfigure}
\end{figure}
\noindent
Das Oszilloskop auf den Fotos bestätigt, dass der RC-Kreis unter der Voraussetzung $\omega >> \frac{1}{RC}$ als Intergrator dienen kann. 
%\end{document}
