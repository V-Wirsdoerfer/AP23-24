\input{../../header.tex}

\begin{document}

\section{Fehlerrechnung}
\label{sec:Fehlerrechnung}

Alle im Protokoll vermerkten Mittelwerte lassen sich über die folgende Formel berechnen:

\begin{equation}
\label{eqn:Mittelwert}
    \bar{x} = \frac{1}{N}\sum_{i=1}^N x_i
\end{equation}

\noindent Zudem lässt sich der dazugehörige Fehler des Mittelwerts wie folgt berechnen:

\begin{equation}
\label{eqn:Mittelwertfehler}
    \increment \bar{x} = \sqrt{\frac{1}{N\left(N-1\right)}\sum_{i=1}^N \left(x_i - \bar{x}\right)²}
\end{equation}

\noindent Entsteht ein neuer Fehler durch bereits fehlerbehaftete Größen, so wird die Gauß'sche Fehlerfortpflanzung angewendet:

\begin{equation}
\label{eqn:Fehlerfortpflanzung}
    \increment f = \sqrt{\sum_{i=1}^N \left(\frac{\partial f}{\partial x_i}\right)²\cdot\left(\increment x_i\right)²}
\end{equation}


\section{Auswertung}
\label{sec:Auswertung}

Zu Beginn der Auswertung, soll die Zeitkonstante $\tau$ eines RC-Gliedes durch die Analyse eines Auf- beziehungsweise Entladevorgangs
des Kondensators bestimmt werden. Dazu werden bei Beobachtung des Entladevorgangs zugehörige Wertepaare der Zeit $t$ und der Kondensatorspannug $U_C(t)$
entnommen und grafisch dargestellt: 

\begin{figure}[H]
    \centering
    \includegraphics[height=5cm]{../build/Entladungskurve.pdf}
    \label{fig:Entladungskurve}
    \caption{Entladungsvorgang des Plattenkondensators}
\end{figure}

\noindent Mit Hilfe von Gleichung \eqref{eqn:Entladung} und der Abbildung \ref{fig:Entladungskurve} lässt sich die lineare Ausgleichsrechnung 

\begin{equation}
    \ln\left(\frac{U_C}{U_0}\right) = \underbrace{-\frac{1}{RC}}_m \cdot t + b
\end{equation}

\noindent nachvollziehen. Der Wert für die Zeitkonstante $\tau$ ergibt sich somit aus dem negativen Kehrwert der Steigung $m$.\\
Der gemessene Parameter der Ausgleichsgerade lauten demnach:

\begin{equation}
    \tau = RC = \left(2.19 \pm 0.18\right) \cdot 10^{-3}\,\unit{\second}\\
\end{equation}

\end{document}
