\input{../../header.tex}

\begin{document}
\section{Versuchsaufbau}
Der Versuch besteht aus einem Oszilloskop und einem Frequenzgenerator.
Der Frequenzgenerator kann bei beliebigen Frequenzen sinusförmige, rechteckige, oder dreieckige Schwingungsmuster erzeugen.
In dem Frequenzgenerator ist außerdem ein Hochpass und ein Tiefpass verbaut. In diesem Experiment wird allerdings nur der 
Tiefpass verwendet, um das Relaxationsverhalten des RC-Kreises zu untersuchen.
Die Spannungsquelle aus dem Frequenzgenerator wird mit zwei Kabeln abgegriffen. Ein Kabel geht direkt in das Oszilloskop, 
um die Spannungsquelle zu Messen und mit der Ausgangsspannung am Kondensator zu vergleichen.
Das andere Kabel geht in den Tiefpass. Um die Spannung am Ausgang des Tiefpasses zu messen, wird dort das Oszilloskop mit 
einem anderen Eingang verbunden. So entsteht der Schaltkreis aus \autoref{fig:Schaltkreis5} 

\begin{figure}[H]
    \includegraphics[width=\textwidth]{v353_Schaltkreise_5.png}
    \caption{Schaltkreis des Experimentes}
    \label{fig:Schaltkreis5}
\end{figure}


\section{Versuchsdurchführung}

\section{Messwerte}

\end{document}

