\input{../../header.tex}

\begin{document}
\section{Diskussion}

Das genaue Ablesen der Werte von dem analogen Oszilloskop bereitet einiges an Fehlerpotential. 
Die Spannungskurven auf dem Oszilloskop bewegen sich teilweise. Dies führt selbst bei bester Ablesung zwangsläufig zu Fehlern. 
\\
Um die Phasendifferenz zu bestimmen, muss der Phasenabstand $a$ ermittelt werden. Die Vorraussetzung dafür ist es, die 
Kurve symmetrisch um die x-Achse zu legen. Das muss per Augenmaß gemacht werden und ist somit auch nicht genau.
\\
Die Entladekurve des Kondensators ist ein exponentiell fallender Verlauf. Aus diesem Grund sind die Werte für große Zeiten 
sehr nah beieinander. Die relativen Änderungen werden so klein, dass es nicht mehr möglich ist verlässliche Werte über die 
Änderung jener anzugeben. Das kann der Grund für die Abweichung von der Ausgleichsgerade sein.
\\
In \autoref{fig:Amplitudenverlauf} wird zum Vergleich die Amplitude mithilfe des berechneten $RC$ aufgetragen. 
Es wird jenes $RC$ genommen, welches durch die lineare Ausgleichsgerade berechnet wurde. Zwar sieht der Plot auf den ersten Blick 
sehr passend aus, jedoch gibt es ein $RC$, welches mathematisch genauer ist. Dieses wird mit der Pythonfunktion curve\_fit 
bestimmt. Die aus dem $RC$ resultierende Kurve fittet zwar die Anfangswerte etwas genauer, lässt jedoch für die Endwerte an 
Genauigkeit nach. Das kann daran liegen, dass es nur wenige Werte für große f gibt und somit kleinere Werte, von denen es 
mehr gibt, stärker ins Gewicht fallen. 
\\
Um die Zeitkonstante aus der Phasenverschiebung zu berechnen, muss nach Gleichung \eqref{eqn:Phase_Amplitude} eine Konstante 
im Argument eines $\text{arctan}(x)$ bestimmt werden. Dazu sollten die Daten von $\varphi$ den Verlauf des 
$\text{arctan}(x)$ ergeben. Jedoch ist der Verlauf dieser Funktion für große Werte eine Konstante mit dem Wert $\frac{1}{2} \pi$.
Diese Konstante wird auch gebildet, wie in \ref{fig:Phasenverschiebung} und \ref{fig:polar} zu sehen ist. 
Jedoch wurden keine Messwerte für ausreichend kleine $\varphi$ aufgezeichnet, weshalb der typische Verlauf des $\text{arctan}(x)$
nicht zu erkennen ist. Legt man jedoch den theoretisch vorausgesagten Verlauf durch die Messwerte, so zeigt sich, dass 
die Messwerte den theoretischen Voraussagen genügen.
Aus diesem Grund können aus der Methode der Phasenverschiebung keine verlässlichen Daten zur Bestimmung 
der Zeitkonstante gewonnen werden und es können nur einzelne Messwerte schemenhaft ausgewertet werden.
\label{sec:Diskussion}


\end{document}
