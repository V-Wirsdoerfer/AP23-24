\input{../../header.tex}

\begin{document}
\section{Zielsetzung}
\label{sec:Zielsetzung}
In dem folgend protokollierten Versuch soll das Relaxationsverhalten des RC-Kreises untersucht werden. Das konkrete
Ziel besteht darin, die Zeitkonstante $\tau \vcentcolon = RC$ des Auf- bzw. Entladevorgangs des Kondensators zu bestimmen.
Ferner wird der Zusammenhang zwischen der Amplitude der Kondensatorspannug, sowie der Phasenverschiebung der Ausgangs- und
Kondensatorspannug und der Kreisfrequenz $\omega$ der Ausgangsspannung zu analysieren. Zusätzlich soll gezeigt werden, unter 
welchen Voraussetzungen der RC-Kreis als Integrator arbeiten kann.

\section{Theorie}
\label{sec:Theorie}

Unter einem \emph{Relaxationsvorgang} wird die zeitliche Entwicklung eines Systems verstanden, welches sich aus seinem 
Ausgangszustand entfernt, nach einer gewissen Zeit jedoch nicht-oszillatorisch wieder in seinen Anfangszustand zurückkehrt.
Oftmals besteht dabei ein proportionaler Zusammenhang zwischen der Änderungsgeschwindigkeit $\frac{\symup{d}A}{\symup{d}t}$
der physikalischen Größe $A$ und der Abweichung von $A$ zum Endzustand $A\left(\infty\right)$:

\begin{equation}
    \frac{\symup{d}A}{\symup{d}t} = c\left(A(t) - A(\infty)\right)
\end{equation}

Die Seperation der Variablen $A$ und $t$ in dieser Differentialgleichung und der Integration beider Seiten vom Zeitpunkt 0
bis zum Zeitpunkt t liefert

\begin{equation}
    \int_{A(0)}^{A(t)}\frac{\symup{d}\tilde{A}}{\tilde{A} - A(\infty)} = \int_{0}^{t}c\symup{d}\tilde{t}
\end{equation}

\begin{equation}
    \flushleft{\Leftrightarrow} \ln\left(\frac{A(t)-A(\infty)}{A(0)-A(\infty)}\right) = ct
\end{equation}

\begin{equation}
\label{eqn:Amplitudenverlauf}
    \flushleft{\Leftrightarrow} A(t) = A(\infty) \left(A(0) - A(\infty)\right)e^{ct}
\end{equation}

Dabei muss in Gleichung \eqref{eqn:Amplitudenverlauf} $c < 0$ sein, damit die Funktion $A(t)$ beschränkt ist.
Ein beispielhafter Vorgang für ein Relaxationsverhalten wird durch die Ent- und Aufladung eines Kondensators in einem 
RC-Kreis dar.

\begin{figure}
    \centering
    \includegraphics
\section{Vorbereitung}

\section{Fehlerrechnung}
\end{document}