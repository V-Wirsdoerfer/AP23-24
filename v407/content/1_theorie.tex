\input{../../header.tex}

\begin{document}

\section{Zielsetzung}
\label{sec:Zielsetzung}

Wie dem Versuchstitel bereits entnommen werden kann, beschäftigt sich das im folgenden protokollierte Experiment mit den \emph{Fresnelschen Formeln}.
Konkret gesagt, werden diese über den Zusammenhang zwischen Reflexionsintensität eines Laserstrahls und dessen Winkel überprüft. Wie dieser Prozess genau 
gelingt, wird im nächsten Kapitel dargelegt. 

\section{Theorie}
\label{sec:Theorie}


\subsection{Grundlagen}
\label{sec:Grundlagen}

Im feld- und materierfreien Raum kann Licht als elektromagnetische Welle aufgefasst werden. Grundlage hierfür sind die \emph{Maxwellschen Gleichungen}, 
welche Aussagen über das Verhalten und die Entstehung elektrischer- und magnetischer Felder machen. Besonders die Maxwell Gleichungen

\begin{align}
\label{eqn:MW34}
    \vec{\nabla}\times\vec{H} &= \vec{j} + \varepsilon\varepsilon_0\partial_{t}\vec{E}\\
    \vec{\nabla}\times\vec{E} &= -\mu\mu_0\partial_{t}\vec{H}
\end{align}

\noindent spielen in der hier verwendeten Herleitung eine essenzielle Rolle. Dabei repräsentieren $\vec{E}$ und $\vec{H}$ die elektrische- bzw. magnetische 
Feldstärke. Die Stromdichte wird mit $\vec{j}$ abgekürzt. Die Faktoren in den obigen Gleichungen haben die folgende Bedeutung:

\begin{align*}
    \varepsilon &: \text{Relative Dielektrizitätskonstante}\\
    \varepsilon_0 &: \text{Influenzkosntante}\\
    \mu &: \text{Permeabilität des Mediums}\\
    \mu_0 &: \text{Induktionskonstante}
\end{align*}

\noindent In diesem Versuch liegt der Schwerpunkt auf nicht-ferromagnetischen und nicht elektrische leitenden Medien, weshalb $\mu \approx 1$ und $\vec{j} = 0$
gesetzt wird.  

\subsection{Strahlungsleistung des elektromagnetischen Feldes}

Anknüpfend an das vorherige Teilkapitel, werden die Gleichungen \eqref{eqn:MW34} skalar mit $\vec{E}$ bzw. $\vec{H}$ multipliziert weshalb sie folglich die Gestalt 

\begin{align}
\label{eqn:MW34c}
    \vec{E}\cdot\left(\vec{\nabla}\times\vec{H}\right) &= \varepsilon\varepsilon_0\vec{E}\cdot\partial_{t}\vec{E}\\
    \vec{H}\cdot\left(\vec{\nabla}\times\vec{E}\right) &= -\mu_0\vec{H}\cdot\partial_{t}\vec{H}
\end{align}
\newpage
\noindent Zudem ist bekannt, dass

\begin{align}
    \vec{\nabla}\cdot\left(\vec{E}\times\vec{H}\right) = \vec{H}\cdot\left(\vec{\nabla}\times\vec{E}\right) - \vec{E}\cdot\left(\vec{\nabla}\times\vec{H}\right),
    \label{eqn:Divpoynt}
\end{align}

\noindent weshalb sich die Gleichungen \eqref{eqn:MW34c} eingesetzt in \eqref{eqn:Divpoynt} schreiben lassen als

\begin{align}
\label{eqn:Divpoyntzsm}
    \vec{\nabla}\cdot\left(\vec{E}\times\vec{H}\right) = -\mu_0\vec{H}\cdot\partial_t\vec{H} - \varepsilon\varepsilon_0\vec{E}\cdot\partial_t\vec{E}.
\end{align}

\noindent Außerdem sind die folgenden Ausdrücke bekannt:

\begin{align}
    W_\text{el} &= \frac{1}{2}\varepsilon\varepsilon_0\vec{E}²\\
    W_\text{mag} &= \frac{1}{2}\mu_0\vec{H}²\\
\label{eqn:}
\end{align}

\noindent Die Energiedichte beschreibt die Energie pro Volumeneinheit und lässt sich im Falle von elektrischen und magnetischen Feldern schreiben als 

\begin{align}
\label{eqn:Energiedichten}
    W_\text{el} &= \frac{1}{2}\varepsilon\varepsilon_0\vec{E}²\\
    W_\text{mag} &= \frac{1}{2}\mu_0\vec{H}²\\
\end{align}

\noindent Durch zeitliches Differenzieren dieser Terme \eqref{eqn:Energiedichten} lässt sich Gleichung \eqref{eqn:Divpoyntzsm} über äquivalentes Umformen 
in der Gestalt 

\begin{align*}
    \vec{\nabla}\cdot\left(\vec{E}\times\vec{H}\right) + \partial_t\left(W_\text{el}+W_\text{mag}\right) = 0
\end{align*}

\noindent Dieser Ausdruck kann nun über ein beliebiges Volumen $V$ integriert. Über den mathematischen Satz von Gauß kann das Volumenintegral über den 
Divergenzterm in ein Oberflächenintegral des Vektorfeldes $\vec{E}\times\vec{H}$ über den Rand des Volumens $\partial{}V=O$ überführt werden. 

\begin{align}
    \int_O \left(\vec{E}\times\vec{H}\right)\,\symup{d}\vec{O} + \partial_t\left(\int_V W_\text{el}\,\symup{d}V + \int_V W_\text{mag}\,\symup{d}V\right) = 0
\label{eqn:Gauss}
\end{align}

\noindent Ähnlich zur wichtigen \emph{Kontinuitätsgleichung} aus diversen Feldtheorien, kann in diesem Fall konstatiert werden, dass der Ausdruck des Oberflächenintegrals
den vollständigen Energiestrom pro Zeiteinheit durch die Oberfläche $O$ angibt. Dies resultiert aus der Tatsache, dass keine außer den genannten Energiefromen 
einen Einfluss auf den energetischen Zustand des Systems hat. Das in dem Integral auftauchende Vektorfeld 

\begin{equation}
    \vec{E}\times\vec{H} \eqcolon \vec{S}  
\end{equation}

\noindent wird als \emph{Poynting-Vektor} bezeichnet und beschreibt die Strahlungsleistung pro Flächeneinheit eines elektromagnetischen Feldes. Ein andere 
Darstellungsweise des Betrags des Poyting-Vektors lässt sich auf Grundlage der Felder 

\begin{align*}
    \vec{E}(\vec{r},t) &= \vec{E}_0\exp\left(i(\vec{k}\cdot\vec{r}-\omega{}t)\right)\\
    \vec{H}(\vec{r},t) &= \vec{H}_0\exp\left(i(\vec{k}\cdot\vec{r}-\omega{}t)\right)
\end{align*}

\noindent finden. Hierbei $\vec{r}$ für den Ortsvektor, $t$ für die Zeit, $\vec{k}$ für die vektorielle Wellenzahl und $\omega$ für die Kreisfrequenz. Über die Maxwell-Gleichungen 
\uproman{1} und \uproman{2} lässt sich die Korrelation 

\begin{equation}
    H_0 = \sqrt{\frac{\varepsilon\varepsilon_0}{\mu_0}}E_0
\label{eqn:Amplituden}
\end{equation}

\noindent der Amplituden $E_0$ und $H_0$ ausdrücken. Im Hinblick auf die Gleichungen \eqref{eqn:Energiedichte} wird durch diese Beziehung, dass die elektrische und magnetische 
Energiedichte im elektromagnetischen Feld äquivalent sind, weshalb sich Gleichung \eqref{eqn:Gauss} zu

\begin{equation*}
    \int_O \vec{S}\,\symup{d}\vec{O} +\partial_t\int_V \varepsilon\varepsilon_0\vec{E}²\,\symup{d}V = 0
\end{equation*}

\noindent umformulieren lässt. Um nun eine direkte Äquivalenz der Terme zu kreieren wird angenommen, dass die Energie mit einer Geschwindigkeit $v$ in einer Zeit d$t$ 
durch die Fläche $O$ strömt und dabei das Volumenelement d$V$ erfüllt. Somit gilt

\begin{equation*}
    \symup{d}V = Ov\,\symup{d}t 
\end{equation*},

\noindent was bedeutet, dass die Strahlungsleistung pro Flächeneinheit auch ausgedrückt werden kann als 

\begin{equation}
    |\vec{S}| = v\varepsilon\varepsilon_0\vec{E}².
\label{eqn:BetragS}
\end{equation}

\subsection{Reflexion an einer Grenzfläche}
\label{sec:Grenzflaeche}

Trifft eine ebene Welle der Amplitude $\vec{E}_\text{e}$ auf die Grenzfläche eines Mediums, so wird ein Brcuhteil der Amplitude reflektiert, wohingegegen der Rest unter einem 
Winkel $\beta$ transmittiert und in somit in das Medium eindringt. Zur besseren Vorstellung dient die untenstehende Abbildung.

\begin{figure}[H]
    \centering
    \includegraphics[height=5cm]{Grenzflaeche.png}
    \caption{Strahlengänge einer ebenen Welle an einer Grenzfläche\cite{Versuchsanleitung_v407}.}
    \label{fig:SkizzeGrenzflaeche}
\end{figure}

\noindent Die Tatsache, dass der Brechungsindex $n$ im Medium größer als 1 ist, impliziert bereits, dass die Lichtgeschwindigkeit $v$ im Medium kleiner ist als die
Vakuumlichtgeschwindigkeit $c$, was über die Beziehung $n = \sfrac{c}{v}$ leicht eingesehen werden kann. Nach dem \emph{Snelliusschem Brechungsgesetz} bedeutet dies 
ausßerdem, dass der Brechungswinkel $\beta$ kleiner als der Einfalls- bzw. Ausfallswinkel $\alpha$. Dementsprechend ändert sich auch der in der Abbildung 
\ref{fig:SkizzeGrenzflaeche} Querschnitt von $F_\text{e}$ auf $F_\text{d}$, was über trigonometrische Beziehungen den Energiesatz

\begin{equation*}
    S_\text{e}\cos(\alpha) = S_\text{r}\cos(\alpha) + S_\text{d}\cos(\beta) 
\end{equation*}

\noindent zur Folge hat. Über den im vorherigen Kapitel hergeleiteten Ausdruck für den Betrag des Poynting-Vektors \eqref{eqn:BetragS} ergibt sich 

\begin{equation*}
    c\varepsilon_0\vec{E}²_\text{e}\cos(\alpha) = c\varepsilon_0\vec{E}²_\text{r}\cos(\alpha) + v\varepsilon_0\vec{E}²_\text{d}\cos(\beta)
\end{equation*}

\noindent und unter Ausnutzung der Beziehung $n = \sfrac{c}{v}$

\begin{equation}
    \left(\vec{E}²_\text{e} - \vec{E}²_\text{r}\right)n\cos(\alpha) = \varepsilon\vec{E}²_\text{d}\cos(\beta).
\label{eqn:ESatz}
\end{equation}

\noindent Die Wellengleichung 

\begin{equation*}
    \increment\vec{E} = \varepsilon\varepsilon_0\mu_0\partial²_t\vec{E},
\end{equation*}

\noindent welche aus den Maxwellgleichungen \uproman{3} und \uproman{4} folgt, zeigt, dass der Faktor $\varepsilon\varepsilon_0\mu_0$ als Kehrwehrt der quadrierten Ausbreitungsgeschwindigkeit 
$v$ der Welle geschrieben werden, was im Vakuum ($\varepsilon \approx 1$) bedeutet

\begin{equation*}
    \varepsilon_0\mu_0 = \frac{1}{c²}.
\end{equation*}

\noindent Hieraus kann die \emph{Maxwellsche Relation} $n² = \varepsilon$ hergeleitet werden, weshalb die Gleichung \eqref{eqn:ESatz} auch in der Form 

\begin{equation}
    \left(\vec{E}²_\text{e} - \vec{E}²_\text{r}\right)\cos(\alpha) = n\vec{E}²_\text{d}\cos(\beta).
\end{equation}

\noindent darstellbar ist. 

\subsection{Polarisationsrichtungen}
\label{sec:Polarisationsrichtungen}

Im letzten Teilkapitel der Theorie wird detaillierter auf die Polarisationsrichtungen eingegangen. Zur fachspezifischen Beschreibung dieser Thematik ist die Einführung der Einfallsebene 
unabdinglich. Trifft die einfallende Welle, beschrieben durch den Vektor $\vec{E}_\text{e}$ auf eine Grenzfläche, so spannt $\vec{E}_\text{e}$ zusammen mit dem Normalenvektor der Grenzfläche
die sogenannte Einfallebene auf, welche senkrecht auf der Grenzebene steht. Dieses Bild wird durch die folgenden Abbildung unterstützt.

\begin{figure}[H]
    \centering
    \includegraphics[height=6cm]{Einfallsebene.png}
    \caption{Darstellung der senkrechten Polarisationsrichtung einer ebenen Welle an einer Grenzfläche\cite{Versuchsanleitung_v407}.}
    \label{fig:SkizzePolarisationsrichtung}
\end{figure}

\noindent Demnach kann der Feldvektor $\vec{E}_\text{e}$ in eine zur Einfallsebene senkrecht- und parallele Komponente zerlegt werden.

\begin{equation*}
    \vec{E}_\text{e} = \vec{E}_\perp + \vec{E}_\parallel
\end{equation*}

\noindent Der orthogonale Zusammenhang zwischen Einfalls- und Grenzebene bewirkt zudem, dass der senkrecht zur Einfallsebene oszillierende Teil 
tangential zur Grenzebene liegt. Die Stetigkeitsbedingung der Tangentialkomponente des E-Feldes aus der Elektrodynamik liefert 

\begin{equation*}
    \vec{E}_{\text{r}\perp} = -\vec{E}_{\text{e}\perp}\frac{n\cos(\beta)-\cos(\alpha)}{n\cos(\beta)+\cos(\alpha)},
\end{equation*}

\noindent was sich durch das bereits angesprochene Brechungsgesetz nach Snellius $n = \sfrac{\sin(\alpha)}{\sin(\beta)}$ auch schreiben lässt als

\begin{equation}
    \vec{E}_{\text{r}\perp} = -\vec{E}_{\text{e}\perp}\frac{\sin(\alpha-\beta)}{\sin(\alpha-\beta)}
    \label{eqn:18}
\end{equation}

oder 

\begin{equation}
    \vec{E}_{\text{r}\perp}(\alpha) = -\vec{E}_{\text{e}\perp}\frac{\left(\sqrt{n²-\sin²(\alpha)}-\cos(\alpha)\right)²}{n²-1},
    \label{eqn:19}
\end{equation}

\noindent um entweder $n$ oder $\beta$ zu eliminieren. Die Gleichungen \eqref{eqn:18} und \eqref{eqn:19} zeigen den gesuchten Zusammenhang zwischen 
Amplitude und Reflexionswinkel.\\

\noindent Ferner besitzt die ebene Welle auch eine Komponente $\vec{E}_\parallel$, welche parallel zur Einfallsebene schwingt. Dieses Fall wird in der 
untenstehenden Abbildung dargestellt.

\begin{figure}
    \centering
    \includegraphics[height=5cm]{senkPolar.png}
    \caption{Darstellung der parallelen Polarisationsrichtung einer ebenen Welle an einer Grenzfläche\cite{Versuchsanleitung_v407}.}
    \label{fig:senkrechtPolar}
\end{figure}

\noindent Der senkrechte Feldvektor $\vec{E}_\parallel$ lässt sich nun in eine tangentiale und normale Komponente zerlegen. Als Resultat von 
Stetigkeitsbedingung und trigonometrischen Zusammenhängen folgt 

\begin{equation*}
    \vec{E}_{\text{r}\parallel} = \vec{E}_{\text{e}\parallel}\frac{n\cos(\alpha)-\cos(\beta)}{n\cos(\alpha)+\cos(\beta)}.
\end{equation*}
 \noindent Dies kann analog zur senkrechten Komponente mittels Snellius umgeschrieben werden als 

 \begin{equation}
    \vec{E}_{\text{r}\parallel} = \vec{E}_{\text{e}\parallel}\frac{\tan(\alpha-\beta)}{\tan(\alpha+\beta)}
\label{eqn:ReflexionBrewster} 
\end{equation}

 \noindent bzw.

\begin{equation}
    \vec{E}_{\text{r}\parallel}(\alpha) = \vec{E}_{\text{e}\parallel}\frac{n²\cos(\alpha)-\sqrt{n²-\sin²(\alpha)}}{n²\cos(\alpha)+\sqrt{n²-\sin²(\alpha)}}.
\label{eqn:21}
\end{equation}

\subsection{Brewsterscher Winkel}
\label{sec:Brewster}

Ein zusätzliches Phänomen, welches ausschließlich bei der zuletzt genannten parallelen Polarisationsrichtung auftritt, ist der sogenannte \emph{Brewstersche Winkel}.
Dieser beschreibt jenen Einfallswinkel der ebenen, an welchem der Lichtstrahl nicht reflektiert wird, sondern ausschließlich in das Medium eindringt.
Unter Betrachtung von Gleichung \eqref{eqn:ReflexionBrewster} lässt sich konstatieren, dass der Zähler des Bruchs für nichttriviale Werte von $\alpha$ und $\beta$ nicht
Null wird. Es jedoch möglich, dass der Nenner bei geeigneter Wahl der Winkel gegen $\infty$ strebt. Dies ist genau, dann der Fall, wenn 

\begin{equation*}
    \alpha_\text{B} + \beta_\text{B} = \qty{90}{\degree}
\end{equation*}

\noindent Über das Brechungsgesetz lautet der Brewstersche Winkel demnach

\begin{equation*}
    \alpha_\text{B} = \arctan(n).
\end{equation*}


\section{Vorbereitung}

\section{Fehlerrechnung}
\end{document}