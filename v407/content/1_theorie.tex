\input{../../header.tex}

\begin{document}

\section{Zielsetzung}
\label{sec:Zielsetzung}

Wie dem Versuchstitel bereits entnommen werden kann, beschäftigt sich das im folgenden protokollierte Experiment mit den \emph{Fresnelschen Formeln}.
Konkret gesagt, werden diese über den Zusammenhang zwischen Reflexionsintensität eines Laserstrahls und dessen Winkel überprüft. Wie dieser Prozess genau 
gelingt, wird im nächsten Kapitel dargelegt. 

\section{Theorie}
\label{sec:Theorie}


\subsection{Grundlagen}
\label{sec:Grundlagen}

Im feld- und materierfreien Raum kann Licht als elektromagnetische Welle aufgefasst werden. Grundlage hierfür sind die \emph{Maxwellschen Gleichungen}, 
welche Aussagen über das Verhalten und die Entstehung elektrischer- und magnetischer Felder machen. Besonders die Maxwell Gleichungen

\begin{align}
\label{eqn:MW34}
    \vec{\nabla}\times\vec{H} &= \vec{j} + \varepsilon\varepsilon_0\partial_{t}\vec{E}\\
    \vec{\nabla}\times\vec{E} &= -\mu\mu_0\partial_{t}\vec{H}
\end{align}

\noindent spielen in der hier verwendeten Herleitung eine essenzielle Rolle. Dabei repräsentieren $\vec{E}$ und $\vec{H}$ die elektrische- bzw. magnetische 
Feldstärke. Die Stromdichte wird mit $\vec{j}$ abgekürzt. Die Faktoren in den obigen Gleichungen haben die folgende Bedeutung:

\begin{align*}
    \varepsilon &: \text{Relative Dielektrizitätskonstante}\\
    \varepsilon_0 &: \text{Influenzkosntante}\\
    \mu &: \text{Permeabilität des Mediums}\\
    \mu_0 &: \text{Induktionskonstante}
\end{align*}

\noindent In diesem Versuch liegt der Schwerpunkt auf nicht-ferromagnetischen und nicht elektrische leitenden Medien, weshalb $\mu \approx 1$ und $\vec{j} = 0$
gesetzt wird.  

\subsection{Strahlungsleistung des elektromagnetischen Feldes}

Anknüpfend an das vorherige Teilkapitel, werden die Gleichungen \ref{eqn:MW34} skalar mit $\vec{E}$ bzw. $\vec{H}$ multipliziert weshalb sie folglich die Gestalt 

\begin{align}
\label{eqn:MW34c}
    \vec{E}\cdot\left(\vec{\nabla}\times\vec{H}\right) &= \varepsilon\varepsilon_0\vec{E}\cdot\partial_{t}\vec{E}\\
    \vec{H}\cdot\left(\vec{\nabla}\times\vec{E}\right) &= -\mu_0\vec{H}\cdot\partial_{t}\vec{H}
\end{align}
\newpage
\noindent Zudem ist bekannt, dass

\begin{align}
    \vec{\nabla}\cdot\left(\vec{E}\times\vec{H}\right) = \vec{H}\cdot\left(\vec{\nabla}\times\vec{E}\right) - \vec{E}\cdot\left(\vec{\nabla}\times\vec{H}\right),
    \label{eqn:Divpoynt}
\end{align}

\noindent weshalb sich die Gleichungen \ref{eqn:MW34c} eingesetzt in \ref{eqn:Divpoynt} schreiben lassen als

\begin{align}
\label{eqn:Divpoyntzsm}
    \vec{\nabla}\cdot\left(\vec{E}\times\vec{H}\right) = -\mu_0\vec{H}\cdot\partial_t\vec{H} - \varepsilon\varepsilon_0\vec{E}\cdot\partial_t\vec{E}.
\end{align}


\section{Vorbereitung}

\section{Fehlerrechnung}
\end{document}