\input{../../header.tex}

\begin{document}
\section{Auswertung}
\label{sec:Auswertung}


Die Intensität $I=\frac1 2 \varepsilon_0 E^2$ mit der nimmt mit dem Winkel ab. Diese Intensität soll nun Einheitenlos gemacht werden, indem durch die 
Eingangsintensität $I_0$ geteilt wird. Diese liegt bei $I_0=\qty{0.48\pm0.01e-3}{\ampere}$. Sie wird bestimmt, indem die Intensität des Lasers ohne 
Filter gemessen wird. 
Der Winkel wird nun gegen die Wurzel der einheitenlosen Intensitäten aufgetragen. So ergibt sich der folgende Plot.

\begin{figure}
    \includegraphics[width=0.9\textwidth]{../build/I0.pdf}
    \caption{Wurzel der Intensitäten gegen den jeweiligen Winkel.}
\end{figure}



Im Folgenden soll nun der Brechungsindex $n$ bestimmt werden. Dazu werden die Gleichungen \eqref{??} und \eqref{??} umgestellt. 
So ergibt sich für senkrechte Polarisation:

\begin{equation}
    n_\bot = \pm\sqrt{\frac{-2 E_{e_\bot} E_{r_\bot} \cos{(2a)} + E_{r_\bot}^2 + E_{e_\bot}^2 }{ 2 E_{e_\bot} E_{r_\bot} + E_{r_\bot}^2 + E_{e_\bot}^2  }}.
\end{equation}


\noindent Ebenso ergibt sich für parallele Polarisation ein Brechungsindex von 





\end{document}
