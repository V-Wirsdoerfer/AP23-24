%\documentclass[
  bibliography=totoc,     % Literatur im Inhaltsverzeichnis
  captions=tableheading,  % Tabellenüberschriften
  titlepage=firstiscover, % Titelseite ist Deckblatt
]{scrartcl}

% Paket float verbessern
\usepackage{scrhack}

% Warnung, falls nochmal kompiliert werden muss
\usepackage[aux]{rerunfilecheck}

% unverzichtbare Mathe-Befehle
\usepackage{amsmath}
% viele Mathe-Symbole
\usepackage{amssymb}
% Erweiterungen für amsmath
\usepackage{mathtools}

% Fonteinstellungen
\usepackage{fontspec}
% Latin Modern Fonts werden automatisch geladen
% Alternativ zum Beispiel:
%\setromanfont{Libertinus Serif}
%\setsansfont{Libertinus Sans}
%\setmonofont{Libertinus Mono}

% Wenn man andere Schriftarten gesetzt hat,
% sollte man das Seiten-Layout neu berechnen lassen
\recalctypearea{}

% deutsche Spracheinstellungen
\usepackage[ngerman]{babel}


\usepackage[
  math-style=ISO,    % ┐
  bold-style=ISO,    % │
  sans-style=italic, % │ ISO-Standard folgen
  nabla=upright,     % │
  partial=upright,   % │
  mathrm=sym,        % ┘
  warnings-off={           % ┐
    mathtools-colon,       % │ unnötige Warnungen ausschalten
    mathtools-overbracket, % │
  },                       % ┘
]{unicode-math}

% traditionelle Fonts für Mathematik
\setmathfont{Latin Modern Math}
% Alternativ zum Beispiel:
%\setmathfont{Libertinus Math}

\setmathfont{XITS Math}[range={scr, bfscr}]
\setmathfont{XITS Math}[range={cal, bfcal}, StylisticSet=1]

% Zahlen und Einheiten
\usepackage[
  locale=DE,                   % deutsche Einstellungen
  separate-uncertainty=true,   % immer Unsicherheit mit \pm
  per-mode=symbol-or-fraction, % / in inline math, fraction in display math
]{siunitx}

% chemische Formeln
\usepackage[
  version=4,
  math-greek=default, % ┐ mit unicode-math zusammenarbeiten
  text-greek=default, % ┘
]{mhchem}

% richtige Anführungszeichen
\usepackage[autostyle]{csquotes}

% schöne Brüche im Text
\usepackage{xfrac}

% Standardplatzierung für Floats einstellen
\usepackage{float}
\floatplacement{figure}{htbp}
\floatplacement{table}{htbp}

% Floats innerhalb einer Section halten
\usepackage[
  section, % Floats innerhalb der Section halten
  below,   % unterhalb der Section aber auf der selben Seite ist ok
]{placeins}

% Seite drehen für breite Tabellen: landscape Umgebung
\usepackage{pdflscape}

% Captions schöner machen.
\usepackage[
  labelfont=bf,        % Tabelle x: Abbildung y: ist jetzt fett
  font=small,          % Schrift etwas kleiner als Dokument
  width=0.9\textwidth, % maximale Breite einer Caption schmaler
]{caption}
% subfigure, subtable, subref
\usepackage{subcaption}

% Grafiken können eingebunden werden
\usepackage{graphicx}

% schöne Tabellen
\usepackage{tabularray}
\UseTblrLibrary{booktabs, siunitx}

% Verbesserungen am Schriftbild
\usepackage{microtype}

% Literaturverzeichnis
\usepackage[
  backend=biber,
]{biblatex}
% Quellendatenbank
\addbibresource{lit.bib}
\addbibresource{programme.bib}

% Hyperlinks im Dokument
\usepackage[
  german,
  unicode,        % Unicode in PDF-Attributen erlauben
  pdfusetitle,    % Titel, Autoren und Datum als PDF-Attribute
  pdfcreator={},  % ┐ PDF-Attribute säubern
  pdfproducer={}, % ┘
]{hyperref}
% erweiterte Bookmarks im PDF
\usepackage{bookmark}

% Trennung von Wörtern mit Strichen
\usepackage[shortcuts]{extdash}

\author{%
  Vincent Wirsdörfer\\%
  \href{mailto:vincent.wirsdoerfer@udo.edu}{authorA@udo.edu}%
  \and%
  Joris Daus\\%
  \href{mailto:joris.daus@udo.edu}{authorB@udo.edu}%
}
\publishers{TU Dortmund – Fakultät Physik}


%\begin{document}
\section{Auswertung}
\label{sec:Auswertung}

Zur Bestimmung des Brewsterwinkels, muss der Winkel mit der minimalen Intensität bestimmt werden. 
Die Intensität $I=\frac1 2 \varepsilon_0 E^2$ mit der Amplitude des E-Feldes $E$ verändert sich mit dem Winkel. 
Durch eine Auftragung soll diese Veränderung verdeutlicht und der Brewsterwinkel bestimmt werden. 
Diese Intensität soll nun Einheitenlos gemacht werden, indem durch die Eingangsintensität $I_0$ geteilt wird. 
Diese liegt bei $I_0=\qty{0.48\pm0.01e-3}{\ampere}$. Sie wird bestimmt, indem die Intensität des Lasers ohne 
Filter gemessen wird. 
Der Winkel wird nun gegen die Wurzel der einheitenlosen Intensitäten aufgetragen, um den Betrag der Amplitude des E-Feldes zu bekommen. 
So ergibt sich der folgende Plot.

\begin{figure}
    \centering
    \includegraphics[width=0.9\textwidth]{build/I0.pdf}
    \caption{Wurzel der Intensitäten gegen den jeweiligen Winkel.}
\end{figure}

\noindent Mithilfe der Auftragung kann nun das Minimum des parallel polarisierten Lichts und der dazugehörige Winkel bestimmt werden. 
Wird dies gemacht, so ergibt sich ein Brewsterwinkel von 

\begin{align}
    \alpha_\text{p} = \qty{74}{\degree}
\end{align}

%%% Brechungsindex
\noindent Im Folgenden soll nun der Brechungsindex $n$ bestimmt werden. Dazu werden die Gleichungen \eqref{eqn:19} und \eqref{eqn:21} umgestellt. 
So ergibt sich für senkrechte Polarisation:

\begin{equation}
    n_\bot = \pm\sqrt{\frac{-2 E_{e_\bot} E_{r_\bot} \cos{(2a)} + E_{r_\bot}^2 + E_{e_\bot}^2 }{ 2 E_{e_\bot} E_{r_\bot} + E_{r_\bot}^2 + E_{e_\bot}^2  }}.
\end{equation}

\noindent Ebenso ergibt sich für parallele Polarisation ein Brechungsindex von 

\begin{equation}
    n = \pm \left( \frac{1}{2 \cos^2(\alpha) (E_{\text{r}_\parallel} - E_\text{e})^2} \pm \sqrt{\frac{1}{4 \cos^4(\alpha) (E_{\text{r}_\parallel} - E_\text{e})^4} - \tan^2(\alpha) \left(\frac{E_{\text{r}_\parallel} + E_\text{e}}{E_{\text{r}_\parallel} - E_\text{e}} \right)^2} \right)
\end{equation}

\noindent Es wird im Folgenden für $E_{\text{r}_\parallel}$ und $E_{r_\bot}$ das E-Feld genutzt, welches vor dem Polarisationsfilter die Welle beschreibt. Also jenes aus $I_0$.
Nun kann der Brechungsindex für jeden Winkel ausgerechnet werden. Idealerweise sollte dieser konstant und unabhängig vom Winkel sein. Dies wird nun überprüft. Dazu werden 
die Brechungsindizes gegen den jeweiligen Winkel aufgetragen. Zunächst wird das orthogonal polarisierte Licht betrachtet.
Der Brechungsindex wird zunächst unverändert in einem halblogarithmischen Diagramm aufgetragen. Jedoch sind deutliche Veränderungen zu erkennen. 
Da der Brechungsindex mit zunehmenden Winkel größer wird, wird überprüft, ob ein exponentielles Verhalten vorliegt. Aus diesem Grund wird der Logarithmus 
des Brechungsindizes gezogen. Wenn ein exponentielles Verhalten vorliegt, sollten nun eine gerade zu erkennen sein. Dies ist nicht der Fall, jedoch ist eine Parabel zu erkennen. 
Aus diesem Grund wird die Wurzel des bereits logarithmierten Wert gezogen und erneut eine Auftragung durchgeführt. 

\begin{figure}[H]
    \centering
    \includegraphics[width=0.9\textwidth]{build/Brechungsindex_orthogonal.pdf}
    \caption{Brechungsindizes von senkrecht polarisiertem Licht unverändert, logarithmiert, und Wurzel des Logarithmus aufgetragen.}
\end{figure}

\noindent Nun ist eine gerade zu erkennen. Der Brechungsindex ist also Proportional zu $\text{e}^{ax^2}$. Die gleiche Auswertung wird für das parallel polarisierte Licht 
wiederholt. 

\begin{figure}[H]
    \centering
    \includegraphics[width=0.9\textwidth]{build/Brechungsindex_parallel.pdf}
    \caption{Brechungsindizes von parallel polarisiertem Licht unverändert, logarithmiert, und Wurzel des Logarithmus aufgetragen.}
\end{figure}

\noindent Hier kann leider keine solche Abhängigkeit festgestellt werden. Auch für verschiedene Wurzeln und Basen des Logarithmus wird keine Abhängigkeit gefunden.
Für die parallele Polarisation kann keine exponentielle Abhängigkeit festgestellt werden. 
Wichtig ist jedoch festzuhalten, dass der Brechungsindex in beiden Fällen von $\alpha$ abhängt und sich stark ändert. 
Weil auf diese Weise es nicht möglich ist den Brechungsindex herauszufinden, müssen die Messdaten mithilfe von \eqref{eqn:19} und \eqref{eqn:21} gefittet werden. 
Zu beachten ist dabei, dass auf der y-Achse $\sqrt{\frac{I}{I_0}}$ aufgetragen wird. Aus diesem Grund müssen die Gleichungen durch 
$\sqrt{I_0} = E_{\text{e}_\bot} = E_{\text{e}_\parallel} =  E_\text{e}$ geteilt werden.
Der Fit dieser Gleichungen führt zu keinem zufriedenstellendem Ergebnis, weshalb ein Korrekturfaktor $a$ eingeführt wird. Da die Wurzel der Intensitäten gefittet wird und diese 
das Vorzeichen nicht ändern kann, muss der Betrag genommen werden. So werden die Gleichungen umgestellt zu

\begin{align}
    \sqrt{\frac{I_{\text{r}_\bot}}{I_0}} &= E_{\text{r}_\bot} = a \left| \frac{\left( \sqrt{n^2 - \sin^2(\alpha)} - \cos(\alpha) \right)^2}{ n^2 -1 } \right| 
    \label{eqn:fit_senk}\\
    \sqrt{\frac{I_{\text{r}_\parallel}}{I_0}} &= E_{\text{r}_\parallel} = a \left| \frac{n^2 \cos(\alpha) - \sqrt{n^2 - \sin^2(\alpha)} }{ n^2 \cos(\alpha) + \sqrt{n^2 - \sin^2(\alpha)}} \right|
    \label{eqn:fit_para}
\end{align}

\noindent Diese Gleichungen genügen nun allen Anforderungen, um einen Fit durchzuführen. Zunächst werden alle Werte mit in den Fit genommen. Es stellt sich allerdings als sinnvoll 
heraus, die letzten beiden Werte nicht zu inkludieren. Die Werte werden also korrigiert, indem die letzten beiden Messpunkte nicht beachtet werden.


\begin{figure}[H]
    \centering
    \includegraphics[width=0.9\textwidth]{build/I0_fit.pdf}
    \caption{Fit von Gleichung \eqref{eqn:fit_senk} und Gleichung \eqref{eqn:fit_para} an die Messdaten, zur Bestimmung des Brechungsindex. }
\end{figure}

\noindent Die berechneten Parameter aus dem Fit ergeben die folgenden Brechungsindizes:

\begin{align*}
    n_\parallel &= \num{3.64 \pm 0.06}\\
    n_\bot      &= \num{3.48 \pm 0.12}
\end{align*}

\noindent Die eingeführten Korrekturfaktoren betragen  

\begin{align*}
    a_\parallel &= \num{0.555 \pm 0.009}\\
    a_\bot      &= \num{0.662 \pm 0.008}.
\end{align*}

\noindent Zum Schluss wird der Brechungsindex über Umstellen der Gleichung \eqref{eqn:brews_tan} bestimmt. Dies ergibt einen Wert von 

\begin{align*}
    n_\text{tan} = \num{3.49\pm0.11}.
\end{align*}


%\end{document}
