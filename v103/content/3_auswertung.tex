%\documentclass[
  bibliography=totoc,     % Literatur im Inhaltsverzeichnis
  captions=tableheading,  % Tabellenüberschriften
  titlepage=firstiscover, % Titelseite ist Deckblatt
]{scrartcl}

% Paket float verbessern
\usepackage{scrhack}

% Warnung, falls nochmal kompiliert werden muss
\usepackage[aux]{rerunfilecheck}

% unverzichtbare Mathe-Befehle
\usepackage{amsmath}
% viele Mathe-Symbole
\usepackage{amssymb}
% Erweiterungen für amsmath
\usepackage{mathtools}

% Fonteinstellungen
\usepackage{fontspec}
% Latin Modern Fonts werden automatisch geladen
% Alternativ zum Beispiel:
%\setromanfont{Libertinus Serif}
%\setsansfont{Libertinus Sans}
%\setmonofont{Libertinus Mono}

% Wenn man andere Schriftarten gesetzt hat,
% sollte man das Seiten-Layout neu berechnen lassen
\recalctypearea{}

% deutsche Spracheinstellungen
\usepackage[ngerman]{babel}


\usepackage[
  math-style=ISO,    % ┐
  bold-style=ISO,    % │
  sans-style=italic, % │ ISO-Standard folgen
  nabla=upright,     % │
  partial=upright,   % │
  mathrm=sym,        % ┘
  warnings-off={           % ┐
    mathtools-colon,       % │ unnötige Warnungen ausschalten
    mathtools-overbracket, % │
  },                       % ┘
]{unicode-math}

% traditionelle Fonts für Mathematik
\setmathfont{Latin Modern Math}
% Alternativ zum Beispiel:
%\setmathfont{Libertinus Math}

\setmathfont{XITS Math}[range={scr, bfscr}]
\setmathfont{XITS Math}[range={cal, bfcal}, StylisticSet=1]

% Zahlen und Einheiten
\usepackage[
  locale=DE,                   % deutsche Einstellungen
  separate-uncertainty=true,   % immer Unsicherheit mit \pm
  per-mode=symbol-or-fraction, % / in inline math, fraction in display math
]{siunitx}

% chemische Formeln
\usepackage[
  version=4,
  math-greek=default, % ┐ mit unicode-math zusammenarbeiten
  text-greek=default, % ┘
]{mhchem}

% richtige Anführungszeichen
\usepackage[autostyle]{csquotes}

% schöne Brüche im Text
\usepackage{xfrac}

% Standardplatzierung für Floats einstellen
\usepackage{float}
\floatplacement{figure}{htbp}
\floatplacement{table}{htbp}

% Floats innerhalb einer Section halten
\usepackage[
  section, % Floats innerhalb der Section halten
  below,   % unterhalb der Section aber auf der selben Seite ist ok
]{placeins}

% Seite drehen für breite Tabellen: landscape Umgebung
\usepackage{pdflscape}

% Captions schöner machen.
\usepackage[
  labelfont=bf,        % Tabelle x: Abbildung y: ist jetzt fett
  font=small,          % Schrift etwas kleiner als Dokument
  width=0.9\textwidth, % maximale Breite einer Caption schmaler
]{caption}
% subfigure, subtable, subref
\usepackage{subcaption}

% Grafiken können eingebunden werden
\usepackage{graphicx}

% schöne Tabellen
\usepackage{tabularray}
\UseTblrLibrary{booktabs, siunitx}

% Verbesserungen am Schriftbild
\usepackage{microtype}

% Literaturverzeichnis
\usepackage[
  backend=biber,
]{biblatex}
% Quellendatenbank
\addbibresource{lit.bib}
\addbibresource{programme.bib}

% Hyperlinks im Dokument
\usepackage[
  german,
  unicode,        % Unicode in PDF-Attributen erlauben
  pdfusetitle,    % Titel, Autoren und Datum als PDF-Attribute
  pdfcreator={},  % ┐ PDF-Attribute säubern
  pdfproducer={}, % ┘
]{hyperref}
% erweiterte Bookmarks im PDF
\usepackage{bookmark}

% Trennung von Wörtern mit Strichen
\usepackage[shortcuts]{extdash}

\author{%
  Vincent Wirsdörfer\\%
  \href{mailto:vincent.wirsdoerfer@udo.edu}{authorA@udo.edu}%
  \and%
  Joris Daus\\%
  \href{mailto:joris.daus@udo.edu}{authorB@udo.edu}%
}
\publishers{TU Dortmund – Fakultät Physik}


%\begin{document}
\section{Auswertung}
Um den Elastizitätsmodul zu berechnen, wird das Verfahren der lineare Regression gewählt. Dieses Verfahren wird gewählt, 
da es weniger anfällig für systematische Messfehler ist. Die Parameter der linearen Funktion werden per polyfit in Python 
bestimmt. Dazu werden die Messdaten 
zunächst linearisiert. Im folgendenden bezeichnet $x$ den Abstand zur Einspannung.
Die einseitige Einspannung wird linearisiert, indem $x_\text{lin} = L x^2 - \frac{x^3}{3}$ 
gewählt wird und die Messdaten gegen $x_\text{lin}$ aufgetragen werden.
Die beidseitige Linearisierung muss in zwei aufgeteilt werden. Für $0 \leq x \leq \frac{L}{2} $ wird 
$x_\text{lin} = 3L^2 x - 4x^3 $ gewählt. Die zweite Hälfte $\frac{L}{2} \leq x \leq L$ wird durch 
$x_\text{lin} = 4x^3 - 12 L x^2 + 9L^2 x -L^3 $ linearisiert.
Diese Linearisierung stellt den Vorteil, dass die Gleichungen \eqref{eqn:Biegung_einseitig}, \eqref{eqn:Ende1} und 
\eqref{eqn:Ende2} nun eine lineare Funktion sind. 

\noindent Um den Elastizitätsmodul für die einseitige Einspannung zu berechnen muss Gleichung \eqref{eqn:Biegung_einseitig} 
umgestellt werden. Dadurch ergibt sich 
\begin{equation}
    E=\frac{F}{2Im}.
    \label{eqn:E_e}
\end{equation}

\noindent Bei der beidseitigen Messung ergibt sich eine leicht andere Formel. Diese Folgt aus den Gleichungen \eqref{eqn:Ende1} 
und \eqref{eqn:Ende2}. Durch die Linearisierung ergeben sie den selben Vorfaktor, weshalb es nur eine Vorschrift für 
den Elastizitätsmodul gibt.

\begin{equation}
    E = \frac{F}{48 I m}
    \label{eqn:E_b}
\end{equation}

\noindent $m$ ist dabei die Steigung der Ausgleichsgrade und $I$ das Flächenträgheitsmoment für den jeweiligen Stab. 
$F$ ist die durch das angehängte Gewicht verursachte Gewichtskraft.

\label{sec:Auswertung}

\subsection{Stab mit kreisförmigem Querschnitt}
\label{sec:rund}
Der kreisförmige Stab besitzt folgende Kenngrößen:
\begin{align*}
    m_\text{Kreis} &= \left(0.4123 \pm 0.00005\right)\,\unit{\kilo \gram} \\
    L_\text{Kreis} &= \left(0.59 \pm 0.005\right)\,\unit{meter}\\
    r &= \left(0.01 \pm 0.000005\right) \, \unit{\meter}
\end{align*}


Zu Beginn wird der runde Stab wie in der Durchführung beschrieben vermessen. 
Die gemessene Auslenkung $D(x)$ wird in 
Abhängigkeit von dem Abstand $x$ zum Einspannort in Tabelle \ref{tab:K_e} aufgetragen.
Demnach gilt für den runden Stab:

\begin{table}[H]
    \centering
    \label{tab:K_e}
    \sisetup{ per-mode=reciprocal}
    \begin{tblr}{
        colspec = {S[table-format=2.1] S[table-format=3.1]},
        row{1} = {guard, mode=math},
        }
        \toprule
        x\, \mathbin{/} \unit{\centi \meter} & 
        D \left(x \right) 10^{-5} \mathbin{/} \unit{\meter}\\
        \midrule
        2.7     &   2.5     \\
        5.0     &   7.0     \\
        7.5     &   13.5    \\
        10.0    &   22.5    \\
        12.5    &   33.5    \\
        15.0    &   45.5    \\
        17.5    &   59.5    \\
        20.0    &   76.0    \\
        22.5    &   94.5    \\
        25.0    &   113.0   \\
        27.5    &   135.5   \\
        30.0    &   157.5   \\
        32.5    &   178.5   \\
        35.0    &   203.0   \\
        37.5    &   225.5   \\
        \bottomrule
    \end{tblr}
    \caption{Einseitige Biegung des runden Stabes.}
\end{table}

\noindent Die Daten aus Tabelle \ref{tab:K_e} werden linearisiert und aufgetragen. Außerdem wird eine Ausgleichsgrade über die 
lineare Regression berechnet und eingezeichnet.

\begin{figure}[H]
    \centering
    \includegraphics[height=8cm]{kreis_e.pdf}       
    \caption{Einseitige Biegung des runden Stabes.}
    \label{fig:K_e}
\end{figure}

\noindent Die lineare Regression liefert für die Steigung $m$ einen Wert von \\
$m_\text{einseitig} = \left( 0.03909 \pm 0.00016 \right) \unit[per-mode=reciprocal]{\per \meter \squared}$. Daraus ergibt sich mithilfe 
von Gleichung \eqref{eqn:E_e} 
\begin{equation}
    E_\text{einseitig} = \left( 127.9 \pm 0.5 \right) \unit{\kilo \newton \per \milli \meter \squared}
\end{equation}
 
\noindent Die beidseitige Messung mit dem Gewicht in der Mitte gibt zwei Messreihen. Eine Messreihe ist für links vom Gewicht, 
die andere Messreihe ist rechts vom Gewicht. Das Gewicht ist in der Mitte. 

\begin{table}[H]
    \centering
    \label{tab:Beidseitig_kreisfoermig}
    \sisetup{table-format=1.2, per-mode=reciprocal}
    \begin{tblr}{
        colspec = {S S S S},
        row{1} = {guard, mode=math},
        }
        \toprule
        x_\text{links} \mathbin{/} \unit{\centi \meter} & 
        D\left(x_\text{links}\right) 10^{-5} \mathbin{/} \unit{\meter}& 
        x_\text{rechts} \mathbin{/} \unit{\centi \meter} & 
        D\left(x_\text{rechts}\right) 10^{-5} \mathbin{/} \unit{\meter}\\    
        \midrule
        30.6    &    23.5   &   2.7    &    2.0     \\   
        32.9    &    24.5   &   5.0    &    2.5     \\   
        35.4    &    24.5   &   7.5    &    5.5     \\   
        37.9    &    22.5   &   10.0   &    8.0     \\    
        40.4    &    20     &   12.5   &    11      \\
        42.9    &    27     &   15.0   &    13      \\
        45.4    &    23     &   17.5   &    16      \\
        47.9    &    19.5   &   20.0   &    19      \\
        50.4    &    10     &   22.5   &    22.5    \\   
        52.7    &    10     &   25.0   &    22.5    \\   
    \end{tblr}
    \caption{Beidseitige Biegung des runden Stabes.}
\end{table}

\noindent Da beide Messreihen unterschiedlich linearisiert werden müssen, werden zwei Auftragungen benötigt. 

\begin{figure}[H]
    \centering
    \includegraphics[height=8cm]{kreis_b_n.pdf}
    \caption{Beidseitige Biegung des runden Stabes.}
    \label{fig:K_b_n}
\end{figure}

\begin{figure}[H]
    \centering
    \includegraphics[height=8cm]{kreis_b_f.pdf}
    \caption{Beidseitige Biegung des runden Stabes.}
    \label{fig:K_b_f}
\end{figure}

\noindent Wieder wird per linearer Regression die Steigung $m$ ermittelt. So ergibt sich aus Abbildung \ref{fig:K_b_n} eine Steigung
von $m_\text{rechts}= \left( 0.00154 \pm 0.00011\right) \unit[per-mode=reciprocal]{\per \meter \squared} $ und aus Abbildung \ref{fig:K_b_f} eine Steigung von 
$m_\text{links}= \left( 0.00095\pm0.00025 \right) \unit[per-mode=reciprocal]{\per \meter \squared}$.

\noindent Durch die Steigungen lassen sich die Elastizitätsmoduln für die jeweilige Seite berechnen.

\begin{align}
    E_\text{links} & = \left( 440 \pm 120\right) \unit{\kilo \newton \per \milli \meter \squared} &
    E_\text{rechts} & = \left(271 \pm 19\right) \unit{\kilo \newton \per \milli \meter \squared}
\end{align}


\subsection{Stab mit quadratischem Querschnitt}
Der Quadratische Stab besitzt die folgenden Messgrößen
\begin{align*}
    m_\text{Quadrat} &= \left(0.5361 \pm 0.00005 \right)\,\unit{\kilo \gram} \\
    L_\text{Quadrat} &= \left(0.6 \pm 0.005 \right)\,\unit{meter}\\
    a &= \left(0.01 \pm 0.000005\right) \, \unit{\meter}
\end{align*}

Für den quadratischen Stab wird das Experiment identisch wiederholt. 
So ergeben sich auf identischem Weg folgenden Werte für die Steigung:

\begin{align}
    m_\text{einseitig} = \left(0.02486 \pm 0.00016\right)\unit{\per \meter \squared}  \\
    m_\text{links} = \left(0.00101\pm0.00003\right)\unit{\per \meter \squared}        \\
    m_\text{rechts} = \left(0.00102\pm0.00008\right)\unit{\per \meter \squared}
\end{align}

\noindent aus diesen Werten werden ebenfalls identisch die Elastizitätsmoduln berechnet.

\begin{align}
    E_\text{einseitig} &=\left(118.5   \pm 8\right)\unit{\kilo \newton \per \milli \meter \squared}    \\
    E_\text{links} &=    \left(243    \pm 7 \right)\unit{\kilo \newton \per \milli \meter \squared}     \\
    E_\text{rechts} &=   \left(241    \pm 18\right)\unit{\kilo \newton \per \milli \meter \squared}    
\end{align}

\noindent Im Folgenden werden die Daten und Abbildungen aufgelistet, aus denen wie in \autoref{sec:rund} die Werte berechnet werden.

\begin{table}[H]
    \centering
    \label{tab:K_e}
    \sisetup{ per-mode=reciprocal}
    \begin{tblr}{
        colspec = {S[table-format=2.1] S[table-format=3.1]},
        row{1} = {guard, mode=math},
        }
        \toprule
        x\, \mathbin{/} \unit{\centi \meter} & 
        D \left(x \right) 10^{-5} \mathbin{/} \unit{\meter}\\
        \midrule
        2.6     &   1.5     \\
        5.0     &   5.0     \\
        7.5     &   8.5     \\
        10.0    &   14.0    \\    
        12.5    &   20.5    \\    
        15.0    &   29.0    \\    
        17.5    &   38.5    \\    
        20.0    &   49.0    \\    
        22.5    &   60.5    \\
        25.0    &   74.0    \\
        27.5    &   85.0    \\
        30.0    &   99.5    \\
        32.5    &   119.0   \\
        35.0    &   131.0   \\
        37.5    &   147.0   \\
        40.0    &   160.5   \\
        42.5    &   174.5   \\
        45.0    &   196.0   \\
        47.5    &   208.0   \\
        49.4    &   225.0   \\
        \bottomrule
    \end{tblr}
    \caption{Einseitige Biegung des quadratischen Stabes.}
\end{table}

\begin{table}[H]
    \centering
    \label{tab:beidseitig_quadratisch}
    \sisetup{table-format=1.2, per-mode=reciprocal}
    \begin{tblr}{
        colspec = {S S S S},
        row{1} = {guard, mode=math},
        }
        \toprule
        x_\text{links} \mathbin{/} \unit{\centi \meter} & 
        D\left(x_\text{links}\right) 10^{-5} \mathbin{/} \unit{\meter}& 
        x_\text{rechts} \mathbin{/} \unit{\centi \meter} & 
        D\left(x_\text{rechts}\right) 10^{-5} \mathbin{/} \unit{\meter}\\    
        \midrule
        32.5    &   16  &   2.6     &   0.0     \\
        35.0    &   16  &   5.0     &   1.0     \\
        37.5    &   15  &   7.5     &   3.0     \\
        40.0    &   13  &   10.0    &   4.5     \\
        42.5    &   11  &   12.5    &   5.5     \\
        45.0    &   9.5 &   15.0    &   8.0     \\
        47.5    &   7   &   17.5    &   9.5     \\
        50.0    &   6   &   20.0    &   12.0    \\
        52.5    &   3   &   22.5    &   13.5    \\
        55.0    &   0   &   25.0    &   15.0    \\
        \bottomrule
    \end{tblr}
    \caption{Beidseitige Biegung des quadratischen Stabes.}
\end{table}

\begin{figure}[H]
    \centering
    \includegraphics[height=8cm]{quadrat_E.pdf}
    \caption{Einseitige Biegung des quadratischen Stabes.}
    \label{fig:Q_b_n}
\end{figure}

\begin{figure}[H]
    \centering
    \includegraphics[height=8cm]{quadrat_b_n.pdf}
    \caption{Beidseitige Biegung des quadratischen Stabes.}
    \label{fig:Q_b_n}
\end{figure}

\begin{figure}[H]
    \centering
    \includegraphics[height=8cm]{quadrat_b_f.pdf}
    \caption{Beidseitige Biegung des quadratischen Stabes.}
    \label{fig:Q_b_f}
\end{figure}



%\end{document}