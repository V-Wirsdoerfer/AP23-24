\documentclass[
  bibliography=totoc,     % Literatur im Inhaltsverzeichnis
  captions=tableheading,  % Tabellenüberschriften
  titlepage=firstiscover, % Titelseite ist Deckblatt
]{scrartcl}

% Paket float verbessern
\usepackage{scrhack}

% Warnung, falls nochmal kompiliert werden muss
\usepackage[aux]{rerunfilecheck}

% unverzichtbare Mathe-Befehle
\usepackage{amsmath}
% viele Mathe-Symbole
\usepackage{amssymb}
% Erweiterungen für amsmath
\usepackage{mathtools}

% Fonteinstellungen
\usepackage{fontspec}
% Latin Modern Fonts werden automatisch geladen
% Alternativ zum Beispiel:
%\setromanfont{Libertinus Serif}
%\setsansfont{Libertinus Sans}
%\setmonofont{Libertinus Mono}

% Wenn man andere Schriftarten gesetzt hat,
% sollte man das Seiten-Layout neu berechnen lassen
\recalctypearea{}

% deutsche Spracheinstellungen
\usepackage[ngerman]{babel}


\usepackage[
  math-style=ISO,    % ┐
  bold-style=ISO,    % │
  sans-style=italic, % │ ISO-Standard folgen
  nabla=upright,     % │
  partial=upright,   % │
  mathrm=sym,        % ┘
  warnings-off={           % ┐
    mathtools-colon,       % │ unnötige Warnungen ausschalten
    mathtools-overbracket, % │
  },                       % ┘
]{unicode-math}

% traditionelle Fonts für Mathematik
\setmathfont{Latin Modern Math}
% Alternativ zum Beispiel:
%\setmathfont{Libertinus Math}

\setmathfont{XITS Math}[range={scr, bfscr}]
\setmathfont{XITS Math}[range={cal, bfcal}, StylisticSet=1]

% Zahlen und Einheiten
\usepackage[
  locale=DE,                   % deutsche Einstellungen
  separate-uncertainty=true,   % immer Unsicherheit mit \pm
  per-mode=symbol-or-fraction, % / in inline math, fraction in display math
]{siunitx}

% chemische Formeln
\usepackage[
  version=4,
  math-greek=default, % ┐ mit unicode-math zusammenarbeiten
  text-greek=default, % ┘
]{mhchem}

% richtige Anführungszeichen
\usepackage[autostyle]{csquotes}

% schöne Brüche im Text
\usepackage{xfrac}

% Standardplatzierung für Floats einstellen
\usepackage{float}
\floatplacement{figure}{htbp}
\floatplacement{table}{htbp}

% Floats innerhalb einer Section halten
\usepackage[
  section, % Floats innerhalb der Section halten
  below,   % unterhalb der Section aber auf der selben Seite ist ok
]{placeins}

% Seite drehen für breite Tabellen: landscape Umgebung
\usepackage{pdflscape}

% Captions schöner machen.
\usepackage[
  labelfont=bf,        % Tabelle x: Abbildung y: ist jetzt fett
  font=small,          % Schrift etwas kleiner als Dokument
  width=0.9\textwidth, % maximale Breite einer Caption schmaler
]{caption}
% subfigure, subtable, subref
\usepackage{subcaption}

% Grafiken können eingebunden werden
\usepackage{graphicx}

% schöne Tabellen
\usepackage{tabularray}
\UseTblrLibrary{booktabs, siunitx}

% Verbesserungen am Schriftbild
\usepackage{microtype}

% Literaturverzeichnis
\usepackage[
  backend=biber,
]{biblatex}
% Quellendatenbank
\addbibresource{lit.bib}
\addbibresource{programme.bib}

% Hyperlinks im Dokument
\usepackage[
  german,
  unicode,        % Unicode in PDF-Attributen erlauben
  pdfusetitle,    % Titel, Autoren und Datum als PDF-Attribute
  pdfcreator={},  % ┐ PDF-Attribute säubern
  pdfproducer={}, % ┘
]{hyperref}
% erweiterte Bookmarks im PDF
\usepackage{bookmark}

% Trennung von Wörtern mit Strichen
\usepackage[shortcuts]{extdash}

\author{%
  Vincent Wirsdörfer\\%
  \href{mailto:vincent.wirsdoerfer@udo.edu}{authorA@udo.edu}%
  \and%
  Joris Daus\\%
  \href{mailto:joris.daus@udo.edu}{authorB@udo.edu}%
}
\publishers{TU Dortmund – Fakultät Physik}


\begin{document}
\section{Diskussion}
\label{sec:Diskussion}

Um nun die im Versuch ermittelten Werte der Elastizitätsmoduln mit der Literatur vergleichen können, muss
zunächst untersucht werden, um welches Material es sich handelt. Dies geschieht durch die Berechnung der Stoffdichte $\rho$.
Dies lässt sich durch das Verhältnis aus der Masse $m$ und des Volumens $V$ der Stäbe bestimmen:

\begin{equation*}
    \rho = \frac{m}{V}
\end{equation*}

\noindent Für den Stab mit kreisförmigem Querschnitt bedeutet dies:

\begin{equation}
    \rho_\text{Kreis} = \frac{m_\text{Kreis}}{\pi\cdot r²\cdot L_\text{Kreis}} = (8900 \pm 80)\,\unit{\kilo\gram\per\cubic\meter}
\end{equation}

\noindent Hierbei beschreibt $r$ den Radius des Querschnitts.\\
Analog kann somit die Dichte des Stabes mit quadratischem Querschnitt wie folgt formuliert werden:

\begin{equation}
    \rho_\text{Quadrat} = \frac{m_\text{Quadrat}}{a²\cdot L_\text{Quadrat}} = (8940 \pm 70)\,\unit{\kilo\gram\per\cubic\meter}
\end{equation}

\noindent Der Audruck $a²$ steht für die Querschnittsfläche.\\
Aus der Literatur \cite{Dichte_Kupfer} wird entnommen, dass die berechneten Werte der Dichte nah an dem Literaturwert von Kupfer liegen.
Der Literaturwert ist $\rho_\text{Kupfer} = 8920\,\unit{\kilo\gram\per\cubic\meter}$. Dementsprechend muss zur Bewertung der Reliabilität der
empirischen Daten, das Elastizitätsmodul von Kupfer näher betrachtet werden. Der dazugehörige Literaturwert \cite{Modul_Kupfer} liegt bei
$E_\text{Kupfer} = 120\,\unit{\kilo\newton\per\milli\meter\squared}$\\\\

\noindent Es fällt auf, dass die ermittelten Werte des Elastizitätsmoduls mittels der einseitigen Einspannung bei beiden Stäben nah am 
tatsächlichen Wert des Moduls liegen:

\begin{gather}
\label{eqn:gute_Werte}
    E_\text{einseitig} = \left(127.9   \pm 0.5  \right)\unit{\kilo \newton \per \milli \meter \squared}\quad \Rightarrow \quad \text{Rel. Abweichung} \equiv 6.58\,\unit{\percent}\\
    E_\text{einseitig} = \left(118.5   \pm 0.008\right)\unit{\kilo \newton \per \milli \meter \squared}\quad \Rightarrow \quad \text{Rel. Abweichung} \equiv 1.25\,\unit{\percent}
\end{gather}

\noindent Diese Ergebnisse können insofern gerechtfertigt werden, da die einzelnen Messungen der Durchbiegung $D(x)$ über die komplette Länge 
des Stabes aufgenommen wird. Zusätzlich wird der hohen Sensitivität der Messuhren mit einer genauen Eichung jener entgegnet.
Nichtsdestotrotz zeigen die Werte aus \eqref{eqn:gute_Werte} auch auf, dass der Wert des Stabes mit quadratischem Querschnitt näher an dem
Literaturwert liegt als der Wert des Stabes mit kreisförmigem Querschnitt. Dies kann durch systematische Fehler begründet werden, da zum Beispiel das
\emph{Tasträdchen} der Messuhren einen stabileren Kontakt zur kantigen, als zur abgerundeten Oberfläche des Stabes hat.\\
Dieser Fehler ist jedoch vergleichsweise klein im Gegensatz zur beidseitigen Einspannungsmethode.\\\\
Für die beidseitige Einspannung des Stabes mit kreisförmiger Querschnittsfläche ergeben sich folgende Abweichungen:

\begin{gather}
\label{eqn:schlechte_Werte_k}
    E_\text{links} =   \left( 243    \pm 7\right)\unit{\kilo \newton \per \milli \meter \squared}\quad \Rightarrow \quad \text{Rel. Abweichung} \equiv 102.5\,\unit{\percent}\\
    E_\text{rechts} =  \left( 241    \pm 18\right)\unit{\kilo \newton \per \milli \meter \squared}\quad \Rightarrow \quad \text{Rel. Abweichung} \equiv 100.8\,\unit{\percent}
\end{gather}
\noindent
Diese signifikanten Abweichungen vom Literaturwert können durch eine Vielzahl von systematischen Fehlern der Versuchsapperatur erklärt werden.
Zum Einen liegt in vielerlei Hinsicht eine Asymmetrie der beidseitigen Einspannung vor. Während der Stab auf der linken Seite der Apperatur nur auf dem
Fußpunkt B aufliegt, so wird er auf der rechten Seite zusätzlich durch die Spannvorrichtung C befestigt. Dies sorgt für einen immensen Unterschied,
was die jeweilige Stabilität und Durchbiegungsfähigkeit des Stabes betrifft. Ferner wird nicht gewährleistet, dass die ohnehin schon hochempfindlichen 
Messuhren, äquivalente Ausschläge anzeigen. Bereits diese Fehler zeigen in den Tabllen \ref{tab:beidseitig_kreisfoermig} und \ref{tab:beidseitig_quadratisch}
markante Unterschiede der Durchbiegung der linken und rechten Einspanung der Stäbe.\\
Außerdem wird das Gewicht in der augenscheinlichen Mitte des Einspannungsabstandes aufgehangen, was nicht unmittelbar mit der Längenskala auf der Apperatur
in Verbindung gebracht werden kann. Um die Messuhren nach jeder einzelnen Messung erneut zu eichen, muss das Gewicht dementsprechend oft abgehangen und aufgehangen 
werden. Diese führt zu wiederholten Parallexenfehlern, welche sich im Endresultat der Elastizitätsmoduln widerspiegeln. 
\end{document}
