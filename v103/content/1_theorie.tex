\input{../../header.tex}

\begin{document}
\section{Zielsetzung}


Das konkrete Ziel des folgend protokollierten Versuchs besteht darin den Elastizitätsmodul von Metallen verschiedener
Legierungen zu bestimmen. Ferner werden die empirisch ermittelten Werte mit der Literatur verglichen und somit deren
Aussagekraft bewertet.

\section{Theorie}
\label{sec:Theorie}
An einem Körper lässt sich genau dann eine Veränderung der Gestalt und des Volumens feststellen, wenn eine Kraft auf die
Oberfläche des Körpers einwirkt. Oftmals bezieht man diese Krafteinwirkung auf Flächeneinheiten, weswegen die physikalisch 
erhaltende Größe in diesem Fall die Spannung ist. Dabei muss jedoch zwischen zwei Spannungskomponenten unterschieden werden.
Einerseits gibt es den senkrecht zur Oberfläche stehenden Anteil der Spannung, welcher als \emph{Normalenspannung} $\sigma$
bezeichnet wird. Andererseit existiert zudem eine parallel zur Oberfläche stehender Anteil der Spannung. Dieser Anteil
heißt \emph{Tangentialspannung}. Ist nun die relative Änderung $\sfrac{\increment L}{L}$ einer linearen Körperdimension
hinreichend klein, so lässt sich ein proportionaler Zusammenhang zwischen dieser Änderung und der angreifenden Normalenspannung
$\sigma$ konstatieren. Präziser ausgedrückt wird dieser Zusammenhang durch das \emph{Hook'sche Gesetz}:

\begin{equation}
\label{eqn:Hook}
    \sigma = E\frac{\increment L}{L}
\end{equation}

\noindent Hierbei wird der Proportionalitätsfaktor dieser Gleichung als \emph{Elastizitätsmodul} beschrieben. Dieser 
repräsentiert eine wichtige Materialkonstante in der Werkstofftechnik.\\
Anhand von Gleichung \eqref{eqn:Hook} lässt sich nun vermuten, dass das Elastizitätsmodul durch eine bestimmte Messvorrichtung
und triviales Umstellen der Gleichung einfach zu errechnen ist. Aufgrund der Tatsache, dass solch eine Vorrichtung nicht immer
gewährleistet werden kann, wird sich bei diesem Versuch einer anderen Methodik, nämlich der \emph{Biegung} von elastischen
Stäben, bedient. Die theoretischen Grundgedanken hinter diesem Ansatz werden im Folgenden näher erläutert.

\subsection{Berechnung des E-Moduls bei einseitiger Einspannung des homogenen Stabes}

Wie bereits in der Theorie angeklungen, bedingt die Dehnung des Stabes nach Gleichung \eqref{eqn:Hook} eine Biegung des Körpers
was letztlich in einer Deformation resultiert. Dementsprechend muss eine Versuchsapperatur gefunden werden, welche es erlaubt,
die Durchbiegung $D(x)$ an verschiedenen Stellen $x$ zu messen. Eine solche Apperatur wird in Abbildung \ref{fig:Durchbiegung}
dargestellt.

\begin{figure}
    \centering
    \includegraphics[height=5cm]{Durchbiegung.png}
    \caption{Durchbiegung eines elastischen Stabes bei einseitiger Aufhängung}
    \label{fig:Durchbiegung}
\end{figure}

%\noindent Es soll somit eine Menge von Wertepaaren $\{D(x)\x}$ bestimmt werden. Diese Messreihe wird dazu verwendet, den E-Modul
des Stabes zu berechnen, da die die Funktion $D(x)$ abhängig von $E$ ist. Die Gleichung muss somit nach $E$ umgesellt werden, um 
einen Wert für den Modul zu erhalten. Doch wie konkretisiert sich die Funktion $D(x)$ mathematisch genau?\\
An der Abbildung \ref{fig:Durchbiegung} ist zu erkennen, dass Kräftepaare auf den Stab einwirken, weswegen eine
\emph{Drehmomentgleichung} aufgestellt werden muss, um einen Ausdruck für $D(x)$ zu finden. Ferner zeigt die Abbildung, dass die Kraft $F$
auf den Querschnitt $Q$ ein Drehmoment $M_F$ bewirkt, welcher den Querschnitt aus seiner Ausgangslage verdreht. Dadurch werden,
die oberen Schichten des Stabes gedehnt und die unteren Schichten gestaucht. Elastische Eigenschaften des Stabes erzeugen jedoch
innere Normalenspannungen, welche dieser Biegung entgegenwirken. Zwischen der oberen und unteren Schicht existiert eine Fläche,
wo sich die Zugspannungen an der oberen Schicht und die Druckspannungen genau ausgleichen, weswegen hier in Summe keine Spannungen 
auftreten. Daher wird diese Fläche auch als \emph{neutrale Faser}\footnote{In Abb. \ref{fig:Durchbiegung} wird ihre Schnittline als gestrichelte Linie dargestellt.} bezeichnet.
Die entgegenwirkenden jedoch betragsmäßig äquivalenten Zug- und Druckspannungen erzeugen ein Drehmoment $M_\sigma$, welches sich wie folgt
berechnet lässt:

\begin{equation}
\label{eqn:Moments}
    M_\sigma = \int_Q y\sigma(y)\symup{d}q 
\end{equation}

\noindent Hierbei bezeichnet $y$ den Abstand des Flächenelements d$q$ zur neutralen Faser.\\
Wenn nun, wie in Abbildung \ref{fig:Durchbiegung} gezeigt, eine Kraft $F$, mit dem Hebelarm $L-x$ an einem Punkt des Stabes angreift, bedeutet dies
ein weiteres Drehmoment $M_F$ mit
\begin{equation}
\label{eqn:MomentF}
    M_F = F\left(L-x\right).
\end{equation}

\noindent Die Deformation stellt sich nun so sein, dass ein Gleichgewicht der Momente \eqref{eqn:Moments} und \eqref{eqn:MomentF} herrscht:

\begin{equation}
    \int_Q y\sigma(y)\symup{d}q =F\left(L-x\right)
\end{equation}



\end{document}