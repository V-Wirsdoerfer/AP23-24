\input{../../header.tex}

\begin{document}
\section{Auswertung}
\label{sec:Auswertung}

\subsection{Gravitations-Methode}

Um den Abstand der Masse zum Mittelpunkt der Kugel zu berechnen, müssen einige Größen ermittelt werden. So wird die Tiefe des Lochs, in dem der Stab eingeführt 
wird, und Höhe des Stiels, der über die Kugel hinausguckt vermessen. So entsteht der offset $r_\text{offset} = \qty{0.0051}{\meter}$ zum Mittelpunkt. 
Dieser offset wird zu dem gemessenen Abstand des Gewichtes hinzuaddiert. Das Gewicht besitzt eine Masse von $m_\text{Gewicht} = \qty{1.4e-3}{\kilo \gram}$.
Gemessen wird der Abstand $r$ zum Gewicht von Stabanfang und der Strom $I$, bei dem ein gerade so starkes Magnetfeld erzeugt wird, sodass die Kugel im 
Gleichgewicht ist. 

\begin{table*}[h]
    \centering
    \sisetup{table-format=1.2}
    \begin{tblr}{
            colspec = {S S},
            row{1} = {guard, mode=math},
        }
        \toprule
        r \mathbin{/} \unit{\centi \meter} & I \mathbin{/} \unit{\ampere} \\
        \midrule
        5.17    &   2.20    \\ 
        5.45    &   2.25    \\    
        5.76    &   2.35    \\    
        6.06    &   2.45    \\    
        6.40    &   2.50    \\    
        6.75    &   2.55    \\    
        7.16    &   2.60    \\    
        8.01    &   2.70    \\    
        8.21    &   2.85    \\    
        8.72    &   3.00    \\    
        \bottomrule
    \end{tblr}
    \caption{Benötigter Strom, um Masse im Abstand $r + r_\text{offset}$ im Gleichgewicht zu halten.}
\end{table*}

Um diese Daten auszuwerten wird Gleichung \eqref{eqn:Gravitation} umgestellt zu

\begin{equation}
    \mu_\text{Dipol} \cdot \frac{B}{m \cdot g} =  r .
    \label{eqn:linGravitation}
\end{equation}

\noindent
Trägt man nun die Werte mithilfe von Gleichung \eqref{eqn:linGravitation}  und \eqref{eqn:B0} auf und bestimmt die Ausgleichsgerade, so entsteht folgender Plot:

\begin{figure}
    \includegraphics[width=\textwidth]{../build/Gravitation.pdf}
    \caption{Lineare Auftragung von B zu r.}  
\end{figure}

\end{document}
