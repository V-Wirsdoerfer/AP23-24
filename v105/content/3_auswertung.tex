\input{../../header.tex}

\begin{document}
\section{Auswertung}
\label{sec:Auswertung}

\subsection{Gravitations-Methode}

Um den Abstand der Masse zum Mittelpunkt der Kugel zu berechnen, müssen einige Größen ermittelt werden. So wird die Tiefe des Lochs, in dem der Stab eingeführt 
wird, und Höhe des Stiels, der über die Kugel hinausguckt vermessen. So entsteht der offset $r_\text{offset} = \qty{0.0051}{\meter}$ zum Mittelpunkt. 
Dieser offset wird zu dem gemessenen Abstand des Gewichtes hinzuaddiert. Das Gewicht besitzt eine Masse von $m_\text{Gewicht} = \qty{1.4e-3}{\kilo \gram}$.
Gemessen wird der Abstand $r$ zum Gewicht von Stabanfang und der Strom $I$, bei dem ein gerade so starkes Magnetfeld erzeugt wird, sodass die Kugel im 
Gleichgewicht ist. 

\begin{table*}[h]
    \centering
    \sisetup{table-format=1.2}
    \begin{tblr}{
            colspec = {S S},
            row{1} = {guard, mode=math},
        }
        \toprule
        r \mathbin{/} \unit{\centi \meter} & I \mathbin{/} \unit{\ampere} \\
        \midrule
        5.17    &   2.20    \\ 
        5.45    &   2.25    \\    
        5.76    &   2.35    \\    
        6.06    &   2.45    \\    
        6.40    &   2.50    \\    
        6.75    &   2.55    \\    
        7.16    &   2.60    \\    
        8.01    &   2.70    \\    
        8.21    &   2.85    \\    
        8.72    &   3.00    \\    
        \bottomrule
    \end{tblr}
    \caption{Benötigter Strom, um Masse im Abstand $r + r_\text{offset}$ im Gleichgewicht zu halten.}
\end{table*}

\noindent Um diese Daten auszuwerten, muss nach Vorbild der Gleichung \eqref{eqn:Gravitation} $r$ gegen $B$ aufgetragen und eine 
Ausgleichsgerade angelegt werden.

\begin{figure}
    \includegraphics[width=\textwidth]{../build/Gravitation.pdf}
    \caption{Lineare Auftragung von B zu r.}  
\end{figure}

\noindent Die Ausgleichsgerade hat die Form 

\begin{equation}
    r = m B + b
\end{equation}

\noindent mit den Werten

\begin{align*}
    m &= \qty{0.0066 \pm 0.0004}{\unit{\meter \per \tesla}} \\
    b &= \qty{-0.048 \pm 0.008}{\unit{\meter}}.
\end{align*}

\noindent Aus der Steigung kann das magnetische Dipolmoment mithilfe von Gleichung \eqref{eqn:Gleichung} 
berechnet werden:

\begin{align}
    m &= \frac{\mu_\text{Dipol}}{m_\text{Gewicht} \cdot g} \\
    \mu_\text{Dipol} &= m \cdot g \cdot m_\text{Gewicht} = \qty{0.480 \pm 0.031}{\unit{\ampere \meter \squared}} 
\end{align}


\subsection{Schwingungs-Methode}

Da eine Schwingungsperiode sehr kurz ist, wird die Zeit von zehn Schwingungen gemessen und anschließend durch 10 geteilt, 
um auf eine Schwingungsdauer zurückzuschließen. Die Messdaten zu verschiedenen Stromstärken sind wie folgt: 

\begin{table*}[h]
    \centering
    \sisetup{table-format=1.1}
    \begin{tblr}{
            colspec = {S S[table-format=1.3]},
            row{1} = {guard, mode=math},
        }
        \toprule
        I \mathbin{/} \unit{\ampere} & T \mathbin{/} \unit{\second} \\
        \midrule
        0.7     &   2.242   \\
        0.9     &   1.866   \\
        1.1     &   1.744   \\
        1.3     &   1.580   \\
        1.5     &   1.490   \\
        1.7     &   1.338   \\
        1.9     &   1.263   \\
        2.1     &   1.185   \\
        2.3     &   1.139   \\
        2.5     &   1.070   \\  
        \bottomrule
    \end{tblr}
    \caption{Dauer einer Schwingung abhängig vom strominduzierten B-Feld.}
\end{table*}

\noindent Diese Messdaten können nun aufgetragen werden und mit Vorbild der 
Gleichung \eqref{eqn:Schwingung} durch die lineare Funktion 

\begin{equation}
    T^2 = m \cdot \frac{1}{B}
\end{equation}

\noindent gefittet werden, wobei $m$ als 


\begin{align}
    m &\coloneqq \frac{4 \pi ^2 J_\text{K}}{\mu_\text{Dipol}}
\end{align}

\noindent definiert ist.

\begin{figure}
    \includegraphics{../build/Schwingung.pdf}
    \caption{Quadrat der Schwingungsdauer aufgetragen gegen das inverse des Magnetfeldes.}
\end{figure}

\noindent Die Ausgleichsgerade besitzt die folgenden Parameter, aus denen sich auch 
das magnetische Moment berechnen lässt

\end{document}
