\documentclass[
  bibliography=totoc,     % Literatur im Inhaltsverzeichnis
  captions=tableheading,  % Tabellenüberschriften
  titlepage=firstiscover, % Titelseite ist Deckblatt
]{scrartcl}

% Paket float verbessern
\usepackage{scrhack}

% Warnung, falls nochmal kompiliert werden muss
\usepackage[aux]{rerunfilecheck}

% unverzichtbare Mathe-Befehle
\usepackage{amsmath}
% viele Mathe-Symbole
\usepackage{amssymb}
% Erweiterungen für amsmath
\usepackage{mathtools}

% Fonteinstellungen
\usepackage{fontspec}
% Latin Modern Fonts werden automatisch geladen
% Alternativ zum Beispiel:
%\setromanfont{Libertinus Serif}
%\setsansfont{Libertinus Sans}
%\setmonofont{Libertinus Mono}

% Wenn man andere Schriftarten gesetzt hat,
% sollte man das Seiten-Layout neu berechnen lassen
\recalctypearea{}

% deutsche Spracheinstellungen
\usepackage[ngerman]{babel}


\usepackage[
  math-style=ISO,    % ┐
  bold-style=ISO,    % │
  sans-style=italic, % │ ISO-Standard folgen
  nabla=upright,     % │
  partial=upright,   % │
  mathrm=sym,        % ┘
  warnings-off={           % ┐
    mathtools-colon,       % │ unnötige Warnungen ausschalten
    mathtools-overbracket, % │
  },                       % ┘
]{unicode-math}

% traditionelle Fonts für Mathematik
\setmathfont{Latin Modern Math}
% Alternativ zum Beispiel:
%\setmathfont{Libertinus Math}

\setmathfont{XITS Math}[range={scr, bfscr}]
\setmathfont{XITS Math}[range={cal, bfcal}, StylisticSet=1]

% Zahlen und Einheiten
\usepackage[
  locale=DE,                   % deutsche Einstellungen
  separate-uncertainty=true,   % immer Unsicherheit mit \pm
  per-mode=symbol-or-fraction, % / in inline math, fraction in display math
]{siunitx}

% chemische Formeln
\usepackage[
  version=4,
  math-greek=default, % ┐ mit unicode-math zusammenarbeiten
  text-greek=default, % ┘
]{mhchem}

% richtige Anführungszeichen
\usepackage[autostyle]{csquotes}

% schöne Brüche im Text
\usepackage{xfrac}

% Standardplatzierung für Floats einstellen
\usepackage{float}
\floatplacement{figure}{htbp}
\floatplacement{table}{htbp}

% Floats innerhalb einer Section halten
\usepackage[
  section, % Floats innerhalb der Section halten
  below,   % unterhalb der Section aber auf der selben Seite ist ok
]{placeins}

% Seite drehen für breite Tabellen: landscape Umgebung
\usepackage{pdflscape}

% Captions schöner machen.
\usepackage[
  labelfont=bf,        % Tabelle x: Abbildung y: ist jetzt fett
  font=small,          % Schrift etwas kleiner als Dokument
  width=0.9\textwidth, % maximale Breite einer Caption schmaler
]{caption}
% subfigure, subtable, subref
\usepackage{subcaption}

% Grafiken können eingebunden werden
\usepackage{graphicx}

% schöne Tabellen
\usepackage{tabularray}
\UseTblrLibrary{booktabs, siunitx}

% Verbesserungen am Schriftbild
\usepackage{microtype}

% Literaturverzeichnis
\usepackage[
  backend=biber,
]{biblatex}
% Quellendatenbank
\addbibresource{lit.bib}
\addbibresource{programme.bib}

% Hyperlinks im Dokument
\usepackage[
  german,
  unicode,        % Unicode in PDF-Attributen erlauben
  pdfusetitle,    % Titel, Autoren und Datum als PDF-Attribute
  pdfcreator={},  % ┐ PDF-Attribute säubern
  pdfproducer={}, % ┘
]{hyperref}
% erweiterte Bookmarks im PDF
\usepackage{bookmark}

% Trennung von Wörtern mit Strichen
\usepackage[shortcuts]{extdash}

\author{%
  Vincent Wirsdörfer\\%
  \href{mailto:vincent.wirsdoerfer@udo.edu}{authorA@udo.edu}%
  \and%
  Joris Daus\\%
  \href{mailto:joris.daus@udo.edu}{authorB@udo.edu}%
}
\publishers{TU Dortmund – Fakultät Physik}


\begin{document}
\section{Auswertung}
\label{sec:Auswertung}

\subsection{Gravitations-Methode}

Um den Abstand der Masse zum Mittelpunkt der Kugel zu berechnen, müssen einige Größen ermittelt werden. So wird die Tiefe des Lochs, in dem der Stab eingeführt 
wird, und Höhe des Stiels, der über die Kugel hinausguckt vermessen. So entsteht der offset $r_\text{offset} = \qty{0.0051}{\meter}$ zum Mittelpunkt. 
Dieser offset wird zu dem gemessenen Abstand des Gewichtes hinzuaddiert. Das Gewicht besitzt eine Masse von $m_\text{Gewicht} = \qty{1.4e-3}{\kilo \gram}$.
Gemessen wird der Abstand $r$ zum Gewicht von Stabanfang und der Strom $I$, bei dem ein gerade so starkes Magnetfeld erzeugt wird, sodass die Kugel im 
Gleichgewicht ist. 

\begin{table*}[h]
    \centering
    \sisetup{table-format=1.2}
    \begin{tblr}{
            colspec = {S S},
            row{1} = {guard, mode=math},
        }
        \toprule
        r \mathbin{/} \unit{\centi \meter} & I \mathbin{/} \unit{\ampere} \\
        \midrule
        5.17    &   2.20    \\ 
        5.45    &   2.25    \\    
        5.76    &   2.35    \\    
        6.06    &   2.45    \\    
        6.40    &   2.50    \\    
        6.75    &   2.55    \\    
        7.16    &   2.60    \\    
        8.01    &   2.70    \\    
        8.21    &   2.85    \\    
        8.72    &   3.00    \\    
        \bottomrule
    \end{tblr}
    \caption{Benötigter Strom, um Masse im Abstand $r + r_\text{offset}$ im Gleichgewicht zu halten.}
\end{table*}

\noindent Um diese Daten auszuwerten, muss nach Vorbild der Gleichung \eqref{eqn:Gravitation} $r$ gegen $B$ aufgetragen und eine 
Ausgleichsgerade angelegt werden.

\begin{figure}[H]
    \includegraphics[width=\textwidth]{../build/Gravitation.pdf}
    \caption{Lineare Auftragung von B zu r.}  
\end{figure}

\noindent Die Ausgleichsgerade hat die Form 

\begin{equation*}
    r = m B + b
\end{equation*}

\noindent mit den Werten

\begin{align*}
    m &= \qty{0.0066 \pm 0.0004}{\unit{\meter \per \tesla}} \\
    b &= \qty{-0.048 \pm 0.008}{\unit{\meter}}.
\end{align*}

\noindent Aus der Steigung kann das magnetische Dipolmoment mithilfe von Gleichung \eqref{eqn:Gleichung} 
berechnet werden:

\begin{align*}
    m &= \frac{\mu_\text{Dipol}}{m_\text{Gewicht} \cdot g} \\
    \mu_\text{Dipol} &= m \cdot g \cdot m_\text{Gewicht} = \qty{0.480 \pm 0.031}{\unit{\ampere \meter \squared}} 
\end{align*}


\subsection{Schwingungs-Methode}

Da eine Schwingungsperiode sehr kurz ist, wird die Zeit von zehn Schwingungen gemessen und anschließend durch 10 geteilt, 
um auf eine Schwingungsdauer zurückzuschließen. Die Messdaten zu verschiedenen Stromstärken sind wie folgt: 

\begin{table*}[h]
    \centering
    \sisetup{table-format=1.1}
    \begin{tblr}{
            colspec = {S S[table-format=1.3]},
            row{1} = {guard, mode=math},
        }
        \toprule
        I \mathbin{/} \unit{\ampere} & T \mathbin{/} \unit{\second} \\
        \midrule
        0.7     &   2.242   \\
        0.9     &   1.866   \\
        1.1     &   1.744   \\
        1.3     &   1.580   \\
        1.5     &   1.490   \\
        1.7     &   1.338   \\
        1.9     &   1.263   \\
        2.1     &   1.185   \\
        2.3     &   1.139   \\
        2.5     &   1.070   \\  
        \bottomrule
    \end{tblr}
    \caption{Dauer einer Schwingung abhängig vom strominduzierten B-Feld.}
\end{table*}

\noindent Diese Messdaten können nun aufgetragen werden und mit Vorbild der 
Gleichung \eqref{eqn:Schwingung} durch die lineare Funktion 

\begin{equation*}
    T^2 = m \cdot \frac{1}{B}
\end{equation*}

\noindent gefittet werden, wobei $m$ als 


\begin{align*}
    m &\coloneqq \frac{4 \pi ^2 J_\text{K}}{\mu_\text{Dipol}}
\end{align*}

\noindent definiert ist.

\begin{figure}[H]
    \includegraphics[width=\textwidth]{../build/Schwingung.pdf}
    \caption{Quadrat der Schwingungsdauer aufgetragen gegen das inverse des Magnetfeldes.}
\end{figure}

\noindent Die Ausgleichsgerade der Form $mx +b$ besitzt die folgenden Parameter, aus denen sich 
ebenfalls das magnetische Moment berechnen lässt:

\begin{align*}
    m &= \qty{0.00494 \pm 0.00015}{\second \squared \tesla}\\
    b &= \qty{-0.32\pm0.09}{\per \tesla}\\
    \mu_\text{Dipol} &= \qty{0.299 \pm 0.009}{\ampere \meter \squared}
\end{align*}

\subsection{Präzessions-Methode}

Das Stroboskop wurde auf einen konstanten Wert von $f=\qty{5.5}{\unit{\hertz}}$ eingestellt. 
Für jede Stromstärke werden drei Messungen mit jeweils einer Umdrehung durchgeführt. So 
entstehen folgende Werte:

\begin{table*}[h]
    \centering
    \sisetup{table-format=2.2}
    \begin{tblr}{
            colspec = {S[table-format=1.1] S S S},
            row{1} = {guard, mode=math},
        }
        \toprule
        I \mathbin{/} \unit{\ampere} & t \mathbin{/} \unit{\second} \\
        \midrule
        1.0     &   13.77   &   13.58   &   13.75   \\
        2.0     &   7.36    &   6.81    &   7.35    \\
        3.0     &   6.49    &   4.78    &   4.65    \\
        3.5     &   3.94    &   3.96    &   3.75    \\
        4.0     &   3.66    &   3.48    &   3.69    \\    
        \bottomrule
    \end{tblr}
    \caption{Umlaufzeiten bei eingestelltem Strom.}
\end{table*}

\newpage %vllt später unnötig
\noindent Diese Werte werden nun grafisch ausgewertet. Außerdem wird eine Ausgleichsgerade der Form $mx+b$ 
durch die Werte gefittet. 

\begin{figure}[H]
    \includegraphics[width=\textwidth]{../build/Präzession.pdf}
\end{figure}

Mit der Definition 

\begin{equation*}
    m \coloneqq \frac{\mu_\text{Dipol}}{2 \pi L_\text{K}}
\end{equation*}

\noindent kann das Dipolmoment berechnet werden. So ergeben sich die Werte:

\begin{align*}
    m &= \qty{51 \pm 4 }{\unit[per-mode=reciprocal]{\per \second \tesla}}\\
    b &= \qty{0.000 \pm 0.017}{\unit{\per \second}}\\
    \mu_\text{Dipol} &= \qty{0.42 \pm 0.04}{\unit{\ampere \meter \squared}}
\end{align*}



\end{document}
