\input{../../header.tex}

\begin{document}

\section{Zielsetzung}
\label{sec:Zielsetzung}

In dem folgend protokollierten Versuch wird das magnetischen Moment eines Permanentmagneten auf verschiedenen 
experimentellen Wegen ermittelt. Zum Einen durch die statische Methode der Gravitation. Zum Anderen über die 
dynamischen Methoden der Schwingungsdauer als auch der Präzession eines Magneten.

\section{Theorie}
\label{sec:Theorie}

\subsection{Grundlagen}
Nach der zweiten \emph{Maxwell-Gleichung}\footnote{Diese Gleichung beschreibt die Divergenzfreiheit des magnetischen Feldes
$\left(\nabla \cdot \symbf{B} = 0\right)$} ist bekannt, dass keine magnetischen Monopole existieren. Die in der 
Multipolentwicklung einfachste Form ist der \emph{magnetische Dipol}. Dieser kann beispielsweise durch einen stromdurchflossenen 
Leiter oder einen Permanentmagneten erzeugt werden. Während die Berechnung des magnetischen Moments $\vec{\mu}$ eines 
stromdurchflossenen Leiters mit Strom $I$ und Normalenvektor der Querschnittsfäche $\vec{A}$ durch 

\begin{equation*}
    \vec{\mu} = I \cdot \vec{A}
\end{equation*}

\noindent trivial, stellt die Berechnung des magnetischen Moments eines Permanentmagneten eine größere Aufgabe dar.\\
\noindent Auf einen magnetischen Dipol wirkt im homogenen Manetfeld solange das \emph{Drehmoment} $\vec{D} = \vec{\mu} \times \vec{B}\label{eqn:D_B}$,
bis die Beziehung $\vec{\mu} \parallel \vec{B}$ gilt. Um ein homogenes Magnetfeld zu realisieren, wird zum Beispiel ein 
Helmholtz-Spulenpaar verwendet, welches aus zwei gleichsinnig vom Strom $I$ durchflossenen Kreisspulen besteht. Durch geeignte 
Symmetrie- und Abstandsanpassung der Spulen kann das Magnetfeld mittels des \emph{Biot-Savart-Gesetz}

\begin{equation*}
    \vec{B}\left(\vec{r}\right) = \frac{\mu_{0}I'}{4\pi}\oint_{\Gamma'} \frac{\symup{d}\vec{l'} \times \left(\vec{r} - \vec{r'}\right)}{| \vec{r} - \vec{r'} |³}
\end{equation*}

\noindent berechnet werden. Für ein Spulenpaar mit je einer Windung, einem Spulenradius $R$ und Spulenabstand $d$ ergibt sich das B-Feld in der Mitte:

\begin{equation}
\label{eqn:B0}
    | \vec{B}(0) | = \frac{\mu_{0}IR²}{\left(R²+\left(\frac{d}{2}\right)²\right)^{\sfrac{3}{2}}}
\end{equation}

\subsection{Bestimmung des magnetischen Momentes unter Beanspruchung der Gravitation}

\noindent Eine Masse $m$ befindet sich im magntischen Feld eines Helmholtz-Spulenpaares. Auf die Masse wirkt die Gravitationskraft 
$\vec{F}_\text{G} = m \cdot \vec{g}$. Die Masse ist durch eine Aluminiumstange mit einer Billiardkugel verbunden. Dementsprechend 
wird ein Drehmoment 

\begin{equation*}
    \vec{D}_{G} = m \cdot \left(\vec{r} \times \vec{g}\right)
\end{equation*}

\noindent ausgeübt. Dabei steht $| \vec{r} |$ für den Abstand zwischen der Masse $m$ und dem Anfang des Stiels der Billiardkugel.
Zusätzlich wirkt das in Gleichung \eqref{eqn:D_B} beschriebene Drehmoment 

\begin{equation*}
    \vec{\mu}_\text{Dipol} \times \vec{B} = m \cdot \left(\vec{r} \times \vec{g}\right).
\end{equation*}

\noindent Bei geeigneter Magnetfeldstärke befinden sich die beiden Drehmomente im Gleichgewicht. Aufgrund der Parallelität von $\vec{B}$ und
$\vec{F}_\text{G}$ gilt der Zusammenhang

\begin{equation}
\label{eqn:Gravitation}
    \mu_\text{Dipol} \cdot B = m \cdot r \cdot g.
\end{equation}

\noindent Aus diesem Zusammenhang lässt sich das magnetische Moment berechnen.

\subsection{Berechnung des magnetischen Momentes über die Schwingungsdauer eines Magneten}

\noindent Wird die Billiardkugel im homogenen Magnetfeld der Helmholtz-Spulen in Schwingung versetzt, verhält sie sich 
ähnlich zu einem harmonischen Oszillator. Die dadurch erzeugte Bewegung kann somit durch die folgende Differentialgleichung
beschrieben werden:

\begin{equation*}
    - | \vec{\mu}_\text{Dipol} \times \vec{B} | = J_\text{K} \cdot \frac{\symup{d}²\theta}{\symup{d}t²}
\end{equation*}

\noindent Mittels der Periodendauer $T$ lässt sich die Lösung dieser Differentialgleichung schreiben als 

\begin{equation*}
    T² = \frac{4\pi²J_\text{K}}{\mu_\text{Dipol}}\frac{1}{B}.
\end{equation*}

\noindent Hierbei bezeichnet $J_\text{K}$ das Trägheitsmoment der Kugel.

\subsection{Bestimmung des magnetischen Momentes durch die Präzession eines Magneten}

\noindent Nun wird die Billiardkugel in Rotation versetzt und um einem kleinen Winkel ausgelenkt. Aufgrund 
der Rotation, bleibt diese Auslenkung jedoch stabil. Wirkt nun durch das Magnetfeld eine äußere Kraft auf die 
Figurenachse $\vec{L}_\text{K}$, so präzediert die Kugel. Anschaulich bedeutet dies, dass die Figurenachse auf einem Kegelmantel 
um die Drehimpulsachse $\vec{L}$. Diese Bewegung lässt sich beschreiben durch:

\begin{equation*}
    \vec{\mu}_\text{Dipol} \times \vec{B} = \frac{\text{d}\vec{L}_\text{K}}{\text{d}t}
\end{equation*}

\noindent mit der Präzessionsfrequenz 

\begin{equation*}
    \Omega_\text{p} = \frac{\mu B}{| L_\text{K} |}
\end{equation*}

\noindent als Lösung der Differentialgleichung. Aus den Beziehungen $L_\text{K} = J_\text{K}\omega$ und $\omega = 2\pi\nu
= \frac{2\pi}{T_\text{p}}$ lässt sich diese Gleichung umschreiben zu

\begin{equation}
\label{eqn:Praezession}
    \frac{1}{T_\text{p}} = \frac{\mu_\text{Dipol}}{2\pi{}L_\text{K}}B, 
\end{equation}

\noindent woraus der Wert für das magnetische Moment des Magneten abgleitet werden kann.

\section{Vorbereitung}

\subsection{Magnetfeld}
Die Berechnung der magnetischen Flußdichte $B$ in der Mitte des Helmholtz-Spulenpaares lässt sich durch die Formel 
\eqref{eqn:B0} berechnen. Die dafür notwendigen Werte der einzelnen Größen werden der Versuchsanleitung entnommen und
lauten wie folgt:

\begin{align*}
    N &= 195 \\
    d &= 0.138\,\unit{\meter} \\
    R &= 0.109\,\unit{\meter} \\
    I &= 1\,\unit{\ampere} \\
\end{align*}

\noindent Für die magnetische Permeabilität wird der Wert $\mu_0 = 1.256\,637\,062\,\unit{\newton\per\ampere\squared}$ 
verwendet, \cite{Magnetische_Feldkonstante}.\\

\noindent Daraus ergibt sich ein Wert von 

\begin{equation*}
    B = ...\,\unit{\tesla}
\end{equation*}

\noindent als magnetische Flußdichte.

\subsection{Trägheitsmoment der Kugel}

Allgemein lässt sich das Trägheitsmoment eines Körpers über das folgende Volumenintegral bestimmen:

\begin{equation*}
    J = \iiint_V \rho \cdot r_\text{a}² \symup{d}V
\end{equation*}

\noindent Hierbei steht $\rho$ für die Dichte des Körpers und $r_\text{a}²$ für den quadrierten Abstand zur Rotationsachse.
Angewandt auf die Symmetrie einer Kugel ergibt sich daraus das Trägheitsmoment einer Kugel

\begin{equation*}
    J_\text{K} = \frac{2}{5}MR²,
\end{equation*}

\noindent wobei $M$ für die Masse und $R$ für den Radius der Kugel steht. Nach Versuchsanleitung werden nun die Werte $M = 0.15\,\unit{\kilo\gram}$
und $R = 0.025\,\unit{\meter}$ eingesetzt woraus sich das Trägheitsmoment

\begin{equation*}
    J_\text{K} = 3.75 \cdot 10^{-5}\,\unit{\kilo\gram\meter\squared}
\end{equation*}

\noindent berechnen lässt.
\section{Fehlerrechnung}
\end{document}