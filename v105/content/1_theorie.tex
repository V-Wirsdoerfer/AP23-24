\documentclass[
  bibliography=totoc,     % Literatur im Inhaltsverzeichnis
  captions=tableheading,  % Tabellenüberschriften
  titlepage=firstiscover, % Titelseite ist Deckblatt
]{scrartcl}

% Paket float verbessern
\usepackage{scrhack}

% Warnung, falls nochmal kompiliert werden muss
\usepackage[aux]{rerunfilecheck}

% unverzichtbare Mathe-Befehle
\usepackage{amsmath}
% viele Mathe-Symbole
\usepackage{amssymb}
% Erweiterungen für amsmath
\usepackage{mathtools}

% Fonteinstellungen
\usepackage{fontspec}
% Latin Modern Fonts werden automatisch geladen
% Alternativ zum Beispiel:
%\setromanfont{Libertinus Serif}
%\setsansfont{Libertinus Sans}
%\setmonofont{Libertinus Mono}

% Wenn man andere Schriftarten gesetzt hat,
% sollte man das Seiten-Layout neu berechnen lassen
\recalctypearea{}

% deutsche Spracheinstellungen
\usepackage[ngerman]{babel}


\usepackage[
  math-style=ISO,    % ┐
  bold-style=ISO,    % │
  sans-style=italic, % │ ISO-Standard folgen
  nabla=upright,     % │
  partial=upright,   % │
  mathrm=sym,        % ┘
  warnings-off={           % ┐
    mathtools-colon,       % │ unnötige Warnungen ausschalten
    mathtools-overbracket, % │
  },                       % ┘
]{unicode-math}

% traditionelle Fonts für Mathematik
\setmathfont{Latin Modern Math}
% Alternativ zum Beispiel:
%\setmathfont{Libertinus Math}

\setmathfont{XITS Math}[range={scr, bfscr}]
\setmathfont{XITS Math}[range={cal, bfcal}, StylisticSet=1]

% Zahlen und Einheiten
\usepackage[
  locale=DE,                   % deutsche Einstellungen
  separate-uncertainty=true,   % immer Unsicherheit mit \pm
  per-mode=symbol-or-fraction, % / in inline math, fraction in display math
]{siunitx}

% chemische Formeln
\usepackage[
  version=4,
  math-greek=default, % ┐ mit unicode-math zusammenarbeiten
  text-greek=default, % ┘
]{mhchem}

% richtige Anführungszeichen
\usepackage[autostyle]{csquotes}

% schöne Brüche im Text
\usepackage{xfrac}

% Standardplatzierung für Floats einstellen
\usepackage{float}
\floatplacement{figure}{htbp}
\floatplacement{table}{htbp}

% Floats innerhalb einer Section halten
\usepackage[
  section, % Floats innerhalb der Section halten
  below,   % unterhalb der Section aber auf der selben Seite ist ok
]{placeins}

% Seite drehen für breite Tabellen: landscape Umgebung
\usepackage{pdflscape}

% Captions schöner machen.
\usepackage[
  labelfont=bf,        % Tabelle x: Abbildung y: ist jetzt fett
  font=small,          % Schrift etwas kleiner als Dokument
  width=0.9\textwidth, % maximale Breite einer Caption schmaler
]{caption}
% subfigure, subtable, subref
\usepackage{subcaption}

% Grafiken können eingebunden werden
\usepackage{graphicx}

% schöne Tabellen
\usepackage{tabularray}
\UseTblrLibrary{booktabs, siunitx}

% Verbesserungen am Schriftbild
\usepackage{microtype}

% Literaturverzeichnis
\usepackage[
  backend=biber,
]{biblatex}
% Quellendatenbank
\addbibresource{lit.bib}
\addbibresource{programme.bib}

% Hyperlinks im Dokument
\usepackage[
  german,
  unicode,        % Unicode in PDF-Attributen erlauben
  pdfusetitle,    % Titel, Autoren und Datum als PDF-Attribute
  pdfcreator={},  % ┐ PDF-Attribute säubern
  pdfproducer={}, % ┘
]{hyperref}
% erweiterte Bookmarks im PDF
\usepackage{bookmark}

% Trennung von Wörtern mit Strichen
\usepackage[shortcuts]{extdash}

\author{%
  Vincent Wirsdörfer\\%
  \href{mailto:vincent.wirsdoerfer@udo.edu}{authorA@udo.edu}%
  \and%
  Joris Daus\\%
  \href{mailto:joris.daus@udo.edu}{authorB@udo.edu}%
}
\publishers{TU Dortmund – Fakultät Physik}


\begin{document}

\section{Zielsetzung}
\label{sec:Zielsetzung}

In dem folgend protokollierten Versuch wird das magnetischen Moment eines Permanentmagneten auf verschiedenen 
experimentellen Wegen ermittelt. Zum Einen durch die statische Methode der Gravitation. Zum Anderen über die 
dynamischen Methoden der Schwingungsdauer als auch der Präzession eines Magneten.

\section{Theorie}
\label{sec:Theorie}

\subsection{Grundlagen}
Nach der zweiten \emph{Maxwell-Gleichung}\footnote{Diese Gleichung beschreibt die Divergenzfreiheit des magnetischen Feldes
$\left(\nabla \cdot \symbf{B} = 0\right)$} ist bekannt, dass keine magnetischen Monopole existieren. Die in der 
Multipolentwicklung einfachste Form ist der \emph{magnetische Dipol}. Dieser kann beispielsweise durch einen stromdurchflossenen 
Leiter oder einen Permanentmagneten erzeugt werden. Während die Berechnung des magnetischen Moments $\vec{\mu}$ eines 
stromdurchflossenen Leiters mit Strom $I$ und Normalenvektor der Querschnittsfäche $\vec{A}$ durch 

\begin{equation*}
    \vec{\mu} = I \cdot \vec{A}
\end{equation*}

\noindent trivial, stellt die Berechnung des magnetischen Moments eines Permanentmagneten eine größere Aufgabe dar.\\
\noindent Auf einen magnetischen Dipol wirkt im homogenen Manetfeld solange das \emph{Drehmoment} $\vec{D} = \vec{\mu} \times \vec{B}\label{eqn:D_B}$,
bis die Beziehung $\vec{\mu} \parallel \vec{B}$ gilt. Um ein homogenes Magnetfeld zu realisieren, wird zum Beispiel ein 
Helmholtz-Spulenpaar verwendet, welches aus zwei gleichsinnig vom Strom $I$ durchflossenen Kreisspulen besteht. Durch geeignte 
Symmetrie- und Abstandsanpassung der Spulen kann das Magnetfeld mittels des \emph{Biot-Savart-Gesetz}

\begin{equation*}
    \vec{B}\left(\vec{r}\right) = \frac{\mu_{0}I'}{4\pi}\oint_{\Gamma'} \frac{\symup{d}\vec{l'} \times \left(\vec{r} - \vec{r'}\right)}{| \vec{r} - \vec{r'} |³}
\end{equation*}

\noindent berechnet werden. Für ein Spulenpaar mit je einer Windung, einem Spulenradius $R$ und Spulenabstand $d$ ergibt sich das B-Feld in der Mitte:

\begin{equation*}
    | \vec{B}(0) | = \frac{\mu_{0}IR²}{\left(R²+\left(\frac{d}{2}\right)²\right)^{\sfrac{3}{2}}}
\end{equation*}

\subsection{Bestimmung des magnetischen Momentes unter Beanspruchung der Gravitation}

\noindent Eine Masse $m$ befindet sich im magntischen Feld eines Helmholtz-Spulenpaares. Auf die Masse wirkt die Gravitationskraft 
$\vec{F}_\text{G} = m \cdot \vec{g}$. Die Masse ist durch eine Aluminiumstange mit einer Billiardkugel verbunden. Dementsprechend 
wird ein Drehmoment 

\begin{equation*}
    \vec{D}_{G} = m \cdot \left(\vec{r} \times \vec{g}\right)
\end{equation*}

\noindent ausgeübt. Dabei steht $| \vec{r} |$ für den Abstand zwischen der Masse $m$ und dem Anfang des Stiels der Billiardkugel.
Zusätzlich wirkt das in Gleichung \eqref{eqn:D_B} beschriebene Drehmoment 

\begin{equation*}
    \vec{\mu}_\text{Dipol} \times \vec{B} = m \cdot \left(\vec{r} \times \vec{g}\right).
\end{equation*}

\noindent Bei geeigneter Magnetfeldstärke befinden sich die beiden Drehmomente im Gleichgewicht. Aufgrund der Parallelität von $\vec{B}$ und
$\vec{F}_\text{G}$ gilt der Zusammenhang

\begin{equation}
\label{eqn:Gravitation}
    \mu_\text{Dipol} \cdot B = m \cdot r \cdot g.
\end{equation}

\noindent Aus diesem Zusammenhang lässt sich das magnetische Moment berechnen.

\subsection{Berechnung des magnetischen Momentes über die Schwingungsdauer eines Magneten}

\noindent Wird die Billiardkugel im homogenen Magnetfeld der Helmholtz-Spulen in Schwingung versetzt, verhält sie sich 
ähnlich zu einem harmonischen Oszillator. Die dadurch erzeugte Bewegung kann somit durch die folgende Differentialgleichung
beschrieben werden:

\begin{equation*}
    - | \vec{\mu}_\text{Dipol} \times \vec{B} | = J_\text{K} \cdot \frac{\symup{d}²\theta}{\symup{d}t²}
\end{equation*}

\noindent Mittels der Periodendauer $T$ lässt sich die Lösung dieser Differentialgleichung schreiben als 

\begin{equation*}
    T² = \frac{4\pi²J_\text{K}}{\mu_\text{Dipol}}\frac{1}{B}.
\end{equation*}

\noindent Hierbei bezeichnet $J_\text{K}$ das Trägheitsmoment der Kugel.

\subsection{Bestimmung des magnetischen Momentes durch die Präzession eines Magneten}

\noindent Nun wird die Billiardkugel in Rotation versetzt und um einem kleinen Winkel ausgelenkt. Aufgrund 
der Rotation, bleibt diese Auslenkung jedoch stabil. Wirkt nun durch das Magnetfeld eine äußere Kraft auf die 
Figurenachse $\vec{L}_\text{K}$, so präzediert die Kugel. Anschaulich bedeutet dies, dass die Figurenachse auf einem Kegelmantel 
um die Drehimpulsachse $\vec{L}$. Diese Bewegung lässt sich beschreiben durch:

\begin{equation*}
    \vec{\mu}_\text{Dipol} \times \vec{B} = \frac{\text{d}\vec{L}_\text{K}}{\text{d}t}
\end{equation*}

\noindent mit der Präzessionsfrequenz 

\begin{equation*}
    \Omega_\text{p} = \frac{\mu B}{| L_\text{K} |}
\end{equation*}

\noindent als Lösung der Differentialgleichung. Aus den Beziehungen $L_\text{K} = J_\text{K}\omega$ und $\omega = 2\pi\nu
= \frac{2\pi}{T_\text{p}}$ lässt sich diese Gleichung umschreiben zu

\begin{equation}
\label{eqn:Praezession}
    \frac{1}{T_\text{p}} = \frac{\mu_\text{Dipol}}2\pi{L_\text{K}}B, 
\end{equation}

\noindent woraus der Wert für das magnetische Moment des Magneten abgleitet werden kann.

\section{Vorbereitung}

\subsection{Magnetfeld}
Die Berechnung der magnetischen Flußdichte $B$ in der Mitte des Helmholtz-Spulenpaare lässt sich durch die Formel aus 
\eqref{eqn:}

\section{Fehlerrechnung}
\end{document}