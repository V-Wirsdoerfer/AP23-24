\input{../../header.tex}

\begin{document}

\section{Zielsetzung}

Das Ziel des im folgend protokollierten Versuches besteht in der Betrachtung und Anwendung der Grundlagen der 
Ultraschalltechnik.

\section{Theorie}
\label{sec:Theorie}

Der als Ultraschall bekannte Frequenzbereich beginnt mit \qty{20}{\kilo\hertz} bei der oberen Schranke des hörbaren 
Frequenspektrums und endet bei etwa \qty{1}{\giga\hertz}. Die Schallwelle ist ein prominentes Beispiel für eine longitudinale 
Welle 

\begin{equation*}
%\label{eqn:Welle}
    p\left(x,t\right) = p_0 + v_0Z\cos\left(\omega{}t - kx\right),
\end{equation*}

\noindent welche sich aufgrund von Dichteschwankungen des zugrundeliegenden Mediums ausbreitet. Im Gegensatz zu EM-Wellen ist 
die Phasengeschwindigkeit von Schallwellen aufgrund der Druckschwankungen materialabhängig. In Liquiden und Gasen breitet sich 
Schall ausschließlich als Longitudinalwelle mit \emph{Schallgeschwindigkeit}

\begin{equation*}
%\label{eqn:vFl}
    c_\text{Fl} = \sqrt{\frac{1}{\kappa\cdot\rho}}
\end{equation*}

aus. Diese ist somit invers proportional zur Kompressibilität $\kappa$ und zur Dichte $\rho$. Verglichen dazu, kann sich Schall 
in Festkörpern auch als Transversaalwelle mit Geschwindigkeit 

\begin{equation*}
    %\label{eqn:vFk}
        c_\text{Fe} = \sqrt{\frac{E}{\rho}}
\end{equation*}

\noindent ausbreiten, wobei der Kehrwert der Kompressibilität durch das Elastizitätsmodul $E$ erstezt wird. Die Schallgeschwindigkeiten
in Festkörpern sind dabei jedoch grundsätzlich anisotrop. Der Intensitätsverlust mit anfänglichem Wert $I_0$ nimmt 
exponentiell nach der Strecke $x$ ab 

\begin{equation*}
%\label{eqn:Intensität}
    I(x) = I_0\cdot\exp{}\left(-\alpha{}x\right)
\end{equation*}

\section{Vorbereitung}


\section{Fehlerrechnung}
\end{document}