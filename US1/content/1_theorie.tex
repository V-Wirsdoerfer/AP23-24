\input{../../header.tex}

\begin{document}

\section{Zielsetzung}

Das Ziel des im folgend protokollierten Versuches besteht in der Betrachtung und Anwendung der Grundlagen der 
Ultraschalltechnik.

\section{Theorie}
\label{sec:Theorie}

\subsection{Schallwellen in Materie}

\noindent Der als Ultraschall bekannte Frequenzbereich beginnt mit \qty{20}{\kilo\hertz} bei der oberen Schranke des hörbaren 
Frequenspektrums und endet bei etwa \qty{1}{\giga\hertz}. Die Schallwelle ist ein prominentes Beispiel für eine longitudinale 
Welle 

\begin{equation*}
%\label{eqn:Welle}
    p\left(x,t\right) = p_0 + v_0Z\cos\left(\omega{}t - kx\right),
\end{equation*}

\noindent welche sich aufgrund von Dichteschwankungen des zugrundeliegenden Mediums ausbreitet. Im Gegensatz zu EM-Wellen ist 
die Phasengeschwindigkeit von Schallwellen aufgrund der Druckschwankungen materialabhängig. In Liquiden und Gasen breitet sich 
Schall ausschließlich als Longitudinalwelle mit \emph{Schallgeschwindigkeit}

\begin{equation*}
%\label{eqn:vFl}
    c_\text{Fl} = \sqrt{\frac{1}{\kappa\cdot\rho}}
\end{equation*}

aus. Diese ist somit invers proportional zur Kompressibilität $\kappa$ und zur Dichte $\rho$. Verglichen dazu, kann sich Schall 
in Festkörpern auch als Transversaalwelle mit Geschwindigkeit 

\begin{equation*}
    %\label{eqn:vFk}
        c_\text{Fe} = \sqrt{\frac{E}{\rho}}
\end{equation*}

\noindent ausbreiten, wobei der Kehrwert der Kompressibilität durch das Elastizitätsmodul $E$ erstezt wird. Die Schallgeschwindigkeiten
in Festkörpern sind dabei jedoch grundsätzlich anisotrop. Der Intensitätsverlust mit anfänglichem Wert $I_0$ nimmt 
exponentiell nach der Strecke $x$ ab 

\begin{equation*}
%\label{eqn:Intensität}
    I(x) = I_0\cdot\exp{}\left(-\alpha{}x\right),
\end{equation*}

\noindent wobei $\alpha$ den Absorptionskoeffizienten der Schallamplitude repräsentiert. Bei Auftreffen einer Schallwelle auf 
eine Grenzfläche wird die Welle sowohl reflektiert als auch transimittiert. Der Reflexionskoeffizient $R$ berechnet sich dabei 
über die akkustischen Impedanzen der angrenzenden Materialien:

\begin{equation*}
%\label{eqn:Rkoeff}
    R = \left(\frac{Z_1 - Z_2}{Z_1 + Z_2}\right)²
\end{equation*}

\noindent Der restliche Anteil gilt demnach dem Transmissionskoeffizienten $T = 1-R$.

\subsection{Erzeugung und Anwendung von Ultraschall}

\noindent Eine Möglichkeit zur Erzeugung von Ultraschall ist der reziproke piezoelektrische-Effekt. Hierbei wird ein piezoelektrischer 
Kristall in einem elektrischen Wechselfeld zur Schwingung angeregt und strahlt dabei Ultraschallwellen aus. Bei Erreichen der
Resonanzfrequenz können ausgesprochen hohe Schallenergiedichten entstehen.\\

In der Medizin wird grundsätzlich zwischen zwei Möglichkeiten unterschieden, um mittels der Ultraschalltechnik, Informationen 
über den durchstralten Körper zu gweinnen. Beim sogenannten \textbf{Durchstrahlungsverfahren} gibt ein Ultraschallsender einen 
kurzzeitigen Schallimpuls ab, welcher nach Durchstrahlung des Materials von einem Ultraschallempfänger ausgewertet wird. 
Potentielle Fehlstellen im Material spiegeln sich über abgeschwächte Intensitäten wieder. Dem gegenüber steht das 
\textbf{Impuls-Echo-Verfahren}, welches den Ultraschallsender als auch Empfänger interpretiert, da der Puls nach dem Matrial an 
einer Grenzfläche reflektiert wird. Im Gegensatz zum Durchstrahlungsverfahren gibt hier sowohl die Höhe des Echos Aufschluss über 
die Größe die Fehlstelle als auch die Laufzeit $t$ über die Lage der Fehlstelle. Letztes erfordert im Rahmen der zugrundeliegenden
Gleichung 

\begin{equation*}
%\label{eqn:Fehllage}
    s = \frac{1}{2}ct 
\end{equation*}

\noindent die Schallgeschwindigkeit $c$. Visualisiert werden die Laufzeitdiagramme in einem \emph{A-Scan}, \emph{B-Scan} oder in 
einem \emph{TM-Scan}. Die untenstehende Abbildung verdeutlicht die unterschiedlichen Grundprinzipien beider Verfahren.

\begin{figure}
    \begin{subfigure}{0.48\textwidth}
        \centering
        \includegraphics[height=6cm]{Durchschallung.png}
        \caption{Durchschallungs-Verfahren.}
        \label{fig:DSV}
    \end{subfigure}
    \hfill
    \begin{subfigure}{0.48\textwidth}
        \centering 
        \includegraphics[height=6cm]{Impuls_Echo.png}
        \caption{Impuls-Echo-Verfahren.}
        \label{fig:IEV}
    \end{subfigure}
    \caption{Vergleich beider Verfahren.}
    \label{fig:Vergleich}
\end{figure}

\section{Vorbereitung}

%https://www.karldeutsch.de/wp-content/uploads/2020/03/Schallgeschwindigkeit-in-Fl%C3%BCssigkeiten-DE.pdf
%https://www.olympus-ims.com/de/ndt-tutorials/thickness-gauge/appendices-velocities/
\section{Fehlerrechnung}
\end{document}