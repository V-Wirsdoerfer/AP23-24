\documentclass[
  bibliography=totoc,     % Literatur im Inhaltsverzeichnis
  captions=tableheading,  % Tabellenüberschriften
  titlepage=firstiscover, % Titelseite ist Deckblatt
]{scrartcl}

% Paket float verbessern
\usepackage{scrhack}

% Warnung, falls nochmal kompiliert werden muss
\usepackage[aux]{rerunfilecheck}

% unverzichtbare Mathe-Befehle
\usepackage{amsmath}
% viele Mathe-Symbole
\usepackage{amssymb}
% Erweiterungen für amsmath
\usepackage{mathtools}

% Fonteinstellungen
\usepackage{fontspec}
% Latin Modern Fonts werden automatisch geladen
% Alternativ zum Beispiel:
%\setromanfont{Libertinus Serif}
%\setsansfont{Libertinus Sans}
%\setmonofont{Libertinus Mono}

% Wenn man andere Schriftarten gesetzt hat,
% sollte man das Seiten-Layout neu berechnen lassen
\recalctypearea{}

% deutsche Spracheinstellungen
\usepackage[ngerman]{babel}


\usepackage[
  math-style=ISO,    % ┐
  bold-style=ISO,    % │
  sans-style=italic, % │ ISO-Standard folgen
  nabla=upright,     % │
  partial=upright,   % │
  mathrm=sym,        % ┘
  warnings-off={           % ┐
    mathtools-colon,       % │ unnötige Warnungen ausschalten
    mathtools-overbracket, % │
  },                       % ┘
]{unicode-math}

% traditionelle Fonts für Mathematik
\setmathfont{Latin Modern Math}
% Alternativ zum Beispiel:
%\setmathfont{Libertinus Math}

\setmathfont{XITS Math}[range={scr, bfscr}]
\setmathfont{XITS Math}[range={cal, bfcal}, StylisticSet=1]

% Zahlen und Einheiten
\usepackage[
  locale=DE,                   % deutsche Einstellungen
  separate-uncertainty=true,   % immer Unsicherheit mit \pm
  per-mode=symbol-or-fraction, % / in inline math, fraction in display math
]{siunitx}

% chemische Formeln
\usepackage[
  version=4,
  math-greek=default, % ┐ mit unicode-math zusammenarbeiten
  text-greek=default, % ┘
]{mhchem}

% richtige Anführungszeichen
\usepackage[autostyle]{csquotes}

% schöne Brüche im Text
\usepackage{xfrac}

% Standardplatzierung für Floats einstellen
\usepackage{float}
\floatplacement{figure}{htbp}
\floatplacement{table}{htbp}

% Floats innerhalb einer Section halten
\usepackage[
  section, % Floats innerhalb der Section halten
  below,   % unterhalb der Section aber auf der selben Seite ist ok
]{placeins}

% Seite drehen für breite Tabellen: landscape Umgebung
\usepackage{pdflscape}

% Captions schöner machen.
\usepackage[
  labelfont=bf,        % Tabelle x: Abbildung y: ist jetzt fett
  font=small,          % Schrift etwas kleiner als Dokument
  width=0.9\textwidth, % maximale Breite einer Caption schmaler
]{caption}
% subfigure, subtable, subref
\usepackage{subcaption}

% Grafiken können eingebunden werden
\usepackage{graphicx}

% schöne Tabellen
\usepackage{tabularray}
\UseTblrLibrary{booktabs, siunitx}

% Verbesserungen am Schriftbild
\usepackage{microtype}

% Literaturverzeichnis
\usepackage[
  backend=biber,
]{biblatex}
% Quellendatenbank
\addbibresource{lit.bib}
\addbibresource{programme.bib}

% Hyperlinks im Dokument
\usepackage[
  german,
  unicode,        % Unicode in PDF-Attributen erlauben
  pdfusetitle,    % Titel, Autoren und Datum als PDF-Attribute
  pdfcreator={},  % ┐ PDF-Attribute säubern
  pdfproducer={}, % ┘
]{hyperref}
% erweiterte Bookmarks im PDF
\usepackage{bookmark}

% Trennung von Wörtern mit Strichen
\usepackage[shortcuts]{extdash}

\author{%
  Vincent Wirsdörfer\\%
  \href{mailto:vincent.wirsdoerfer@udo.edu}{authorA@udo.edu}%
  \and%
  Joris Daus\\%
  \href{mailto:joris.daus@udo.edu}{authorB@udo.edu}%
}
\publishers{TU Dortmund – Fakultät Physik}


\begin{document}

\section{Zielsetzung}

Das Ziel des im folgend protokollierten Versuches besteht in der Betrachtung und Anwendung der Grundlagen der 
Ultraschalltechnik.

\section{Theorie}
\label{sec:Theorie}

Der als Ultraschall bekannte Frequenzbereich beginnt mit \qty{20}{\kilo\hertz} bei der oberen Schranke des hörbaren 
Frequenspektrums und endet bei etwa \qty{1}{\giga\hertz}. Die Schallwelle ist ein prominentes Beispiel für eine longitudinale 
Welle 

\begin{equation*}
%\label{eqn:Welle}
    p\left(x,t\right) = p_0 + v_0Z\cos\left(\omega{}t - kx\right),
\end{equation*}

\noindent welche sich aufgrund von Dichteschwankungen des zugrundeliegenden Mediums ausbreitet. Im Gegensatz zu EM-Wellen ist 
die Phasengeschwindigkeit von Schallwellen aufgrund der Druckschwankungen materialabhängig. In Liquiden und Gasen breitet sich 
Schall ausschließlich als Longitudinalwelle mit \emph{Schallgeschwindigkeit}

\begin{equation*}
%\label{eqn:vFl}
    c_\text{Fl} = \sqrt{\frac{1}{\kappa\cdot\rho}}
\end{equation*}

aus. Diese ist somit invers proportional zur Kompressibilität $\kappa$ und zur Dichte $\rho$. Verglichen dazu, kann sich Schall 
in Festkörpern auch als Transversaalwelle mit Geschwindigkeit 

\begin{equation*}
    %\label{eqn:vFk}
        c_\text{Fe} = \sqrt{\frac{E}{\rho}}
\end{equation*}

\noindent ausbreiten, wobei der Kehrwert der Kompressibilität durch das Elastizitätsmodul $E$ erstezt wird. Die Schallgeschwindigkeiten
in Festkörpern sind dabei jedoch grundsätzlich anisotrop. Der Intensitätsverlust mit anfänglichem Wert $I_0$ nimmt 
exponentiell nach der Strecke $x$ ab 

\begin{equation*}
%\label{eqn:Intensität}
    I(x) = I_0\cdot\exp{-\alpha{}x}
\end{equation*}

\section{Vorbereitung}


\section{Fehlerrechnung}
\end{document}