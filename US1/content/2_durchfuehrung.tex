\input{../../header.tex}

\begin{document}
\section{Versuchsdurchführung}

In diesem Versuch wird mit der Echo-Methode gearbeitet. Daher wird nur ein Schallkopf pro Messung benötigt.
Es gibt zwei verschiedene Ultraschallköpfe, mit denen experimentiert wird. Sie unterscheiden sich durch 
ihre Frequenz. Ein Kopf mit roter Markierung arbeitet mit einer Frequenz von \qty{2}{\mega \hertz}. Der 
andere mit blauer Markierung arbeitet mit \qty{1}{\mega \hertz}. Beide werden an das Echoskop angeschlossen.
Aus technischen Gründen ist es jedoch nur möglich einen pro Zeit anzuschließen. Wenn also der Kopf gewechselt 
werden muss, so muss er auch umgesteckt werden.

\subsection{Bestimmung der Schallgeschwindigkeit}

\noindent Um die Schallgeschwindigkeit zu bestimmen, werden Zylinder gleicher Radien und unterschiedlicher 
Länge vermessen. Zunächst werden die Höhen der Zylinder mit einem Messschieber vermessen.
Anschließend wird bidestilliertes Wasser auf die Zylinder getan, welches als Kontaktmittel dient. 
Auf das Kontaktmittel werden die Ultraschallköpfe gestellt. 
Die Aluminiumzylinder werden mit dem \qty{2}{\mega \hertz} Ultraschallkopf vermessen. Es wird die Laufzeit 
des Schalls und die Amplitude der Peaks notiert, indem im Computerprogramm die Curser auf die Peaks gelegt werden.
Die Differenzen der Peaks berechnet der Computer automatisch.   




\section{Messwerte}

\end{document}

