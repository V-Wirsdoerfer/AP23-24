%\documentclass[
  bibliography=totoc,     % Literatur im Inhaltsverzeichnis
  captions=tableheading,  % Tabellenüberschriften
  titlepage=firstiscover, % Titelseite ist Deckblatt
]{scrartcl}

% Paket float verbessern
\usepackage{scrhack}

% Warnung, falls nochmal kompiliert werden muss
\usepackage[aux]{rerunfilecheck}

% unverzichtbare Mathe-Befehle
\usepackage{amsmath}
% viele Mathe-Symbole
\usepackage{amssymb}
% Erweiterungen für amsmath
\usepackage{mathtools}

% Fonteinstellungen
\usepackage{fontspec}
% Latin Modern Fonts werden automatisch geladen
% Alternativ zum Beispiel:
%\setromanfont{Libertinus Serif}
%\setsansfont{Libertinus Sans}
%\setmonofont{Libertinus Mono}

% Wenn man andere Schriftarten gesetzt hat,
% sollte man das Seiten-Layout neu berechnen lassen
\recalctypearea{}

% deutsche Spracheinstellungen
\usepackage[ngerman]{babel}


\usepackage[
  math-style=ISO,    % ┐
  bold-style=ISO,    % │
  sans-style=italic, % │ ISO-Standard folgen
  nabla=upright,     % │
  partial=upright,   % │
  mathrm=sym,        % ┘
  warnings-off={           % ┐
    mathtools-colon,       % │ unnötige Warnungen ausschalten
    mathtools-overbracket, % │
  },                       % ┘
]{unicode-math}

% traditionelle Fonts für Mathematik
\setmathfont{Latin Modern Math}
% Alternativ zum Beispiel:
%\setmathfont{Libertinus Math}

\setmathfont{XITS Math}[range={scr, bfscr}]
\setmathfont{XITS Math}[range={cal, bfcal}, StylisticSet=1]

% Zahlen und Einheiten
\usepackage[
  locale=DE,                   % deutsche Einstellungen
  separate-uncertainty=true,   % immer Unsicherheit mit \pm
  per-mode=symbol-or-fraction, % / in inline math, fraction in display math
]{siunitx}

% chemische Formeln
\usepackage[
  version=4,
  math-greek=default, % ┐ mit unicode-math zusammenarbeiten
  text-greek=default, % ┘
]{mhchem}

% richtige Anführungszeichen
\usepackage[autostyle]{csquotes}

% schöne Brüche im Text
\usepackage{xfrac}

% Standardplatzierung für Floats einstellen
\usepackage{float}
\floatplacement{figure}{htbp}
\floatplacement{table}{htbp}

% Floats innerhalb einer Section halten
\usepackage[
  section, % Floats innerhalb der Section halten
  below,   % unterhalb der Section aber auf der selben Seite ist ok
]{placeins}

% Seite drehen für breite Tabellen: landscape Umgebung
\usepackage{pdflscape}

% Captions schöner machen.
\usepackage[
  labelfont=bf,        % Tabelle x: Abbildung y: ist jetzt fett
  font=small,          % Schrift etwas kleiner als Dokument
  width=0.9\textwidth, % maximale Breite einer Caption schmaler
]{caption}
% subfigure, subtable, subref
\usepackage{subcaption}

% Grafiken können eingebunden werden
\usepackage{graphicx}

% schöne Tabellen
\usepackage{tabularray}
\UseTblrLibrary{booktabs, siunitx}

% Verbesserungen am Schriftbild
\usepackage{microtype}

% Literaturverzeichnis
\usepackage[
  backend=biber,
]{biblatex}
% Quellendatenbank
\addbibresource{lit.bib}
\addbibresource{programme.bib}

% Hyperlinks im Dokument
\usepackage[
  german,
  unicode,        % Unicode in PDF-Attributen erlauben
  pdfusetitle,    % Titel, Autoren und Datum als PDF-Attribute
  pdfcreator={},  % ┐ PDF-Attribute säubern
  pdfproducer={}, % ┘
]{hyperref}
% erweiterte Bookmarks im PDF
\usepackage{bookmark}

% Trennung von Wörtern mit Strichen
\usepackage[shortcuts]{extdash}

\author{%
  Vincent Wirsdörfer\\%
  \href{mailto:vincent.wirsdoerfer@udo.edu}{authorA@udo.edu}%
  \and%
  Joris Daus\\%
  \href{mailto:joris.daus@udo.edu}{authorB@udo.edu}%
}
\publishers{TU Dortmund – Fakultät Physik}


%\begin{document}
\section{Versuchsdurchführung}

In diesem Versuch wird mit der Echo-Methode gearbeitet. Daher wird nur ein Schallkopf pro Messung benötigt.
Es gibt zwei verschiedene Ultraschallköpfe, mit denen experimentiert wird. Sie unterscheiden sich durch 
ihre Frequenz. Ein Kopf mit roter Markierung arbeitet mit einer Frequenz von \qty{2}{\mega \hertz}. Der 
andere mit blauer Markierung arbeitet mit \qty{1}{\mega \hertz}. Beide werden an das Echoskop angeschlossen.
Aus technischen Gründen ist es jedoch nur möglich einen pro Zeit anzuschließen. Wenn also der Kopf gewechselt 
werden muss, so muss er auch umgesteckt werden.

\subsection{Bestimmung der Schallgeschwindigkeit}

\noindent Um die Schallgeschwindigkeit zu bestimmen, werden Zylinder gleicher Radien und unterschiedlicher 
Länge vermessen. Zunächst werden die Höhen der Zylinder mit einem Messschieber vermessen.
Anschließend wird bidestilliertes Wasser auf die Zylinder getan, welches als Kontaktmittel dient. 
Auf das Kontaktmittel werden die Ultraschallköpfe gestellt. 
Die Aluminiumzylinder werden mit dem \qty{2}{\mega \hertz} Ultraschallkopf vermessen. Es wird die Laufzeit 
des Schalls und die Amplitude der Peaks notiert, indem im Computerprogramm die Cursor auf die Peaks gelegt werden.
Die Differenzen der Peaks berechnet der Computer automatisch.   
Es sind nur sieben Zylinder vorhanden, jedoch müssen acht Messungen aufgenommen werden. Aus diesem Grund werden 
die beiden kleinsten Zylinder aufeinander gestapelt und die Übergangsstelle mit dem Kontaktmittel gefüllt

\noindent Dieses Vorgehen wird für alle sieben Acryl- und Aluminiumzylinder wiederholt. Zusätzlich werden die Acrylzylinder 
mit dem \qty{1}{\mega \hertz} Ultraschallkopf vermessen.

\noindent Des Weiteren soll eine Kalibrierkurve aufgenommen werden. Dazu wird ein Erlenmeyerkolben in einem Stativ eingespannt,
unter welchem sich eine \qty{2}{\mega \hertz} Ultraschallsonde befindet. Zwischen dem Glas und der Sonde wird ein 
Ultraschallgel als Kontaktmittel aufgetragen. Zum Beginn der Messung werden \qty{50}{\milli \liter} in den 
Erlenmeyerkolben eingefüllt. Es werden nun die Messwerte notiert. Um den Füllstand zu erhöhen, werden in einem 
Messzylinder \qty{10}{\milli \liter} abgemessen und diese in den Erlenmeyerkolben gefüllt. Anschließend werden die 
Daten erneut aufgenommen. Der Vorgang wiederholt sich so lange, bis \qty{200}{\milli \liter} im Erlenmeyerkolben 
enthalten sind. 


\section{Messwerte}
Zunächst werden die Daten für die Aluminiumzylinder aufgenommen. Es wird die Höhe des jeweiligen Zylinders gemessen und die Zeit, 
die der Schall braucht um hin und zurück durch den Zylinder zu kommen. Dabei werden keine Spannungsdaten aufgenommen, da 
sich diese als nicht plausibel herauskristallisieren, weil die Spannungen mit zunehmender Höhe steigen, obwohl diese 
der Theorie nach abnehmen sollten. So gehen die folgenden Daten in die Auswertung hinein. 

\begin{table}[H]
    \centering 
    \caption{Messdaten der Alumniniumzylinder mit einer \qty{2}{\mega \hertz} Sonde.}
    \begin{tblr}{
        colspec = {S[table-format=3.2] S[table-format=2.1] },
        row{1} = {guard, mode=math},
        }
        \toprule
        \text{Höhe der Zylinder} \mathbin{/} \unit{\milli \meter} & \text{Zeitdifferens des Schalls} \mathbin{/} \unit{\micro \second} \\
        \midrule
        31.10   &   8.8     \\
        40.14   &   12.5    \\        
        53.00   &   15.6    \\        
        62.16   &   18.4    \\        
        71.24   &   21.5    \\        
        81.10   &   24.6    \\        
        110.10  &   33.4    \\        
        120.06  &   36.7    \\        
        \bottomrule
    \end{tblr}    
    \label{tab:Aluminium}
\end{table}

\noindent Nun werden auf gleiche Art und Weise die Acrylzylinder vermessen. Hierbei wird jedoch die Anfangsspannung $U_0$, mit der das 
Signal in den Zylinder geht und die Spannung $U$ mit der das Signal aus dem Zylinder wieder herauskommt, mit aufgenommen. So entstehen 
die folgenden Messdaten 


\begin{table}[H]
    \centering 
    \caption{Messdaten der Acrylzylinder mit einer \qty{2}{\mega \hertz} Sonde.}
    \sisetup{table-format = 1.3}
    \begin{tblr}{
        colspec = {S[table-format=3.2] S[table-format=2.2] S S},
        row{1} = {guard, mode=math},
        }
        \toprule
        \text{Höhe der Zylinder} \mathbin{/} \unit{\milli \meter} & \text{Zeitdifferenz des Schalls} \mathbin{/} \unit{\micro \second} & U_0 \mathbin{/} \unit{\volt} & U \mathbin{/} \unit{\volt} \\
        \midrule
        31.80   &   23.20   &   0.454   &   0.447   \\
        40.36   &   29.80   &   0.487   &   0.528   \\
        52.00   &   39.10   &   0.491   &   0.359   \\
        61.56   &   45.90   &   0.541   &   0.231   \\
        72.16   &   52.70   &   0.503   &   0.184   \\
        80.44   &   59.30   &   0.501   &   0.179   \\
        102.10  &   72.80   &   0.502   &   0.014   \\
        120.62  &   88.10   &   0.503   &   0.013   \\        
        \bottomrule
    \end{tblr}    
    \label{tab:Acryl2MHz}
\end{table}

\noindent Zur Bestimmung des Dämpfungswiderstand werden die Acrylzylinder erneut mit einem \qty{1}{\mega \hertz} Ultraschallkopf untersucht. 
Dabei wird darauf verichtet die Zeitdifferenz zu nehmen, da dies bereits gemacht wurde. Es werden also nur die jeweiligen Ein- und 
Ausgangsspannungen notiert. So gehen die folgenden Werte in die Auswertung mit hinein.

\begin{table}[H]
    \centering 
    \caption{Messdaten der Acrylzylinder mit einer \qty{1}{\mega \hertz} Sonde.}
    \sisetup{table-format = 1.3}
    \begin{tblr}{
        colspec = {S[table-format=3.2] S S},
        row{1} = {guard, mode=math},
        }
        \toprule
        \text{Höhe der Zylinder} \mathbin{/} \unit{\milli \meter}  & U_0 \mathbin{/} \unit{\volt} & U \mathbin{/} \unit{\volt} \\
        \midrule
        31.80   &   0.744   &   0.148   \\
        40.36   &   0.744   &   0.138   \\
        52.00   &   0.735   &   0.142   \\
        61.56   &   0.738   &   0.071   \\
        72.16   &   0.740   &   0.013   \\
        80.44   &   0.746   &   0.009   \\  
        \bottomrule
    \end{tblr}    
    \label{tab:Acryl1MHz}
\end{table}

\noindent Es sind nun alle benötigten Werte aufgenommen. 

%\end{document}

