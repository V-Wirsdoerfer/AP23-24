%\input{../../header.tex}

%\begin{document}
\section{Diskussion}
\label{sec:Diskussion}

Wie bereits in den vorherigen Kapiteln des Protokolls dargestellt, werden diverse Untersuchungen der Ultraschallausbreitung 
in verschiedenen Materialien durchgeführt. Zu Beginn des Versuchs soll über die Geräte- und Programmeinstellung die Frequenz 
und Wellenlänge des Ultraschalls in Acryl berechnet. Wie dem Kapitel \ref{sec:Auswertung} zu entnehmen ist, lauten die 
aufgenommenen Werte wie folgt:

\begin{align*}
    f &= \sfrac{1}{T} = \qty{2}{\mega\hertz}\\
    \lambda &= c_\text{Acryl}\cdot{}f = \qty{1.831e-10}{\meter}
\end{align*}

\noindent Die Tatsache, dass bei allen fünf Messungen der Periodendauer ein Wert von $T = \qty{0.5}{\micro\second}$ gemessen 
wird, minimiert die Standardabweichung des Mittelwerts auf null und begünstigt gleichermaßen die Präzision der Ergebnisse 
für die Wellenlänge und die Frequenz. Somit entspricht die empirisch bestimmte Frequenz exakt der Frequenz der verwendeten 
Sonde von $f = \qty{2}{\mega\hertz}$.\\

\noindent Im Anschluss werden die Schallgeschwindigkeiten in Aluminium und Acryl gemessen. Die dafür aufgenommenen Werte lauten 

\begin{align*}
    v_\text{Alu} &= \qty{6.46\pm0.08e3}{\meter\per\second} \\
    v_\text{Acryl} &= \qty{2.79\pm0.04e3}{\meter \per \second}.
\end{align*}

\noindent Diese Werte werden im Folgenden mit den recherchierten Literaturwerten

\begin{align*}
    c_\text{Aluminium} &= \qty{6320}{\meter \per \second}\\
    c_\text{Acryl}  &= \qty{2730}{\meter \per \second}
\end{align*}

\noindent abgeglichen. Wie zu erkennen ist, weicht der gemessene Mittelwert der Schallgeschwindigkeit von Aluminium um etwa \qty{2.22}{\percent}
vom Literaturwert ab. Bei Acryl beträgt die prozentuale Abweichung \qty{2.20}{\percent}. Beide Ergebnisse implizieren, dass 
keine signifikanten Fehler die Messung beeinträchtigen. Zusätzlich beschränkt sich die Versuchsdurchführung im Wesentlichen auf 
die Bedienung technischer Geräte, was den Eigenanteil an Fehlern minimiert und die Benennung potentieller Fehlerquellen
erschwert. Die trotzdem vorhandene Abweichung der Werte kann daher beispielsweise auf zu wenig Kontaktmittel zwischen Sonde 
und Zylinder oder auf mikroskopische Fehlerquellen im Zylinder hinweisen, welche ein ungewolltes Fehlerecho reflektieren 
und die Messwerte verfälschen. Zusätzlich könnte es sein, dass der Cursor zur Bestimmung der Laufzeiten nicht vollständig den 
gewollten Zeitpeak trifft, sondern leicht daneben liegt, was leicht veränderte Schallgeschwindigkeiten der jeweiligen 
Materialien zur Folge hat.\\

\noindent Die Messung des Dämpfungskoeffizienten ist nur für Acryl möglich, da die Aluminiumzylinder bei wiederholten Messungen keinen 
exponentiellen Trend aufweisen und somit aus der Messung gestrichen werden! Für Acryl wird dagegen jedoch eine Dämpfungskurve 
mit der \qty{1}{\mega\hertz}- und mit der \qty{2}{\mega\hertz}-Sonde aufgenommen. Wie in der Auswertung bereits behandelt, bildet
die mit der \qty{2}{\mega\hertz}-Sonde tatsächlich, wie nach Gleichung \eqref{eqn:Intensitaet} postuliert, einen exponentiellen 
Abfall ab. Der Dämpfungskoeffizient in Acryl ist ein Parameter des \emph{curve\_fit} und lautet 

\begin{align*}
    \alpha_{\qty{2}{\mega\hertz}} = \qty{7.92\pm4.62}{\per\meter}
\end{align*}

\noindent Obwohl die gemessenen Werte durch eine abfallende Exponentialfunktion \emph{gefittet} werden können, ist die 
Aussagekräftigkeit des Wertes aufgrund des großen Fehlers mit Vorsicht zu behandeln. Die hohe Fehleranfälligkeit des Parameters
ist durch starke Abweichung der Daten vom \emph{fit} zu erklären. Dies ist mittels Abbildung \ref{fig:Kurve2MHz} gut nachvollziehbar.
Systematisch betrachtet schlägt sich an dieser Stelle die Schwierigkeit wieder, speziell im Bereich kleiner Amplituden, dem 
Datenprogramm die korrekten Amplitudenwerte zu entnehmen. Dies wird aufgrund mehrerer immer kleiner werdender Maxima erschwert.\\

\noindent Im Vergleich dazu gestaltet sich die Konstruktion des exponentiellen \emph{fits} bei der \qty{1}{\mega\hertz}-Sonde wesentlich 
schwieriger. Hierbei können keine Parameter ausgegeben werden, welche die Messdaten entsprechen \emph{fitten}. Die alternative Lösung des 
logarithmischen Ansatzes wird bereits in der Auswertung näher behandelt und liefert den Dämpfungskoeffizienten

\begin{align*}
    \alpha_{\qty{1}{\mega\hertz}} = \qty{31.19\pm7.15}{\per\meter}.
\end{align*}

\noindent Einerseits ist die Korrektheit dieses Wertes wegen der alternativen Methode nur schwer zu beurteilen. Andererseits 
ergibt die Erhöhung des Koeffizienten bei niedriger Frequenz der Sonde im Sachzusammenhang Sinn, da niedrigfrequente Wellen 
eine geringere Reichweite aufweisen. Dies ist zu erwarten, da Wellen mit höherer Frequenz häufiger schwingen und somit statistisch 
gesehen eine größere Stoßwahrscheinlichkeit haben. Dies führt dazu, dass Wellen höherer Frequenz eine geringere Reichweite haben. Dies 
impleziert direkt einen höheren Dämmpfungsfaktor. Dies entspricht den Messergebnissen.\\

\noindent Die Abbildung \ref{fig:Kalibrierkurve} zeigt, dass die Daten der Füllstandsmessung gut an einen \emph{polyfit} 2.
Ordnung gelegt werden können. Die zugehörigen der allgemeinen Funktionsgleichung

\begin{equation*}
    f(x) = ax² + bx + c
\end{equation*}

\noindent lauten wie folgt:

\begin{align*}
    a &= \qty{6.073\pm0.479e-4}{\micro\second\per\milli\liter\squared}\\  
    b &= \qty{0.107\pm0.012}{\micro\second\per\milli\liter}\\
    c &= \qty{10.71\pm0.698}{\micro\second }
\end{align*}

\noindent Der quadratische Zusammenhang kann in diesem Bereich der Messung insofern gerechtfertigt werden, als das eine konstante 
Zugabe von Wasservolumen im letzten Bereich der Messung zu einer quadratischen Steigung der Höhe und somit auch der Laufzeit 
führt, was Resultat der besonderen Form des Erlenmeyerkolbens ist. 

\section{Anhang}

\begin{figure}
    \centering
    \includegraphics[width=0.9\textwidth]{content/Laborbuch1.jpg}
\end{figure}

\begin{figure}
    \centering
    \includegraphics[width=0.9\textwidth]{content/Laborbuch2.jpg}
\end{figure}

%\end{document}
