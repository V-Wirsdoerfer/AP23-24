%\input{../../header.tex}

%\begin{document}
\section{Diskussion}
\label{sec:Diskussion}

\subsection{Die Bragg Bedingung}

Die Überprüfung der Bragg Bedingung kann durchaus als gelungen betrachtet werden. Mit einem gemessen Winkel von $\theta_\text{B,exp} = \qty{27.7}{\degree}$
beträgt die Abweichung lediglich \qty{0.3}{\degree} vom Sollwert $\theta_\text{B,soll} = \qty{28}{\degree}$. Anhand dieses empirische Befundes kann 
die Bragg Bedingung als verifiziert angenommen werden. 

\subsection{Das Emissionsspektrum von Kupfer}

Die mittels des Experiments bestimmten Werte der minimalen Wellenlänge $\lambda_\text{min}$ sowie die dazu korrespondierende maximale Energie $E_\text{max}$
sind 

\begin{align*}
    \lambda_\text{min} &= \qty{35.1}{\pico\meter} \\
    E_\text{max} &= \qty{35.3}{\kilo\electronvolt}, \\
\end{align*}

\noindent was die folgenden relativen Abweichungen zu den berechneten Theoriewerten mit sich bringt:

\begin{align*}
    \increment\lambda_\text{min} &= \qty{1.14}{\percent}\\
    \increment{}E &= \qty{0.92}{\percent}
\end{align*}

\noindent Auch hier sind eher geringfügige Abweichungen der Theoriewerte festzustellen.\\

\noindent Weitere Messgrößen, welche durch die Aufnahme des Emissionsspektrum gewonnen werden können, sind die in der Vorbereitung bestimmten  
\enquote{charakteristischen Energien} und die dazugehörigen Winkel der $\text{K}_\alpha$- und $\text{K}_\beta$-Linien. Zusätzlich können mittels 
dieser Größen die Abschirmkonstanten $\sigma_\text{i}$ ($i \in \{1,2,3\}$) berechnet werden. Die experimentell bestimmten Werte sowie die zu erwartenden 
Werte aus der Literatur\cite{nist} werden im Folgenden aufgelistet. 

\begin{align*}
    E_\text{\alpha,exp} &= \qty{8.08}{\electronvolt} &E_\text{,theo} &= \qty{8.0}{\electronvolt} \\
    E_\text{\beta,exp} &= \qty{8.92}{\electronvolt} &E_\text{,theo} &= \qty{8.9}{\electronvolt}  \\
    \theta_\text{\alpha,exp} &= \qty{22.4}{\degree} &\theta_\text{,theo} &= \qty{22.68}{\degree} \\ 
    \theta_\text{\beta,exp} &= \qty{20.2}{\degree} &\theta_\text{,theo} &= \qty{20.25}{\degree} \\  
    \sigma_\text{1,exp} &= 3.28  &\sigma_\text{1,theo} &= 3.28 \\  
    \sigma_\text{2,exp} &= 12.57  &\sigma_\text{2,theo} &= 12.14 \\      
    \sigma_\text{3,exp} &= 21.67  &\sigma_\text{3,theo} &= 21.41 \\ 
\end{align*}

\noindent Die dadurch resultierenden prozentualen Abweichungen lauten:

\begin{align*}
    \increment{}E_\text{\alpha} &= \qty{0.07}{\percent} \\
    \increment{}E_\text{\beta} &= \qty{0.61}{\percent}  \\
    \increment{}\theta_\text{\alpha} &= \qty{1.23}{\percent}  \\
    \increment{}\theta_\text{\beta} &= \qty{0.25}{\percent}   \\
    \increment{}\sigma_\text{1,exp} &= \qty{0}{\percent}        \\ 
    \increment{}\sigma_\text{2,exp} &= \qty{3.56}{\percent}        \\   
    \increment{}\sigma_\text{3,exp} &= \qty{1.23}{\percent}        \\   
\end{align*}

\subsection{Das Absorptionsspektrum}

Mit Blick auf die Theorietabelle in der Vorbereitung \ref{tab:BraggBedingungTab} wird die identische Struktur mit den experimentell bestimmten 
Werten gefüllt und dargestellt.

\begin{table}[H]
    \centering
    \caption{Vorbereitungstabelle.}
    \label{tab:BraggBedingungTabexp}
    \begin{tblr}{
        colspec = {S S[table-format=2.0] S[table-format=2.2] S[table-format=2.1] S[table-format=1.2]},
        row{1} = {guard, mode=math},
    }
    \toprule
    \text{Element} & \text{Z} & E_\text{K,Lit} \mathbin{/} \unit{\kilo\electronvolt} & \theta_\text{K,Lit} \mathbin{/} \unit{\degree} & \sigma_\text{K} \\
    \midrule
        Zn  &  30  &  9.75  &  18.4  &  3.42  \\
        Br  &  35  &  13.69 &  13.0  &  3.59  \\
        Ga  &  31  &  10.47 &  17.1  &  3.48  \\
        Sr  &  38  &  16.28 &  10.9  &  3.80  \\
        Zr  &  40  &  18.09 &  9.8   &  4.00  \\
    \bottomrule
    \end{tblr}
\end{table}

\noindent Der Vegrleich beider Tabellen lässt auf großteilig ähnliche Ergebnisse schließen und wir durch die prozentualen Abweichungen konkretisiert.

\subsubsection{Zink}

\begin{align*}
    \increment{}E_\text{K} &= \qty{0.97}{\percent} \\
    \increment{}\theta_\text{K} &= \qty{0.99}{\percent} \\ 
    \increment{}\sigma_\text{K} &= \qty{3.65}{\percent} \\
\end{align*}  
      
\subsubsection{Brom}

\begin{align*}
    \increment{}E_\text{K} &= \qty{1.60}{\percent} \\
    \increment{}\theta_\text{K} &= \qty{1.61}{\percent} \\ 
    \increment{}\sigma_\text{K} &= \qty{6.59}{\percent} \\
\end{align*}

\subsubsection{Gallium}

\begin{align*}
    \increment{}E_\text{K} &= \qty{0.97}{\percent} \\
    \increment{}\theta_\text{K} &= \qty{0.99}{\percent} \\ 
    \increment{}\sigma_\text{K} &= \qty{3.71}{\percent} \\
\end{align*}

\subsubsection{Strontium}

\begin{align*}
    \increment{}E_\text{K} &= \qty{1.12}{\percent} \\
    \increment{}\theta_\text{K} &= \qty{1.12}{\percent} \\ 
    \increment{}\sigma_\text{K} &= \qty{4.88}{\percent} \\
\end{align*}

\subsubsection{Zirkonium}

\begin{align*}
    \increment{}E_\text{K} &= \qty{0.54}{\percent} \\
    \increment{}\theta_\text{K} &= \qty{0.54}{\percent} \\ 
    \increment{}\sigma_\text{K} &= \qty{2.43}{\percent} \\
\end{align*}

\noindent Der aus dem Moseleyschen Gesetz ermittelte Wert der Rydberg-Konstante beträgt 

\begin{align*}
    R_\infty  =\qty{12.85\pm0.16}{\electronvolt}
\end{align*}

\noindent und weicht daher um $\qty{5.5\pm1.2}{\percent}$ vom Literaturwert $R_{\infty,\text{theo}}  =\qty{12.85\pm0.16}{\electronvolt}$\cite{Rydberg} ab.\\

\noindent Aufgrund der Tatsache, dass Großteile des Experiments nicht aktiv beeinflusst werden können, ist es schwierig, Ursachen 
und systematische Hintergründe zu finden, weche Messabweichungen begünstigen. Einerseits könnten jedoch nicht perfekt positionierte bzw. 
zueinander ausgerichtete Bauelemente im Röntgenapparat oder um den LiF-Kristall Gründe für Fehler sein. Andererseits ist anzuzweifeln, 
dass die vom Computerprogrammm voraussgesetzten Werte eine optimale Auflösung aller Ergebnisse garantieren, was die Integartionszeit oder 
den Winkelzuwachs betrifft. Detailliertere Parameter könnten durchaus zu präziseren Ergebnissen führen. Dessen ungeachtet, liegen 
alle berechneten Werte in einem kleinen Fehlerintervall um die tatsächlichen Literaturwerte.


%\end{document}




%Quelle: https://xdb.lbl.gov/Section1/Table_1-2.pdf
