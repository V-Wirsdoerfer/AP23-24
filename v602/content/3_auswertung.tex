\documentclass[
  bibliography=totoc,     % Literatur im Inhaltsverzeichnis
  captions=tableheading,  % Tabellenüberschriften
  titlepage=firstiscover, % Titelseite ist Deckblatt
]{scrartcl}

% Paket float verbessern
\usepackage{scrhack}

% Warnung, falls nochmal kompiliert werden muss
\usepackage[aux]{rerunfilecheck}

% unverzichtbare Mathe-Befehle
\usepackage{amsmath}
% viele Mathe-Symbole
\usepackage{amssymb}
% Erweiterungen für amsmath
\usepackage{mathtools}

% Fonteinstellungen
\usepackage{fontspec}
% Latin Modern Fonts werden automatisch geladen
% Alternativ zum Beispiel:
%\setromanfont{Libertinus Serif}
%\setsansfont{Libertinus Sans}
%\setmonofont{Libertinus Mono}

% Wenn man andere Schriftarten gesetzt hat,
% sollte man das Seiten-Layout neu berechnen lassen
\recalctypearea{}

% deutsche Spracheinstellungen
\usepackage[ngerman]{babel}


\usepackage[
  math-style=ISO,    % ┐
  bold-style=ISO,    % │
  sans-style=italic, % │ ISO-Standard folgen
  nabla=upright,     % │
  partial=upright,   % │
  mathrm=sym,        % ┘
  warnings-off={           % ┐
    mathtools-colon,       % │ unnötige Warnungen ausschalten
    mathtools-overbracket, % │
  },                       % ┘
]{unicode-math}

% traditionelle Fonts für Mathematik
\setmathfont{Latin Modern Math}
% Alternativ zum Beispiel:
%\setmathfont{Libertinus Math}

\setmathfont{XITS Math}[range={scr, bfscr}]
\setmathfont{XITS Math}[range={cal, bfcal}, StylisticSet=1]

% Zahlen und Einheiten
\usepackage[
  locale=DE,                   % deutsche Einstellungen
  separate-uncertainty=true,   % immer Unsicherheit mit \pm
  per-mode=symbol-or-fraction, % / in inline math, fraction in display math
]{siunitx}

% chemische Formeln
\usepackage[
  version=4,
  math-greek=default, % ┐ mit unicode-math zusammenarbeiten
  text-greek=default, % ┘
]{mhchem}

% richtige Anführungszeichen
\usepackage[autostyle]{csquotes}

% schöne Brüche im Text
\usepackage{xfrac}

% Standardplatzierung für Floats einstellen
\usepackage{float}
\floatplacement{figure}{htbp}
\floatplacement{table}{htbp}

% Floats innerhalb einer Section halten
\usepackage[
  section, % Floats innerhalb der Section halten
  below,   % unterhalb der Section aber auf der selben Seite ist ok
]{placeins}

% Seite drehen für breite Tabellen: landscape Umgebung
\usepackage{pdflscape}

% Captions schöner machen.
\usepackage[
  labelfont=bf,        % Tabelle x: Abbildung y: ist jetzt fett
  font=small,          % Schrift etwas kleiner als Dokument
  width=0.9\textwidth, % maximale Breite einer Caption schmaler
]{caption}
% subfigure, subtable, subref
\usepackage{subcaption}

% Grafiken können eingebunden werden
\usepackage{graphicx}

% schöne Tabellen
\usepackage{tabularray}
\UseTblrLibrary{booktabs, siunitx}

% Verbesserungen am Schriftbild
\usepackage{microtype}

% Literaturverzeichnis
\usepackage[
  backend=biber,
]{biblatex}
% Quellendatenbank
\addbibresource{lit.bib}
\addbibresource{programme.bib}

% Hyperlinks im Dokument
\usepackage[
  german,
  unicode,        % Unicode in PDF-Attributen erlauben
  pdfusetitle,    % Titel, Autoren und Datum als PDF-Attribute
  pdfcreator={},  % ┐ PDF-Attribute säubern
  pdfproducer={}, % ┘
]{hyperref}
% erweiterte Bookmarks im PDF
\usepackage{bookmark}

% Trennung von Wörtern mit Strichen
\usepackage[shortcuts]{extdash}

\author{%
  Vincent Wirsdörfer\\%
  \href{mailto:vincent.wirsdoerfer@udo.edu}{authorA@udo.edu}%
  \and%
  Joris Daus\\%
  \href{mailto:joris.daus@udo.edu}{authorB@udo.edu}%
}
\publishers{TU Dortmund – Fakultät Physik}


\begin{document}
\section{Auswertung}
\label{sec:Auswertung}

Im Folgenden soll die Bragg Bedingung, das Emissionsspektrum einer Cu-Röntgenröhre, und das Absorptionsspektrum verschiedener Substanzen überprüft 
werden. 

\subsection{Bragg Bedingung}
Die Messung findet bei einem festen Kristallwinkel von \qty{14}{\degree} statt.

\begin{figure}[H]
    \centering
    \includegraphics[width=0.9\textwidth]{../build/BraggBed.pdf}
    \caption{Überprüfen der Bragg Bedingung.}
\end{figure}    
    
    
\subsection{Emissionsspektrum der Cu-Röntgenröhre}
Nun wird das Emissionsspektrum der Cu-Röntgenröhre näher betrachtet. Dabei werden die charakteristischen Linien $K_\alpha$ und $K_\beta$ bestimmt 
und markiert. Des Weiteren wird der Bremsberg farblich unterlegt. SO entsteht alles in allem die folgende Auftragung. 

\begin{figure}[H]
    \centering
    \includegraphics[width=0.9\textwidth]{../build/EmissionCu.pdf}
    \caption{Emissionsspektrum einer Cu-Röntgenröhre.}
\end{figure}    

\noindent Zur Bestimmung des Grenzwinkels wird der Bereich von kleinen Winkeln und somit von kleinen Wellenlängen betrachtet. 

\begin{figure}[H]
    \centering
    \includegraphics[width=0.9\textwidth]{../build/Grenzwinkel.pdf}
    \caption{Emissionsspektrum beim Grenzwinkel.}
    \label{Grenzwinkel}
\end{figure}    

\noindent In \autoref{Grenzwinkel} lässt sich ein Grenzwinkel von $\theta_\text{Grenz} \qty{5}{\degree}$ ablesen. Der Grenzwinkel ist der Winkel, 
bei dem die gemessene Intensität anfängt zu steigen. Die dazugehörige minimale 
Wellenlänge wird mithilfe der Bragg Bedingung in erster Ordnung $\lambda=2 d \sin(\theta)$ bestimmt. Daraus lässt sich die korrespondierende maximale 
Energie $E=\frac{hc}{\lambda}$ bestimmen. So ergeben sich 

\begin{align}
    \lambda_\text{min} &=\qty{35.1}{\pico \meter}\\
    \label{eqn:E_theta}
    E &= \frac{hc}{2 d \sin(\theta)}\\
    E_\text{max} &= \qty{35.3}{\kilo \electronvolt}
\end{align}

\noindent Bei genauerer Betrachtung kann der entsprechende Winkel für die $K_\alpha$ und die $K_\beta$ Linien bestimmt werden. So tritt die treten die 
Linien bei den folgenden Winkeln auf:

\begin{align}
    \theta_{K_\beta} &= \qty{20.2}{\degree}\\ 
    \theta_{K_\alpha}  &= \qty{22.4}{\degree} 
\end{align}
    
\noindent Mithilfe der Winkel kann nun die Energie der $K_\alpha$- und  $K_\beta$-Übergänge bestimmt werden. Dies geschieht über \autoref{eqn:E_theta}.

\begin{align}
    E_{K_\beta} &= \qty{8.92}{\kilo \electronvolt}\\ 
    E_{K_\alpha}  &= \qty{8.08}{\kilo \electronvolt} 
\end{align}

\noindent Zur besseren Visualisierung wird nun nur der für die Linien wichtige Bereich dargestellt. 

\begin{figure}[H]
    \centering
    \includegraphics[width=0.9\textwidth]{../build/EmissionCu_zoom.pdf}
    \caption{Emissionsspektrum bei der $K_\alpha$ und der $K_\beta$ Linie.}
    \label{zoom}
\end{figure}    

\noindent Aus \autoref{zoom} lassen sich nun die Halbwertsbreiten bestimmen. Dazu werden zunächst die Punkte der Messungen miteinander verbunden. 
Anschließend wird die halbe Höhe ermittelt, indem die Hälfte der Zählrate bei der K-Linie genommen wird. Es wird nun eine horizontale auf Höhe der 
halben ganzen Höhe angesetzt und die Schnittpunkte mit den vorher eingezeichneten Linien als Schnittpunkt konstatiert. Die Differenz der Schnittpunkte 
ist nun nie Breite bei halber Höhe.

\begin{figure}[H]
    \centering
    \includegraphics[width=0.9\textwidth]{../build/Halfheight.pdf}
    \caption{Emissionsspektrum bei der $K_\alpha$ und der $K_\beta$ Linie und deren Breite.}
\end{figure}    

\noindent Die Breite des $K_\beta$-Peaks beträgt \qty{0.45}{\degree} und die des $K_\alpha$-Peaks \qty{0.49}{\degree}. Die Schrittweite der Apparatur 
ist also kleiner als die Breite der Peaks, weshalb diese aufgelöst werden können, auch wenn die Auflösung nicht um ein vielfaches besser ist. 
Es können nur in etwa vier Werte aufgelöst werden, weshalb es nicht möglich ist aussagekräftige und genaue Fehler und Werte zu erreichen. \\

\noindent Nun sollen die Abschirmkonstanten $\sigma_1$, $\sigma_2$ und $\sigma_3$ bestimmt werden. Dies geschieht mithilfe der folgenden Gleichungen. 

\begin{align}
    \label{eqn:Eabs}
    E_{K\text{, abs}} &= R_\infty \left(Z-\sigma_1\right)^2 \\
    E_{K,\alpha}    &= R_\infty\left(\frac{1}{n}\right)^2 \cdot \left(Z- \sigma_1\right)^2 - R_\infty\left(\frac{1}{m}\right)^2 \cdot \left(Z- \sigma_2\right)^2 \\
    E_{K,\beta}     &= R_\infty\left(\frac{1}{n}\right)^2 \cdot \left(Z- \sigma_1\right)^2 - R_\infty\left(\frac{1}{l}\right)^2 \cdot \left(Z- \sigma_3\right)^2 
\end{align}
    
\noindent Dabei gilt hier $n=1, m=2, L=3$ und $R_\infty=\qty{13.6}{\electronvolt}$  \cite{Versuchsanleitung_v602}.
Mit dem gegebenen Versuchsaufbau ist es nicht möglich $E_{K\text{, abs}}$ bzw. $\sigma_1$ zu berechnen. Daher muss auf den Literaturwert von 
$E_{K\text{, abs}} = \qty{8.99}{\kilo \electronvolt}$ \cite{nist} zurückgegriffen werden. Zur Bestimmung von $\sigma_1$ wird \autoref{eqn:Eabs} umgestellt, sowie 
der Literaturwert und die gemessenen Übergangsenergien eingesetzt. Gleiches gilt für $\sigma_2$ und $\sigma_3$. So ergeben sich die folgenden Formeln und Werte:

\begin{align}
    \sigma_1 &= Z - \sqrt{\frac{E_{K\text{, abs}}}{R_\infty}}   &= 3.28 \\
    \sigma_2 &= Z - m \sqrt{\frac{E_{K\text{, abs}}-E_{K,\alpha}}{R_\infty}}    &= 12.57\\
    \sigma_3 &= Z - l \sqrt{\frac{E_{K\text{, abs}}-E_{K,\beta}}{R_\infty}}     &= 21.67
\end{align}

\subsection{Absorptionsspektrum}
Zu Beginn werden nun alle Messergebnisse der verschiedenen Absorber graphisch dargestellt. Es wird außerdem bei dem Anstieg von einem Plateau zum 
anderen eine Ausgleichsgerade hineingelegt, um die halbe Höhe des Anstiegs zu bestimmen.

\begin{figure}[H]
    \centering
    \includegraphics[width=0.8\textwidth]{../build/AbsorptionBr.pdf}
    \caption{Absorptionsspektrum von Brom.}
\end{figure}

\begin{figure}[H]
    \centering
    \includegraphics[width=0.8\textwidth]{../build/AbsorptionGa.pdf}
    \caption{Absorptionsspektrum von Gallium.}
\end{figure}

\begin{figure}[H]
    \centering
    \includegraphics[width=0.8\textwidth]{../build/AbsorptionSr.pdf}
    \caption{Absorptionsspektrum von Strontium.}
\end{figure}

\begin{figure}[H]
    \centering
    \includegraphics[width=0.8\textwidth]{../build/AbsorptionZn.pdf}
    \caption{Absorptionsspektrum von Zink.}
\end{figure}

\begin{figure}[H]
    \centering
    \includegraphics[width=0.8\textwidth]{../build/AbsorptionZr.pdf}
    \caption{Absorptionsspektrum von Zirkonium.}
\end{figure}
    
\noindent Die durch die Auftragung bestimmten halben Höhen sind bei den folgenden Winkeln.

\begin{align*}
    \theta_\text{Br} &= \qty{13.0}{\degree}\\
    \theta_\text{Ga} &= \qty{17.1}{\degree}\\
    \theta_\text{Sr} &= \qty{10.9}{\degree}\\
    \theta_\text{Zn} &= \qty{18.4}{\degree}\\
    \theta_\text{Zr} &= \qty{9.8}{\degree}
\end{align*}

\noindent Bei diesen Winkeln ist ein Sprung in den Abbildungen zu sehen, wie viele Photonen durch das Metall hindurch kommen bzw. abgeschirmt 
werden. Es liegt also ein Sprung in der Energie der Photonen vor. Den Winkel, bei dem dieser Sprung auftritt, ist nun bekannt. Es kann nun über 
\autoref{eqn:E_theta} die dazugehörige Energie berechnet werden. So ergeben sich die folgenden Energien für die K-Kante.

\begin{align*}
    E_\text{k,Br} &= \qty{13.69}{\kilo \electronvolt}\\
    E_\text{k,Ga} &= \qty{10.47}{\kilo \electronvolt}\\
    E_\text{k,Sr} &= \qty{16.28}{\kilo \electronvolt}\\
    E_\text{k,Zn} &= \qty{9.75}{\kilo \electronvolt}\\
    E_\text{k,Zr} &= \qty{18.09}{\kilo \electronvolt}
\end{align*}

\noindent Nun kann mithilfe der Absorptionsenergien $E_K$ die Abschirmkonstante berechnet werden. Dies geschieht über 

\begin{equation*}
    \sigma_K = Z - \sqrt{\frac{E_K}{R_\infty} - \frac{\alpha^2 Z^4}{4}}
\end{equation*}

\noindent Dabei ist $Z$ die jeweilige Ordnungszahl des Elements und $\alpha=\num{7.297e-3}$ die Sommerfeldsche Feinstrukturkonstante.
Die Abschirmkonstanten ergeben sich dementsprechend zu 

\begin{align*}
    \sigma_\text{k,Br} &= \num{3.59}   \\
    \sigma_\text{k,Ga} &= \num{3.48}   \\
    \sigma_\text{k,Sr} &= \num{3.80}   \\
    \sigma_\text{k,Zn} &= \num{3.42}    \\
    \sigma_\text{k,Zr} &= \num{4.00}   
\end{align*}


\subsection{Moosleysches-Gesetz}
Das Moosleysche Gesetz sagt aus, dass die Absorptionsenergie  proportional zum Quadrat der Ordnungszahl ist: $E_K \propto Z^2.$
Dabei gilt insbesondere auch, dass diese über die Rydberg-Energie $R_\infty$ verbunden sind. Werden nun $\sqrt{E_K}$ gegen die jeweiligen 
Ordnungszahlen $Z$ aufgetragen, so sollte sich eine lineare Funktion ergeben, dessen Steigung die Wurzel der Rydberg-Energie entspricht. 

\begin{figure}[H]
    \centering
    \includegraphics[width=0.9\textwidth]{../build/Moseley.pdf}
    \caption{Überprüfen des Moseleyschen Gesetzes.}
\end{figure}

\noindent Die Ausgleichsgerade besitzt die Form 

\begin{equation*}
    (\num{3.59\pm0.02})\sqrt{\unit{\electronvolt}}x - (\num{8.7\pm0.8})\sqrt{\unit{\electronvolt}}.
\end{equation*}

\noindent Wird nun die Steigung quadriert, so ergibt sich ein experimenteller Wert für die Rydberg-Energie von 

\begin{equation*}
    R_\infty = \qty{12.85\pm0.16}{\electronvolt}.
\end{equation*}
















\end{document}