%\input{../../header.tex}

%\begin{document}
%\section{Zielsetzung}
%\label{sec:Theorie}

%\section{Theorie}

\section{Vorbereitung}

Die Literaturwerte\cite{nist} der Energien von Cu-$\text{K}_\alpha$ und Cu-$\text{K}_\beta$ sowie die dazu korrespondierenden Winkel lauten wie folgt

\begin{align*}
    E_\alpha &= \qty{8}{\kilo\electronvolt}  &\theta_\alpha &= \qty{22.4}{\degree} \\
    E_\beta &= \qty{8.9}{\kilo\electronvolt}  &\theta_\beta &= \qty{20.2}{\degree}  \\
\end{align*}

\noindent Die Elementspezifische Tabelle des Bragg-Winkels wie durch Einsatz der Literaturwerte\cite{last} ergänzt.

\begin{table}[H]
    \centering
    \caption{Vorbereitungstabelle.}
    \label{tab:BraggBedingungTab}
    \begin{tblr}{
        colspec = {S S[table-format=2.0] S[table-format=2.2] S[table-format=2.1] S[table-format=1.2]},
        row{1} = {guard, mode=math},
    }
    \toprule
    \text{Element} & \text{Z} & E_\text{K,Lit} \mathbin{/} \unit{\kilo\electronvolt} & \theta_\text{K,Lit} \mathbin{/} \unit{\degree} & \sigma_\text{K} \\
    \midrule
        Zn  &  30  &  9.65  &  18.6  &  3.56  \\
        Br  &  35  &  13.47 &  13.2  &  3.85  \\
        Ga  &  31  &  10.37 &  17.3  &  3.61  \\
        Sr  &  38  &  16.10 &  11.0  &  4.00  \\
        Zr  &  40  &  17.99 &  9.9   &  4.10  \\
    \bottomrule
    \end{tblr}
\end{table}

%\section{Fehlerrechnung}
%\end{document}