\input{../../header.tex}

\begin{document}
\section{Diskussion}
\label{sec:Diskussion}

In diesem Experiment stellt die Überprüfbarkeit der Ergebnisse Herausforderungen dar. 
Das Ziel dieses Experimentes ist nicht wie im klassischen Sinne Konstanten oder Größen 
herauszufinden, sondern den Umgang und Anwendungsmöglichkeiten der Ultraschalltechnik 
zu betrachten. Es wird im Folgenden den Umständen geschuldet also mehr auf die 
Techniken als auf konkrete Werte eingegangen. 

\subsection{Abstandsmessungen mit einem Acrylblock}
Die Durchführung der Durchmessermessung hat konkret keine bemerkenswerten Schwierigkeiten 
bereitet. Die Lokalisierung der Löcher in dem Computerprogramm ist sehr eindeutig. Zur 
Verifizierung, ob der ausgewählte Peak von dem Loch stammt, wird die Ultraschallsonde 
leicht nach links und rechts verschoben. Verringert sich dann die  ausgewählte Amplitude, 
kann mit großer Sicherheit gesagt werden, dass der Peak im A-Scan von dem zu beobachtenden 
Loch stammt. Jedoch mussten bei der Auswertung annahmen getroffen werden, welche zu einer 
Ungenauigkeit in den Ergebnissen führen können. So muss die Schallgeschwindigkeit in der 
Anpassungsschicht auf die von Acryl abgeschätzt werden. Des Weiteren wäre es genauer, wenn 
die Dicke der Anpassungsschicht von vornherein bekannt wäre, sodass sich keine Fehler aus 
jener Berechnung fortpflanzen können. Außerdem wir pro Loch nur eine Messung durchgeführt. 
Dies führt dazu, dass statistische Fehler nicht ausgeglichen werden können. In der Auswertung 
wird eine Schallgeschwindigkeit in Acryl von \qty{273,78}{\meter \per \second} konstituiert. 
Verglichen mit dem Literaturwert von \qty{2730}{\meter\per\second} ist dies eine Abweichung 
von lediglich \qty{1,1}{\percent}. Dies deutet darauf hin, dass obwohl nur eine Messung pro 
Loch aufgenommen wird, dennoch sehr genaue Werte bestimmt werden. Die über die 
Schallgeschwindigkeit bestimmten Durchmesser der Löcher $D_\text{Schall}$ werden nun mit den 
über den Messchieber bestimmten Durchmesser $D_\text{Mess}$ verglichen.

\begin{table}[H]
    \centering 
    \caption{Durchmesser des Acrylblocks.}
    \label{tab:Durchmesser}
    \begin{tblr}{
        colspec = {S[table-format=2.2] S[table-format=2.2]},
        row{1} = {guard, mode=math},
        }
        \toprule 
        \text{} D_\text{Schall} \mathbin{/} \unit{\milli \meter} & \text{Laufzeit } t_\text{Bottom} \mathbin{/} \unit{\micro\second} \\
        \midrule 
        11.76  &  43.45 \\
        10.35  &  44.68 \\
        41.69  &  09.86 \\
        36.30  &  16.05 \\
        30.70  &  22.42 \\
        25.20  &  28.59 \\
        19.35  &  34.43 \\
        13.41  &  40.33 \\
        07.53  &  46.17 \\
        04.72  &  52.22 \\
        40.02  &  11.32 \\
        \bottomrule
    \end{tblr}
\end{table}




%Anpassungsschicht muss selbst bestimmt werden
%Geschwindigkeit über Tumore nocht konstant
%Schallgeschwindigkeit in wasser angenommen und nicht in silikon oder Tumor
%Reindrücken in Silikon verändert evtl auch Lage des Tumors
%Ultraschallsonde nicht immer orthogonal auf Brust
%Werte wie Größe des Tumors, und Herzzeitvolumen von der Apparatur abhängig und daher keine Literaturwerte
%Messungen gut, da schallgeschwindigkeit 1% von Literaturwert abweicht



\end{document}
