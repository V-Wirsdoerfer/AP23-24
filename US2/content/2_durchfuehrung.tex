\input{../../header.tex}

\begin{document}
\section{Versuchsaufbau und Versuchsdurchführung}

Es sollen verschiedene Anwendungsbeispiele der Ultraschalltechnik untersucht und erprobt werden.
Dazu gehören unter anderem die Tiefenmessung, die medizinische Diagnostik und die zeitliche Auflösung.
Für die Tiefenmessung wird ein Acrylblock mit Löchern von zwei Seiten vermessen. Die medizinische 
Diagnostik bezieht sich auf die Erkennung von Brustkrebs anhand eines Brustmodells. Die zeitliche 
Auflösung wird mithilfe eines Herzmodells ermittelt, wobei das Herzzeitvolumen ermittelt werden kann.

\subsection{Abstandsmessung mithilfe von Ultraschalltechnik}
Es kann mithilfe von Ultraschalltechnik der Abstand zu einem Objekt in guter Näherung bestimmt werden. 
Dies beruht auf der Laufzeitmessung des Ultraschalls und wird sich nun zunutze gemacht. Über die 
Schallgeschwindigkeit kann der Abstand allein mit der Laufzeit bestimmt werden. So muss nur die 
Laufzeit gemessen werden. Erprobt wird diese Methode anhand eines Acrylblocks mit elf Löchern. Dieser 
hat die Löcher, wie in der folgenden Graphik zusehen ist, angeordnet.

\begin{figure}[H]
    \centering
    \includegraphics[width=0.6\textwidth]{Acrylblock.png}
    \caption{Skizze des Acrylblocks.}
\end{figure}

Die Ultraschallsonde wird zunächst auf die obere Kante mit Kontaktmittel gelegt. Von dort wird mithilfe 
des A-Scans die Laufzeit zu jedem Loch gemessen. Dabei ist zu erwähnen, dass es sich um die Laufzeit 
des Hin- und Rückwegs handelt, also sozusagen die zweifache Strecke gemessen wird. Die Löcher werden 
chronologisch vermessen. Nachdem das letzte Loch vermessen worden ist, wird der Acrylblock umgedreht 
und die Ultraschallsonde mit Kontaktmittel auf die Unterseite gestellt. Die Unterseite ist also durch 
die Drehung nun oben. Alle Löcher werden wie vorher der Reiche nach mit dem A-Scan vermessen. \\

\noindent Um Referenzwerte für die Durchmesser der Löcher zu bekommen, werden die gemessenen Strecken 
mit dem Messschieber ausgemessen. Es wird also analog die Dicke des Acryls zwischen Loch und Ober- 
beziehungsweise Unterkante vermessen. Damit der Lochdurchmesser bestimmt werden kann, wird die 
Gesamthöhe des Acrylblocks ebenfalls bestimmt. 


\subsection{Medizinische Diagnostik}
Die Ultraschalltechnik ist in der Medizin weit verbreitet und wird als bildgebendes Verfahren häufig 
verwendet. Dies wird am Beispiel von Brustkrebs erprobt. Untersucht wird ein Brustmodell aus Silikon 
mit zwei Tumoren. Zunächst muss der Brustkrebs lokalisiert werden, wofür das Modell abgetastet wird. 
Es können per Hand zwei Tumore festgestellt werden. Diese befinden sich oberhalb und rechts neben 
der Brustwarze und werden in der folgenden Abbildung umkreist. 

\begin{figure}[H]
    \centering
    \includegraphics[width = 0.8\textwidth]{Tumororientierung2.jpg}
    \caption{Lokalisierung der Tumoren am Modell.}
\end{figure}
 
\noindent Diese beiden Tumore werden nun mithilfe des B-Scans visualisiert. Es wird dafür mit der 
Ultraschallsonde ein Schnittbild der Brust aufgenommen, auf dem der Tumor zu erkennen ist. Zwischen 
Ultraschallkopf und Silikon wird Ultraschallgel als Kontakt- und Gleitmittel aufgetragen. Die Sonde 
muss mit konstanter Geschwindigkeit in einer Linie über den Tumor gefahren werden. Anschließend wird 
die verstrichene Strecke mit dem Messschieber vermessen und in das Programm eingetragen. Die Skala 
auf den Scans wird so in eine Strecke umgewandelt, damit die Größe des Tumors ausgewertet werden kann.


%\section{Versuchsdurchführung}

\section{Messwerte}

\end{document}

