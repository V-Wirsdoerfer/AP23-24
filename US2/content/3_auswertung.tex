\input{../../header.tex}

\begin{document}

\section{Messwerte}
\label{sec:Messwerte}

Zu Beginn des Experiments werden die geometrischen Abmessungen der Acrylblocks aufgenommen. In Analogie zu Abbildung \ref{fig:Acrylblock}
werden hierbei die Distanzen zwischen der oberen Blockkante und der näherliegenden Grenzfläche der Löcher als \enquote{Top} und die Strecken
von der unteren Kante bis zu den unteren Grenzflächen als \enquote{Bottom} bezeichnet. Mittels einer Schieblehre werden somit 
die folgenden Messungen dokumentiert:

\begin{table}[H]
    \centering 
    \caption{Geometrische Abmessung des Acrylblocks.}
    \begin{tblr}{
        colspec = {S[table-format=2.2] S[table-format=2.2]},
        row{1} = {guard, mode=math},
        }
        \toprule 
        \text{Top} \mathbin{/} \unit{\milli\meter} & \text{Bottom} \mathbin{/} \unit{\milli\meter} \\
        \midrule
        19.15  &  59.05 \\
        17.50  &  60.80 \\
        60.50  &  13.20 \\
        53.10  &  21.60 \\
        45.50  &  30.20 \\
        37.95  &  38.60 \\
        29.90  &  46.65 \\
        21.80  &  54.70 \\
        13.80  &  62.70 \\
        05.75  &  70.65 \\
        54.45  &  15.35 \\
        \bottomrule
    \end{tblr}
    \label{tab:AbmessungenBlock}
\end{table} 

\noindent Die Gesamthöhe des Acrylblocks beträgt \qty{79}{\milli\meter}. Zusätzlich beinhalten alle zuvor aufgeführten Streckenwerte
aufgrund der Schieblehre einen Fehler von $\pm\qty{0.025}{\milli\meter}$.\\

\noindent Im Folgenden wird mittels des in der Theorie \ref{sec:Theorie} angesprochnenen \emph{Impuls-Echo Verfahrens} ein \emph{A-Scan}
durchgeführt, um die Laufzeiten zu messen. Das an der Grenzfläche reflektierte Echo wird hierbei durch einen erhöhten Amplitudenausschlag 
auf dem Programm angezeigt, woraus sich die Laufzeit $t$ in $\unit{\micro\second}$ ableiten lässt. Entsprechend der geometrischen Messung 
wird auch hierbei zwischen \enquote{Top} und \enquote{Bottom} unterschieden und dies tabellarische kenntlich gemacht.

\begin{table}[H]
    \centering 
    \caption{Laufzeitmessung des Acrylblocks.}
    \label{tab:Laufzeitmessung}
    \begin{tblr}{
        colspec = {S[table-format=2.2] S[table-format=2.2]},
        row{1} = {guard, mode=math},
        }
        \toprule 
        \text{Laufzeit} t_\text{Top} \mathbin{/} \unit{\micro\second} & \text{Laufzeit} t_\text{Bottom} \mathbin{/} \unit{\micro\second} \\
        \midrule 
        11.76  &  43.45 \\
        10.35  &  44.68 \\
        41.69  &  09.86 \\
        36.30  &  16.05 \\
        30.70  &  22.42 \\
        25.20  &  28.59 \\
        19.35  &  34.43 \\
        13.41  &  40.33 \\
        07.53  &  46.17 \\
        04.72  &  52.22 \\
        40.02  &  11.32 \\
        \bottomrule
    \end{tblr}
\end{table}

\noindent Zuletzt wird das Herzmodell untersucht, um Rückschlüsse auf das Herzzeitvolumen (HZV) ziehen zu können. Hierzu wird 
die Herzfrequenz mittel eines \emph{TM-Scans} aufgezeichnet und auf Grundlage dieser Daten das endsystolische Volumen (ESV) sowie die 
Frequenz $f$ berechnet. Die aufgezeichneten Messwerte lauten wie folgt:

\begin{table}[H]
    \centering 
    \caption{Messung der Herzfrequenz mittels eines \emph{TM-Scans}.}
    \begin{tblr}{
        colspec = {S[table-format=1.2] S[table-format=2.2]},
        row{1} = {guard, mode=math},
        }
        \toprule 
        \text{Periodendauer} T  \mathbin{/} \unit{\second} & \text{Auslenkung} \mathbin{/} \unit{\micro\second} \\
        \midrule
        2.17  &  12.24 \\
        2.25  &  10.30 \\
        1.73  &  10.30 \\
        1.73  &  10.30 \\
        1.66  &  10.78 \\
        1.79  &  11.26 \\
        1.50  &  11.26 \\
        1.73  &  11.26 \\
        1.47  &  10.90 \\
        \bottomrule
    \end{tblr}
    \label{tab:AbmessungenBlock}
\end{table}

\section{Auswertung}


\label{sec:Auswertung}

\end{document}
