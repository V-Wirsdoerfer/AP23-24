\documentclass[
  bibliography=totoc,     % Literatur im Inhaltsverzeichnis
  captions=tableheading,  % Tabellenüberschriften
  titlepage=firstiscover, % Titelseite ist Deckblatt
]{scrartcl}

% Paket float verbessern
\usepackage{scrhack}

% Warnung, falls nochmal kompiliert werden muss
\usepackage[aux]{rerunfilecheck}

% unverzichtbare Mathe-Befehle
\usepackage{amsmath}
% viele Mathe-Symbole
\usepackage{amssymb}
% Erweiterungen für amsmath
\usepackage{mathtools}

% Fonteinstellungen
\usepackage{fontspec}
% Latin Modern Fonts werden automatisch geladen
% Alternativ zum Beispiel:
%\setromanfont{Libertinus Serif}
%\setsansfont{Libertinus Sans}
%\setmonofont{Libertinus Mono}

% Wenn man andere Schriftarten gesetzt hat,
% sollte man das Seiten-Layout neu berechnen lassen
\recalctypearea{}

% deutsche Spracheinstellungen
\usepackage[ngerman]{babel}


\usepackage[
  math-style=ISO,    % ┐
  bold-style=ISO,    % │
  sans-style=italic, % │ ISO-Standard folgen
  nabla=upright,     % │
  partial=upright,   % │
  mathrm=sym,        % ┘
  warnings-off={           % ┐
    mathtools-colon,       % │ unnötige Warnungen ausschalten
    mathtools-overbracket, % │
  },                       % ┘
]{unicode-math}

% traditionelle Fonts für Mathematik
\setmathfont{Latin Modern Math}
% Alternativ zum Beispiel:
%\setmathfont{Libertinus Math}

\setmathfont{XITS Math}[range={scr, bfscr}]
\setmathfont{XITS Math}[range={cal, bfcal}, StylisticSet=1]

% Zahlen und Einheiten
\usepackage[
  locale=DE,                   % deutsche Einstellungen
  separate-uncertainty=true,   % immer Unsicherheit mit \pm
  per-mode=symbol-or-fraction, % / in inline math, fraction in display math
]{siunitx}

% chemische Formeln
\usepackage[
  version=4,
  math-greek=default, % ┐ mit unicode-math zusammenarbeiten
  text-greek=default, % ┘
]{mhchem}

% richtige Anführungszeichen
\usepackage[autostyle]{csquotes}

% schöne Brüche im Text
\usepackage{xfrac}

% Standardplatzierung für Floats einstellen
\usepackage{float}
\floatplacement{figure}{htbp}
\floatplacement{table}{htbp}

% Floats innerhalb einer Section halten
\usepackage[
  section, % Floats innerhalb der Section halten
  below,   % unterhalb der Section aber auf der selben Seite ist ok
]{placeins}

% Seite drehen für breite Tabellen: landscape Umgebung
\usepackage{pdflscape}

% Captions schöner machen.
\usepackage[
  labelfont=bf,        % Tabelle x: Abbildung y: ist jetzt fett
  font=small,          % Schrift etwas kleiner als Dokument
  width=0.9\textwidth, % maximale Breite einer Caption schmaler
]{caption}
% subfigure, subtable, subref
\usepackage{subcaption}

% Grafiken können eingebunden werden
\usepackage{graphicx}

% schöne Tabellen
\usepackage{tabularray}
\UseTblrLibrary{booktabs, siunitx}

% Verbesserungen am Schriftbild
\usepackage{microtype}

% Literaturverzeichnis
\usepackage[
  backend=biber,
]{biblatex}
% Quellendatenbank
\addbibresource{lit.bib}
\addbibresource{programme.bib}

% Hyperlinks im Dokument
\usepackage[
  german,
  unicode,        % Unicode in PDF-Attributen erlauben
  pdfusetitle,    % Titel, Autoren und Datum als PDF-Attribute
  pdfcreator={},  % ┐ PDF-Attribute säubern
  pdfproducer={}, % ┘
]{hyperref}
% erweiterte Bookmarks im PDF
\usepackage{bookmark}

% Trennung von Wörtern mit Strichen
\usepackage[shortcuts]{extdash}

\author{%
  Vincent Wirsdörfer\\%
  \href{mailto:vincent.wirsdoerfer@udo.edu}{authorA@udo.edu}%
  \and%
  Joris Daus\\%
  \href{mailto:joris.daus@udo.edu}{authorB@udo.edu}%
}
\publishers{TU Dortmund – Fakultät Physik}


\begin{document}

\section{Messwerte}
\label{sec:Messwerte}

Zu Beginn des Experiments werden die geometrischen Abmessungen der Acrylblocks aufgenommen. In Analogie zu Abbildung \ref{fig:Acrylblock}
werden hierbei die Distanzen zwischen der oberen Blockkante und der näherliegenden Grenzfläche der Löcher als \enquote{Top} und die Strecken
von der unteren Kante bis zu den unteren Grenzflächen als \enquote{Bottom} bezeichnet. Mittels einer Schieblehre werden somit 
die folgenden Messungen dokumentiert:

\begin{table}[H]
    \centering 
    \caption{Geometrische Abmessung des Acrylblocks.}
    \begin{tblr}{
        colspec = {S[table-format=2.2] S[table-format=2.2]},
        row{1} = {guard, mode=math},
        }
        \toprule 
        \text{Top} \mathbin{/} \unit{\milli\meter} & \text{Bottom} \mathbin{/} \unit{\milli\meter} \\
        \midrule
        19.15  &  59.05 \\
        17.50  &  60.80 \\
        60.50  &  13.20 \\
        53.10  &  21.60 \\
        45.50  &  30.20 \\
        37.95  &  38.60 \\
        29.90  &  46.65 \\
        21.80  &  54.70 \\
        13.80  &  62.70 \\
        05.75  &  70.65 \\
        54.45  &  15.35 \\
        \bottomrule
    \end{tblr}
    \label{tab:AbmessungenBlock}
\end{table} 

\noindent Die Gesamthöhe des Acrylblocks beträgt \qty{79}{\milli\meter}. Zusätzlich beinhalten alle zuvor aufgeführten Streckenwerte
aufgrund der Schieblehre einen Fehler von $\pm\qty{0.025}{\milli\meter}$.\\

\noindent Im Folgenden wird mittels des in der Theorie \ref{sec:Theorie} angesprochnenen \emph{Impuls-Echo Verfahrens} ein \emph{A-Scan}
durchgeführt, um die Laufzeiten zu messen. Das an der Grenzfläche reflektierte Echo wird hierbei durch einen erhöhten Amplitudenausschlag 
auf dem Programm angezeigt, woraus sich die Laufzeit $t$ in $\unit{\micro\second}$ ableiten lässt. Entsprechend der geometrischen Messung 
wird auch hierbei zwischen \enquote{Top} und \enquote{Bottom} unterschieden und dies tabellarische kenntlich gemacht.

\begin{table}[H]
    \centering 
    \caption{Laufzeitmessung des Acrylblocks.}
    \begin{tblr}{
        colspec = {S[table-format=2.2] S[table-format=2.2]},
        row{1} = {guard, mode=math},
        }
        \toprule 
        \text{Laufzeit} t_\text{Top} \mathbin{/} \unit{\micro\second} & \text{Laufzeit} t_\text{Bottom} \mathbin{/} \unit{\micro\second} \\
        \midrule 
        11.76  &  43.45 \\
        10.35  &  44.68 \\
        41.69  &  09.86 \\
        36.30  &  16.05 \\
        30.70  &  22.42 \\
        25.20  &  28.59 \\
        19.35  &  34.43 \\
        13.41  &  40.33 \\
        07.53  &  46.17 \\
        04.72  &  52.22 \\
        40.02  &  11.32 \\
        \bottomrule
        \label{tab:Laufzeitmessung}
    \end{tblr}
\end{table}

\noindent Zuletzt wird das Herzmodell untersucht, um Rückschlüsse auf das Herzzeitvolumen (HZV) ziehen zu können. Hierzu wird 
die Herzfrequenz mittel eines \emph{TM-Scans} aufgezeichnet und auf Grundlage dieser Daten das endsystolische Volumen (ESV) sowie die 
Frequenz $f$ berechnet. Die aufgezeichneten Messwerte lauten wie folgt:

\begin{table}[H]
    \centering 
    \caption{Messung der Herzfrequenz mittels eines \emph{TM-Scans}.}
    \begin{tblr}{
        colspec = {S[table-format=1.2] S[table-format=2.2]},
        row{1} = {guard, mode=math},
        }
        \toprule 
        \text{Periodendauer} T  \mathbin{/} \unit{\second} & \text{Auslenkung} \mathbin{/} \unit{\micro\second} \\
        \midrule
        2.17  &  12.24 \\
        2.25  &  10.30 \\
        1.73  &  10.30 \\
        1.73  &  10.30 \\
        1.66  &  10.78 \\
        1.79  &  11.26 \\
        1.50  &  11.26 \\
        1.73  &  11.26 \\
        1.47  &  10.90 \\
        \bottomrule
    \end{tblr}
    \label{tab:AbmessungenBlock}
\end{table}

\section{Auswertung}


\label{sec:Auswertung}

\end{document}
