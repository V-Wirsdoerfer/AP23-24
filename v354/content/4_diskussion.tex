\input{../../header.tex}

\begin{document}
\section{Diskussion}
\label{sec:Diskussion}

Der berechnete Wert des effektiven Widerstandes $R_\text{eff}$ ist mehr als doppelt so groß, wie der eingebaute Widerstand. 
Dies kann durch nicht beachtete Innenwiderstände im Tastkopf, in der Spule oder den Kabeln erklärt werden. 
Hinzu kommt der Blindwiderstand der Spule und des Kondensators. Außerdem kann das Bandpassverhalten der Schaltung die Messung verzerren.

\noindent Der berechnete Wert für den Widerstand, bei dem der aperiodische Grenzfall auftritt, weicht um 20 \% ab. Dies kann an systematischen 
Fehlern liegen, da es keine klare Grenze gibt, bei dem der Fall eintritt. Es wurde nach Augenmaß entschieden und auf einem Oszilloskop mit kleiner Anzeige 
abgelesen. Dies sind Faktoren, die systematische Fehler begünstigen.

\noindent Die experimentell bestimmten Frequenzen $\omega_-$ und $\omega_+$ sind in einer anderen Größenordnung, als die theoretisch bestimmten. 
Jedoch sind die daraus resultierenden Güten mit 6,3 \% Abweichung sehr nah beieinander. Da zur Bestimmung der Güte die Differenz berechnet werden muss, 
mitteln sich systematische Fehler raus. Das lässt darauf schließen, dass massive systematische Fehler begangen worden sind.  
 


\end{document}
