\input{../../header.tex}

\begin{document}
\section{Versuchsdurchführung}

Zu Beginn des Versuchs werden die einzelnen Bestandteile des Oszilloskops, sowie des Schwingkreises genauer untersucht 
und auf ihre Funktion überprüft. Im Anschluss soll der effektive Dämpfungswiderstand der Schwingung durch die 
Zeitabhängigkeit der Amplitude berechnet werden. Dazu wird folgende Schaltung aufgebaut:

\begin{figure}
    \centering
    \includegraphics[height=4cm]{5a.png}
    \caption{Schaltung zur Untersuchung des effektiven Dämpfungswiderstandes \cite{Versuchsanleitung_v354}.}
    \label{fig:5a}
\end{figure}

\noindent Hierbei ist jedoch zu erwähnen, dass der in Abbildung \ref{fig:5a} zu erkennende Nadelimpulsgenerator \textbf{kein}
Bestandteil des Versuchs ist. Er wird in diesem Fall durch eine Rechteckspannung ersetzt.\\
Auf der digitalen Anzeige des Oszilloskops zeigt sich nun eine gedämpfte Schwingung. Um die zeitliche Abhängigkeit der Amplitude 
zu analysieren, werden im Folgenden Wertepaare \{$t$ , $U_\text{C}(t)$\} abgelesen und notiert. Um dieses Bild zu komplettieren, 
werden sowohl positive als auch negative Amplituden berücksichtigt.\\\\

\noindent Im nächsten Teilabschnitt des Versuchs soll der aperiodische Grenzwiderstand $R_\text{ap}$ bestimmt werden. Dazu wird 
die folgende Schaltung aufgebaut:

\begin{figure}[H]
    \centering
    \includegraphics[height=4cm]{5b.png}
    \caption{Schaltung zur Untersuchung des aperiodischen Grenzwiderstandes \cite{Versuchsanleitung_v354}.}
    \label{fig:5b}
\end{figure}

\noindent Zuerst wird der regelbare Widerstand auf sein Maximum gedreht, um ein Relaxationsverhalten (stetige Abnahme der
Kondensatorspannung) zu simulieren. Im Anschluss wird dieser Widerstand behutsam verringert, bis auf der digitalen Anzeige 
des Oszilloskops ein \enquote{Überschwingverhalten} erkennbar ist. Somit ist der aperiodische Grenzwiderstand bereits überschritten.
Dementsprechend muss der Widerstand erneut erhöht werden, sodass das Überschwingverhalten gerade verschwindet. Retrospektiv 
betrachtet wird also versucht, die Abschätzungen aus Gleichung \eqref{eqn:reell} und \eqref{eqn:imaginaer} in die Äquivalenz 
aus Gleichung \eqref{eqn:Grenzfall} zu verwandeln.\\\\

\noindent Im letzten Teil des Versuchs wird die Frequenzabhängigkeit der Kondensatorspannung beobachtet. Die dazu 
benötigte Schaltung wird in der folgenden Abbildung dargestellt.

\begin{figure}[H]
    \centering
    \includegraphics[height=4cm]{5c.png}
    \caption{Schaltung zur Untersuchung der Frequenzabhängigkeit der Kondensatorspannung \cite{Versuchsanleitung_v354}.}
    \label{fig:5c}
\end{figure}

\noindent Zunächst wird die Erregerspannung gemessen. Im Anschluss wird die Frequenz am Sinusgenerator variiert, um 
diverse Wertepaare \{$f$, $U_\text{C}(f)$\} zu ermitteln. Diese Datenpaare aus Frequenz und Kondensatorspannung werden letztlich in 
das Heft übertragen.

\section{Messwerte}

\subsection{Zeitabhängigkeit der Amplitude}

Die Untersuchung der Frequenzabhängigkeit der Amplitude zur Bestimmung des effektiven Dämpfungswiderstandes liefert
folgende Wertepaare:

\begin{table}[H]
    \centering
    \caption{Messdaten zur Bestimmung des effektiven Dämpfungswiderstandes.}
    \sisetup{table-format=1.1}
    \begin{tblr}{
        colspec = {S[table-format=3.0] S S[table-format=3.0] S},
        row{1} = {guard}, row{2} = {guard, mode=math},
    }
    \toprule
    \SetCell[c=2]{c} Positive Amplituden & & \SetCell[c=2]{c} Negative Amplituden & \\
    \cmidrule[lr]{1-2}\cmidrule[lr]{3-4}
    t \mathbin{/} \unit{\micro \second} & U_\text{C} \mathbin{/} \unit{\volt} &  t \mathbin{/} \unit{\micro \second} & U_\text{C} \mathbin{/} \unit{\volt} \\
    \midrule
    5e-6    & 3.8 & 15e-6   & -3.4 \\
    35e-6   & 3.2 & 45e-6   & -2.9 \\
    65e-6   & 2.7 & 75e-6   & -2.4 \\
    90e-6   & 2.3 & 105e-6  & -2.0 \\
    120e-6  & 1.9 & 135e-6  & -1.7 \\
    150e-6  & 1.6 & 165e-6  & -1.4 \\
    180e-6  & 1.4 & 190e-6  & -1.2 \\
    210e-6  & 1.2 & 220e-6  & -1.0 \\
    240e-6  & 1.0 & 260e-6  & -0.9 \\
    275e-6  & 0.8 & 285e-6  & -0.8 \\
    305e-6  & 0.7 & 315e-6  & -0.6 \\
    330e-6  & 0.6 & 345e-6  & -0.6 \\
    360e-6  & 0.5 & 375e-6  & -0.5 \\
    420e-6  & 0.4 & 435e-6  & -0.3 \\
    480e-6  & 0.3 & 495e-6  & -0.2 \\
    \bottomrule
    \end{tblr}
\end{table}

\subsection{Frequenzabhängigkeit der Kondensatorspannung}

Durch die Beobachtung der Frequenzabhängigkeit der Kondensatorspannung werden 
folgende Datenpaare aus Frequenz und Spannung ermittelt:

\begin{table}
    \centering 
    \caption{Messdaten zur Bestimmung der Frequenzabhängigkeit der Kondensatorspannung.}
    \sisetup{table-format=1.1}
    \begin{tblr}{
        colspec = {S[table-format=2.0] S},
        row{1} = {guard, mode=math},
    }
    \toprule
    f \mathbin{/} \unit{\kilo\hertz} & U_\text{C} \mathbin{/} \unit{\volt} \\
    \midrule 
    5  & 2,5  \\
    10 & 2,8  \\
    15 & 3,0  \\
    20 & 3,6  \\
    22 & 4,0  \\
    24 & 4,6  \\
    26 & 5,2  \\
    28 & 6,4  \\
    30 & 7,6  \\
    32 & 9,0  \\
    34 & 9,0  \\
    36 & 8,0  \\
    38 & 6,5  \\
    40 & 4,8  \\
    42 & 4,0  \\
    45 & 2,9  \\
    50 & 2,0  \\
    \bottomrule
    \end{tblr}
\end{table}

\end{document}

