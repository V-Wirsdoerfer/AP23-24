%\documentclass[
  bibliography=totoc,     % Literatur im Inhaltsverzeichnis
  captions=tableheading,  % Tabellenüberschriften
  titlepage=firstiscover, % Titelseite ist Deckblatt
]{scrartcl}

% Paket float verbessern
\usepackage{scrhack}

% Warnung, falls nochmal kompiliert werden muss
\usepackage[aux]{rerunfilecheck}

% unverzichtbare Mathe-Befehle
\usepackage{amsmath}
% viele Mathe-Symbole
\usepackage{amssymb}
% Erweiterungen für amsmath
\usepackage{mathtools}

% Fonteinstellungen
\usepackage{fontspec}
% Latin Modern Fonts werden automatisch geladen
% Alternativ zum Beispiel:
%\setromanfont{Libertinus Serif}
%\setsansfont{Libertinus Sans}
%\setmonofont{Libertinus Mono}

% Wenn man andere Schriftarten gesetzt hat,
% sollte man das Seiten-Layout neu berechnen lassen
\recalctypearea{}

% deutsche Spracheinstellungen
\usepackage[ngerman]{babel}


\usepackage[
  math-style=ISO,    % ┐
  bold-style=ISO,    % │
  sans-style=italic, % │ ISO-Standard folgen
  nabla=upright,     % │
  partial=upright,   % │
  mathrm=sym,        % ┘
  warnings-off={           % ┐
    mathtools-colon,       % │ unnötige Warnungen ausschalten
    mathtools-overbracket, % │
  },                       % ┘
]{unicode-math}

% traditionelle Fonts für Mathematik
\setmathfont{Latin Modern Math}
% Alternativ zum Beispiel:
%\setmathfont{Libertinus Math}

\setmathfont{XITS Math}[range={scr, bfscr}]
\setmathfont{XITS Math}[range={cal, bfcal}, StylisticSet=1]

% Zahlen und Einheiten
\usepackage[
  locale=DE,                   % deutsche Einstellungen
  separate-uncertainty=true,   % immer Unsicherheit mit \pm
  per-mode=symbol-or-fraction, % / in inline math, fraction in display math
]{siunitx}

% chemische Formeln
\usepackage[
  version=4,
  math-greek=default, % ┐ mit unicode-math zusammenarbeiten
  text-greek=default, % ┘
]{mhchem}

% richtige Anführungszeichen
\usepackage[autostyle]{csquotes}

% schöne Brüche im Text
\usepackage{xfrac}

% Standardplatzierung für Floats einstellen
\usepackage{float}
\floatplacement{figure}{htbp}
\floatplacement{table}{htbp}

% Floats innerhalb einer Section halten
\usepackage[
  section, % Floats innerhalb der Section halten
  below,   % unterhalb der Section aber auf der selben Seite ist ok
]{placeins}

% Seite drehen für breite Tabellen: landscape Umgebung
\usepackage{pdflscape}

% Captions schöner machen.
\usepackage[
  labelfont=bf,        % Tabelle x: Abbildung y: ist jetzt fett
  font=small,          % Schrift etwas kleiner als Dokument
  width=0.9\textwidth, % maximale Breite einer Caption schmaler
]{caption}
% subfigure, subtable, subref
\usepackage{subcaption}

% Grafiken können eingebunden werden
\usepackage{graphicx}

% schöne Tabellen
\usepackage{tabularray}
\UseTblrLibrary{booktabs, siunitx}

% Verbesserungen am Schriftbild
\usepackage{microtype}

% Literaturverzeichnis
\usepackage[
  backend=biber,
]{biblatex}
% Quellendatenbank
\addbibresource{lit.bib}
\addbibresource{programme.bib}

% Hyperlinks im Dokument
\usepackage[
  german,
  unicode,        % Unicode in PDF-Attributen erlauben
  pdfusetitle,    % Titel, Autoren und Datum als PDF-Attribute
  pdfcreator={},  % ┐ PDF-Attribute säubern
  pdfproducer={}, % ┘
]{hyperref}
% erweiterte Bookmarks im PDF
\usepackage{bookmark}

% Trennung von Wörtern mit Strichen
\usepackage[shortcuts]{extdash}

\author{%
  Vincent Wirsdörfer\\%
  \href{mailto:vincent.wirsdoerfer@udo.edu}{authorA@udo.edu}%
  \and%
  Joris Daus\\%
  \href{mailto:joris.daus@udo.edu}{authorB@udo.edu}%
}
\publishers{TU Dortmund – Fakultät Physik}


%\begin{document}
\section{Auswertung}
\label{sec:Auswertung}

\subsection{Gedämpftes Schwingverhalten}

\noindent Zu Beginn wird das Schwingverhalten des RLC-Kreises untersucht. Dazu wird die Spannung am Kondensator gegen die Zeit aufgetragen. 
Wenn die einzelnen Spannungsspitzen nun mit einer Kosinusfunktion gefittet werden, ergibt sich folgende Darstellung.

\begin{figure}
    \includegraphics[width=\textwidth]{a.pdf}
    \caption{Spannung gegen Zeit im RCL-Kreis.}
\end{figure}

\noindent Die Parameter der fit-Funktion

\begin{equation}
    U(t)= A \text e ^{-2\pi\mu t} \cos{ \left(2 \pi \nu t + \eta \right)}
\end{equation}

\noindent werden per curve\_fit in Python berechnet.
So ergeben sich folgende Parameter:

\begin{align}
    & A     = \qty{4.33}{\volt}       \\
    &\mu    = \qty{926.19}{\per \second}   \\
    &\nu    = \qty{33684.92}{\per \second} \\
    &\eta   = \qty{-0.6}{}
\end{align}

\noindent Neben den berechneten Werten, sind folgende Werte vorgegeben:

\begin{align}
    &R_1 =  \qty{48.1 \pm 0.1}{\ohm}    \\
    &R_2 =  \qty{509.9 \pm 0.5}{\ohm}   \\
    &L  =   \qty{10.11 \pm 0.03}{\milli \henry}\\
    &C  =   \qty{5 \pm 0.02}{\nano \farad}
\end{align}

\noindent Nun wird mithilfe von Gleichung \eqref{eqn:mu} der effektive Widerstand $R_{\text{eff}}$ berechnet werden. Die Abklingdauer $T_{\text{ex}}$ wird 
durch Gleichung \eqref{eqn:abklingdauer} ermittelt. So ergeben sich folgende Werte:

\begin{align}
    &R_{\text{eff}} = \qty{118 \pm 7}{\ohm} \\
    &T_{\text{ex}} = \qty{172 \pm 10}{\micro \second}
\end{align}    

\noindent In die Schaltung ist der Widerstand $R_1$ eingebaut. Der berechnete Widerstand weißt zu dem eingebauten Widerstand eine Differenz 
von $\qty{70 \pm 7}{\ohm}$ auf. 
%Diskussion: R_eff Ist allerdings mehr als dopplet so groß wie R_1 was dadurch nicht erklärt werden kann.


\subsection{Aperiodischer Grenzfall}

Das aperiodische Schwingverhalten wird wie in der Durchführung beschrieben, vermessen. Der theoretische Wert für den Widerstand, bei dem der aperiodische
Grenzfall auftritt, wird über Gleichung \eqref{eqn:Grenzfall} mit $R_\text{theo} = \qty{4396 \pm 7}{\ohm}$ berechnet.
Gemessen wird der Wert $R_\text{exp} = \qty{3500 \pm 50}{\ohm}$.

\subsection{Resonanzfrequenz}

Gemessen wird die Spannung am Kondensator. Diese wird gegen die eingestellte Frequenz aufgetragen. Um die Spannung zu normieren, wird durch 
die Eingangsspannung geteilt. Die normierte Spannung kann nun gegen die Frequenz aufgetragen werden. 
Dabei wird die y-Achse logarithmiert. 

\begin{figure}[H]
    \includegraphics[width=\textwidth]{c_log.pdf}
    \caption{Normierte Spannung gegen Frequenz halblogarithmisch aufgetragen.}
    \label{fig:c_log}
\end{figure}

\noindent In Abbildung \ref{fig:c_log} ist zu sehen, dass die Daten einen quadratischen Verlauf bis zu einem Maximum haben und dann in einem 
quadratischen Verlauf wieder fallen. Dies spricht für eine Gaußverteilung.
Die genaue Kurve kann über ein Curve\_fit mit der Gaußschen Normalverteilung bestimmt werden. Wenn nun mithilfe der Kurve die 
Frequenz bestimmt, bei der die Spannung auf das $\frac{1}{\sqrt 2}$ -fache des Maximalwertes abfällt, berechnen sich die Frequenzen $w_+$ und 
$w_-$. Zur Veranschaulichung wird dies im Folgenden aufgetragen.

\begin{figure}[H]
    \includegraphics[width=\textwidth]{c_lin.pdf}
    \caption{Normierte Spannung gegen Frequenz mit Gaußglocke als Ausgleichskurve.}
\end{figure}

\noindent Anhand der Parameter für die Gaußfunktion lässt sich das Maximum der Funktion, also die Resonanzfrequenz, auf einen Wert von 
$\omega_\text{res} = \qty{32789.00 \pm 231.04}{\hertz}$ bestimmen.
Die Werte von $w_+$ und $w_-$, sowie die daraus berechnete Güte $q$ lassen sich über die eingezeichnete Schnittgerade und der Gleichung

\begin{equation*}
    q = \frac{\omega_0}{\omega_+ - \omega_-}
\end{equation*}

\noindent zu 

\begin{align}
    &w_- = \qty{27344.79}{\hertz} \\
    &w_+ = \qty{38233.21}{\hertz} \\
    &q_\text{exp}   = \qty{19.965 \pm 0.033}{}
\end{align}

\noindent bestimmen. \\
\noindent Diese Werte lassen sich auch theoretisch berechnen. Die Güte wird über Gleichung \eqref{eqn:güte} und die Frequenzen über

\begin{equation*}
    \omega_{+,-}= \pm \frac{R}{2L} + \sqrt{\frac{R^2}{4L^2} +\frac{1}{LC}}
\end{equation*}

berechnet. So ergeben sich die theoretischen Werte zu:

\begin{align}
    &w_- = \qty{223.3 \pm 0.5 e3}{\hertz} \\
    &w_+ = \qty{211.6 \pm 0.5 e3}{\hertz} \\
    &q_\text{theo} = \qty{18.7 \pm 1.1}{}
\end{align}


%\end{document}
