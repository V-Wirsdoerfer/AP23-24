\input{../../header.tex}

\begin{document}
\section{Diskussion}
\label{sec:Diskussion}

%Messmethode konstanter Probenstrom und B-Feld geändert: nur Aufladen bzw nur Entladung des Magneten, hätten eigentlich beides machen müssen
%Literaturwert Ladungsträgerdichte Kupfer: 8,4*10^28 https://13db.de/wissen/geschwindigkeit-von-elektronen/
\cite{leitfaehigkeiten}
%wir haben: konstanten B-Feld:  (1.06+/-0.18)e+28 1/m^3 | konstanten Probenstrom:  (1.29+/-0.22)e+29 1/m^3 
%Extrapolation der linearen Regression geht bei nicht auf 0 für 0T (plot 1-3)
%hat auch zur Folge: nur positive Hallspannungen, daher keine Rückschlüsse auf positive oder negative Ladungsträger bzw Lücken
%Literaturwert: Quelle Geschke physikalischen Praktikum(schon in lit.bib) Seite 292 µ für Kupfer ist 32cm^2/(Vs) umgerechnet 3,2 10^-3 m^2/(Vs) wir haben 3.1+-0,5m^2/(Vs)!!!!!!
\cite[292]{Physikalisches_Praktikum}
%sehr super sexy dass das das selbe ist njom njom
%Literaturwert Driftgeschwindigkeit in Kupfer 9,8e-5m/s https://13db.de/wissen/geschwindigkeit-von-elektronen/
\cite{leitfaehigkeiten}
%wir haben 4,8+-0,8e-5m/s
%Messfehler: extrem kleine Spannungen, weshalb sehr empfindlich. außerdem hat die angezeigte Spannung sich durch bloßes berühren geändert, auch kein stabiler Wert angezeigt
%probenhalterung wackelig und daher wsl nt ganz orthogonal auf B-Feld
%Magnetspulen sind heiß geworden und daher verluste entstanden -> Magnetfeld wurde schwächer (vor allem bei Messung mit konstanten B-Feld wichtig)



\end{document}
