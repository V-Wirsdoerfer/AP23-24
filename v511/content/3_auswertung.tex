%\input{../../header.tex}

%\begin{document}
\section{Auswertung}
\label{sec:Auswertung}

\noindent Zu Beginn der Auswertung werden die Daten gesichtet, um mithilfe einer linearen Regression 
die Ladungsträgerdichte $n$ für die jeweiligen Proben zu bestimmen. Die lineare Regression besitzt die Form $ax+b$, wobei $a$ 
die ermittelte Steigung ist. 
Zunächst wird die Messung bei gleichbleibendem Probenstrom ausgewertet. 
Dazu wird die gemessene Hall-Spannung gegen das anliegende B-Feld aufgetragen. 
Der physikalische Ausdruck für die Ladungsträgerdichte in Abhängigkeit von der Steigung lautet nach 
Gleichung \eqref{eqn:Spannung} 

\begin{equation*}
    n = - \frac{I}{a e_0 d}.
\end{equation*}

\noindent Die Messung der Kupferfolie ergibt durch eine Steigung \\ 
von $a_\text{Kupfer}=$\qty{1.129\pm 0.027e-5}{\volt\per\tesla} eine Ladungsträgerdichte von 

\begin{equation*}
    n_\text{Kupfer} = \qty{1.29\pm0.22e29}{\per\cubic\meter}
\end{equation*}

\begin{figure}[H]
    \centering
    \label{Kupfer}
    \includegraphics[height=7cm]{Kupfer.pdf}
    \caption{Hall-Spannung von Kupfer in Abhängigkeit des anliegenden Magnetfeldes bei 7A anliegendem Probenstrom.}
\end{figure}

\noindent Die Auswertung der Messwerte der Silberfolie ergeben mit einer Steigung \\
von $a_\text{Silber}=$\qty{1.763\pm 0.032e-5}{\volt\per\tesla} eine Ladungsträgerdichte von 

\begin{equation*}
    n_\text{Silber} = \qty{1.18\pm0.2e29}{\per\cubic\meter}.
\end{equation*}

\begin{figure}[H]
    \centering
    \label{Silber}
    \includegraphics[height=7cm]{Silber.pdf}
    \caption{Hall-Spannung von Silber in Abhängigkeit des anliegenden Magnetfeldes bei 10A anliegendem Probenstrom.}
\end{figure}

\noindent Die Ladungsträgerdichte der Zinkfolie ergeben sich aus der Steigung \\
von $a_\text{Zink}=$\qty{1.36\pm0.05e-5}{\volt\per\tesla} zu 

\begin{equation*}
    n_\text{Zink} = \qty{8.9\pm1.3e28}{\per\cubic\meter}.
\end{equation*}

\begin{figure}[H]
    \centering
    \label{Zink}
    \includegraphics[height=7cm]{Zink.pdf}
    \caption{Hall-Spannung von Zink in Abhängigkeit des anliegenden Magnetfeldes bei 7A anliegendem Probenstrom.}
\end{figure}

\noindent Die zweite Messmethode mit konstantem B-Feld wird nun ausgewertet. Dazu wird die gemessene Hall-Spannung 
gegen den anliegenden Probenstrom aufgetragen. Zur Vermeidung von systematischen Fehlern werden beide Richtungen des 
Probestroms vermessen.

\begin{figure}[H]
    \centering
    \includegraphics[height=7cm]{constB.pdf}
    \caption{Hall-Spannung von Kupfer in Abhängigkeit des Probenstroms bei \qty{576}{\milli\tesla}.}
\end{figure}

\noindent Die Ladungsträgerdichte lässt sich ähnlich wie vorher mithilfe von \eqref{eqn:Spannung} berechnen:

\begin{equation*}
    n = - \frac{B}{e_0 d a}.
\end{equation*}

\noindent Folgende Werte lassen sich so für die Steigung und die Ladungsträgerdichte ermitteln:
\begin{align*}
    a_\text{Silber2}=\qty{1.744 \pm 0.017}{\volt\per\ampere} && n_\text{Silber2} = \qty{6.8\pm 1.1e27}{\per\cubic\meter}.
\end{align*}

\noindent Alle weiteren Größen lassen sich mithilfe von Tabelle \ref{tab:Geometrie} und den Ladungsträgerdichten 
berechnen. Da für Zink nicht genügend Werte gemessen werden konnten, musste für die mittlere Flugzeit, die Beweglichkeit und 
die mittlere freie Weglänge auf den Literaturwert des spezifischen Widerstandes $\rho = \qty{5.916e-8}{\ohm \meter}$ 
\cite{spezifischer_Widerstand} zurückgegriffen werden.\\
\noindent Diese weiteren Größen werden nun in der folgenden Tabelle aufgeführt:

\hspace{-200pt}{
    \centering
    \begin{table}[H]
        \centering
    \caption{mikroskopische Leitfähigkeitsparameter.}
    \label{tab:Ergebnisse}
%    \sisetup{scientific-notation=true}
    \begin{tblr}{
        colspec = {S S 
        S[separate-uncertainty=true, table-format=1.2(1), table-format=1.2(2)] S[separate-uncertainty=true, table-format=1.2(1), table-format=1.2(2)] 
        S[separate-uncertainty=true, table-format=1.2(1), table-format=1.2(2)] S[separate-uncertainty=true, table-format=1.2(1), table-format=1.2(2)]},
    }
    \toprule
    \text{Größe}                             & \text{Gleichung}     & \text{Kupfer} & \text{Silber} & \text{Zink}    & \text{Tantal} \\
    \midrule
    \text{mittlere Flugzeit } $\overline{\tau}_{d} \mathbin{/}$ \qty{e-14}{\second}  & \eqref{eqn:ELS} 
                                                                    & 3.50\pm 0.60    & 3.10 \pm 0.50    & \num{1.35\pm 0.19} & /             \\
    \text{mitt. Driftgeschw. } $v_{d} \mathbin{/}$ \qty{e-5}{\meter\per\second}  & \eqref{eqn:velocity}\text{;} \eqref{eqn:drift} 
                                                                    & 4.80\pm 0.80    & 5.30\pm 0.90    & 7.00\pm 1.00         & /             \\
    \text{Beweglichkeit } $\mu \mathbin{/} $ \qty{e-3}{\square\meter\per\volt\per \second}     & \eqref{eqn:Beweglichkeit}                      
                                                                    & 3.10\pm 0.50    & 2.70\pm 0.50    & \num{1.20\pm 0.20}    & /             \\
    \text{Totalgeschw. } $|\overline{v}| \mathbin{/}$ \qty{e6}{\meter\per\second}              & \eqref{eqn:totv}                               
                                                                    & 1.81\pm 0.10   & 1.76\pm 0.10   & 1.60\pm 0.08    & /             \\
    \text{mittlere freie Weglänge } $\overline{l} \mathbin{/}$ \qty{e-8}{\meter}                       & \eqref{eqn:mitlaenge}                          
                                                                    & 6.30\pm 0.70    & 5.50\pm 0.60    & \num{2.16\pm 0.21}              & /             \\
    \bottomrule
    \end{tblr}
\end{table}
}

\noindent Zum Schluss werden die Ladungsträger $z$ pro Atom der jeweiligen Metalle berechnet, 
indem die Ladungsträgerdichte durch die Avogadro-Konstante $N=\qty{6.022e23}{\per \mol}$ 
geteilt wird. So ergeben sich folgende Werte:

\begin{align*}
    z_\text{Kupfer} &= \qty{2.10\pm0.40e5}{\per \mol} \\
    z_\text{Silber} &= \qty{1.96\pm0.33e5}{\per \mol} \\
    z_\text{Zink}   &= \qty{1.47\pm0.21e5}{\per \mol} \\
\end{align*}

%\end{document}