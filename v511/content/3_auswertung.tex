\documentclass[
  bibliography=totoc,     % Literatur im Inhaltsverzeichnis
  captions=tableheading,  % Tabellenüberschriften
  titlepage=firstiscover, % Titelseite ist Deckblatt
]{scrartcl}

% Paket float verbessern
\usepackage{scrhack}

% Warnung, falls nochmal kompiliert werden muss
\usepackage[aux]{rerunfilecheck}

% unverzichtbare Mathe-Befehle
\usepackage{amsmath}
% viele Mathe-Symbole
\usepackage{amssymb}
% Erweiterungen für amsmath
\usepackage{mathtools}

% Fonteinstellungen
\usepackage{fontspec}
% Latin Modern Fonts werden automatisch geladen
% Alternativ zum Beispiel:
%\setromanfont{Libertinus Serif}
%\setsansfont{Libertinus Sans}
%\setmonofont{Libertinus Mono}

% Wenn man andere Schriftarten gesetzt hat,
% sollte man das Seiten-Layout neu berechnen lassen
\recalctypearea{}

% deutsche Spracheinstellungen
\usepackage[ngerman]{babel}


\usepackage[
  math-style=ISO,    % ┐
  bold-style=ISO,    % │
  sans-style=italic, % │ ISO-Standard folgen
  nabla=upright,     % │
  partial=upright,   % │
  mathrm=sym,        % ┘
  warnings-off={           % ┐
    mathtools-colon,       % │ unnötige Warnungen ausschalten
    mathtools-overbracket, % │
  },                       % ┘
]{unicode-math}

% traditionelle Fonts für Mathematik
\setmathfont{Latin Modern Math}
% Alternativ zum Beispiel:
%\setmathfont{Libertinus Math}

\setmathfont{XITS Math}[range={scr, bfscr}]
\setmathfont{XITS Math}[range={cal, bfcal}, StylisticSet=1]

% Zahlen und Einheiten
\usepackage[
  locale=DE,                   % deutsche Einstellungen
  separate-uncertainty=true,   % immer Unsicherheit mit \pm
  per-mode=symbol-or-fraction, % / in inline math, fraction in display math
]{siunitx}

% chemische Formeln
\usepackage[
  version=4,
  math-greek=default, % ┐ mit unicode-math zusammenarbeiten
  text-greek=default, % ┘
]{mhchem}

% richtige Anführungszeichen
\usepackage[autostyle]{csquotes}

% schöne Brüche im Text
\usepackage{xfrac}

% Standardplatzierung für Floats einstellen
\usepackage{float}
\floatplacement{figure}{htbp}
\floatplacement{table}{htbp}

% Floats innerhalb einer Section halten
\usepackage[
  section, % Floats innerhalb der Section halten
  below,   % unterhalb der Section aber auf der selben Seite ist ok
]{placeins}

% Seite drehen für breite Tabellen: landscape Umgebung
\usepackage{pdflscape}

% Captions schöner machen.
\usepackage[
  labelfont=bf,        % Tabelle x: Abbildung y: ist jetzt fett
  font=small,          % Schrift etwas kleiner als Dokument
  width=0.9\textwidth, % maximale Breite einer Caption schmaler
]{caption}
% subfigure, subtable, subref
\usepackage{subcaption}

% Grafiken können eingebunden werden
\usepackage{graphicx}

% schöne Tabellen
\usepackage{tabularray}
\UseTblrLibrary{booktabs, siunitx}

% Verbesserungen am Schriftbild
\usepackage{microtype}

% Literaturverzeichnis
\usepackage[
  backend=biber,
]{biblatex}
% Quellendatenbank
\addbibresource{lit.bib}
\addbibresource{programme.bib}

% Hyperlinks im Dokument
\usepackage[
  german,
  unicode,        % Unicode in PDF-Attributen erlauben
  pdfusetitle,    % Titel, Autoren und Datum als PDF-Attribute
  pdfcreator={},  % ┐ PDF-Attribute säubern
  pdfproducer={}, % ┘
]{hyperref}
% erweiterte Bookmarks im PDF
\usepackage{bookmark}

% Trennung von Wörtern mit Strichen
\usepackage[shortcuts]{extdash}

\author{%
  Vincent Wirsdörfer\\%
  \href{mailto:vincent.wirsdoerfer@udo.edu}{authorA@udo.edu}%
  \and%
  Joris Daus\\%
  \href{mailto:joris.daus@udo.edu}{authorB@udo.edu}%
}
\publishers{TU Dortmund – Fakultät Physik}


\begin{document}
\section{Auswertung}
\label{sec:Auswertung}

\noindent Zu Beginn der Auswertung werden die Daten gesichtet, um mithilfe einer linearen Regression 
die Ladungsträgerdichte $n$ für die jeweiligen Proben zu bestimmen. Die lineare Regression besitzt die Form $ax+b$, wobei $a$ 
die ermittelte Steigung ist. 
Zunächst wird die Messung bei gleichbleibendem Probenstrom ausgewertet. 
Dazu wird die gemessene Hall-Spannung gegen das anliegende B-Feld aufgetragen. 
Der physikalische Ausdruck für die Ladungsträgerdichte in Abhängigkeit von der Steigung lautet nach 
Gleichung \eqref{eqn:Spannung} 

\begin{equation*}
    n = - \frac{I}{a e_0 d}.
\end{equation*}

\noindent Die Messung der Kupferfolie ergibt durch eine Steigung von $a_\text{Kupfer}=$\qty{1.129\pm 0.027e-5}{\volt\per\tesla} eine 
Ladungsträgerdichte von 

\begin{equation*}
    n_\text{Kupfer} = \qty{1.29\pm0.22e29}{\per\cubic\meter}
\end{equation*}

\begin{figure}[H]
    \centering
    \label{Kupfer}
    \includegraphics[height=7cm]{../build/Kupfer.pdf}
    \caption{Hall-Spannung von Kupfer in Abhängigkeit des anliegenden Magnetfeldes bei 7A anliegendem Probenstrom.}
\end{figure}

\noindent Die Auswertung der Messwerte der Silberfolie ergeben mit einer Steigung von $a_\text{Silber}=$\qty{1.763\pm 0.032e-5}{\volt\per\tesla} eine 
Ladungsträgerdichte von 

\begin{equation*}
    n_\text{Silber} = \qty{1.18\pm0.2e29}{\per\cubic\meter}.
\end{equation*}

\begin{figure}[H]
    \centering
    \label{Silber}
    \includegraphics[height=7cm]{../build/Silber.pdf}
    \caption{Hall-Spannung von Silber in Abhängigkeit des anliegenden Magnetfeldes bei 10A anliegendem Probenstrom.}
\end{figure}

\noindent Die Ladungsträgerdichte der Zinkfolie ergeben sich aus der Steigung von $a_\text{Zink}=$\qty{1.36\pm0.05e-5}{\volt\per\tesla}
zu 

\begin{equation*}
    n_\text{Zink} = \qty{8.9\pm1.3e28}{\per\cubic\meter}.
\end{equation*}

\begin{figure}[H]
    \centering
    \label{Zink}
    \includegraphics[height=7cm]{../build/Zink.pdf}
    \caption{Hall-Spannung von Zink in Abhängigkeit des anliegenden Magnetfeldes bei 7A anliegendem Probenstrom.}
\end{figure}

\noindent Die zweite Messmethode mit konstantem B-Feld wird nun ausgewertet. Dazu wird die gemessene Hall-Spannung 
gegen den anliegenden Probenstrom aufgetragen. Zur Vermeidung von systematischen Fehlern werden beide Richtungen des 
Probestroms vermessen.

\begin{figure}[H]
    \centering
    \includegraphics[height=7cm]{../build/constB.pdf}
    \caption{Hall-Spannung von Kupfer in Abhängigkeit des Probenstroms bei \qty{576}{\milli\tesla}.}
\end{figure}

\noindent Die Ladungsträgerdichte lässt sich ähnlich wie vorher mithilfe von \eqref{eqn:Spannung} berechnen:

\begin{equation*}
    n = - \frac{B}{e_0 d a}.
\end{equation*}

\noindent Folgende Werte lassen sich so für die Steigung und die Ladungsträgerdichte ermitteln:
\begin{align*}
    a_\text{Kupfer}=\qty{1.744 \pm 0.017}{\volt\per\ampere} && n_\text{Kupfer2} = \qty{1.06\pm0.18e28}{\per\cubic\meter}.
\end{align*}

\noindent Alle weiteren Größen lassen sich mithilfe von \autoref{tab:Geometrie} und den Ladungsträgerdichten 
berechnen.\\
\noindent Diese weiteren Größen werden nun in der folgenden Tabelle aufgeführt:

\hspace{-200pt}{
\centering
\begin{table}[H]
    \centering
    \label{tab:Ergebnisse}
%    \sisetup{scientific-notation=true}
    \begin{tblr}{
        colspec = {S S 
        S[separate-uncertainty=true, table-format=1.2(1), table-format=1.2(2)] S[separate-uncertainty=true, table-format=1.2(1), table-format=1.2(2)] 
        S[separate-uncertainty=true, table-format=1.2(1), table-format=1.2(2)] S[separate-uncertainty=true, table-format=1.2(1), table-format=1.2(2)]},
    }
    \toprule
    \text{Größe}                             & \text{Gleichung}     & \text{Kupfer} & \text{Silber} & \text{Zink}    & \text{Tantal} \\
    \midrule
    \text{mittlere Flugzeit } $\overline{\tau}_{d} \mathbin{/}$ \qty{e-14}{\second}  & \eqref{eqn:ELS} 
                                                                    & 3.5\pm 0.6    & 3.1 \pm 0.5    & /             & /             \\
    \text{mitt. Driftgeschw. } $v_{d} \mathbin{/}$ \qty{e-5}{\meter\per\second}  & \eqref{eqn:velocity}\text{;} \eqref{eqn:drift} 
                                                                    & 4.8\pm 0.8    & 5.3\pm 0.9    & 7\pm 1         & /             \\
    \text{Beweglichkeit } $\mu \mathbin{/} $ \qty{e-3}{\square\meter\per\volt\per \second}     & \eqref{eqn:Beweglichkeit}                      
                                                                    & 3.1\pm 0.5    & 2.7\pm 0.5    & /              & /             \\
    \text{Totalgeschw. } $|\overline{v}| \mathbin{/}$ \qty{e6}{\meter\per\second}              & \eqref{eqn:totv}                               
                                                                    & 1.81\pm 0.1   & 1.76\pm 0.1   & 1.6\pm 0.08    & /             \\
    \text{mittlere freie Weglänge } $\overline{l} \mathbin{/}$ \qty{e-8}{\meter}                       & \eqref{eqn:mitlaenge}                          
                                                                    & 6.3\pm 0.7    & 5.5\pm 0.6    & /              & /             \\
    \bottomrule
    \end{tblr}
    \caption{mikroskopische Leitfähigkeitsparameter.}
\end{table}
}
\end{document}
.