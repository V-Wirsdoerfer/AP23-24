\input{../../header.tex}

\begin{document}


\section{Versuchsdurchführung}
\label{sec:Versuchsdurchfuehrung}

Zu Beginn des Experiments werden die geometrischen Maße der vorliegenden Versuchsmaterialien gesammelt. Hierzu 
werden jeweils die Dicken der Drähte und der Folien aller metallischer Proben gemessen. Die Längen der Spulendrähte
von Kupfer, Silber und Tantal können abgelesen werden.\\

\noindent Im Folgenden werden die elektrischen Widerstände der Metalle aufgenommen. Dazu wird die untenstehende Abbildung 
nachgebaut.

\begin{figure}
    \centering
    \includegraphics[height=6cm]{Widerstand.png}
    \caption{Messung der Drahtwiderstände \cite{Versuchsanleitung_v511}.}
    \label{fig:Widerstand}
\end{figure}

\noindent Wie in der Abbildung \ref{fig:Widerstand} gezeigt, wird zunächst ein konstanter Strom durch die Probe geschickt
und darufhin der Spannungsabfall an einem digitalen Voltmeter abgelesen. Diese Prozedur wird für alle weiteren Proben und 
inverse Stromrichtungen wiederholt, um systematische Fehler zu verringern.\\

\noindent Mit Blick auf Gleichung \eqref{eqn:Spannung} ist es nun von Interesse die durch den Hall-Effekt entstehende 
Hall-Spannung zu messen, um Rückschlüsse auf die Ladungsträgerdichte zu ziehen. Hierfür wird nun eine Versuchsapparatur 
nach Abbildung \ref{fig:Hall-Effekt} benötigt. Zuerst wird die elektrische Leiterplatte von Silber an der Apparatur 
angebracht, sodass die Silberfolie senkrecht zum annähernd homogenen Magnetfeld steht. Um das B-Feld einzuschalten wird 
der Elektromagnet mit einem Konstantstromgerät verkabelt. Auch die Probenplatte wird mit einem separaten Stromgerät versorgt.
Im Folgenden wird der Querstrom auf ein Maximum von $10\,\unit{\ampere}$ hochgeregelt. Ziel ist nun für 10 verschiedene 
Magnetfeldstärken die zugehörige Hall-Spannung zu messen. Dementsprechend wird die Spulenstrom in kleinen Intervallschritten 
erhöht und die Spannung digital abgelesen. Zwischen den Messungen wird mittles einer Hall-Sonde das B-Feld gemessen. Tabellarisch 
werden die Werte in das Experimentierheft überführt. Diese Messungen werden für die Leiterplatten aus Kupfer und Zink analog 
durchgeführt. Aufgrund einer Erhitzung des Konstantstromgerätes bei Silber wird der Querstrom jedoch auf nur $7\,\unit{\ampere}$
anstatt von anfänglichen $10\,\unit{\ampere}$ eingestellt. Ferner wird der Spulenstrom nach jeder letzten Messung der jeweiligen 
Probe langsam heruntergereglt, um Transistoren und das Konstantstromgerät nicht zu beschädigen.\\

\noindent Zuletzt wird erneut die Silberprobe angebracht und an ein Konstantstromgerät angeschlossen. Im Vergleich zum vorherigen 
Versuchsteil bleibt nun jedoch der Spulenstrom und dementsprechend auch die Magnetfelstärke während der Versuchsreihe unverändert.
Für unterschiedliche Hall-Spannungen sorgt nun die Variation des Querstroms, welcher durch die Probe fließt. Es wird also für 10 
verschiedene Probenströme die Hall-Spannung gemessen und notiert. Im Anschluss wird der Spulenstrom umgepolt und der Versuch wiederholt.\\

\noindent 

\section{Messwerte}
\label{sec:Messwerte}


\end{document}

