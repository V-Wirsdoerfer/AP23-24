\input{../../header.tex}

\begin{document}
\section{Auswertung}
\label{sec:Auswertung}

%
%Mit Theorie vielleicht weg machen, wenn es in der Theorie schon vorgekommen ist.
%
%
Die Zentrale Gleichung lautet 
\begin{equation}
    \theta_\text{frei} = \frac{e_0^3}{8 \pi ^2 \varepsilon_0c^3} \frac{NB}{n} \lambda ^2
    \label{eqn:Winkel_frei}
\end{equation}
%
%
%
%

\subsection{Magnetfeld an der Probe}
\noindent Für \autoref{eqn:Winkel_frei} wird das Magnetfeld an der Probe benötigt. Um dies zu bestimmen, 
werden die Daten aus \autoref{tab:magnetfeld} aufgetragen.

\begin{figure}[H]
    \centering
    \includegraphics[width = 0.9\textwidth]{../Magnetfeld.pdf}
    \caption{Magnetfeld im Inneren des Magneten in Abhängigkeit vom Abstand zur Mitte der Probe.}
    \label{fig:magnetfeld}
\end{figure}

\noindent Wie zu sehen ist, ist das Maximum des Magnetfelds genau am Ort der Probe. Das Magnetfeld 
direkt an der Probe beträgt somit 
%
%
% gleichen Wert wie in Rechnung nehmen
$B=\qty{429}{\milli \tesla}$.
%
%
%

\subsection{Winkelmessungen}
\noindent Im Folgenden werden nun die rohen Messdaten der Winkel gegen $\lambda^2$ aufgetragen, um diese 
zu sichten.

\begin{figure}[H]
    \centering
    \includegraphics[width=0.9\textwidth]{../Kristallwinkel.pdf}
    \caption{Messwerte aller Proben.}
    \label{fig:roh}
\end{figure}

\noindent Es ist zu sehen, dass es schon einige Ausreißer gibt. Dementsprechend wird später überprüft, welche 
Werte in die Rechnung genommen werden.\\
\noindent Da die absoluten Winkel des Goniometers nicht von Interesse sind, sondern nur der relative 
Winkel, um den die Polarisation gedreht wird, müssen diese noch bestimmt werden. Dieser lässt sich 
bestimmen, indem die Differenz der Winkel gebildet und halbiert wird.

\begin{equation}
    \centering
    \theta_\text{frei}=\frac{1}{2} \left(\theta_2 - \theta 1 \right)
    \label{eqn:theta_frei}
\end{equation}

\noindent Dies wird für alle drei Proben gebildet, da nur die relative Drehung wichtig ist. 
Die Drehung muss außerdem auf eine Drehung pro Strecke umgerechnet werden. Dies geschieht, indem der Drehwinkel 
durch die Dicke der Probe geteilt wird.
Um allen Einfluss, welcher nicht auf die Ladungsträgerdichte zurückzuführen ist, herauszurechnen, wird 
$\theta_\text{KR}$ des undotierten GaAs von $\theta_\text{KR}$ der dotierten GaAs Proben abgezogen. 
Dies geschieht für beide Proben und wird nun aufgetragen, um ausreißende Datenpunkte zu finden.

\begin{figure}[H]
    \centering
    \includegraphics[width=0.9\textwidth]{../Rohplot.pdf}
    \caption{Differenz der Winkel vor Ausfilterung.}
    \label{fig:pre}
\end{figure}

\noindent Hier ist zu sehen, dass der zweite und der achte Wert starke Abweichungen von den anderen Werten 
aufzeigen. Daher werden diese in den folgenden Berechnungen nicht mit gewertet. Es kann nun eine Ausgleichsrechnung 
durchgeführt werden, in der \autoref{eqn:Winkel_frei} gefittet wird. 

\begin{figure}[H]
    \centering
    \includegraphics[width=0.9\textwidth]{../LinRegress.pdf}
    \caption{Lineare Ausgleichsrechnung der freien Winkel.}
    \label{fig:LinRegress}
\end{figure}

\noindent Die Steigung der stark dotierten Probe beträgt 

\begin{equation}
    m_\text{stark}=\qty{9.6\pm0.9e12}{\radiant \per \squared \meter}
\end{equation}


\end{document}
