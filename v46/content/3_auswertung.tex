\input{../../header.tex}

\begin{document}
\section{Auswertung}
\label{sec:Auswertung}

%
%Mit Theorie vielleicht weg machen, wenn es in der Theorie schon vorgekommen ist.
%
%
Die Zentrale Gleichung lautet 
\begin{equation}
    \theta_\text{frei} = \frac{e_0^3}{8 \pi ^2 \varepsilon_0c^3} \frac{NB}{n} \lambda ^2
    \label{eqn:Winkel_frei}
\end{equation}
%
%
%
%

\subsection{Magnetfeld an der Probe}
\noindent Für \autoref{eqn:Winkel_frei} wird das Magnetfeld an der Probe benötigt. Um dies zu bestimmen, 
werden die Daten aus \autoref{tab:magnetfeld} aufgetragen.

\begin{figure}
    \centering
    \includegraphics[width = 0.9\textwidth]{../Magnetfeld.pdf}
    \caption{Magnetfeld im Inneren des Magneten in Abhängigkeit vom Abstand zur Mitte der Probe.}
    \label{fig:magnetfeld}
\end{figure}

\noindent Wie zu sehen ist, ist das Maximum des Magnetfelds genau am Ort der Probe. Das Magnetfeld 
direkt an der Probe beträgt somit 
%
%
% gleichen Wert wie in Rechnung nehmen
$B=\qty{429}{\milli \tesla}$.
%
%
%
\subsection{}


\end{document}
