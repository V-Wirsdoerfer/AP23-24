%\input{../../header.tex}

%\begin{document}
\section{Zielsetzung}
\label{sec:Theorie}

\section{Theorie}

\section{Vorbereitung}

\section{Fehlerrechnung}
\label{sec:Fehlerrechnung}

Alle im Protokoll vermerkten Mittelwerte lassen sich über die folgende Formel berechnen:

\begin{equation}
\label{eqn:Mittelwert}
    \bar{x} = \frac{1}{N}\sum_{i=1}^N x_i
\end{equation}

Zudem lässt sich der dazugehörige Fehler des Mittelwerts wie folgt berechnen:

\begin{equation}
\label{eqn:Mittelwertfehler}
    \increment \bar{x} = \sqrt{\frac{1}{N\left(N-1\right)}\sum_{i=1}^N \left(x_i - \bar{x}\right)²}
\end{equation}

Entsteht ein neuer Fehler durch bereits fehlerbehaftete Größen, so wird die Gauß'sche Fehlerfortpflanzung angewendet:

\begin{equation}
\label{eqn:Fehlerfortpflanzung}
    \increment f = \sqrt{\sum_{i=1}^N \left(\frac{\partial f}{\partial x_i}\right)²\cdot\left(\increment x_i\right)²}
\end{equation}

%\end{document}