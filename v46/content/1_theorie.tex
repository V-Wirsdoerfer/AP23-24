\input{../../header.tex}

\begin{document}
\section{Zielsetzung}
\label{sec:Theorie}

\section{Theorie}

Das Ziel des im folgend protokollierten Versuchs besteht in der Bestimmung der effektiven Masse der Leitungselektronen in 
n-dotiertem Galliumarsenid mittels des Faraday-Effekts. 

\subsection{Bandstruktur von Festkörpern}

\noindent Eine Möglichkeit um die Struktur von Festkörpern genauer zu beschreiben ist das sogenannte \textit{Bandstrukturmodell}. Betrachtet man das 
freie Elektronengas, so ergibt sich für die Elektronen im Festkörper eine quadratische Dispersionsrelation:

\begin{equation}
    E(k) = \frac{\hbar²{}k²}{2m}
\end{equation}

\noindent Hierbei steht $\hbar$ für das reduzierte Planck'sche Wirkungsquantum, $k$ für die Wellenzahl und $m$ für die Masse des Elektrons.
Wird nun zusätzlich die Anwesenheit eines perioden Gitterpotentials, ausgelöst durch positive Ionenrümpfe, betrachtet, so ergeben sich verbotene
elektronische Zustände. Jene Energieintervalle, dessen Besetzung erlaubt sind werden als \textit{Bänder} bezeichnet. Andere Zustände hingegen, welche
energetisch verboten sind, werden als \textit{Bandlücke} beschrieben. Um einen Eindruck zu gewinnen, wie sich das Bändermodell für spezifische
Elektronenkonfigurationen konkretisiert, ist im Folgenden die Bandstrauktur von Magnesium abgebildet. 

\begin{figure}[H]
    \centering
    \includegraphics[width=0.7\textwidth]{BandstrukturMagn.png}
    \caption{Bandstruktur von Magnesium.}
    \label{fig:Magnesium}
\end{figure}

\noindent Besonders entscheidend um verschiedene Materialen bzw. Atome zu klassifizieren ist das \textit{Valenz-} und \textit{Leitungsband}. Die Bestzung 
oder Nicht-Besetzung dieser Bänder gibt Aufschluss über die elektrische Leitfähigekit der betrachteten Materialen und wird maßgeblich von der 
\textit{Fermi-Energie} sowie der Größe der \text{Bandlücke} zwischen ihnen beeinflusst. \\

\noindent Bei \textit{Metallen} liegt die Fermi-Energie im Leitungsband und aufgrund der elektronischen Struktur metallischer Stoffe ist der Übergang beider 
Bänder fließend. Die damit verbundene makroskopische Besetzung des Leitungsbandes führt zu einer hohern elektrischen Leitfähigkeit von Metallen.
Im Gegensatz dazu liegt bei \textit{Isolatoren} eine besonders große Bandlücke vor, was die Bestzung des Leitungsbandes ohne starke äußere Anregung verbietet.
Dies erklärt die geringe elektrische Leitfähigekit von Isolatoren. Die Bandstruktur von \textit{Halbleitern} ist grundsätzlich nicht einfach von jener der 
Isolatoren zu unterscheiden. Auch hier liegt die Fermi-Energie unterhalb des Leitungsbandes, jedoch ist die Bandlücke in der Regel geringer, was neue 
Möglichkeiten zur Veränderung elektronischer Eigenschaften bietet. Eine Option ist die sogenannte \textit{Dotierung}, welche jedoch im Späteren 
näher erklärt wird. Die folgende Abbildung zeigt, wie sich die verschiedenen Material-Spezifikationen in der Bandstruktur unterscheiden. \\

\begin{figure}[H]
    \centering
    \includegraphics[width=0.7\textwidth]{MetIsoHalb.png}
    \caption{Bandstruktur von Isolatoren, Halbleitern und Metallen.}
    \label{fig:MetIsoHalb}
\end{figure}

\subsection{Dotierung}

\noindent Unter Dotierung versteht man in der Halbleitertechnik die bewusste Verunreinigung von Materialien zur Veränderung
der elektronischen Eigenschaften des Ausgangsmaterials. Die in diesem Experiment verwendete Methode der \textit{n-Dotierung}
bezeichnet das Einfügen von Fremdatomen in das Ausgangsmaterial, welche jeweils über ein zusätzliches Valenzelektron verfügen. 
Die dadurch vorhandenen Elektronen sind nicht statisch an das periodische Gitterpotential gebunden, sondern können sich \enquote{frei} 
durch das Material bewegen. Energetisch liegen die freien Elektronen leicht unter dem Leitungsband.

\begin{figure}[H]
    \centering
    \includegraphics[width=0.4\textwidth]{nDot.png}
    \caption{Energetisches Niveau von Donatoren.}
    \label{fig:nDot}
\end{figure}

\noindent Die effektive Bandlücke zwischen den freien Elektronen und dem Leitungsband ist wesentlich kleiner als die Bandlücke 
zwischen Valenz- und Leitungsband, welche bei üblichen Halbleitermaterialien im Bereich von \qty{1.4}{\electronvolt} liegt. Temperaturen 
in der Nähe von \qty{300}{\kelvin} (Raumtemperatur) können ausreichen, um diese Elektronen in das Leitungsband anzuregen. 
Auf diese Art und Weise kann die elektrische Leitfähigekit von Halbleitermaterialien signifikant gesteigert werden. 

\subsection{Effektive Masse und Faraday-Effekt}

\noindent Die effektive Masse beschreibt die Wirkung des periodischen Gitterpotentials auf die Masse der Elektronen im Material.
In anisotropen optischen Materialien lässt sich der Tensor der effektiven Masse über die folgende Formel berechnen:

\begin{equation}
    (m^{\ast})_{\text{ij}} = \hbar²(\frac{\partial²E(\vec{k})}{\partial{}k_{\text{i}}\partial{}k_{\text{j}}})^{-1}
    \label{eqn:meff}
\end{equation}

\section{Fehlerrechnung}
\label{sec:Fehlerrechnung}

Alle im Protokoll vermerkten Mittelwerte lassen sich über die folgende Formel berechnen:

\begin{equation}
\label{eqn:Mittelwert}
    \bar{x} = \frac{1}{N}\sum_{i=1}^N x_i
\end{equation}

\noindent Zudem lässt sich der dazugehörige Fehler des Mittelwerts wie folgt berechnen:

\begin{equation}
\label{eqn:Mittelwertfehler}
    \increment \bar{x} = \sqrt{\frac{1}{N\left(N-1\right)}\sum_{i=1}^N \left(x_i - \bar{x}\right)²}
\end{equation}

\noindent Entsteht ein neuer Fehler durch bereits fehlerbehaftete Größen, so wird die Gauß'sche Fehlerfortpflanzung angewendet:

\begin{equation}
\label{eqn:Fehlerfortpflanzung}
    \increment f = \sqrt{\sum_{i=1}^N \left(\frac{\partial f}{\partial x_i}\right)²\cdot\left(\increment x_i\right)²}
\end{equation}

\end{document}