\input{../../header.tex}

\begin{document}
\section{Versuchsaufbau}
Im Folgenden wird der Versuchsaufbau erklärt, indem zunächst der Strahlengang erläutert wird.\\
Die Lichtquelle stellt eine Halogenlampe dar, welche mit \qty{7}{\volt} und \qty{3.2}{\ampere} betrieben wird.
Das Licht der Lampe wird durch eine Linse parallelisiert, sodass die Intensität über die Länge des Versuchsaufbaus erhalten bleibt. 
Hinter der Linse steht ein Chopper, welcher das Lichtsignal zerhackt. Im Anschluss steht das erste Glan-Thompson-Prisma, welches mit 
einem Goniometer verbunden ist und das durchgehende Licht linear polarisiert. Die Drehachse liegt entlang der Ausbreitungsrichtung. 
Der linear polarisierte Lichtstrahl geht längs des Magnetes durch eine Bohrung. Die Magnetfeldlinien liegen somit parallel zum 
Lichtstrahl. Wie in der \autoref{fig:Schlitz} zu sehen ist, ist in der Mitte des Magnets ein fig:Schlitz, in welchem die GaAs Probe eingelegt 
werden kann.

\begin{figure}
    \includegraphics[width=0.5\textwidth]{Schlitz.jpg}
    \caption{Schlitz für die Probe im Magneten.}
    \label{fig:Schlitz}
\end{figure}


\section{Versuchsdurchführung}

\section{Messwerte}

\end{document}

