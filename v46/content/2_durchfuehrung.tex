\input{../../header.tex}

\begin{document}
\section{Versuchsaufbau}
Im Folgenden wird der Versuchsaufbau aus \autoref{fig:Aufbau} erklärt, indem zunächst der Strahlengang erläutert wird.\\

\begin{figure}[H]
    \includegraphics[width=0.9\textwidth]{Aufbau.png}
    \caption{Schematischer Aufbau des Versuches \cite{Versuchsanleitung_v46}.}
    \label{fig:Aufbau}
\end{figure}

\noindent Die Lichtquelle stellt eine Halogenlampe dar, welche mit \qty{7}{\volt} und \qty{3.2}{\ampere} betrieben wird.
Das Licht der Lampe wird durch eine Linse parallelisiert, sodass die Intensität über die Länge des Versuchsaufbaus erhalten bleibt. 
Hinter der Linse steht ein Chopper, welcher das Lichtsignal zerhackt. Im Anschluss steht das erste Glan-Thompson-Prisma, welches mit 
einem Goniometer verbunden ist und das durchgehende Licht linear polarisiert. Die Drehachse liegt entlang der Ausbreitungsrichtung. 
Der linear polarisierte Lichtstrahl geht längs des Magnetes durch eine Bohrung. Die Magnetfeldlinien liegen somit parallel zum 
Lichtstrahl. Wie in der \autoref{fig:Schlitz} zu sehen ist, ist in der Mitte des Magnets ein Schlitz, in welchem die GaAs Probe eingelegt 
werden kann.

\begin{figure}[H]
    \includegraphics[width=0.5\textwidth]{Schlitz.jpg}
    \caption{Schlitz für die Probe im Magneten.}
    \label{fig:Schlitz}
\end{figure}

\noindent Hinter dem Magneten steht nun ein Interferenzfilter, welcher alle Wellenlängen bis auf eine herausfiltert. So kann die 
Wellenlängen eingestellt werden. Das Licht trifft nun auf das zweite Glan-Thompson-Prisma und wird dort in einen ordentlichen und einen 
außerordentlichen Strahl aufgeteilt. Die Strahlen sind jeweils orthogonal zueinander polarisiert und sind nun räumlich getrennt. Sie 
treffen auf den Photowiderstand und werden dort detektiert. Die Detektoren geben eine Wechselspannung aus, welche in einen Differenzverstärker 
eingespeist wird. Das Signal des Differenzverstärkers wird über einen Selektivverstärker in ein Oszilloskop eingespeist. Der Selektivverstärker 
filtert alle Frequenzen heraus, die nicht im Bereich von der Frequenz des Choppers liegen. So wird Rauschen unterdrückt. 


\section{Versuchsdurchführung}
\noindent Im Folgenden wird zunächst auf die Justage eingegangen und im Anschluss auf das eigentliche Erheben der Messwerte.

\subsection{Justage}
\noindent Der Strahlengang muss justiert werden. Dies geschieht, indem die Sammellinsen, das Goniometer die Glan-Thompson-Prisma und 
die Detektoren richtig ausgerichtet werden. \\
Die Sammellinse wird so verschoben, dass das Licht auf die Bohrung des Magnets trifft und so fokussiert, dass die Lichtkanten scharf sind. 



\section{Messwerte}

\end{document}

