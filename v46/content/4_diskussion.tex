%\documentclass[
  bibliography=totoc,     % Literatur im Inhaltsverzeichnis
  captions=tableheading,  % Tabellenüberschriften
  titlepage=firstiscover, % Titelseite ist Deckblatt
]{scrartcl}

% Paket float verbessern
\usepackage{scrhack}

% Warnung, falls nochmal kompiliert werden muss
\usepackage[aux]{rerunfilecheck}

% unverzichtbare Mathe-Befehle
\usepackage{amsmath}
% viele Mathe-Symbole
\usepackage{amssymb}
% Erweiterungen für amsmath
\usepackage{mathtools}

% Fonteinstellungen
\usepackage{fontspec}
% Latin Modern Fonts werden automatisch geladen
% Alternativ zum Beispiel:
%\setromanfont{Libertinus Serif}
%\setsansfont{Libertinus Sans}
%\setmonofont{Libertinus Mono}

% Wenn man andere Schriftarten gesetzt hat,
% sollte man das Seiten-Layout neu berechnen lassen
\recalctypearea{}

% deutsche Spracheinstellungen
\usepackage[ngerman]{babel}


\usepackage[
  math-style=ISO,    % ┐
  bold-style=ISO,    % │
  sans-style=italic, % │ ISO-Standard folgen
  nabla=upright,     % │
  partial=upright,   % │
  mathrm=sym,        % ┘
  warnings-off={           % ┐
    mathtools-colon,       % │ unnötige Warnungen ausschalten
    mathtools-overbracket, % │
  },                       % ┘
]{unicode-math}

% traditionelle Fonts für Mathematik
\setmathfont{Latin Modern Math}
% Alternativ zum Beispiel:
%\setmathfont{Libertinus Math}

\setmathfont{XITS Math}[range={scr, bfscr}]
\setmathfont{XITS Math}[range={cal, bfcal}, StylisticSet=1]

% Zahlen und Einheiten
\usepackage[
  locale=DE,                   % deutsche Einstellungen
  separate-uncertainty=true,   % immer Unsicherheit mit \pm
  per-mode=symbol-or-fraction, % / in inline math, fraction in display math
]{siunitx}

% chemische Formeln
\usepackage[
  version=4,
  math-greek=default, % ┐ mit unicode-math zusammenarbeiten
  text-greek=default, % ┘
]{mhchem}

% richtige Anführungszeichen
\usepackage[autostyle]{csquotes}

% schöne Brüche im Text
\usepackage{xfrac}

% Standardplatzierung für Floats einstellen
\usepackage{float}
\floatplacement{figure}{htbp}
\floatplacement{table}{htbp}

% Floats innerhalb einer Section halten
\usepackage[
  section, % Floats innerhalb der Section halten
  below,   % unterhalb der Section aber auf der selben Seite ist ok
]{placeins}

% Seite drehen für breite Tabellen: landscape Umgebung
\usepackage{pdflscape}

% Captions schöner machen.
\usepackage[
  labelfont=bf,        % Tabelle x: Abbildung y: ist jetzt fett
  font=small,          % Schrift etwas kleiner als Dokument
  width=0.9\textwidth, % maximale Breite einer Caption schmaler
]{caption}
% subfigure, subtable, subref
\usepackage{subcaption}

% Grafiken können eingebunden werden
\usepackage{graphicx}

% schöne Tabellen
\usepackage{tabularray}
\UseTblrLibrary{booktabs, siunitx}

% Verbesserungen am Schriftbild
\usepackage{microtype}

% Literaturverzeichnis
\usepackage[
  backend=biber,
]{biblatex}
% Quellendatenbank
\addbibresource{lit.bib}
\addbibresource{programme.bib}

% Hyperlinks im Dokument
\usepackage[
  german,
  unicode,        % Unicode in PDF-Attributen erlauben
  pdfusetitle,    % Titel, Autoren und Datum als PDF-Attribute
  pdfcreator={},  % ┐ PDF-Attribute säubern
  pdfproducer={}, % ┘
]{hyperref}
% erweiterte Bookmarks im PDF
\usepackage{bookmark}

% Trennung von Wörtern mit Strichen
\usepackage[shortcuts]{extdash}

\author{%
  Vincent Wirsdörfer\\%
  \href{mailto:vincent.wirsdoerfer@udo.edu}{authorA@udo.edu}%
  \and%
  Joris Daus\\%
  \href{mailto:joris.daus@udo.edu}{authorB@udo.edu}%
}
\publishers{TU Dortmund – Fakultät Physik}


%\begin{document}
\section{Diskussion}
\label{sec:Diskussion}

%
%Frage: Welcher Literaturwert ist der richtige?
Der Literaturwert der effektiven Masse von GaAs beträgt $m^*_\text{lit}=\qty{0.083\pm0.005}m_e$ \cite{v46_paper}.
%
%
Die Abweichungen zum Literaturwert betragen somit


%
%berechnet mit 8.3%m_e
\begin{align*}
    \Delta_\text{stark} &= \qty{19\pm9}{\percent}\\
    \Delta_\text{schwach} &= \qty{2\pm7}{\percent}\\
\end{align*}

\noindent Zu sehen ist eine kleine, jedoch nicht verschwindende Abweichung zum Literaturwert. Dieser kann durch mehrere Probleme bei der Durchführung erklärt werden.\\
\noindent Zum einen ist die genaue Einstellung des Goniometers herausfordernd. So ist auf dem Oszilloskop um den richtigen Winkel herum kaum Veränderung zu 
erkennen. Es kann also das Goniometer verstellt werden, ohne dass die Einstellung verifiziert oder falsifiziert werden kann. Des Weiteren bereitet das 
Ablesen des Goniometers Schwierigkeiten, da die Feinskala sehr klein ist und teilweise mehrere Striche als das richtige Messergebnis interpretiert werden 
könnten. Das Oszilloskop liefert kein konstantes Bild, was zu Problemen beim Finden führt. \\
Die größte experimentelle Einschränkung war jedoch die Temperatur des Elektromagnets. Da durchgängig in etwa \qty{300}{\watt} durch den Magneten fließen, 
heizt er sich dementsprechend auf. Dies hatte zur Folge, dass die \qty{20}{\volt} Spannungsquelle am Ende einer Messreihe \qty{20.5}{\volt} 
liefern muss, um den Strom von \qty{10}{\ampere} aufrechtzuerhalten. Zu Beginn einer Messreihe ist die Stromquelle in der Strombegrenzung und passt die 
Spannung dem benötigten Strom von \qty{10}{\ampere} an. Es fließt also der Strom, welcher für ein konstantes Magnetfeld benötigt wird. Zum Ende der 
Messreihe ist die Spannungsquelle jedoch in der Spannungsbegrenzung, was zu Folge hat, dass der Strom limitiert wird. Dementsprechend ist das 
Magnetfeld auch nicht mehr konstant, was zu falschen Annahmen in der Berechnung führt.\\
Der Selektivverstärker zeigt während des Experimentierens gelegentlich einen Overload an. Sobald das Minimum eingestellt ist, erlischt das 
Lämpchen allerdings.\\
Abschließend kann gesagt werden, dass das Ergebnis trotz experimenteller Herausforderungen als eine grobe Orientierung dienen kann.
%\end{document}
