\input{../../header.tex}

\begin{document}
\section{Diskussion}
\label{sec:Diskussion}

%
%Frage: Welcher Literaturwert ist der richtige?
Der Literaturwert der effektiven Masse von GaAs beträgt $m^*_\text{lit}=\qty{0.083\pm0.005}m_e$ \cite{v46_paper}.
%
%
Die Abweichungen zum Literaturwert betragen somit


%
%berechnet mit 8.3%m_e
\begin{align*}
    \Delta_\text{stark} &= \qty{19\pm9}{\percent}\\
    \Delta_\text{schwach} &= \qty{2\pm7}{\percent}\\
\end{align*}

\noindent Zu sehen ist eine kleine, aber signifikanten Abweichung zum Literaturwert. Dieser kann durch mehrere Probleme bei der Durchführung erklärt werden.\\
\noindent Zum einen ist die genaue Einstellung des Goniometers herausfordernd. So ist auf dem Oszilloskop um den richtigen Winkel herum kaum Veränderung zu 
erkennen. Es kann also das Goniometer verstellt werden, ohne dass die Einstellung verifiziert oder falsifiziert werden kann. Des Weiteren bereitet das 
Ablesen des Goniometers Schwierigkeiten, da die Feinskala sehr klein ist und teilweise mehrere Striche als das richtige Messergebnis interpretiert werden 
könnten. Das Oszilloskop liefert kein konstantes Bild, was zu Problemen beim Finden führt. \\
Die größte experimentelle Einschränkung war jedoch die Temperatur des Elektromagnets. Da durchgängig in etwa \qty{300}{\watt} durch den Magneten fließen, 
heizt er sich dementsprechend auf. Dies hatte zur Folge, dass die \qty{20}{\volt} Spannungsquelle am Ende einer Messreihe \qty{20.5}{\volt} 
liefern muss, um den Strom von \qty{10}{\ampere} aufrechtzuerhalten. Zu Beginn einer Messreihe ist die Stromquelle in der Strombegrenzung und passt die 
Spannung dem benötigten Strom von \qty{10}{\ampere} an. Es fließt also der Strom, welcher für ein konstantes Magnetfeld benötigt wird. Zum Ende der 
Messreihe ist die Spannungsquelle jedoch in der Spannungsbegrenzung, was zu Folge hat, dass der Strom limitiert wird. Dementsprechend ist das 
Magnetfeld auch nicht mehr konstant, was zu falschen Annahmen in der Berechnung führt.\\
Der Selektivverstärker zeigt während des Experimentierens gelegentlich einen Overload an. Sobald das Minimum eingestellt ist, erlischt das 
Lämpchen allerdings.\\
Abschließend kann gesagt werden, dass das Ergebnis trotz experimenteller Herausforderungen als eine grobe Orientierung dienen kann.
\end{document}
