%\documentclass[
  bibliography=totoc,     % Literatur im Inhaltsverzeichnis
  captions=tableheading,  % Tabellenüberschriften
  titlepage=firstiscover, % Titelseite ist Deckblatt
]{scrartcl}

% Paket float verbessern
\usepackage{scrhack}

% Warnung, falls nochmal kompiliert werden muss
\usepackage[aux]{rerunfilecheck}

% unverzichtbare Mathe-Befehle
\usepackage{amsmath}
% viele Mathe-Symbole
\usepackage{amssymb}
% Erweiterungen für amsmath
\usepackage{mathtools}

% Fonteinstellungen
\usepackage{fontspec}
% Latin Modern Fonts werden automatisch geladen
% Alternativ zum Beispiel:
%\setromanfont{Libertinus Serif}
%\setsansfont{Libertinus Sans}
%\setmonofont{Libertinus Mono}

% Wenn man andere Schriftarten gesetzt hat,
% sollte man das Seiten-Layout neu berechnen lassen
\recalctypearea{}

% deutsche Spracheinstellungen
\usepackage[ngerman]{babel}


\usepackage[
  math-style=ISO,    % ┐
  bold-style=ISO,    % │
  sans-style=italic, % │ ISO-Standard folgen
  nabla=upright,     % │
  partial=upright,   % │
  mathrm=sym,        % ┘
  warnings-off={           % ┐
    mathtools-colon,       % │ unnötige Warnungen ausschalten
    mathtools-overbracket, % │
  },                       % ┘
]{unicode-math}

% traditionelle Fonts für Mathematik
\setmathfont{Latin Modern Math}
% Alternativ zum Beispiel:
%\setmathfont{Libertinus Math}

\setmathfont{XITS Math}[range={scr, bfscr}]
\setmathfont{XITS Math}[range={cal, bfcal}, StylisticSet=1]

% Zahlen und Einheiten
\usepackage[
  locale=DE,                   % deutsche Einstellungen
  separate-uncertainty=true,   % immer Unsicherheit mit \pm
  per-mode=symbol-or-fraction, % / in inline math, fraction in display math
]{siunitx}

% chemische Formeln
\usepackage[
  version=4,
  math-greek=default, % ┐ mit unicode-math zusammenarbeiten
  text-greek=default, % ┘
]{mhchem}

% richtige Anführungszeichen
\usepackage[autostyle]{csquotes}

% schöne Brüche im Text
\usepackage{xfrac}

% Standardplatzierung für Floats einstellen
\usepackage{float}
\floatplacement{figure}{htbp}
\floatplacement{table}{htbp}

% Floats innerhalb einer Section halten
\usepackage[
  section, % Floats innerhalb der Section halten
  below,   % unterhalb der Section aber auf der selben Seite ist ok
]{placeins}

% Seite drehen für breite Tabellen: landscape Umgebung
\usepackage{pdflscape}

% Captions schöner machen.
\usepackage[
  labelfont=bf,        % Tabelle x: Abbildung y: ist jetzt fett
  font=small,          % Schrift etwas kleiner als Dokument
  width=0.9\textwidth, % maximale Breite einer Caption schmaler
]{caption}
% subfigure, subtable, subref
\usepackage{subcaption}

% Grafiken können eingebunden werden
\usepackage{graphicx}

% schöne Tabellen
\usepackage{tabularray}
\UseTblrLibrary{booktabs, siunitx}

% Verbesserungen am Schriftbild
\usepackage{microtype}

% Literaturverzeichnis
\usepackage[
  backend=biber,
]{biblatex}
% Quellendatenbank
\addbibresource{lit.bib}
\addbibresource{programme.bib}

% Hyperlinks im Dokument
\usepackage[
  german,
  unicode,        % Unicode in PDF-Attributen erlauben
  pdfusetitle,    % Titel, Autoren und Datum als PDF-Attribute
  pdfcreator={},  % ┐ PDF-Attribute säubern
  pdfproducer={}, % ┘
]{hyperref}
% erweiterte Bookmarks im PDF
\usepackage{bookmark}

% Trennung von Wörtern mit Strichen
\usepackage[shortcuts]{extdash}

\author{%
  Vincent Wirsdörfer\\%
  \href{mailto:vincent.wirsdoerfer@udo.edu}{authorA@udo.edu}%
  \and%
  Joris Daus\\%
  \href{mailto:joris.daus@udo.edu}{authorB@udo.edu}%
}
\publishers{TU Dortmund – Fakultät Physik}


%\begin{document}
\section{Auswertung}
\label{sec:Auswertung}

Die Alpha-Teilchen verlieren mit zurückgelegter Strecke Energie. Dieser Energieverlust $-\frac{\text{d} E}{\text{d} x}$ 
kann bestimmt werden, indem die Energie der Teilchen nach der effektiven Weglänge aufgetragen wird. \\
Die effektive Weglänge kann mithilfe von Gleichung \eqref{eqn:effLaenge} bestimmt werden.
\noindent Die Energie ist direkt proportional zu dem Channel, in dem am meisten Teilchen gemessen werden. Der Channel, der bei einem Druck von 
$\qty{0}{\milli \bar}$ am meisten Teilchen gemessen hat, besitzt die Energie von \qty{4}{\mega \eV}. Der Channel \num{0} 
besitzt eine Energie von \qty{0}{\eV}. So kann über einen einfachen Dreisatz die Energie von jedem Channel berechnet 
werden. Werden diese Werte nun gegeneinander aufgetragen, so ergeben sich die folgenden Plots.
Zunächst wird die Auftragung für die Messung bei einem Abstand von \qty{4}{\centi \meter} dargestellt.

\begin{figure}[H]
    \includegraphics[width=\textwidth]{4cm.pdf}
    \caption{Effektive Weglänge gegen am meist gemessene Energie bei 4 cm Abstand.}
\end{figure}

\noindent Die Auftragung für die zweite Messung mit einem Abstand von \qty{5}{\centi \meter} sieht folgendermaßen aus.

\begin{figure}[H] 
    \includegraphics[width=\textwidth]{5cm.pdf}
    \caption{Effektive Weglänge gegen am meist gemessene Energie bei 5 cm Abstand.}
\end{figure}

\noindent Nun werden die Ausgleichsgerade beider Auftragungen ausgewertet. Die Steigungen der Geraden ergeben sich zu 

\begin{align*}
    -\frac{\text{d} E_{\alpha, 4}}{\text{d} x} &= \qty{1.12 \pm 0.05 e8}{\eV \per \meter} \\
    -\frac{\text{d} E_{\alpha, 5}}{\text{d} x} &= \qty{1.11 \pm 0.09 e8}{\eV \per \meter}. \\
\end{align*}

\noindent Um die mittlere freie Weglänge zu bestimmen, muss die Gesamtzahl der gezählten Pulse gegen die effektive Weglänge aufgetragen werden. 
Die effektive Weglänge wurde bereits vorher bestimmt. Die Gesamtzahl der Pulse ist eine Messgröße. 
Es wird nun eine Horizontale Linie auf Höhe der Hälfte der maximal gemessenen Pulse gezeichnet. Des Weiteren wird eine Ausgleichsgerade über 
den Abfall der Messwerte gelegt. Nun wird der Schnittpunkt beider geraden analytisch bestimmt und mit einem grünen Punkt markiert.\\
\noindent Dieses Verfahren wird für \qty{4}{\centi \meter} und für \qty{5}{\centi \meter} durchgeführt. So ergeben sich die folgenden Grafiken.

\begin{figure}[H] 
    \includegraphics[width=\textwidth]{sumpulses4.pdf}
    \caption{Effektive Weglänge gegen am meist gemessene Energie bei 5 cm Abstand.}
\end{figure}

\begin{figure}[H] 
    \includegraphics[width=\textwidth]{sumpulses5.pdf}
    \caption{Effektive Weglänge gegen am meist gemessene Energie bei 5 cm Abstand.}
\end{figure}

\noindent Der Schnittpunkt der Hilfslinien gibt die mittlere Reichweite $R_m$ der 
Alphastrahlung an. Diese werden mit der erläuterten Methode auf 

\begin{align}
    R_{m, \qty{4}{\centi \meter}} &= \qty{2.33}{\centi \meter} \\
    R_{m, \qty{5}{\centi \meter}} &= \qty{2.41}{\centi \meter} \\
\end{align}

\noindent bestimmt.\\

\noindent Welche genaue Energie ein Alpha-Teilchen besitzt ist zufällig. Es ist jedoch möglich mithilfe einer 
Wahrscheinlichkeitsverteilung Aussagen über die Häufigkeit gewisser Energien zu treffen. Im Folgenden soll nun die Art 
der Wahrscheinlichkeitsverteilung bestimmt werden. Dazu werden die 100 Messungen der häufigsten Energie in einem 
Histogramm aufgetragen. \\
\noindent Für die Poisson-Verteilung muss der Erwartungswert und die Varianz berechnet werden. Diese sind bei der 
Poisson-Verteilung allerdings identisch. Dieser Wert $\lambda$ lässt sich mithilfe des Maximum-Likelihood-Schätzers 
berechnen. So gilt, dass $\lambda$ das arithmetische Mittel ist. Dieses wird zu $\lambda = 1842.91$ berechnet.
Die Standardabweichung wird mit \emph{numpy.std} berechnet und beträgt $72.74$. Das Quadrat aus der Standardabweichung ist 
die Varianz und beträgt dementsprechend $72.47^2 = 5291.74$. \\
\noindent Es kann nun die gemessene Verteilung mit einer Gauß-Verteilung und einer Poisson-Verteilung verglichen werden. 
Die zu vergleichenden Zufallsverteilungen werden beide mit derselben Stichprobengröße von 100 erstellt. Es können nun 
alle drei Wahrscheinlichkeitsverteilungen in Histogrammen mit verschiedenen Bingrößen dargestellt werden.

\begin{figure}
    \includegraphics[width=\textwidth]{Verteilung.pdf}
    \caption{Vergleich von Gauß-, Poisson- und gemessener Verteilung.}
\end{figure}

%\end{document}
