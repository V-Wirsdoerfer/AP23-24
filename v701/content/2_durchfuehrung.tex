\input{../../header.tex}

\begin{document}
\section{Versuchsaufbau}
\label{sec:Versuchsaufbau}

Das Kernelement des Versuches ist ein Glaszylinder mit aufgedruckter Längenskala. In diesem Glaszylinder befindet 
sich eine radioaktive Strahlungsquelle und ein Halbleiter-Detektor. Die Strahlungsquelle ist auf einem verschiebbaren Halter 
positioniert, sodass der Abstand zwischen Präparat und Detektor variiert werden kann. Bei dem hier betrachteten Detektor 
handelt es sich um einen Halbleiter-Sperrschichtzähler, welcher ähnlich zu einer Diode funktioniert. Dieser besteht im 
Wesentlichen aus einer p- und n-dotierten Schicht, welche durch eine \enquote{Verunreinigung} des Halbleitermaterials 
hervorgerufen wird. Werden beispielsweise Bohr- und Phosphor-Atome in einen Silizium-Halbleiter eingesetzt, so entshen aufgrund 
der jeweiligen Elektronenkonfiguration der Fremdatome Bohr und Phosphor zusätzliche \enquote{Elektronen-Loch-Paare} (p-Dotierung)
und Elektronen (n-Dotierung). Zwischen diesen Schichten befindet sich eine ladungsfreie Zone, die sogenannte 
\emph{Verarmungszone}. Ähnlich einer Diode gibt es hierbei eine Sperrrichtung und eine Durchlassrichtung für sich bewegende Elektronen.
Wird nun eine Spannung im Sinne der Sperrrichtung angelegt, entsteht ein zusätzliches elektrisches Feld, welches die Verarmungszone 
vergrößert. Fallen nun die ionisierten Teilchen der $\alpha$-Strahlungsquelle in die eben genannte Zone ein, werden weitere
Elektronen-Loch-Paare erzeugt, welche in Folge des elektrischen Feldes einen Stromimpuls bedingen. Dieser Impuls wird nun von 
einem Vorverstärker verstärkt und an einen Vielkanalanalysator weitergeleitet. Hier werden die detektierten Impule in Form eines Histogramms visualisiert. 
Die genaue Verknüpfung der Geräte wird in der unten stehenden Abbildung \ref{fig:Aufbau} dargestellt. Wie in Kapitel \ref{sec:Theorie} bereits erläutert,
wird zur Bestimmung der Reichweite der Teilchen die Proportionalität zum herrschenden Druck angewendet. Zur Regulierung des Drucks dient eine Vakuumpumpe 
und zwei Ventile. Der Betrag des Drucks lässt sich am Druckmessgerät navigieren und ablesen.

\section{Versuchsdurchführung}
\label{sec:Versuchsdurchführung}



\section{Messwerte}
\label{sec:Messwerte}



\end{document}

