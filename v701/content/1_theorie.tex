\input{../../header.tex}

\begin{document}
\section{Zielsetzung}

Wie bereits der Vorzugsbezeichnung entnommen werden kann, besteht das Ziel des im folgend protokollierten 
Versuch darin, die Reichweite von $\alpha$-Strahlung zu bestimmen.

\section{Theorie}
\label{sec:Theorie}

Konzeptionell ist der Grundgedanke des Experiments die Reichweite der Teilchen über die mathematische Korrelation 
zwischen Reichweite und Energie zu ermitteln. Die $\alpha$-Teilchen verlieren beim Durchdringen von Materie aufgrund 
von elastischen Stößen, Ionisationsprozessen sowie Anregung und Dissoziation von Molekülen eine gewisse Energie. Mittels
der sog. \emph{Bethe-Bloch-Gleichung} lässt sich der Energieverlust $\sfrac{-\symup{d}E_\alpha}{\symup{d}x}$ ausdrücken durch 

\begin{equation}
\label{eqn:BBG}
    -\frac{\symup{d}E_\alpha}{\symup{d}x} = \frac{z²e⁴}{4\pi\varepsilon_{0}m_\text{e}}\frac{nZ}{v²}\ln\left(\frac{2m_\text{e}v²}{I}\right).
\end{equation}

\noindent Hierbei bezeichnet $z$ die Ladung und $v$ die Geschwindigkeit der $\alpha$-Strahlung. $Z$ beschreibt die Ordnungszahl,
$n$ die Ladungsträgerdichte und $I$ die Ionisierungsenergie des Targetgases.\\
\noindent Die Wegstrecke eines $\alpha$-Teilchens bis zur vollständigen Abbremsung wird über das Integral 

\begin{equation*}
%\label{eqn:Wegstrecke}
    R = \int_{0}^{E_\alpha} \frac{...}{...}\,\symup{d}x
\end{equation*}

\noindent berechnet und wird im Folgenden als Reichweite bezeichnet. Bei abnehmender Energie der Teilchen verliert die Bethe-Bloch-Gleichung
\eqref{eqn:BBG} aufgrund von vermehrt auftretenden Ladungsaustauschprozessen ihre Gültigkeit. Dies hat zur Folge, dass die mittlere 
Reichweite der Teilchen über eine empirische Kurve bestimmt wird. Hierbei darf für die mittlere Reichweite von $\alpha$-Strahlung 
in Luft mit Energien $E_\alpha\leq{}2.5\,\unit{\mega\eV}$ die Beziehung $R_\text{m} = 3.1\cdot{}\left(E_\alpha\right)^{\sfrac{3}{2}}$
voraussgesetzt werden.\\

\noindent Zudem ist die Reichweite von $\alpha$-Teilchen unter isothermen und isochoren Voraussetzungen proportional zum Druck $p$, 
weshalb die Reichweite über eine Absorptionsmessung mit variierendem Druck ermittelt werden kann. Bei einem Normaldruck 
$p_0 = 1013\,\unit{\milli\bar}$ und einem festen Abstand $x_0$ zwischen $\alpha$-Strahler und Detektor gilt für die effektive Länge $x$

\begin{equation}
\label{eqn:effLaenge}
    x = x_0\frac{p}{p_0}.
\end{equation}

\section{Vorbereitung}

\section{Fehlerrechnung}
\end{document}