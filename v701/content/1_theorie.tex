\documentclass[
  bibliography=totoc,     % Literatur im Inhaltsverzeichnis
  captions=tableheading,  % Tabellenüberschriften
  titlepage=firstiscover, % Titelseite ist Deckblatt
]{scrartcl}

% Paket float verbessern
\usepackage{scrhack}

% Warnung, falls nochmal kompiliert werden muss
\usepackage[aux]{rerunfilecheck}

% unverzichtbare Mathe-Befehle
\usepackage{amsmath}
% viele Mathe-Symbole
\usepackage{amssymb}
% Erweiterungen für amsmath
\usepackage{mathtools}

% Fonteinstellungen
\usepackage{fontspec}
% Latin Modern Fonts werden automatisch geladen
% Alternativ zum Beispiel:
%\setromanfont{Libertinus Serif}
%\setsansfont{Libertinus Sans}
%\setmonofont{Libertinus Mono}

% Wenn man andere Schriftarten gesetzt hat,
% sollte man das Seiten-Layout neu berechnen lassen
\recalctypearea{}

% deutsche Spracheinstellungen
\usepackage[ngerman]{babel}


\usepackage[
  math-style=ISO,    % ┐
  bold-style=ISO,    % │
  sans-style=italic, % │ ISO-Standard folgen
  nabla=upright,     % │
  partial=upright,   % │
  mathrm=sym,        % ┘
  warnings-off={           % ┐
    mathtools-colon,       % │ unnötige Warnungen ausschalten
    mathtools-overbracket, % │
  },                       % ┘
]{unicode-math}

% traditionelle Fonts für Mathematik
\setmathfont{Latin Modern Math}
% Alternativ zum Beispiel:
%\setmathfont{Libertinus Math}

\setmathfont{XITS Math}[range={scr, bfscr}]
\setmathfont{XITS Math}[range={cal, bfcal}, StylisticSet=1]

% Zahlen und Einheiten
\usepackage[
  locale=DE,                   % deutsche Einstellungen
  separate-uncertainty=true,   % immer Unsicherheit mit \pm
  per-mode=symbol-or-fraction, % / in inline math, fraction in display math
]{siunitx}

% chemische Formeln
\usepackage[
  version=4,
  math-greek=default, % ┐ mit unicode-math zusammenarbeiten
  text-greek=default, % ┘
]{mhchem}

% richtige Anführungszeichen
\usepackage[autostyle]{csquotes}

% schöne Brüche im Text
\usepackage{xfrac}

% Standardplatzierung für Floats einstellen
\usepackage{float}
\floatplacement{figure}{htbp}
\floatplacement{table}{htbp}

% Floats innerhalb einer Section halten
\usepackage[
  section, % Floats innerhalb der Section halten
  below,   % unterhalb der Section aber auf der selben Seite ist ok
]{placeins}

% Seite drehen für breite Tabellen: landscape Umgebung
\usepackage{pdflscape}

% Captions schöner machen.
\usepackage[
  labelfont=bf,        % Tabelle x: Abbildung y: ist jetzt fett
  font=small,          % Schrift etwas kleiner als Dokument
  width=0.9\textwidth, % maximale Breite einer Caption schmaler
]{caption}
% subfigure, subtable, subref
\usepackage{subcaption}

% Grafiken können eingebunden werden
\usepackage{graphicx}

% schöne Tabellen
\usepackage{tabularray}
\UseTblrLibrary{booktabs, siunitx}

% Verbesserungen am Schriftbild
\usepackage{microtype}

% Literaturverzeichnis
\usepackage[
  backend=biber,
]{biblatex}
% Quellendatenbank
\addbibresource{lit.bib}
\addbibresource{programme.bib}

% Hyperlinks im Dokument
\usepackage[
  german,
  unicode,        % Unicode in PDF-Attributen erlauben
  pdfusetitle,    % Titel, Autoren und Datum als PDF-Attribute
  pdfcreator={},  % ┐ PDF-Attribute säubern
  pdfproducer={}, % ┘
]{hyperref}
% erweiterte Bookmarks im PDF
\usepackage{bookmark}

% Trennung von Wörtern mit Strichen
\usepackage[shortcuts]{extdash}

\author{%
  Vincent Wirsdörfer\\%
  \href{mailto:vincent.wirsdoerfer@udo.edu}{authorA@udo.edu}%
  \and%
  Joris Daus\\%
  \href{mailto:joris.daus@udo.edu}{authorB@udo.edu}%
}
\publishers{TU Dortmund – Fakultät Physik}


\begin{document}
\section{Zielsetzung}

Wie bereits der Versuchsbeziechnung entnommen werden kann, besteht das Ziel des im folgend protokollierten 
Versuch darin, die Reichweite von $\alpha$-Strahlung zu bestimmen.

\section{Theorie}
\label{sec:Theorie}

Konzeptionell ist der Grundgedanke des Experiments die Reichweite der Teilchen über die mathematische Korrelation 
zwischen Reichweite und Energie zu ermitteln. Die $\alpha$-Teilchen verlieren beim Durchdringen von Materie aufgrund 
von elastischen Stößen, Ionisationsprozessen sowie Anregung und Dissoziation von Molekülen eine gewisse Energie. Mittels
der sog. \emph{Bethe-Bloch-Gleichung} lässt sich der Energieverlust $\sfrac{-\symup{d}E_\alpha}{\symup{d}x}$ ausdrücken durch 

\begin{equation}
\label{eqn:BBG}
    -\frac{\symup{d}E_\alpha}{\symup{d}x} = \frac{z²e⁴}{4\pi\varepsilon_{0}m_\text{e}}\frac{nZ}{v²}\ln\left(\frac{2m_\text{e}v²}{I}\right).
\end{equation}

\noindent Hierbei bezeichnet $z$ die Ladung und $v$ die Geschwindigkeit der $\alpha$-Strahlung. $Z$ beschreibt die Ordnungszahl,
$n$ die Ladungsträgerdichte und $I$ die Ionisierungsenergie des Targetgases.\\
\noindent Die Wegstrecke eines $\alpha$-Teilchens bis zur vollständigen Abbremsung wird über das Integral 

\begin{equation*}
%\label{eqn:Wegstrecke}
    R = \int_{0}^{E_\alpha} \frac{...}{...}\,\symup{d}x
\end{equation*}

\noindent berechnet und wird im Folgenden als Reichweite bezeichnet. Bei abnehmender Energie der Teilchen verliert die Bethe-Bloch-Gleichung
\eqref{eqn:BBG} aufgrund von vermehrt auftretenden Ladungsaustauschprozessen ihre Gültigkeit. Dies hat zur Folge, dass die mittlere 
Reichweite der Teilchen über eine empirischen Kurve bestimmt wird. Hierbei darf für die mittlere Reichweite von $\alpha$-Strahlung 
in Luft mit Energien $E_\alpha\leq{}2.5\,\unit{\mega\eV}$ die Beziehung $R_\text{m} = 3.1\cdot{}\left(E_\alpha\right)^{\sfrac{3}{2}}$
voraussgesetzt werden.\\

\noindent Zudem ist die Reichweite von $\alpha$-Teilchen unter isothermen und isochoren Voraussetzungen proportional zum Druck $p$, 
weshalb die Reichweite über eine Absorptionsmessung mit variirendem Druck ermittelt werden kann. Bei einem Normaldruck 
$p_0 = 1013\,\unit{\milli\bar}$ und einem festen Abstand $x_0$ zwischen $\alpha$-Strahler und Detektor gilt für die effektive Länge 

\begin{equation*}
%\label{eqn:}
    x = x_0\frac{p}{p_0}.
\end{equation*}

\section{Vorbereitung}

\section{Fehlerrechnung}
\end{document}