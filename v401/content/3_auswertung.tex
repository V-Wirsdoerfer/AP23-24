%\documentclass[
  bibliography=totoc,     % Literatur im Inhaltsverzeichnis
  captions=tableheading,  % Tabellenüberschriften
  titlepage=firstiscover, % Titelseite ist Deckblatt
]{scrartcl}

% Paket float verbessern
\usepackage{scrhack}

% Warnung, falls nochmal kompiliert werden muss
\usepackage[aux]{rerunfilecheck}

% unverzichtbare Mathe-Befehle
\usepackage{amsmath}
% viele Mathe-Symbole
\usepackage{amssymb}
% Erweiterungen für amsmath
\usepackage{mathtools}

% Fonteinstellungen
\usepackage{fontspec}
% Latin Modern Fonts werden automatisch geladen
% Alternativ zum Beispiel:
%\setromanfont{Libertinus Serif}
%\setsansfont{Libertinus Sans}
%\setmonofont{Libertinus Mono}

% Wenn man andere Schriftarten gesetzt hat,
% sollte man das Seiten-Layout neu berechnen lassen
\recalctypearea{}

% deutsche Spracheinstellungen
\usepackage[ngerman]{babel}


\usepackage[
  math-style=ISO,    % ┐
  bold-style=ISO,    % │
  sans-style=italic, % │ ISO-Standard folgen
  nabla=upright,     % │
  partial=upright,   % │
  mathrm=sym,        % ┘
  warnings-off={           % ┐
    mathtools-colon,       % │ unnötige Warnungen ausschalten
    mathtools-overbracket, % │
  },                       % ┘
]{unicode-math}

% traditionelle Fonts für Mathematik
\setmathfont{Latin Modern Math}
% Alternativ zum Beispiel:
%\setmathfont{Libertinus Math}

\setmathfont{XITS Math}[range={scr, bfscr}]
\setmathfont{XITS Math}[range={cal, bfcal}, StylisticSet=1]

% Zahlen und Einheiten
\usepackage[
  locale=DE,                   % deutsche Einstellungen
  separate-uncertainty=true,   % immer Unsicherheit mit \pm
  per-mode=symbol-or-fraction, % / in inline math, fraction in display math
]{siunitx}

% chemische Formeln
\usepackage[
  version=4,
  math-greek=default, % ┐ mit unicode-math zusammenarbeiten
  text-greek=default, % ┘
]{mhchem}

% richtige Anführungszeichen
\usepackage[autostyle]{csquotes}

% schöne Brüche im Text
\usepackage{xfrac}

% Standardplatzierung für Floats einstellen
\usepackage{float}
\floatplacement{figure}{htbp}
\floatplacement{table}{htbp}

% Floats innerhalb einer Section halten
\usepackage[
  section, % Floats innerhalb der Section halten
  below,   % unterhalb der Section aber auf der selben Seite ist ok
]{placeins}

% Seite drehen für breite Tabellen: landscape Umgebung
\usepackage{pdflscape}

% Captions schöner machen.
\usepackage[
  labelfont=bf,        % Tabelle x: Abbildung y: ist jetzt fett
  font=small,          % Schrift etwas kleiner als Dokument
  width=0.9\textwidth, % maximale Breite einer Caption schmaler
]{caption}
% subfigure, subtable, subref
\usepackage{subcaption}

% Grafiken können eingebunden werden
\usepackage{graphicx}

% schöne Tabellen
\usepackage{tabularray}
\UseTblrLibrary{booktabs, siunitx}

% Verbesserungen am Schriftbild
\usepackage{microtype}

% Literaturverzeichnis
\usepackage[
  backend=biber,
]{biblatex}
% Quellendatenbank
\addbibresource{lit.bib}
\addbibresource{programme.bib}

% Hyperlinks im Dokument
\usepackage[
  german,
  unicode,        % Unicode in PDF-Attributen erlauben
  pdfusetitle,    % Titel, Autoren und Datum als PDF-Attribute
  pdfcreator={},  % ┐ PDF-Attribute säubern
  pdfproducer={}, % ┘
]{hyperref}
% erweiterte Bookmarks im PDF
\usepackage{bookmark}

% Trennung von Wörtern mit Strichen
\usepackage[shortcuts]{extdash}

\author{%
  Vincent Wirsdörfer\\%
  \href{mailto:vincent.wirsdoerfer@udo.edu}{authorA@udo.edu}%
  \and%
  Joris Daus\\%
  \href{mailto:joris.daus@udo.edu}{authorB@udo.edu}%
}
\publishers{TU Dortmund – Fakultät Physik}


%\begin{document}
\section{Auswertung}
\label{sec:Auswertung}

Zunächst wird die Wellenlänge des Lasers berechnet. Dazu wird Gleichung \eqref{eqn:Laenge} verwendet. 
$x$ bezeichnet die Strecke, um die der Spiegel verschoben wird. Ein Motor drückt dafür an einem Hebel, 
welcher den Spiegel nach vorne stellt. Das Drücken des Motors kann an der Mikrometerschraube abgelesen 
werden und beträgt \qty{5\pm0.01}{\milli \meter}. Der Hebel besitzt eine Übersetzung von $1:5.017$. Setzt man 
dies nun in \eqref{eqn:Laenge} ein, so ergibt sich 

\begin{equation}
    \label{eqn:Wellenlaenge}
    \lambda = \frac{2 \cdot \qty{5\pm0.01e-3}{\meter} }{z \cdot 5.017}
\end{equation}

\noindent Bevor die Wellenlänge berechnet wird, wird erst die durchschnittliche Zählrate berechnet. Diese 
beträgt $z = \num{2991\pm26}$. Nun kann die Wellenlänge über \eqref{eqn:Wellenlaenge} berechnet werden. So 
ergibt sich für den Laser eine Wellenlänge von 

\begin{align}
    \lambda = \qty{666\pm6e-7}{\nano \meter}.
\end{align}

\noindent Nun kann der Brechungsindex von Luft bestimmt werden. Dies geschieht, indem die Gleichung \eqref{eqn:delta_n} 
nach $\Delta n$ umgestellt wird und in Gleichung \eqref{eqn:Brechungsindex} eingesetzt wird. So ergibt sich

\begin{equation*}
    n = 1 + \frac{z \lambda}{2 D} \frac{T}{T_0}\frac{p_0}{\Delta p}.
\end{equation*}

\noindent In dieser Berechnung wird die Wellenlänge des Lasers mit $\lambda = \qty{635}{\nano \meter}$ angenommen. 
Dies ist der Wert, welcher auf dem Laser steht. Es wird darauf verzichtet den bereits errechneten Wert zu 
nehmen, da so Fehlerfortpflanzungen und Folgefehler verhindert werden. $p_0$ bezeichnet den Luftdruck der Umgebung.
Die Differenz der Luftdrucks in der Gaszelle zum Umgebungsluftdruck wird mit $\Delta p$ bezeichnet. $z$ bezeichnet 
den Mittelwert der Zählraten. Werden nun alle Werte eingesetzt, so ergibt sich ein Brechungsindex von 

\begin{align}
    n = \qty{1.000291\pm0.000020}.
\end{align}




%\end{document}
