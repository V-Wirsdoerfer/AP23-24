%\input{../../header.tex}

%\begin{document}
\section{Auswertung}
\label{sec:Auswertung}

Zunächst wird die Wellenlänge des Lasers berechnet. Dazu wird Gleichung \eqref{eqn:Laenge} verwendet. 
$x$ bezeichnet die Strecke, um die der Spiegel verschoben wird. Ein Motor drückt dafür an einem Hebel, 
welcher den Spiegel nach vorne stellt. Das Drücken des Motors kann an der Mikrometerschraube abgelesen 
werden und beträgt \qty{5\pm0.01}{\milli \meter}. Der Hebel besitzt eine Übersetzung von $1:5.017$. Setzt man 
dies nun in \eqref{eqn:Laenge} ein, so ergibt sich 

\begin{equation}
    \label{eqn:Wellenlaenge}
    \lambda = \frac{2 \cdot \qty{5\pm0.01e-3}{\meter} }{z \cdot 5.017}
\end{equation}

\noindent Bevor die Wellenlänge berechnet wird, wird erst die durchschnittliche Zählrate berechnet. Diese 
beträgt $z = \num{2991\pm26}$. Nun kann die Wellenlänge über \eqref{eqn:Wellenlaenge} berechnet werden. So 
ergibt sich für den Laser eine Wellenlänge von 

\begin{align}
    \lambda = \qty{666\pm6e-7}{\nano \meter}.
\end{align}

\noindent Nun kann der Brechungsindex von Luft bestimmt werden. Dies geschieht, indem die Gleichung \eqref{eqn:delta_n} 
nach $\Delta n$ umgestellt wird und in Gleichung \eqref{eqn:Brechungsindex} eingesetzt wird. So ergibt sich

\begin{equation*}
    n = 1 + \frac{z \lambda}{2 D} \frac{T}{T_0}\frac{p_0}{\Delta p}.
\end{equation*}

\noindent In dieser Berechnung wird die Wellenlänge des Lasers mit $\lambda = \qty{635}{\nano \meter}$ angenommen. 
Dies ist der Wert, welcher auf dem Laser steht. Es wird darauf verzichtet den bereits errechneten Wert zu 
nehmen, da so Fehlerfortpflanzungen und Folgefehler verhindert werden. $p_0$ bezeichnet den Luftdruck der Umgebung.
Die Differenz der Luftdrucks in der Gaszelle zum Umgebungsluftdruck wird mit $\Delta p$ bezeichnet. $z$ bezeichnet 
den Mittelwert der Zählraten. Werden nun alle Werte eingesetzt, so ergibt sich ein Brechungsindex von 

\begin{align}
    n = \qty{1.000291\pm0.000020}.
\end{align}




%\end{document}
