\input{../../header.tex}

\begin{document}
\section{Versuchsaufbau}

Die Funktionsweise des Michelson Interferometers beschränkt sich  im Wesentlichen auf die Reflexe und Interferenzen 
von Laserstrahlen. An einem teildurchlässigen Spiegel wird das kohärente Licht eines Diodenlasers partiell reflektiert 
und transmittiert. Dabei befindet sich der Spiegel in Winkel von \qty{45}{\degree} zur Propagationsrichtung des Laserstrahls.
Dies lässt sich anhand der untenstehenden Abbildung \ref{fig:Aufbau} nachvollziehen. Spiegel \uproman{1} dient zu Reflexion des 
am Spiegel reflektierten Teilstrahl des Lasers. Analog wird der transmittierte Teilstrahl an Spiegel \uproman{2} reflektiert.
Beide Strahlen laufen somit erneut zusammen und können miteinander interferieren. Auf einen Schirm im Strahlenweg oder mit einer
Photodiode kann das Interferenzmuster direkt visualisiert bzw. gezeigt werden.\\

\noindent Unter Zuhilfenahme eines Schmitt-Triggers wird das Signal der Photozelle in einen Impuls umgewandelt. Dieser bewirkt,
dass die analogen Signale der Photozelle verstärkt werden und bei Überschreiten eine gewissen Schwellwertes als digitale 1 von 
einem Zähler registriert werden. Die Spannungen der Photozelle unterhalb der Schwelle erzeugen im Gegensatz eine 0.\\

\noindent Die in der Abbildung erkennbare Gaszelle kann alternierend evakuiert und belüftet werden und dient zur Ermittlung 
von Brechungsindizes verschiedener Gase.

\begin{figure}[H]
    \centering
    \includegraphics[height=5cm]{Aufbau.png}
    \caption{Versuchaufbau des Michelson Interferometers\cite{Versuchsanleitung_v401}.}
    \label{fig:Aufbau}
\end{figure}

\section{Versuchsdurchführung}

\section{Messwerte}

\end{document}

