\input{../../header.tex}

\begin{document}
\section{Diskussion}
\label{sec:Diskussion}

Das Michelson Interferometer arbeitet über Lichtempfindlichkeit, da der Zähler Lichtveränderungen detektiert. 
Außerdem gibt es keine lichtundurchlässige Abschirmung zum Experiment. Dies führt dazu, dass Lichtänderungen 
der Außenwelt das Experiment beeinflussen kann. Gibt es beispielsweise einen Lichtimpuls durch Spiegelungen 
oder eine sich öffnende Tür, so kann der Zähler ein falsch positives Ergebnis zählen. 
Die Lichtveränderungen, welche detektiert werden besitzen eine kleine räumliche Ausdehnung. Aus diesem Grund 
können kleine räumliche Änderungen der Apparatur falsch positive Zählungen zur Folge haben. Beispielsweise 
führen kleine Stöße am Tisch oder dem Messaufbau zu falsch positiven Zählungen. Ein Schlag auf den Tisch kann 
so zu zehn bis 20 Zählungen zu viel führen. Um den Zähler auf die Umgebungshelligkeit zu justieren und so 
falsch positive Ergebnisse zu minimieren, wird der Vorverstärker per Hand kalibriert. Es gibt keine genauen 
Werte dafür und der Drehknopf wird nach Gefühl eingestellt. So kann es sein, dass er zu empfindlich ist, 
was zu erhöhten Werten führt, oder er zu unempfindlich ist, was zu kleinen Werte zur Folge hat. 
Der Motor ist über einen Keilriemen mit der Messschraube verbunden. Der Keilriemen ist allerdings nicht ideal 
gespannt, weshalb Schlupf zwischen Keilriemen und Zahnrad existiert. Aus diesem Grund wird die Mikrometerschraube 
nicht gleichmäßig verstellt und zu schnelle Intensitätswechsel werden ggf. nicht gezählt. Die Skala der 
Messschraube ist leicht verschoben, was ein genaues Ablesen erschwert. Die Zählung wird per Hand gestartet, 
wenn die Mikrometerschraube \qty{2}{\milli \meter} beziehungsweise \qty{5}{\milli \meter} anzeigt und per Hand 
gestoppt, wenn \qty{5}{\milli \meter} verstrichen sind. Dies schließt einen menschlichen Fehler durch nicht 
ideale Reaktionszeiten mit ein.\\

\noindent Als Referenzwert wird die Wellenlänge genommen, welche auf dem Laser steht. Diese beträgt 
$\lambda_\text{Lit} = \qty{635}{\nano \meter}$. Berechnet wurde eine Wellenlänge von 
$\lambda_\text{Lit} = \qty{666\pm6}{\nano \meter}$. Dies ist eine Abweichung von \qty{5\pm0.9}{\percent}.
Grund dafür können die bereits erwähnten Schwierigkeiten bei der Durchführung sein.




%Umwelteinflüsse in Form von Lichtänderungen und Stößen verfälschen Ergebnis
%Keine gleichmäßige Bewegung der Messschraube aufgrund von unflüssiger Bewegung des Keilriemen
%Skala der Schraube ist aufgrund von Abnutzung leicht verschoben
%Der Selektivverstärker und Schmitt-Trigger wird manuell eingestellt ohne harte Faktoren zu kennen
%Schlupf zwischen Keilriemen und Zahnrad

\end{document}
