%\input{../../header.tex}

%\begin{document}
\section{Versuchsaufbau}
Das Herzstück des Versuches ist der Lock-In-Verstärker. Dieser kann eine Referenz- und Signalspannung erzeugen, erzeugte Spannungen mit 
Noise versehen und verstärken. Außerdem ist er dazu in der Lage, Signale durch einen Phasenschieber zu verzögern und somit die Phase zu 
verändern. Verschiedene Filter können zur Bereinigung und Integration eines Signales verwendet werden.\\
\noindent
Das Experiment besteht aus zwei Versuchsaufbauten. 
Ein Versuchsaufbau dient dazu, den Einfluss eines Noisegenerators auf den Lock-In-Verstärker zu analysieren. Der zweite Aufbau testet 
den Lock-In-Verstärker unter einem real erzeugten Rauschsignal.

\subsection{Künstlich generierter Noise}
\label{sec:kunstnoise}
Die Referenzspannung wird über einen Noisegenerator verrauscht und anschließend in einem Pre-Amplifier verstärkt. Das verstärkte Signal 
wird dann in einem Bandpass von sehr tiefen und hohen Frequenzen bereinigt. Anschließend geht dieses Signal dann in einen Detektor.
Die Signalspannung geht durch einen Phasenschieber in den Lock-In-Verstärker. Das Ausgangssignal des Lock-In-Verstärkers wird in ein digitales 
Oszilloskop eingespeist. So entsteht die folgende Schaltung.

\begin{figure}[H]
    \includegraphics[width=\textwidth]{./content/Schaltung_kunstliche.png}
    \caption{Schaltung des Lock-In-Verstärkers mit künstlich generiertem Noise \cite{Versuchsanleitung_v303}}
    \label{fig:kunstnoise}
\end{figure}

\subsection{Noise durch Photodiode}
\noindent
Bei diesem Experimentaufbau wurde der Noisegenerator aus \autoref{sec:kunstnoise} durch eine LED mit Photodiode ausgetauscht. So ergibt sich 
die Schaltung \ref{fig:photonoise}

\begin{figure}[H]
    \includegraphics[width=\textwidth]{./content/Schaltung_Photodiode.png}
    \caption{Schaltung des Lock-In-Verstärkers mit Noise durch Photodiode \cite{Versuchsanleitung_v303}.}
    \label{fig:photonoise}
\end{figure}

\noindent
Der Photodetektor ist auf einer Schiene montiert, sodass der Abstand $r$ aus \autoref{fig:photonoise} variiert werden kann.
Außerdem ist zu erwähnen, dass der Output an einem Integrator (also einem Tiefpass) angeschlossen ist, sodass das Signal über eine 
gewisse Zeitspanne integriert wird.


\section{Versuchsdurchführung}
Zunächst wird der Versuch mit dem Noisegenerator durchgeführt. Anschließend wird der Noise durch den Aufbau mit der Photidiode erzeugt.

\subsection{Künstlich generierter Noise}
\label{sec:durchf_kunst}
Es werden zwei Messreihen zur Überprüfung von Gleichung \eqref{eqn:U_out} aufgenommen. Zu Beginn wird der Noisegenerator auf \emph{OFF} gestellt.
Die Signalspannung ist sinusförmig und geht ungestört in den Lock-In-Verstärker. Dort wird sie mit der Referenzspannung, die vorher den 
Phasenschieber durchläuft, gemischt. Die am Verstärker entstehende Spannung wird gegen die am Phasenschieber eingestellte Phase gemessen.
Anschließend wird der Noisegenerator eingeschaltet und somit die Signalspannung verunreinigt. Nun wird wieder die Spannung gegen die eingestellte 
Phase gemessen.

\subsection{Noise durch Photodiode}
Im ersten Experimentteil wird die Signalspannung an eine LED angeschlossen. Die LED wird mit der angeschlossenen Frequenz der Signalspannung an- und 
ausgeschaltet. Dieses Oszillation wird von dem Photodetektor detektiert und in eine Spannung umgewandelt. Diese Spannung ist aufgrund von dem 
Umgebungslicht und Streuungen von der ursprünglichen Spannung verschieden und somit verunreinigt. Sie geht somit als Signalspannung in den 
Lock-In-Verstärker. Die Referenzspannung wird wie im anderen Experimentaufbau über einen Phasenschieber in den Verstärker eingespeist. 
Der Ausgang des Verstärkers ist an einem Tiefpassfilter zur Integration angeschlossen. Es wird ein bestimmtes Zeitfenster eingestellt, über das 
integriert wird. Dieses wird während des Versuches nicht verändert.\\
\noindent Nun wird der Photodetektor auf der Schiene nach hinten verschoben und nach jeder Verschiebung das integrierte Signal gemessen.  



\section{Messwerte}
Im Folgenden werden die Messwerte beider Experimentaufbauten getrennt voneinander aufgelistet.

\subsection{Künstlich generierter Noise}
Die Messung des nicht verrauschten Signales ergibt folgende Messdaten:

\begin{table}[H]
    \centering
    \sisetup{per-mode=reciprocal}
    \begin{tblr}{colspec = {S[table-format=3] S[table-format=1.2]}, row{1} = {guard, mode=math}}
    \toprule
    \phi \mathbin{/} \unit{\degree} &
    U \mathbin{/} \unit{\volt} \\
    \midrule
    0   &   0.34 \\
    15  &   0.38 \\
    45  &   0.56 \\
    60  &   0.62 \\
    90  &   0.62 \\
    135 &   0.52 \\
    150 &   0.4  \\
    180 &   0.24 \\
    225 &   0.04 \\
    270 &   -0.04\\
    315 &   0.08 \\
    330 &   0.2  \\
    360 &   0.34 \\
    \end{tblr}
    \caption{Phasenverschiebung gegen Spannung ohne Verrauschung.}
    \label{tab:no_noise}
\end{table}

\noindent
Im Anschluss wurde die Messung wiederholt, jedoch wurde dabei die Signalspannung durch einen Noisegenerator geführt. So ergeben sich folgende 
Messdaten:

\begin{table}[H]
    \centering
    \sisetup{per-mode=reciprocal}
    \begin{tblr}{colspec = {S[table-format=3] S[table-format=1.1]}, row{1} = {guard, mode=math}}
    \toprule
    \phi \mathbin{/} \unit{\degree} &
    U \mathbin{/} \unit{\volt} \\
    \midrule
    0   &   0   \\
    15  &   0   \\
    45  &   0.3 \\
    60  &   0.6 \\
    90  &   1   \\
    135 &   1.5 \\
    150 &   1.7 \\
    180 &   1.6 \\
    210 &   1.6 \\
    225 &   1.4 \\
    315 &   0.2 \\
    330 &   0   \\
    360 &   0   \\
    \end{tblr}
    \caption{Phasenverschiebung gegen Spannung mit Verrauschung.}
    \label{tab:mit_noise}
\end{table}


\subsection{Noise durch Photodiode}
Nun werden die Messdaten für den zweiten Experimentaufbau dargelegt. Dabei wird der Abstand $x$ des Photodetektors zur LED gegen die 
integrierte Spannung $U$ aufgetragen:

\begin{table}[H]
    \centering
    \sisetup{per-mode=reciprocal}
    \begin{tblr}{colspec = {S[table-format=2] S[table-format=1.4]}, row{1} = {guard, mode=math}}
    \toprule
    x \mathbin{/} \unit{\centi \meter} &
    U \mathbin{/} \unit{\volt} \\
    \midrule
    5   &   7.9     \\
    6   &   5.95    \\
    7   &   4.8     \\
    8   &   3.9     \\
    9   &   3.25    \\
    10  &   2.6     \\
    11  &   2.4     \\
    12  &   1.95    \\
    13  &   1.6     \\
    14  &   1.45    \\
    15  &   1.25    \\
    16  &   1.05    \\
    17  &   0.95    \\
    18  &   0.825   \\
    19  &   0.75    \\
    20  &   0.7     \\
    25  &   0.4375  \\
    30  &   0.2775  \\
    35  &   0.1875  \\
    40  &   0.125   \\
    45  &   0.1275  \\
    50  &   0.0975  \\
    55  &   0.0775  \\
    60  &   0.065   \\
    65  &   0.0525  \\
    70  &   0.0475  \\
    75  &   0.04    \\
    80  &   0.035   \\
    85  &   0.03    \\
    90  &   0.025   \\
    \end{tblr}
    \caption{Abstand gegen die Spannung am Photodetektor.}
    \label{tab:photonoise}
\end{table}

\noindent
Die Spannung am Photodetektor fällt der Theorie nach schnell ab, weshalb für kleine Abstände die Messabstände geringer sind als für 
große Abstände.

%\end{document}

