\input{../../header.tex}

\begin{document}
\section{Diskussion}
\label{sec:Diskussion}

Die Rauschunterdrückung des Lock-In-Verstärkers kann ebenfalls durch eine Photodetektorschaltung bestätigt. 
Die gemessenen Daten zeigen deutlich den postulierten Trend. Bei niedrigen Abständen $x$ zwischen der LED und 
der Photodiode sind die Intensitätswerte am größten. Bei steigendem Abstand wird die Intensität immer geringer 
und konvergiert gegen Null.
Nichtsdestoweniger lässt sich mittels des \emph{curve-fit} konstatieren, dass die Intensitätswerte bei kleinen 
Abständen über der Kurve liegen und bei großen Abständen tendenziell unterhalb der Kurve liegen. Ferner sind 
kleinere Ausreißer über das komplette Intervall erkennbar. Dafür könnten folgende Ursachen verantwortlich sein:\\
Da die Photodiode ein lichtempfindliches Element darstellt, ist es wahrscheinlich, dass bei der Intensitätsmessung
neben der LED auch externe Lichteinflüsse das Ergebnis beeinträchtigen. So können beispielsweise Personen im Raum 
kurzfristige Schatten auslösen, welche bestimmte Werte beeinflussen. Außerdem können beim Ablesen des analogen 
Zeigers Parallaxenfehler nicht vermieden werden. Die kleinen Ausreißer in der Abbildung lassen sich jedoch 
hauptsächlich durch das Umstellen des Skalierung des Zeigers erklären. Sobald feine Änderungen der Intensität nicht 
mehr abgelesen werden können, muss eine empfindlichere Skalierung gewählt werden. Dabei kann es passieren, dass die 
Werte in der neuen Skalierung besser abgelesen werden können und somit teilweise Werte in der vorherigen Skalierung
leicht übersteigen beziehungsweise nicht den \emph{Fit} forsetzen. 
\end{document}
