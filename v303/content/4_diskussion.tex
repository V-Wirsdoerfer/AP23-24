%\documentclass[
  bibliography=totoc,     % Literatur im Inhaltsverzeichnis
  captions=tableheading,  % Tabellenüberschriften
  titlepage=firstiscover, % Titelseite ist Deckblatt
]{scrartcl}

% Paket float verbessern
\usepackage{scrhack}

% Warnung, falls nochmal kompiliert werden muss
\usepackage[aux]{rerunfilecheck}

% unverzichtbare Mathe-Befehle
\usepackage{amsmath}
% viele Mathe-Symbole
\usepackage{amssymb}
% Erweiterungen für amsmath
\usepackage{mathtools}

% Fonteinstellungen
\usepackage{fontspec}
% Latin Modern Fonts werden automatisch geladen
% Alternativ zum Beispiel:
%\setromanfont{Libertinus Serif}
%\setsansfont{Libertinus Sans}
%\setmonofont{Libertinus Mono}

% Wenn man andere Schriftarten gesetzt hat,
% sollte man das Seiten-Layout neu berechnen lassen
\recalctypearea{}

% deutsche Spracheinstellungen
\usepackage[ngerman]{babel}


\usepackage[
  math-style=ISO,    % ┐
  bold-style=ISO,    % │
  sans-style=italic, % │ ISO-Standard folgen
  nabla=upright,     % │
  partial=upright,   % │
  mathrm=sym,        % ┘
  warnings-off={           % ┐
    mathtools-colon,       % │ unnötige Warnungen ausschalten
    mathtools-overbracket, % │
  },                       % ┘
]{unicode-math}

% traditionelle Fonts für Mathematik
\setmathfont{Latin Modern Math}
% Alternativ zum Beispiel:
%\setmathfont{Libertinus Math}

\setmathfont{XITS Math}[range={scr, bfscr}]
\setmathfont{XITS Math}[range={cal, bfcal}, StylisticSet=1]

% Zahlen und Einheiten
\usepackage[
  locale=DE,                   % deutsche Einstellungen
  separate-uncertainty=true,   % immer Unsicherheit mit \pm
  per-mode=symbol-or-fraction, % / in inline math, fraction in display math
]{siunitx}

% chemische Formeln
\usepackage[
  version=4,
  math-greek=default, % ┐ mit unicode-math zusammenarbeiten
  text-greek=default, % ┘
]{mhchem}

% richtige Anführungszeichen
\usepackage[autostyle]{csquotes}

% schöne Brüche im Text
\usepackage{xfrac}

% Standardplatzierung für Floats einstellen
\usepackage{float}
\floatplacement{figure}{htbp}
\floatplacement{table}{htbp}

% Floats innerhalb einer Section halten
\usepackage[
  section, % Floats innerhalb der Section halten
  below,   % unterhalb der Section aber auf der selben Seite ist ok
]{placeins}

% Seite drehen für breite Tabellen: landscape Umgebung
\usepackage{pdflscape}

% Captions schöner machen.
\usepackage[
  labelfont=bf,        % Tabelle x: Abbildung y: ist jetzt fett
  font=small,          % Schrift etwas kleiner als Dokument
  width=0.9\textwidth, % maximale Breite einer Caption schmaler
]{caption}
% subfigure, subtable, subref
\usepackage{subcaption}

% Grafiken können eingebunden werden
\usepackage{graphicx}

% schöne Tabellen
\usepackage{tabularray}
\UseTblrLibrary{booktabs, siunitx}

% Verbesserungen am Schriftbild
\usepackage{microtype}

% Literaturverzeichnis
\usepackage[
  backend=biber,
]{biblatex}
% Quellendatenbank
\addbibresource{lit.bib}
\addbibresource{programme.bib}

% Hyperlinks im Dokument
\usepackage[
  german,
  unicode,        % Unicode in PDF-Attributen erlauben
  pdfusetitle,    % Titel, Autoren und Datum als PDF-Attribute
  pdfcreator={},  % ┐ PDF-Attribute säubern
  pdfproducer={}, % ┘
]{hyperref}
% erweiterte Bookmarks im PDF
\usepackage{bookmark}

% Trennung von Wörtern mit Strichen
\usepackage[shortcuts]{extdash}

\author{%
  Vincent Wirsdörfer\\%
  \href{mailto:vincent.wirsdoerfer@udo.edu}{authorA@udo.edu}%
  \and%
  Joris Daus\\%
  \href{mailto:joris.daus@udo.edu}{authorB@udo.edu}%
}
\publishers{TU Dortmund – Fakultät Physik}


%\begin{document}
\section{Diskussion}
\label{sec:Diskussion}
Die Form der Kosinusfunktion lässt sich zwar nicht eindeutig auf einen Kosinus zurückführen, jedoch beschreibt 
ein Kosinus mit einer gewissen Phasenverschiebung die zu sehende Funktion sehr gut. Daher kann der theoretische 
Zusammenhang aus Gleichung \eqref{eqn:U_out} als bestätigt angesehen werden.\\
\noindent
Die Rauschunterdrückung des Lock-In-Verstärkers kann ebenfalls durch eine Photodetektorschaltung bestätigt werden.
Die gemessenen Daten zeigen deutlich den postulierten Trend. Bei niedrigen Abständen $x$ zwischen der LED und 
der Photodiode sind die Intensitätswerte am größten. Bei steigendem Abstand wird die Intensität immer geringer 
und konvergiert gegen Null.
Nichtsdestoweniger lässt sich mittels des \emph{curve-fit} konstatieren, dass die Intensitätswerte bei kleinen 
Abständen über der Kurve liegen und bei großen Abständen tendenziell unterhalb der Kurve liegen. Ferner sind 
kleinere Ausreißer über das komplette Intervall erkennbar. Dafür könnten folgende Ursachen verantwortlich sein:\\
Da die Photodiode ein lichtempfindliches Element darstellt, ist es wahrscheinlich, dass bei der Intensitätsmessung
neben der LED auch externe Lichteinflüsse das Ergebnis beeinträchtigen. So können beispielsweise Personen im Raum 
kurzfristige Schatten auslösen, welche bestimmte Werte beeinflussen. Außerdem können beim Ablesen des analogen 
Zeigers Parallaxenfehler nicht vermieden werden. Die kleinen Ausreißer in der Abbildung lassen sich jedoch 
hauptsächlich durch das Umstellen des Skalierung des Zeigers erklären. Sobald feine Änderungen der Intensität nicht 
mehr abgelesen werden können, muss eine empfindlichere Skalierung gewählt werden. Dabei kann es passieren, dass die 
Werte in der neuen Skalierung besser abgelesen werden können und somit teilweise Werte in der vorherigen Skalierung
leicht übersteigen beziehungsweise nicht den \emph{Fit} fortsetzen. 
%\end{document}
