\input{../../header.tex}

\begin{document}
\section{Zielsetzung}
\label{sec:Zielsetzung}

Im folgend protokollierten Versuch wird die Funktionsweise eines \emph{Lock-In-Verstärkers} untersucht.
Das konkrete Ziel besteht darin, die Funktion eines phasenempfindlichen Gleichrichters und eines Lock-In-Verstärkers
zu verifizieren. Ferner soll die Rauschunterdrückung eines Lock-In-Verstärkers mittels einer Photodetektorschaltung
überprüft werden.

\section{Theorie}
\label{sec:Theorie}

Hauptbestandteil eines Lock-In-Verstärkers ist ein phasenempfindlicher Detektor, weswegen der Verstärker zumeist zur 
Verarbeitung stark verrauschter Signale dient. Der allgemeine Aufbau, sowie die Funktionsweise des Lock-In-Verstärker lässt
sich mittels Abbildung \ref{fig:Versuchsaufbau} genauer nachvollziehen.

\begin{figure}[H]
    \centering
    \includegraphics[width=\textwidth]{Versuchsaufbau.png}
    \caption{Aufbau eines Lock-In-Verstärkers}
    \label{fig:Versuchsaufbau}
\end{figure}

\noindent Die Abbildung zeigt, dass das verauschte Nutzsignal $U_\text{Sig}$ zunächst einen Bandpassfilter durchläuft,
welcher sowohl hohe Frequenzen $\left(\omega >> \omega_0\right)$ als auch niedrige Frequenzen $\left(\omega << \omega_0\right)$
\textbf{nicht} passieren lässt. Hierbei bezeichnet $\omega_0$ jene Referenzfrequenz des Referenzsignals $U_\text{Ref}$, mit welcher 
das Netzsignal $U_\text{Sig}$ in dem Mischer multipliziert und somit moduliert wird. Hierbei kann durch den Phasenschieber die Phase  
$\varphi$ des Referenzsignals variiert werden, sodass beide Signale $U_\text{Ref}$ und $U_\text{Sig}$ in eine Phase gebracht werden 
$\left(\increment \varphi = 0\right)$. Im Anschluss wird der Tiefpassfilter unter der Voraussetzung $\tau = RC >> \sfrac{1}{\omega_0}$ 
$\left(\tau: \text{Zeitkonstante}\right)$ als Integrator des Mischsignals $U_\text{Sig} \times U_\text{Ref}$ verwendet. Dabei werden 
Rauschbeiträge weitgehend weggemittelt, weshalb ein proportionaler Zusammenhang zwischen der Eingansspannung und der Ausgangsspannung $U_\text{Out} \propto U_0\cos(\varphi)$ 
existiert. Zusätzlich entscheidet der Tiefpassfilter unmittelbar über die Bandbreite $\increment\nu$ des
Restrauschens. Durch sehr große Zeitkonstanten können beliebig kleine Bandbreiten $\increment\nu = \sfrac{1}{\pi\tau}$ erzeugt werden,
wodurch Güten von $Q = 100000$ erreicht werden.\\
Eine beispielhafte Anwendung des Lock-In-Verstärkers wird in Abbildung \ref{fig:AWD} aufgezeigt. Hierbei wird das Eingangssignals 
mittels einer sinusförmigen Wechselspannung $U_\text{Sig}(t) = U_{0}\sin(\omega t)$ realisert. Dieses Signal wird nun duch eine
Rechteckspannung $U_\text{Ref}$ der gleichen Frequenz moduliert, welche, wie in der Abbildung zu sehen, bei positiven Signalspannungen
von $U_\text{Sig}(t)$ auf $1$ und bei negativen Signalspannungen auf $-1$ steht.

\begin{figure}
    \centering
    \includegraphics[height=7cm]{AWD.png}
    \caption{Beispielhafte Signalverläufe}
    \label{fig:AWD}
\end{figure}

\noindent Näherungsweise kann der zeitliche Verlauf der Rechteckspannung durch eine Fourierreihe der ungeraden harmonischen
Grundfrequenzen dargstellt werden:

\begin{equation*}
    U_\text{Ref} = \frac{4}{\pi}\left(\sin(\omega t) + \frac{1}{3}\sin(3\omega t) + \frac{1}{5}\sin(5\omega t) + \dotsc\right)
\end{equation*}

\noindent Das Produkt aus Signal- und Referenzspannung beinhaltet nun die geraden Oberwellen der Grundfrequenz $\omega$ und lässt
sich somit schreiben als

\begin{equation*}
    U_\text{Sig} \times U_\text{Ref} = \frac{2}{\pi}U_0\left(1 - \frac{2}{3}\cos(2\omega t) - \frac{2}{15}\cos(4\omega t) - \frac{2}{35}\cos(6\omega t) + \dotsc\right).
\end{equation*}

\noindent Durch eine geeignete Wahl des Tiefpassfilters werden die Oberwellen unterdrückt, sodass die Ausgangsspannung eine zur Eingansspannung
proportionale Spannung darstellt:

\begin{equation}
    U_\text{Out} = \frac{2}{\pi}U_0\cos(\varphi)
    \label{eqn:U_out}
\end{equation}

\noindent Sind beide Signale in Phase $\left(\increment \varphi = 0\right)$, so wird die Ausgangsspannung maximal:

\begin{equation*}
    \hat{U}_\text{Out} = \frac{2}{\pi}U_0
\end{equation*}

\section{Vorbereitung}

\end{document}