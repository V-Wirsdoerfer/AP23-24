%\input{../../header.tex}

%\begin{document}

\section{Zielsetzung}
\label{sec:Zielsetzung}

Das Ziel des im folgend protokollierten Versuchs besteht darin, die Strom- Spannungskennlinie einer Photozelle aufzunehmen. Zusätzlich soll das \emph{Plancksche
Wirkungsquantum}, eine fundamentale Naturkonstante, über die Gegenfeldmethode ermittelt werden.

\section{Theorie}
\label{sec:Theorie}

Das grundlegende Prinzip des Photoeffekts ist die Emission bzw. das Herauslösen von Elektronen aus einem Material aufgrund von Lichteinwirkung. Über den äußeren 
Lichtelektrischen Effekt soll im Folgenden die Plancksche Konstante $h$ bestimmt werden. Diese beschreibt den wesentlichen Zusammenhang zwischen Strahlungsenergie
$E_\gamma$ und der Frequenz $f$ über die Gleichung 

\begin{equation*}
    E_\gamma = hf.
\end{equation*}

\noindent Das Produkt $hf$ ist dabei Ausdruck für die Interpretation des Photons als gequanteltes Energiepaket. Dies impliziert bereits, dass diverse Phänomene des 
Photoeffekts nicht vollständig durch die klassische Wellentheorie des Lichts erklärt werden können. Einerseits kann die Beobachtung, dass die abgestrahlte Lichtintensität 
proportional zur Anzahl der von der Oberfläche herausgelösten Elektronen ist, gut von der Theorie von Licht als elektromagnetische Welle erklärt werden. 
Andererseits ist der instantane Einsatz eines Photostroms mit der Bestrahlung des Materials, die materialabhängige Grenzfrequenz und die Frequenzabhängigkeit der 
kinetischen Energie nicht mit dem Formalismus der Wellentheorie vereinbar.\\

\noindent Beobachtet werden hier ausschließlich Metalle, welche im Kern aus positiven Ionenrümpfen bestehen, die von frei beweglichen Elektronen umgeben sind. Energetisch
beschrieben werden die Elektronen durch eine \emph{Fermi-Dirac-Verteilung}. Diese gibt an, mit welcher Wahrscheinlichkeit sich ein Zustand mit Energie $E$ im thermischen 
Gleichgewicht befindet. Visualisiert wird diese Verteilung in der untenstehenden Abbildung.

\begin{figure}
    \centering
    \includegraphics[height=5cm]{content/FDV.png}
    \caption{Fermi-Dirac-Verteilung\cite{Versuchsanleitung_v500}.}
    \label{fig:FDV}
\end{figure}

\noindent Bei einer Temperatur von $T = \qty{0}{\kelvin}$ sind somit alle Zustände unterhalb der Fermi-Energie $E_\text{F}$ besetzt. Sowohl über- als auch unterhalb 
von $E_\text{F}$ befinden sich die Zustände bei einer Systempräparation von $T > \qty{0}{\kelvin}$. Daher muss zusätzliche Energie aufgewendet werden, um ein Elektron 
aus dem Metall zu lösen. Die dafür benötigte Energie wird als \emph{Austrittsarbeit} bezeichnet. Die hier betrachtete Energiezufuhr erfolgt mittels elektromagnetischer 
Strahlung mit Energie $E_\gamma = hf$. Ist diese Energie gleich der Austrittsarbeit, so werden Elektronen herausgelöst. Bei noch größeren Energiemengen, wird die
überschüssige Energie oberhalb der Austrittsarbeit $\Phi$ in Form von kinetischer Energie an das Elektron abgegeben:

\begin{equation}
    E_\gamma = \Phi + \frac{1}{2}mv²
\label{eqn:Energiebilanz1}
\end{equation}

\noindent Um einen konkreteren Ausdruck für die kinetische Energie der Elektronen zu erhalten, wird die Gegenfeldmethode angewendet. Diese funktioniert folgenderweise.
Gegenüber der Photokathode befindet sich eine Anode, welche durch Anlegen einer Spannung ein elektrisches Feld erzeugt. Dieses Feld kann je nach Vorzeichen der Spannung 
die aus der Kathode emittierten Elektronen beschleunigen oder abbremsen. Im Falle einer Abbremsung kann die Spannung so angepasst werde, sodass die Elektronen gerade nicht 
die Anode erreichen. Die dafür benötigte Spannung wird \emph{Grenzspannung} $U_\text{G}$ tituliert. Aus der Energieäquivalenz $E_\text{kin} =  E_\text{el}$ kann somit die 
Energiebilanz aus \eqref{eqn:Energiebilanz1} umgeschrieben werden als 

\begin{equation}
    E_\gamma = hf = \Phi + eU_\text{G},
\label{eqn:Energiebilanz2}
\end{equation}

\noindent wobei $\Phi = hf_\text{G}$ die Austrittsarbeit mit $f_\text{G}$ als Grenzfrequenz bezeichnet. Eine Darstellung dieser Idee liefert die folgende Abbildung.

\begin{figure}
    \centering
    \includegraphics[height=5cm]{content/Gegenfeldmethode.png}
    \caption{Gegenfeldmethode zur Ermittlung des Planck'schen Wirkungsquantum\cite{Versuchsanleitung_v500}.}
    \label{fig:Gegenfeldmethode}
\end{figure}

%\section{Vorbereitung}

%\section{Fehlerrechnung}
%\end{document}