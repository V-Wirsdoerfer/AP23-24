\documentclass[
  bibliography=totoc,     % Literatur im Inhaltsverzeichnis
  captions=tableheading,  % Tabellenüberschriften
  titlepage=firstiscover, % Titelseite ist Deckblatt
]{scrartcl}

% Paket float verbessern
\usepackage{scrhack}

% Warnung, falls nochmal kompiliert werden muss
\usepackage[aux]{rerunfilecheck}

% unverzichtbare Mathe-Befehle
\usepackage{amsmath}
% viele Mathe-Symbole
\usepackage{amssymb}
% Erweiterungen für amsmath
\usepackage{mathtools}

% Fonteinstellungen
\usepackage{fontspec}
% Latin Modern Fonts werden automatisch geladen
% Alternativ zum Beispiel:
%\setromanfont{Libertinus Serif}
%\setsansfont{Libertinus Sans}
%\setmonofont{Libertinus Mono}

% Wenn man andere Schriftarten gesetzt hat,
% sollte man das Seiten-Layout neu berechnen lassen
\recalctypearea{}

% deutsche Spracheinstellungen
\usepackage[ngerman]{babel}


\usepackage[
  math-style=ISO,    % ┐
  bold-style=ISO,    % │
  sans-style=italic, % │ ISO-Standard folgen
  nabla=upright,     % │
  partial=upright,   % │
  mathrm=sym,        % ┘
  warnings-off={           % ┐
    mathtools-colon,       % │ unnötige Warnungen ausschalten
    mathtools-overbracket, % │
  },                       % ┘
]{unicode-math}

% traditionelle Fonts für Mathematik
\setmathfont{Latin Modern Math}
% Alternativ zum Beispiel:
%\setmathfont{Libertinus Math}

\setmathfont{XITS Math}[range={scr, bfscr}]
\setmathfont{XITS Math}[range={cal, bfcal}, StylisticSet=1]

% Zahlen und Einheiten
\usepackage[
  locale=DE,                   % deutsche Einstellungen
  separate-uncertainty=true,   % immer Unsicherheit mit \pm
  per-mode=symbol-or-fraction, % / in inline math, fraction in display math
]{siunitx}

% chemische Formeln
\usepackage[
  version=4,
  math-greek=default, % ┐ mit unicode-math zusammenarbeiten
  text-greek=default, % ┘
]{mhchem}

% richtige Anführungszeichen
\usepackage[autostyle]{csquotes}

% schöne Brüche im Text
\usepackage{xfrac}

% Standardplatzierung für Floats einstellen
\usepackage{float}
\floatplacement{figure}{htbp}
\floatplacement{table}{htbp}

% Floats innerhalb einer Section halten
\usepackage[
  section, % Floats innerhalb der Section halten
  below,   % unterhalb der Section aber auf der selben Seite ist ok
]{placeins}

% Seite drehen für breite Tabellen: landscape Umgebung
\usepackage{pdflscape}

% Captions schöner machen.
\usepackage[
  labelfont=bf,        % Tabelle x: Abbildung y: ist jetzt fett
  font=small,          % Schrift etwas kleiner als Dokument
  width=0.9\textwidth, % maximale Breite einer Caption schmaler
]{caption}
% subfigure, subtable, subref
\usepackage{subcaption}

% Grafiken können eingebunden werden
\usepackage{graphicx}

% schöne Tabellen
\usepackage{tabularray}
\UseTblrLibrary{booktabs, siunitx}

% Verbesserungen am Schriftbild
\usepackage{microtype}

% Literaturverzeichnis
\usepackage[
  backend=biber,
]{biblatex}
% Quellendatenbank
\addbibresource{lit.bib}
\addbibresource{programme.bib}

% Hyperlinks im Dokument
\usepackage[
  german,
  unicode,        % Unicode in PDF-Attributen erlauben
  pdfusetitle,    % Titel, Autoren und Datum als PDF-Attribute
  pdfcreator={},  % ┐ PDF-Attribute säubern
  pdfproducer={}, % ┘
]{hyperref}
% erweiterte Bookmarks im PDF
\usepackage{bookmark}

% Trennung von Wörtern mit Strichen
\usepackage[shortcuts]{extdash}

\author{%
  Vincent Wirsdörfer\\%
  \href{mailto:vincent.wirsdoerfer@udo.edu}{authorA@udo.edu}%
  \and%
  Joris Daus\\%
  \href{mailto:joris.daus@udo.edu}{authorB@udo.edu}%
}
\publishers{TU Dortmund – Fakultät Physik}


\begin{document}
\section{Auswertung}
\label{sec:Auswertung}

%%%hier steht wsl crap
Die Photozelle erzeugt wenn man  sie mit licht beschießt, eine von der Intensität abhängigen Strom. Diese Abhängigkeit ist 
eine bestimmte Kurve. Diese soll untersucht werden.

Für verschiedene Farben werden im Folgenden die jeweiligen Grenzspannungen $U_G$ bestimmt. Dies geschieht, indem über Gerdengleichung 
der Ausgleichsgerade die Nullstelle über 

\begin{equation}
    U_G=\frac{-b}{m}
    \label{eqn:Nullstelle}
\end{equation}

\noindent bestimmt wird. Dabei ist $b$ der y-Achsenabschnitt und $m$ die Steigung der Gerade.

%%% hier steht wsl crap
Zunächst wird der Spalt der Blende so weit geöffnet, dass die maximale Lichtintensität auf die Photozelle fällt. Die Gegenspannung 
wird auf die Grenzspannung eingestellt, sodass kein Strom fließt. Die Spannung wird allmälich erhöht und dabei werden die gemssenen 
Ströme notiert. Diese Daten aus Tabelle \ref{tab:PraeziseBlauGlobal} werden nun zur Sichtung in einem Diagramm aufgetragen. Da für 
kleine Spannungen eine höhere Datendichte existiert, und der Verlauf für kleine Spannungen von Bedeutung ist, wird dieser explizit 
vergrößert. 
%%%

\begin{figure}[H]
    \centering
    \includegraphics[width=0.9\textwidth]{../build/blau_voll.pdf}
    \caption{Globaler Verlauf der blauen Spannung bei voller Intensität mit feiner Messung.}
\end{figure}

\noindent Zu sehen ist, der zunächst recht lineare Anstieg der Intensität mit der Spannung. Dieser ist in etwa so lang, wie er im 
vergrößerten Schaubild dargestellt wird, also erstreckt sich über die ersten 20 Werte. Anschließend steigt der Verlauf quadratisch 
an und geht anschließend asymptotisch gegen das Maximum von etwa \qty{21}{\nano\ampere}. Um die Grenzspannung systematisch zu 
ermitteln werden durch die ersten 20 Werte eine Ausgleichsgerade gelegt und der Nullpunkt dieser Gerade ermittelt. 

\begin{figure}[H]
    \centering
    \includegraphics[width=0.9\textwidth]{../build/blau_voll_fit.pdf}
    \caption{Globaler Verlauf der blauen Spannung bei voller Intensität mit feiner Messung und fit.}
\end{figure}

\noindent Zu sehen ist hier die Notwendigkeit der vergrößerten Abbildung, da der gefittete Bereich so klein ist, dass er im 
Ursprünglichen Diagramm nicht mehr zu erkennen ist. Die gefittete Gerade besitzt die Funktion $\num{3.59\pm0.1 e-10} x + \num{4.07\pm0.1e-10}$. 
Somit ergibt sich eine Grenzspannung bei dem blauen Licht mit einer Wellenlänge von \qty{435.8}{\nano\meter} von 

\begin{align}
    U_{\text{G, blau}} = \qty{-1.13\pm0.04}{Volt}
\end{align}

\noindent Nun wird die Blende so verringert, dass nur die Hälfte der Maximalen Intensität hindurchkommt. Es werden nun die Daten 
aus \ref{tab:BlauGrobGlobal} aufgetragen zur Sichtung.

\begin{figure}[H]
    \centering
    \includegraphics[width=0.9\textwidth]{../build/blau_halb.pdf}
    \caption{Globaler Verlauf der blauen Spannung bei halber Intensität mit grober Messung.}
\end{figure}

\noindent Auf den ersten Blick sehen die Verläufe sehr ähnlich ist. Um dies genauer beurteilen zu können, werden beide Daten nun 
im selben Diagramm aufgetragen.

\begin{figure}[H]
    \centering
    \includegraphics[width=0.9\textwidth]{../build/blau_voll_halb.pdf}
    \caption{Globaler Verlauf der blauen Spannung bei voller und halber Intensität.}
\end{figure}

\noindent Hier ist nun eindeutig zu erkennen, dass die Änderung der Intensität keinen Einfluss auf den Verlauf des Photostroms hat.\\
\noindent Im folgenden werden nun die Messdaten für das violette Licht \qty{404.7e-9}{\nano\meter}, das grüne Licht \qty{546e-9}{\nano\meter} 
und das gelbe Licht \qty{577e-9}{\nano\meter} aufgetragen. 


\begin{figure}[H]
    \centering
    \includegraphics[width=0.9\textwidth]{../build/bunt.pdf}
    \caption{Photostrom des violetten, grünen und gelben Lichts bei der jeweiligen Gegenspannung.}
\end{figure}

\noindent Die Messdaten sind weitesgehend linear angeordnet. Bei der folgenden Auswertung werden jedoch die letzten beiden Werte des 
gelben Lichts nicht mit beachtet, da diese nicht mehr einem linearen Verlauf entsprechen. Nun werden wie schon bei dem blauen Licht 
Ausgleichsgeraden durch die Messwerte gelegt.

\begin{figure}[H]
    \centering
    \includegraphics[width=0.9\textwidth]{../build/bunt_fit.pdf}
    \caption{Photostrom des violetten, grünen und gelben Lichts bei der jeweiligen Gegenspannung mit fit.}
\end{figure}

\noindent So entstehen die folgenden Geradengleichungen des Ausgleichsgeraden:

\begin{align*}
    I_\text{violett} &= \num {3.59\pm0.10e-10} x + \num{4.07\pm0.10e-10}\\
    I_\text{grün}    &= \num {3.29\pm0.14e-10} x + \num{1.86\pm0.07e-10}\\
    I_\text{gelb}    &= \num {1.00\pm0.00e-10} x + \num{4.50\pm0.00e-10}\\
\end{align*}

\noindent Hier ist zu erwähnen, dass die Daten des gelben Lichts eine perfekte Gerade bilden, weshalb keine Abweichungen entstehen. 
Alle Messdaten des gelben Lichts liegen also perfekt auf einer Gerade. \\
\noindent Nun werden die Grenzspannungen über \autoref{eqn:Nullstelle} bestimmt.

\begin{align}
    U_\text{G, violett} &= \qty{-1.32 \pm 0.04}{\volt}\\
    U_\text{G, grün} &= \qty{-0.565 \pm 0.033}{\volt}\\
    U_\text{G, gelb} &= \qty{-0.45 \pm 0.00}{\volt}
\end{align}

\noindent Auch hier ist wieder kein Fehler für das Gelbe Licht anzugeben, da diese Werte eine direkte Konsequenz der verherigen Werte sind.\\
\noindent Es sind nun alle Grenzspannungen bekannt, weshalb mit der Berechnung der Planckkonstante begonnen werden kann. Zunächst werden die 
Grenzspannungen $\cdot (-1)$ gerechnet, damit diese positiv werden und die Planckkonstante ebenfalls positiv ist. Dies darf getan werden, da 
das Vorzeichen nur eine Konvention im Schaltplan ist und diese ohne Probleme geändert werden kann. Außerdem muss die Frequenz über 
$f=\frac{c}{\lambda}$ berechnet werden. So wird nun die Frequenz gegen die Grenzspannung aufgetragen. Dabei muss aus Dimensionstechnischen 
Gründen gegen eV aufgetragen werden. Dies verändert allerdings die Werte nicht, da hier e$=1$ gesetzt wird. 

\begin{figure}[H]
    \centering
    \includegraphics[width=0.9\textwidth]{../build/Planck.pdf}
    \caption{Berechnete Grenzspannungen gegen die jeweilige Frequenz aufgetragen.}
\end{figure}

%evtl noch Grenzfrequenz
\noindent Aus der Ausgleichsgerade ergibt sich das Plancksche Wirkungsquantum als Steigung. Der y-Achsenabschnitt ist equivalent zur 
Austrittsarbeit des Elektrons aus dem Material \ref{eqn:Energiebilanz2}. So lassen sich durch die Ausgleichsgerade die folgenden Werte 
bestimmen.

\begin{align}
    h_\text{berechnet} &=\qty{3.98\pm0.05 e-15}{e \volt \second} \\
    E_{0 \text{abgelesen}} &= \qty{1.620 \pm 0.028}{e \volt}
\end{align}

\noindent Das Planckschewirkungsquantum wird nun auf eine weniger systematische Methode bestimmt. Dazu werden die Grenzspannungen nicht 
berechnet, sondern aus den Messdaten abgelesen. Dies geschieht, indem der größte Werte der Gegenspannung genommen wird, bei dem noch kein 
Photostrom fließt. Dies ist dann die Grenzspannung für eine Farbe. Jetzt wird wie vorhin auch der Messwert für die Grenzspannung gegen die 
dazugehörige Frequenz aufgetragen. Und eine Ausgleichsgerade durch diese Werte gelegt.

\begin{figure}
    \centering
    \includegraphics[width=0.9\textwidth]{../build/Planck_read.pdf}
    \caption{Abgelesene Grenzspannungen gegen die jeweilige Frequenz aufgetragen.}
\end{figure}

\noindent Nun kann wie vorher bereits erwähnt der Nullpunkt der Geraden bestimmt werden, um die Grenzfrequenz zu bestimmen. Die Steigung 
entspricht erneut dem Planckschen Wirkungsquantum. So ergeben sich die folgenden Werte

\begin{align}
    h_\text{abgelesen} &= \qty{4.00 \pm 0.07}{e \volt \second}\\
    E_{0 \text{abgelesen}} &= \qty{1.63 \pm 0.04}{e \volt}
\end{align}

















\end{document}
