\documentclass[
  bibliography=totoc,     % Literatur im Inhaltsverzeichnis
  captions=tableheading,  % Tabellenüberschriften
  titlepage=firstiscover, % Titelseite ist Deckblatt
]{scrartcl}

% Paket float verbessern
\usepackage{scrhack}

% Warnung, falls nochmal kompiliert werden muss
\usepackage[aux]{rerunfilecheck}

% unverzichtbare Mathe-Befehle
\usepackage{amsmath}
% viele Mathe-Symbole
\usepackage{amssymb}
% Erweiterungen für amsmath
\usepackage{mathtools}

% Fonteinstellungen
\usepackage{fontspec}
% Latin Modern Fonts werden automatisch geladen
% Alternativ zum Beispiel:
%\setromanfont{Libertinus Serif}
%\setsansfont{Libertinus Sans}
%\setmonofont{Libertinus Mono}

% Wenn man andere Schriftarten gesetzt hat,
% sollte man das Seiten-Layout neu berechnen lassen
\recalctypearea{}

% deutsche Spracheinstellungen
\usepackage[ngerman]{babel}


\usepackage[
  math-style=ISO,    % ┐
  bold-style=ISO,    % │
  sans-style=italic, % │ ISO-Standard folgen
  nabla=upright,     % │
  partial=upright,   % │
  mathrm=sym,        % ┘
  warnings-off={           % ┐
    mathtools-colon,       % │ unnötige Warnungen ausschalten
    mathtools-overbracket, % │
  },                       % ┘
]{unicode-math}

% traditionelle Fonts für Mathematik
\setmathfont{Latin Modern Math}
% Alternativ zum Beispiel:
%\setmathfont{Libertinus Math}

\setmathfont{XITS Math}[range={scr, bfscr}]
\setmathfont{XITS Math}[range={cal, bfcal}, StylisticSet=1]

% Zahlen und Einheiten
\usepackage[
  locale=DE,                   % deutsche Einstellungen
  separate-uncertainty=true,   % immer Unsicherheit mit \pm
  per-mode=symbol-or-fraction, % / in inline math, fraction in display math
]{siunitx}

% chemische Formeln
\usepackage[
  version=4,
  math-greek=default, % ┐ mit unicode-math zusammenarbeiten
  text-greek=default, % ┘
]{mhchem}

% richtige Anführungszeichen
\usepackage[autostyle]{csquotes}

% schöne Brüche im Text
\usepackage{xfrac}

% Standardplatzierung für Floats einstellen
\usepackage{float}
\floatplacement{figure}{htbp}
\floatplacement{table}{htbp}

% Floats innerhalb einer Section halten
\usepackage[
  section, % Floats innerhalb der Section halten
  below,   % unterhalb der Section aber auf der selben Seite ist ok
]{placeins}

% Seite drehen für breite Tabellen: landscape Umgebung
\usepackage{pdflscape}

% Captions schöner machen.
\usepackage[
  labelfont=bf,        % Tabelle x: Abbildung y: ist jetzt fett
  font=small,          % Schrift etwas kleiner als Dokument
  width=0.9\textwidth, % maximale Breite einer Caption schmaler
]{caption}
% subfigure, subtable, subref
\usepackage{subcaption}

% Grafiken können eingebunden werden
\usepackage{graphicx}

% schöne Tabellen
\usepackage{tabularray}
\UseTblrLibrary{booktabs, siunitx}

% Verbesserungen am Schriftbild
\usepackage{microtype}

% Literaturverzeichnis
\usepackage[
  backend=biber,
]{biblatex}
% Quellendatenbank
\addbibresource{lit.bib}
\addbibresource{programme.bib}

% Hyperlinks im Dokument
\usepackage[
  german,
  unicode,        % Unicode in PDF-Attributen erlauben
  pdfusetitle,    % Titel, Autoren und Datum als PDF-Attribute
  pdfcreator={},  % ┐ PDF-Attribute säubern
  pdfproducer={}, % ┘
]{hyperref}
% erweiterte Bookmarks im PDF
\usepackage{bookmark}

% Trennung von Wörtern mit Strichen
\usepackage[shortcuts]{extdash}

\author{%
  Vincent Wirsdörfer\\%
  \href{mailto:vincent.wirsdoerfer@udo.edu}{authorA@udo.edu}%
  \and%
  Joris Daus\\%
  \href{mailto:joris.daus@udo.edu}{authorB@udo.edu}%
}
\publishers{TU Dortmund – Fakultät Physik}


\begin{document}
\section{Auswertung}
\label{sec:Auswertung}

\noindent Zunächst werden die Messwerte gesichtet. Wenn Anomalien in den Zeiten erkennen zu sind, die auf einen Ladungswechsel der 
hindeuten, so werden diese Messwerte nicht mit in die Auswertung genommen.
Anschließend werden die plausiblen Messwerte für die Berechnung der Ladung herangezogen.

\subsection{Messwerte}
\noindent Für ein Tröpfchen wurden drei bis fünf Messwerte aufgenommen, sowie der Thermistorwiderstand. Es werden nun von allen Tröpfchen die 
Messergebnisse aufgelistet. Der Thermistorwiderstand wird außerdem in \unit{\celsius} umgerechnet und ebenfalls angegeben.


%Tropfen 1 bis 3
\begin{table}[H]
    \centering
    \sisetup{table-format=1.2}
    \begin{tblr}{
        colspec={S S S S S S S S S},
        row{1}={guard, mode=math}, row{2}={guard, mode=math}, row{3}={guard, mode=math},
        }
        \toprule
        \SetCell[c=3]{c} \text{Tröpfchen 1}              & & & \SetCell[c=3]{c} \text{Tröpfchen 2}         & & & \SetCell[c=3]{c} \text{Tröpfchen 3}                            &   \\
        \cmidrule[lr]{1-3}                                      \cmidrule[lr]{4-6}                               \cmidrule[lr]{7-9}                                                 \\
        t_\text{auf} \mathbin{/} \unit{\second} & t_\text{ab} \mathbin{/} \unit{\second} & T & t_\text{auf} \mathbin{/} \unit{\second} & t_\text{ab} \mathbin{/} \unit{\second} & T & 
        t_\text{auf} \mathbin{/} \unit{\second} & t_\text{ab} \mathbin{/} \unit{\second} & T \\
        \midrule
        5.95 & 3.12 & \qty{23}{\celsius}    & 4.26 & 10.80 & /                      & 14.80 & 6.65 & /                        \\
        6.46 & 3.75 & \qty{2.1}{\mega \ohm} & 4.88 & 10.01 & \qty{2.07}{\mega \ohm} & 17.21 & 3.01 & \qty{2.038}{\mega \ohm}   \\                                        
        6.37 & 3.31 & /                     & 4.75 & 12.87 & /                      & 17.65 & 4.81 & /                        \\                 
        /    & /    & /                     & 6.75 & /     & /                      & /     & /    & /                        \\         
    \end{tblr}
    \caption{Temperatur der Kammer, Fall- und Steigzeiten von Tröpfchen 1 bis 3}
\end{table}


%Tropfen 4 bis 6
\begin{table}[H]
    \centering
    \sisetup{table-format=1.2}
    \begin{tblr}{
        colspec={S S S S S S S S S},
        row{1}={guard, mode=math}, row{2}={guard, mode=math}, row{3}={guard, mode=math},
        }
        \toprule
        \SetCell[c=3]{c} \text{Tröpfchen 4}              & & & \SetCell[c=3]{c} \text{Tröpfchen 5}         & & & \SetCell[c=3]{c} \text{Tröpfchen 6}                            &   \\
        \cmidrule[lr]{1-3}                                      \cmidrule[lr]{4-6}                               \cmidrule[lr]{7-9}                                                 \\
        t_\text{auf} \mathbin{/} \unit{\second} & t_\text{ab} \mathbin{/} \unit{\second} & T & t_\text{auf} \mathbin{/} \unit{\second} & t_\text{ab} \mathbin{/} \unit{\second} & T & 
        t_\text{auf} \mathbin{/} \unit{\second} & t_\text{ab} \mathbin{/} \unit{\second} & T \\
        \midrule
        5.26 & 2.44 & \qty{25}{\celsius}      & 9.75 & 3.84 & /                       & 11.53 & 9.29 & \qty{25}{\celsius}       \\             
        5.95 & 2.63 & \qty{2.017}{\mega \ohm} & 6.69 & 3.76 & \qty{2.001}{\mega \ohm} & 11.19 & 8.57 & \qty{2.0}{\mega \ohm}    \\                                        
        5.81 & 2.55 & /                       & 6.68 & 3.41 & /                       & 11.94 & 9.37 & /                        \\                 
        /    & /    & /                       & /    & /    & /                       & 8.762 & /    & /                        \\         
    \end{tblr}
    \caption{Temperatur der Kammer, Fall- und Steigzeiten von Tröpfchen 4 bis 6}
\end{table}


%Tröpfchen 7 bis 9
\begin{table}[H]
    \centering
    \sisetup{table-format=1.2}
    \begin{tblr}{
        colspec={S S S S S S S S S},
        row{1}={guard, mode=math}, row{2}={guard, mode=math}, row{3}={guard, mode=math},
        }
        \toprule
        \SetCell[c=3]{c} \text{Tröpfchen 7}              & & & \SetCell[c=3]{c} \text{Tröpfchen 8}         & & & \SetCell[c=3]{c} \text{Tröpfchen 9}                            &   \\
        \cmidrule[lr]{1-3}                                      \cmidrule[lr]{4-6}                               \cmidrule[lr]{7-9}                                                 \\
        t_\text{auf} \mathbin{/} \unit{\second} & t_\text{ab} \mathbin{/} \unit{\second} & T & t_\text{auf} \mathbin{/} \unit{\second} & t_\text{ab} \mathbin{/} \unit{\second} & T & 
        t_\text{auf} \mathbin{/} \unit{\second} & t_\text{ab} \mathbin{/} \unit{\second} & T \\
        \midrule
        4.94 & 5.04 & \qty{25}{\celsius}      & 2.67 & 3.33 & \qty{26}{\celsius}      & 7.44 &8.25/ & \qty{26}{\celsius}        \\    
        4.58 & 4.60 & \qty{1.991}{\mega \ohm} & 3.11 & 3.50 & \qty{1,991}{\mega \ohm} & 9.22 &8.45/ & \qty{1,93}{\mega \ohm}    \\    
        4.82 & 4.55 & /                       & 3.10 & 3.06 & /                       & 8.40 &7.80/ & /                         \\    
        4.50 & 4.58 & /                       & 2.84 & 3.56 & /                       & 8.27 &7.90/ & /                         \\    
        4.10 & 4.51 & /                       & 2.78 & 3.02 & /                       & /    & /    & /                         \\     
    \end{tblr}
    \caption{Temperatur der Kammer, Fall- und Steigzeiten von Tröpfchen 7 bis 9}
\end{table}


%Tröpfchen 10 bis 12
\begin{table}[H]
    \centering
    \sisetup{table-format=1.2}
    \begin{tblr}{
        colspec={S S S S S S S S S},
        row{1}={guard, mode=math}, row{2}={guard, mode=math}, row{3}={guard, mode=math},
        }
        \toprule
        \SetCell[c=3]{c} \text{Tröpfchen 10}              & & & \SetCell[c=3]{c} \text{Tröpfchen 11}         & & & \SetCell[c=3]{c} \text{Tröpfchen 12}                            &   \\
        \cmidrule[lr]{1-3}                                      \cmidrule[lr]{4-6}                                 \cmidrule[lr]{7-9}                                                 \\
        t_\text{auf} \mathbin{/} \unit{\second} & t_\text{ab} \mathbin{/} \unit{\second} & T & t_\text{auf} \mathbin{/} \unit{\second} & t_\text{ab} \mathbin{/} \unit{\second} & T & 
        t_\text{auf} \mathbin{/} \unit{\second} & t_\text{ab} \mathbin{/} \unit{\second} & T \\
        \midrule
        3.69 & 3.35 & \qty{26}{\celsius}     & 4.99 & 5.12 & \qty{27}{\celsius}     & 5.85 & 6.73 & \qty{27}{\celsius}      \\    
        2.99 & 3.47 & \qty{1.93}{\mega \ohm} & 5.41 & 5.95 & \qty{1.92}{\mega \ohm} & 6.21 & 6.79 & \qty{1.92}{\mega \ohm}  \\    
        3.25 & 3.72 & /                      & 5.35 & 5.47 & /                      & 6.73 & 6.17 & /                       \\    
        2.97 & 3.42 & /                      & 5.34 & 5.58 & /                      & 5.85 & 6.24 & /                       \\    
        3.29 & 3.85 & /                      & 5.55 & 5.53 & /                      & 5.38 & 7.28 & /                       \\     
    \end{tblr}
    \caption{Temperatur der Kammer, Fall- und Steigzeiten von Tröpfchen 10 bis 12}
\end{table}


%Tröpfchen 13 bis 15
\begin{table}[H]
    \centering
    \sisetup{table-format=1.2}
    \begin{tblr}{
        colspec={S S S S S S S S S},
        row{1}={guard, mode=math}, row{2}={guard, mode=math}, row{3}={guard, mode=math},
        }
        \toprule
        \SetCell[c=3]{c} \text{Tröpfchen 13}              & & & \SetCell[c=3]{c} \text{Tröpfchen 14}         & & & \SetCell[c=3]{c} \text{Tröpfchen 15}                            &   \\
        \cmidrule[lr]{1-3}                                      \cmidrule[lr]{4-6}                               \cmidrule[lr]{7-9}                                                 \\
        t_\text{auf} \mathbin{/} \unit{\second} & t_\text{ab} \mathbin{/} \unit{\second} & T & t_\text{auf} \mathbin{/} \unit{\second} & t_\text{ab} \mathbin{/} \unit{\second} & T & 
        t_\text{auf} \mathbin{/} \unit{\second} & t_\text{ab} \mathbin{/} \unit{\second} & T \\
        \midrule
        5.30 & 5.05 & \qty{27}{\celsius}     & 2.50 & 2.99 & \qty{27}{\celsius}    & 9.12 & 8.49 & \qty{27}{\celsius}       \\    
        4.48 & 5.09 & \qty{1.91}{\mega \ohm} & 2.90 & 2.68 & \qty{1.9}{\mega \ohm} & 4.62 & 3.84 & \qty{1.9}{\mega \ohm}    \\    
        4.83 & 4.78 & /                      & 3.72 & 3.04 & /                     & 4.30 & 4.00 & /                        \\    
        5.41 & 4.75 & /                      & 3.06 & 2.94 & /                     & 4.02 & 4.46 & /                        \\    
        5.13 & 5.72 & /                      & 2.89 & 2.96 & /                     & 3.69 & 4.36 & /                        \\     
    \end{tblr}
    \caption{Temperatur der Kammer, Fall- und Steigzeiten von Tröpfchen 13 bis 15}
\end{table}

\noindent Die Tröpfchen 2, 3 und 5 werden nicht mit in die Auswertung genommen, da diese weniger als drei plausible Werte besitzen. 
Die dritte Aufstiegszeit von Tröpfchen 6 wird ebenfalls nicht mit in die Auswertung genommen, da sie diese deutlich von den anderen drei Aufstiegszeiten unterscheidet. 
Gleiches gilt für die erste Aufstiegs- und Abstiegszeit von Tröpfchen 15. \\
\noindent Das 8. Tröpfchen wurde über eine Strecke von \qty{1.5}{\milli\meter} gemessen. Alle anderen Tröpfchen wurden über eine Strecke von \qty{0.5}{\milli\meter} gemessen. 

\subsection{Berechnung der Ladung}
Der erste Schritt der Auswertung ist die Zeitmittelung der Tröpfchen. Dazu wird zu jedem Tröpfchen der Mittelwert inklusive Abweichung berechnet. Mit diesem Mittelwert und der 
zu den Tröpfchen gehörenden Strecke kann die Geschwindigkeit des Auf- und Absteigens bestimmt werden.
Die zu dem jeweiligen Tröpfchen gehörende Viskosität wird Mithilfe der Versuchsanleitung über die Temperatur bestimmt. 
Es sind nun alle Werte bekannt, um mithilfe von Gleichung \eqref{eqn:Ladung} und \eqref{eqn:Radius} die Ladung $q$ und den Radius $r$ aller Tropfen zu 
bestimmen. \\
\noindent Anschließend wird eine korrigierte Ladung $q_\text{eff}$ über den Cunningham-Korrekturterm bestimmt. 
Dazu wird die bereits errechnete Ladung $q$ und der Radius $r$ benutzt, um mithilfe von \eqref{eqn:cunningLad} jene korrigierte Ladung zu errechnen.
Zur groben Einordnung der Daten können diese nun einmal visualisiert werden.

\begin{figure}
    \centering
    \includegraphics[width=\textwidth]{../build/Ladungsauftragung.pdf}
    \caption{Verteilung korrigierte und unkorrigierter Ladungen}
    \label{fig:Ladungsverteilung}
\end{figure}

\noindent Zu sehen sind hier korrigierten und unkorrigierten Ladungen mit den jeweiligen Fehlern. Die vier vertikalen Linien sind jeweils eine 
Elementarladung auseinander. Wichtig zu erwähnen ist, dass die Nummer der Ladung nicht der Nummer des Tröpfchens entspricht.
Lässt man den fünften Wert beiseite, kann die Anschauung eine Größenordnung kleiner werden. 

\begin{figure*}
    \centering
    \includegraphics[width=\textwidth]{../build/Ladung_exkludiert.pdf}
    \caption{Verteilung korrigierte und unkorrigierter Ladungen}
\end{figure*}

\noindent Auch hier sind die vier vertikalen Balken wieder eine Elementarladung auseinander.
Auffällig sind die Fehlerbalken, die sich teilweise über \qty{4e-18}{\coulomb} erstrecken. \\
\noindent Diese Ladungen sind allerdings absolut zu sehen. Sie geben an sich noch keinen Aufschluss über die Elementarladung.
Um jedoch die Elementarladung zu bestimmen, wird ermittelt, was der größte Teiler der Ladung ist, welcher voraussetzt, dass der Quotient 
immer noch größer als die Elementarladung ist. Dies darf gemacht werden, da es nur ganzzahlige Vielfache der Elementarladung gibt. 
Mehr zu diesem Verfahren in \autoref{sec:Diskussion}. 
Dies wird für jeden Wert der korrigierten, sowie der unkorrigierten Ladung durchgeführt. 
Anschließend wird jede Ladung durch den jeweiligen Teiler geteilt. Jede Ladung gibt nun die Elementarladung an.
Um den Mittelwert aller Elementarladungen zu berechnen wird eine lineare Regression angewandt, welche den Fehler der einzelnen Elementarladungen 
berücksichtigt. Grafisch kann die veranschaulicht werden:

\begin{figure*}
    \includegraphics[width=\textwidth]{../build/e0_mean.pdf}
    \caption{Mittelwert berechnen unter Rücksicht auf Fehler}
\end{figure*}

\noindent Die Fehlerbalken des internen 10. und 12. Tropfen sind so groß, dass die y-Skala zu groß skaliert ist. Aus diesem Grund wird 
dieselbe lineare Regression ohne jene Tropfen noch einmal aufgetragen. Es bleibt also der Mittelwert gleich. Die Tropfen werden nur zur 
besseren Visualisierung weggelassen.

\begin{figure*}
    \includegraphics[width=\textwidth]{../build/e0_reduziert.pdf}
    \caption{Mittelwert berechnen unter Rücksicht auf Fehler}
\end{figure*}

\noindent Jetzt ist deutlich, dass der berechnete Mittelwert innerhalb der Fehlertoleranzen liegt. Der y-Achsenabschnitt dieser horizontalen 
Ausgleichskurve gibt die berechnete Elementarladung an:

\begin{align*}
    q            &= \left(1.629\pm0.012\right)\cdot10^{-19}\,\unit{\coulomb} \\
    q_\text{eff} &= \left(1.590 \pm0.040\right)\cdot10^{-19}\,\unit{\coulomb} 
\end{align*}



\end{document}
