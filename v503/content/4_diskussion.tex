\input{../../header.tex}

\begin{document}
\section{Diskussion}
\label{sec:Diskussion}

%Ladung ändert sich durch Korrektur teilweise um ein e
%durch große Fehlerbalken keine Häufungswerte erkennbar
%Objektiv bei Versuchsdurchführung beschlagen
%Gegenlicht hat Tröpfchenbeobachtung erschwert
%systematischer Fehler durch Start-Stop ansagen
%für Auswertung quantisierung der Elementarladung und deren Wert benutzt, da sonst nicht anders möglich (weil zu große Fehler)
Wie in \autoref{fig:Ladungsverteilung} zu sehen ist, erstreckt sich der Fehler einiger Ladungen um über das 16-fache der Elementarladung. 
Diese großen Unsicherheiten haben zur Folge, dass keine Häufungswerte ersichtlich werden. Bei einer theoretischen Zuordnung zu einem Häufungswert wären 
die Ladungen in mehreren Häufungswerten gleichzeitig. Dies ist keine Sinnvolle Methode.
Aufgrund der eben genannten Probleme können keine verlässlichen Werte durch gemeinsame Teiler bestimmt werden. Es muss daher auf den 
Literaturwert der Elementarladung zurückgegriffen werden, um weiter Ergebnisse zu bestimmen. Dieser wird jedoch nur als kleinster Teiler genommen. 


\end{document}
