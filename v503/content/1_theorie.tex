\input{../../header.tex}

\begin{document}

\section{Zielsetzung}
\label{sec:Zielsetzung}

Das Ziel des im folgend protokollierten Versuchs besteht darin die Elementarladung $e_0$ eines Elektrons
zu bestimmen. Dazu wird die Öltröpfchenmethode von Robert Andrew Millikan durchgeführt.

\section{Theorie}
\label{sec:Theorie}

Zum Zwecke der Bestimmung der Elementarladung werden feine Öltröpfchen in das vertikale und approximativ 
homogene elektrische Feld eines Plattenkondensators zerstäubt. Bei diesem Vorgang werden die Tröpfchen
aufgrund von Reibung elektrisch geladen. Zudem wirken bei abgeschaltetem Kondensator zwei Kräfte auf die 
Öltröpfchen. Zum einen bewegt sich ein Tröpfchen der Masse $m$ angesichts der Gravitationskraft
$\vec{F}_\text{G} = m\vec{g}$ nach unten, zum anderen bewirkt die Viskosität der Luft $\eta_\text{L}$ eine 
Reibungskraft $\vec{F}_\text{R} = -6\pi{}r\eta{}_\text{L}\vec{v}$ entgegen der Bewegungsrichtung der Masse. 
Hierbei bezeichnet $r$ den Radius eines Öltröpfchens. Wenn das Tröpfchen nach einer Phase der Beschleunigung 
eine konstante Geschwindigkeit $v_0$ annimmt, gilt das Kräftegleichgewicht

\begin{equation}
\label{eqn:Gewicht1}
    \frac{4\pi}{3}r³\left(\rho_\text{Oel}-\rho_\text{L}\right)g = 6\pi{}\eta_\text{L}rv_0.
\end{equation}

\noindent Der Korrekturterm $\rho_\text{L}$ repräsentiert den Auftrieb, den das Öltröpfchen bei der
Fallbewegung erhält. Nach Umstellen der Gleichung \eqref{eqn:Gewicht1} folgt der Tröpfchenradius:

\begin{equation*}
%\label{eqn:Radius}
    r = \sqrt{\frac{9\eta_\text{L}v_0}{2g\left(\rho_\text{Oel}-\rho_\text{L}\right)}}
\end{equation*}

\noindent Beim Anschluss des Kondensators an eine Spannungsquelle entsteht ein, je nach Polung, gerichtetes 
elektrisches Feld $\vec{E}$. Dieses Feld bewirkt eine zusätzliche elektrische Kraft $\vec{F}_\text{el} = g\vec{E}$
auf die Öltröpfchen, welche abhängig von der Polung der Platten über die Bewegungsrichtung der Tröpfchen entscheidet.
Diese Situation wird in Abbildung \ref{fig:GewichtV} veranschaulicht.

\begin{figure}
    \centering
    \includegraphics[width=\textwidth]{GewichtV.png}
    \caption{Gleichgewichtslage eines Öltröpfchens im homogenen E-Feld \cite{Versuchsanleitung_v503}.}
    \label{fig:GewichtV}
\end{figure}

\noindent Bei einer positiv geladenen unteren Platte wirkt die elektrische Kraft $\vec{F}_\text{el}$
in Richtung der Gravitationskraft. Dementsprechend sinken die Öltröpfchen mit einer Geschwindigkeit $\vec{v}_\text{ab}$
nach unten, was die Kräftegleichung 

\begin{equation}
\label{eqn:Gewicht2}
    \frac{4\pi}{3}r³\left(\rho_\text{Oel}-\rho_\text{L}\right)g - 6\pi{}\eta_\text{L}rv_\text{ab} = -qE
\end{equation}

\noindent zur Folge hat. Bei entgegengesetztem elektrischen Feld steigen die Tröpfchen auf, weshalb die
elektrische Kraft der Gravitation und Reibung entgegenwirkt. Die Kräftegleichung ändert sich somit zu 

\begin{equation}
\label{eqn:Gewicht3}
\frac{4\pi}{3}r³\left(\rho_\text{Oel}+\rho_\text{L}\right)g + 6\pi{}\eta_\text{L}rv_\text{ab} = qE.
\end{equation}

\noindent Aus den Gleichungen \eqref{eqn:Gewicht2} und \eqref{eqn:Gewicht3} folgt der Ausdruck

\begin{equation}
\label{eqn:Ladung}
    q = 3\pi{}\eta_\text{L}\sqrt{\frac{9\eta_\text{L}\left(v_\text{ab}-v_\text{auf}\right)}{4g\left(\rho_\text{Oel}-\rho_\text{L}\right)}}\cdot\frac{\left(v_\text{ab}+v_\text{auf}\right)}{E}
\end{equation}

\noindent für die Ladung und daraus

\begin{equation}
\label{eqn:Radius}
    r = \sqrt{\frac{9\eta_\text{L}\left(v_\text{ab}-v_\text{auf}\right)}{2g\left(\rho_\text{Oel}-\rho_\text{L}\right)}}
\end{equation}

\noindent für den Tröpfchenradius.\\

\noindent Zuletzt muss jedoch angebracht werden, dass die Stokes'sche Reibung ausschließlich für Tröpfchen gilt, deren Abmessung
größer als die mittlere freie Weglänge $\bar{l}$ in Luft ist. Diese Voraussetzung kann jedoch nicht ohne Weiteres
angenommen werden, weshalb die Viskosität $\eta_\text{L}$ von Luft durch den \emph{Cunningham-Korrekturterm} 

\begin{equation*}
%\label{eqn:cunningVisk}
    \eta_\text{eff} = \eta_\text{L}\left(\frac{1}{1 + \frac{B}{pr}}\right)
\end{equation*}

\noindent verändert werden muss. Hierbei ist $B = 822.599\cdot{}10{-5}\,\unit{\pascal\meter}$. Daraus folgt die korrigierte 
Ladung $q_\text{eff}$ mit 

\begin{equation}
\label{eqn:cunningLad}
    q_\text{eff} = q\left(\frac{1}{\left(1+\frac{B}{pr}\right)^{\frac{3}{2}}}\right).
\end{equation}

\section{Versuchsaufbau}
\label{sec:Versuchsaufbau}

\noindent Der allgemeine Versuchsaufbau wird in der untenstehenden Abbildung \ref{fig:Versuchsaufbau} dargestellt. Hauptbestandteil des Experiments
ist die Millikan-Kammer, welche den Kondensator mit Plattenabstand $d = \left(7.6250\pm0.0051\right)
\,\unit{\milli\meter}$ enthält. Die Öltröpfchen, welche ,durch eine Öffnung in der oberen Platte der Kammer,
in den Kondensator gelangen, besitzen eine Dichte von $\rho_\text{Oel}= 886\,\unit[per-mode=reciprocal]{\kilo\gram\per\cubic\meter}$ und 
werden durch eine Halogenlampe angestrahlt. Somit ist die Tröpfchenbewegung mittels des Mikroskops einfacher zu erkennen. Wie bereits 
in Kapitel \ref{sec:Theorie} erwähnt, sind die meisten Tröpfchen bereits durch Reibung elektrisch geladen. Falls dies 
jedoch vereinzelnd nicht der Fall ist, kann ggf. ein radioaktives Präparat angewendet werden, um die Ladung zu ändern.

\begin{figure}
    \centering 
    \includegraphics[height=6cm]{Aufbau.png}
    \caption{Experimenteller Versuchsaufbau des Millikan-Versuchs \cite{Versuchsanleitung_v503}.}
    \label{fig:Aufbau}
\end{figure}

\section{Versuchsdurchführung}
\label{sec:Versuchsdurchführung}

\noindent Vor Durchführung des Experiments wird mit Hilfe der Libelle die waagerechte Einstellung der Apparatur überprüft. Im 
Folgenden kann bei korrekter Anlegung der Spannungsquelle das E-Feld über einen Hebel aufgebaut werden. Zusätzlich wird sich mit 
dem Multimeter vertraut gemacht, welcher den Widerstand des Thermistors bei potentieller Erwärmung misst.\\

Im ersten Schritt des Versuchs werden die Öltröpfchen in das elektrische Feld des Kondensators zerstäubt. Im Rahmen der Zeitmessung 
müssen jedoch zunächst mehrere Einstellungen verändert werden, um die bestmögliche Betrachtungsperspektive gewährleisten zu 
können. Dazu zählt zum einen das Scharfstellen der Tröpfchenebene mit einem Draht wie in Abb. \ref{fig:Aufbau} beschrieben und zum
anderen die manuelle Anpassung des Mikroskops. Nachdem ein hinreichend langsames Tröpfchen gefunden wird, kann die Messung beginnen.
Hierzu wird eine geeignete Strecke auserkiesen, welche das Tröpfchen zurücklegt. Gleichzeitig wird die dafür benötigte Zeit gemessen
und das Experimentierheft übertragen, sodass die Geschwindigkeit ermittelt. Nach Überschreiten dieser Strecke werden die Platten
des Kondensators umgepolt, was dem Bewegungstrend des Tröpfchens entgegensetzt. Erneut wird für dieses spezielle Tröpfchen die für 
die Strecke benötigte Zeit gemessen und notiert. Dieser Vorgang wird so lange wiederholt bis maximal fünf Auf- und 
Abwärtsbewegungen betrachtet und gemessen werden. Zusätzlich wird nach jeder erfolgreichen Messung eines Tröpfchens der Widerstand 
des Thermistors abgelesen und aufgeschrieben.\\

\noindent Insgesamt werden die Geschwindigkeiten von 15 verschiedenen Tröpfchen genauer untersucht, welche zur Berechnung der
Elementarladung konsultiert werden können. Im Anschluss werden die Platten des Kondensators gesäubert und der Arbeitsplatz geordnet
hinterlassen.
\end{document}